\documentclass[answers]{exam}
\usepackage[english]{babel}
\usepackage[utf8x]{inputenc}
\usepackage{amsmath,amssymb,amsthm}
\usepackage{graphicx}

\begin{document}

\noindent{OPER 610 - Linear Programming %
	% Homework 2 %
	(Due Jan 10 at 10am)}
\hspace{\fill} { B. Hosley}
\bigskip

\begin{questions}

%%%%%%%%%%%%%%%%%%%%%%%%%%%
%	\begin{ Question 31}  %
%%%%%%%%%%%%%%%%%%%%%%%%%%%
\setcounter{question}{30}
\question 
Consider the following problem:
\begin{flalign*}
	\operatorname{Maximize} \hspace{4ex} x_1 \ - \ x_2 &&& \\ \text{subject to} \hspace{2ex}
	-x_1 + 3x_2 &\leq  0  && \\
	-3x_1 +2x_2 &\geq 35 && \\
	x_1, \ \quad x_2 &\geq 0 && \\
\end{flalign*}
\begin{parts}
	\part Sketch the feasible region in the \((x_1, x_2 )\) space.
	\part Identify the regions in the \((x_1, x_2)\) space where the slack variables \(x_3\) and \(x_4\) , say, are equal to zero.
	\part Solve the problem geometrically.
	\part Draw the requirement space and interpret feasibility.
\end{parts}

	

\begin{solution}
	\begin{parts}
		\part		
			The feasible region is the (greyish) area bound by the red, blue, orange triangle
			\\ \includegraphics[width=0.7\linewidth]{"images/screen1"}
		\part
			If a slack \(x_3\) is added to the first constraint, then \(x_3=0\) along the red line.
			\\ \includegraphics[width=0.7\linewidth]{"images/Screenshot 2023-01-07 at 6.38.18 PM"} \\
			If a slack \(x_4\) is added to the second constraint, then \(x_4=0\) along the blue line.
			\\ \includegraphics[width=0.7\linewidth]{"images/Screenshot 2023-01-07 at 6.38.37 PM"}
		\part
			The maximum value is 1, obtained by the right-most iso-contribution line. 
			Which gives the solution \((x_1=1,x_2=0)\).
			\\ \includegraphics[width=0.9\linewidth]{"images/Screenshot 2023-01-07 at 8.26.43 PM"}
		\part
			The requirement space is shown below. (Due to graphing tool limitation the arrows are rendered as dots)
			Because \(a_1\) and \(a_2\) are in opposite quadrants, together they span the whole 2D space.
			\\ \includegraphics[width=0.4\linewidth]{"images/Screenshot 2023-01-07 at 10.17.35 PM"}
	\end{parts}
\end{solution}
%\end{ Question 31}

%%%%%%%%%%%%%%%%%%%%%%%%%%%
%	\begin{ Question 41}  %
%%%%%%%%%%%%%%%%%%%%%%%%%%%
\question 
Consider the problem: Minimize \(cx\) subject to 
\(Ax \geq b, x \geq 0\) Suppose that a new constraint is added to the problem.
\begin{parts}
	\part What happens to the feasible region?
	\part What happens to the optimal objective value \(z*\)?
\end{parts}
\begin{solution}
	\begin{parts}
		\part
			Adding a new constraint will either:
			do nothing, if the constraint's region overlaps the original feasible region;
			shrink the feasible region, if the constraint crosses the original feasible region;
			or make the problem infeasible, if the new constraint's region does not intersect with the original feasible region at all.
		\part
			If the new constraint's region overlaps original optimal objective,
			then the optimal objective value will remain the same.
			If the new constraint's region intersects with the original feasible region,
			but does not include the original optima, then the new optimal objective
			will shift to be within the new feasible region.
			If the new constraint's region does not intersect with the original feasible region,
			then the problem becomes infeasible.
	\end{parts}
\end{solution}
%\end{ Question 41}

%%%%%%%%%%%%%%%%%%%%%%%%%%%
%	\begin{ Question 42}	  %
%%%%%%%%%%%%%%%%%%%%%%%%%%%
\question 
Consider the problem: Minimize \(cx\) subject to 
\(Ax \geq b, x \geq 0\) Suppose that a new variable is added to the problem.
\begin{parts}
	\part What happens to the feasible region?
	\part What happens to the optimal objective value \(z*\)?
\end{parts}
\begin{solution}
	\begin{parts}
		\part
			The addition of another variable increases the dimensionality of constraints and the feasible region by one.
		\part
			The optimal objective value may improve if the new higher dimension feasible region containsa more optimal point.
			If it does not, the optimal will remain the same.
			The problem will not be made infeasible with the addition of another variable, it is sufficient to set the new variable's value to zero to retain the original optimal value.	
	\end{parts}
\end{solution}
%\end{ Question 42}

\end{questions}
\end{document}