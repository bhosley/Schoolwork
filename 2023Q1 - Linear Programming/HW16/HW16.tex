\documentclass[12pt]{amsart}
\usepackage[left=0.5in, right=0.5in, bottom=0.75in, top=0.75in]{geometry}
\usepackage[english]{babel}
\usepackage[utf8x]{inputenc}
\usepackage{amsmath,amssymb,amsthm}
\usepackage{enumerate}
\usepackage{graphicx}

\begin{document}
\raggedbottom

\noindent{\large OPER 610 - Linear Programming %
	% Lesson 14 %
	(Due Mar 3 at 10am)}
\hspace{\fill} {\large B. Hosley}
\bigskip


%%%%%%%%%%%%%%%%%%%%%%%
\setcounter{section}{1}
\setcounter{subsection}{24}
\subsection{} 

Suppose that there are \(m\) sources that generate waste and \(n\) disposal sites. 
The amount of waste generated at source \(i\) is \(a_i\) and the capacity of site \(j\) is \(b_j\). 
It is desired to select appropriate transfer facilities from among \(K\) candidate 
facilities. Potential transfer facility \(k\) has a fixed cost \(f_k\), capacity \(q_k\), and unit 
processing cost \(a_k\) per ton of waste. Let \(c_{ik}\) and \(\bar c_{kj}\) be the unit shipping costs 
from source \(i\) to transfer station \(k\) and from transfer station \(k\) to disposal site  \(j\), 
respectively. The problem is to choose the transfer facilities and the shipping 
pattern that minimize the total capital and operating costs of the transfer stations 
plus the transportation costs. Formulate this \textit{distribution problem}. (Hint: Let \(y_k\) 
be \(1\) if transfer station \(k\) is selected and \(0\) otherwise.) \\ \bigskip

% Define sets, parameters, DVs, assumptions!
% 	Note: you will need binary decision variables in addition to 
% 	non-negative decision variables.  See the authors’ hint.
% Second, use set-based notation and indexing of DVs and constraints to present a compact formulation

\textbf{(Phase 1) Solution:} \\

\textbf{\underline{Assumptions}:}

\begin{itemize}
	\item The normal LP assumptions, and;
	\item The whole cost of a processing facility is incurred even at minimal usage.
	\item It is assumed that while the facilities are bound by capacity, the transportation
	between facilities is not.
	\item That the cost of transport is always the same on each route.
	\item That the cost of transportation scales perfectly linearly.
	\item Capacity of disposal and processing must be greater than what is produced, 
	otherwise the problem is will necessarily change to a multi-objective problem.
	\item That the value of the processes that generate the waste produces equal revenue
	per units.
\end{itemize} \bigskip

\textbf{\underline{Sets}:} 
\begin{alignat*}{2}
	I &= \{1,2,\ldots,m\}\ & &:\ m \text{ sources } i \hspace{30ex} \\
	J &= \{1,2,\ldots,n\}\ & &:\ n \text{ disposal sites } j  \\
	K &= \{1,2,\ldots,o\}\ & &:\ o \text{ transfer facilities } k 
\end{alignat*} 

\textbf{\underline{Decision Variables}:} \\ 

\noindent
\(x_{i,k}\quad\) units of waste transferred from source \(m\) to processing facility \(k\). \\
\(x_{k,j}\quad\) units of waste transferred from processing facility \(k\) to disposal site \(j\). \\
\(y_k\qquad\!\) wherein \(y_k=1\) implies that facility \(k\) is being used,
and \(y_k=0\) that facility \(k\) is not being used. \\

\textbf{\underline{Parameters}:} \\

\noindent
\(a_i\quad\) units of waste generated at source \(i\). \\
\(\alpha_k\quad\) cost to process unit weight at facility \(k\). \\
\(b_j\quad\) units of waste disposal site \(j\) can accept. \\
\(c_{i,k}\quad\) shipping cost to transfer waste from source \(i\) to processing facility \(k\). \\
\(c_{k,j}\quad\) shipping cost to transfer waste from processing facility \(k\) to disposal site \(j\). \\
\(f_k\quad\) the cost of utilizing processing facility \(k\). \\
\(q_k\quad\) the capacity for processing at facility \(k\). \\

\clearpage

\textbf{\underline{Instantiation Parameters}:} \\

\noindent
To be valid under a single objective the following must be true
\begin{align*}
	\sum_{i\in I}a_i \leq \sum_{k\in K}q_k \quad\text{and}\quad
	\sum_{i\in I}a_i \leq \sum_{j\in J}b_j .
\end{align*}
It will likely be best to consider these when developing the
constraint instantiation functions. \\

\textbf{\underline{Program}:} \\

\[ \min \qquad \sum_{k\in K} \left[ y_kf_k + \sum_{i\in I} x_{i,k}(c_{i,k}+\alpha_k) + \sum_{j\in J} x_{k,j}c_{k,j} \right]
 	\hspace{45ex} \]
\begin{align*} 
	\text{s.t.}\
	\sum_{i\in I}x_{i,k} &= \sum_{j\in J} x_{k,j} & & \forall k\in K & & \text{\footnotesize(Conservation of waste mass),}  \\
	\sum_{k\in K} x_{i,k} &= a_i & & \forall\ i\in I & & \text{\footnotesize(All generated waste flows out of source),}  \\
	\sum_{i\in I} x_{i,k} &\leq y_kq_k & & \forall\ k\in K & & \text{\scriptsize(Waste leaving sources does not exced processing facility capacity),}  \\
	\sum_{k\in K} x_{k,j} &\leq b_j & & \forall\ j\in J & & \text{\footnotesize(Waste arriving at disposal does not exceed capacity),}  \\
	x_{i,k},\ x_{k,j} &\geq 0 & & \forall i\in I,\ j\in J,\ k\in K & & \text{\footnotesize(Non-negativity),} \\
	y_k \in \{0,&1\} & &\forall\ k\in K & & (y\text{ \footnotesize must be binary}).
\end{align*}


\end{document}