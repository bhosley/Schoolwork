\documentclass[12pt]{amsart}
\usepackage[left=0.5in, right=0.5in, bottom=0.75in, top=0.75in]{geometry}
\usepackage[english]{babel}
\usepackage[utf8x]{inputenc}
\usepackage{amsmath,amssymb,amsthm}
\usepackage{enumerate}
\usepackage{graphicx}

\begin{document}
\raggedbottom

\noindent{\large OPER 610 - Linear Programming %
	% Lesson X %
	(Due Feb X at 10am)}
\hspace{\fill} {\large B. Hosley}
\bigskip


%%%%%%%%%%%%%%%%%%%%%%%
\setcounter{section}{6}
\setcounter{subsection}{26} % Note Errata
\subsection{}
Two players are involved in a competitive game. One player, called the 
row player, has two strategies available; the other player, called the column 
player, has three strategies available. If the row player selects strategy \(i\) and the 
column player selects strategy  \(j\) , the payoff to the row player is \(c_{ij}\) and the 
payoff to the column player is \(-c_{ij}\). Thus, the column player loses what the row 
player wins and vice versa; this is a \textit{two—person zero-sum game}. The following 
matrix gives the payoffs to the row player: 

\bigskip
\begin{center}
	\begin{tabular}{cccc}
		& 1                       & 2                       & 3                      \\ \cline{2-4} 
		\multicolumn{1}{c|}{1} & \multicolumn{1}{c|}{2}  & \multicolumn{1}{c|}{-1} & \multicolumn{1}{c|}{0} \\ \cline{2-4} 
		\multicolumn{1}{c|}{2} & \multicolumn{1}{c|}{-3} & \multicolumn{1}{c|}{2}  & \multicolumn{1}{c|}{1} \\ \cline{2-4} 
	\end{tabular}
\end{center}
\bigskip

Let \(x_1, x_2\), and \(x_3\) be probabilities with which the column player will select the 
various strategies over many plays of the game. Thus \(x_1 + x_2 + x_3 = 1, x_1, x_2, 
x_3 > 0\). If the column player applies these probabilities to the selection of her 
strategy for any play of the game, consider the row player's options. If the row 
player selects row 1, then her expected payoff is \(2x_1 - x_2\). If the row player 
selects row 2, her payoff is \(-3x_1 + 2x_2 + x_3\). Wishing to minimize the maximum 
expected payoff to the row player, the column player should solve the following 
linear program: 

\begin{alignat*}{6}
	\text{Minimize}\quad\ z\ \ & {} {} &\qquad& {} {} &\qquad& {} {} &\qquad&{}\qquad    {}  &   & \\
	\text{subject to}\quad\,\ 
	  x_1& {}+{} &  x_2& {}+{} & x_3  &{} =  {}  & 1 & \\
	 2x_1& {}-{} &  x_2& {} {} &      &{}\leq{}  & z & \\
	-3x_1& {}+{} & 2x_2& {}+{} & x_3  &{}\leq{}  & z & \\
	  x_1&,{} {} &  x_2&,{} {} & x_3  &{}\geq{}  & 0 & \\
	     & {} {} &     & {} {} &   z  &{}    {}  &   & \text{unrestricted}. 
\end{alignat*}

Transposing the variable z to the left-hand-side, we get the column player's 
problem: 

\begin{alignat*}{6}
\text{Minimize}\quad   z  & {} {} &\qquad& {} {} &\qquad& {} {} &\qquad&{}\qquad    {}  &   & \\
\text{subject to}\quad\,\ & {} {} &  x_1& {}+{} &  x_2& {}+{} & x_3  &{} =  {}  & 1 & \\
	                   z  & {}-{} & 2x_1& {}+{} &  x_2& {} {} &      &{}\geq{}  & 0 & \\
	                   z  & {}+{} & 3x_1& {}-{} & 2x_2& {}-{} & x_3  &{}\geq{}  & 0 & \\
	                      & {} {} &  x_1&,{} {} &  x_2&,{} {} & x_3  &{}\geq{}  & 0 & \\
	                      & {} {} &     & {} {} &     & {} {} &   z  &{}    {}  &   & \text{unrestricted}. 
\end{alignat*}

\begin{enumerate}[a.]
	\item Give the dual of this linear program. 
	\item Interpret the dual problem in Part (a). (\textit{Hint}: Consider the row player's problem.) 
	\item Solve the dual problem of Part (a). (\textit{Hint}: This problem may be solved graphically.) 
	\item Use the optimal dual solution of Part (c) to compute the column player's probabilities. 
	\item Interpret the complementary slackness conditions for this two person zero-sum game.
\end{enumerate}

\clearpage
\textbf{Solution:}
\begin{enumerate}[a.]
	\item
		\begin{alignat*}{6}
			\text{Maximize}\quad\ w & {} {} &\qquad& {} {} &\qquad& {} {} &\qquad&{}\qquad    {}  &   & \\
			\text{subject to}\qquad\,
		 	  & {} {} &  y_1& {}+{} &  y_2 &{} =  {}  & 1 & \\
			w & {}-{} & 2y_1& {}+{} & 3y_2 &{}\leq{}  & 0 & \\
			w & {}-{} &  y_1& {}-{} & 2y_2 &{}\leq{}  & 0 & \\
			w & {} {} &     & {}-{} & y_2  &{}\leq{}  & 0 & \\
			  & {} {} &  y_1&,{} {} & y_2  &{}\geq{}  & 0 & \\
			  & {} {} &     & {} {} &   w  &{}    {}  &   & \text{unrestricted}. 
		\end{alignat*} \\
	
	\item 
		The dual represents the optimal moves for the column player to take,
		if the column player wishes to maximize the row player's score
		and minimize their own score. \\
		
	\item
		\(\quad y_1 = \frac{2}{3},\quad y_2 = \frac{1}{3},\quad w = \frac{1}{3}\) \\
		
	\item 
		\(\quad x_1 = \frac{1}{6},\quad x_2 = 0,\quad x_3 = \frac{5}{6},\quad z = \frac{1}{3}\) \\
	
	\item
		Although the slack variables are only implied above, let the subscript \(_s\) 
		represent the corresponding slack variables for both systems.
		Then we have
		\[x_iy_{si}=0, \quad\text{and}\quad x_{si}yi = 0.\]		
		
		These represent that the strategy is optimal,
		the slack variables represent the expected points that the opponent may gain 
		through the implementation of the suboptimal strategy.
		
\end{enumerate}

\end{document}