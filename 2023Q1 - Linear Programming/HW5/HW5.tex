\documentclass[12pt]{amsart}
\usepackage[left=0.5in, right=0.5in, bottom=0.75in, top=0.75in]{geometry}
\usepackage[english]{babel}
\usepackage[utf8x]{inputenc}
\usepackage{amsmath,amssymb,amsthm}
\usepackage{enumerate}
\usepackage{graphicx}

\usepackage{booktabs}
\usepackage{ragged2e}
\def\labelitemi{-}

\begin{document}
\raggedbottom

\noindent{\large OPER 610 - Linear Programming %
	% Lesson 5 %
	(Due Jan 24 at 10am)}
\hspace{\fill} {\large B. Hosley}
\bigskip


%%%%%%%%%%%%%%%%%%%%%%%
\setcounter{section}{1}
\setcounter{subsection}{12}
\subsection{}
A production manager is planning the scheduling of three products on four machines. 
Each product can be manufactured on each of the machines. 
The unit production costs (in \$) are summarized below. \\

\begin{center}
	\begin{tabular}{c|cccc}
		& \multicolumn{4}{c}{Machine} \\
		\midrule
		PRODUCT & 1 & 2 & 3 & 4 \\
		\midrule
		1 & 4 & 4 & 5 & 7 \\
		2 & 6 & 7 & 5 & 6 \\
		3 &12 &10 & 8 &11 
	\end{tabular}
\end{center} \phantom{text}

The time (in hours) required to produce a unit of each product on each of the machines is summarized below.

\begin{center}
	\begin{tabular}{c|cccc}
		& \multicolumn{4}{c}{Machine} \\
		\midrule
		PRODUCT & 1 & 2 & 3 & 4 \\
		\midrule
		1 & 0.3 & 0.25 & 0.2 & 0.2 \\
		2 & 0.2 & 0.3 & 0.2 & 0.25 \\
		3 & 0.8 & 0.6 & 0.6 & 0.5  
	\end{tabular} 
\end{center} \phantom{text}

Suppose that 3000, 6000, and 4000 units of the products are required, 
and that the available machine-hours are 1500, 1200, 1500, and 2000, respectively. 
Formulate the scheduling problem as a linear program. \\

\hrule
\medskip

\textit{Solution:} \\

\begin{enumerate}[a)]
\item Assumptions:
	\begin{itemize}
		\item The objective is to meet (not exceed) demand while minimizing operating costs.
		\item Non-negativity for all variables.
		\item Linearity of the functional output of the machines.
	\end{itemize}
	
\item Sets:
	\begin{itemize}
		\item Products: \(P = \{1,2,3\}\)
		\item Machines: \(M = \{1,2,3,4\}\)
	\end{itemize}
	
\item Parameters:
	\begin{itemize}
		\item \(c_{pm}\): Production Cost (a matrix drawn from the first table)
		\item \(t_{pm}\): Hours to produce units of product (a matrix drawn from the second table)
		\item \(u_p\)   : Units of product \(p\) required
		\item \(h_m\)   : Hours machines \(m\) available
	\end{itemize}
	
\item Decision variables:
	\begin{itemize}
		\item Number of products \(p\) manufactured on machine \(m\): \(x_{pm}\)
	\end{itemize}

\item Explicit Formulation
	\begin{flalign*}
		\operatorname{min} \quad
		 4x_{11} +\,\ 4x_{12} + 5x_{13} +\,\ 7x_{14} +    & \\
		 6x_{21} +\,\ 7x_{22} + 5x_{23} +\,\ 6x_{24} +    & \\
		12x_{31} +   10x_{32} + 8x_{33} +   11x_{34} \ \  &\\
		\text{s.t.} \hspace{12ex}
		x_{11} + x_{12} + x_{13} + x_{14} &= 3000 \\
		x_{21} + x_{22} + x_{23} + x_{24} &= 6000 \\
		x_{31} + x_{32} + x_{33} + x_{34} &= 4000 \\
		0.3x_{11}  +\,\ 0.2x_{21}  + 0.8x_{31} &\leq 1500 \\
		0.25x_{12} +\,\ 0.3x_{22}  + 0.6x_{32} &\leq 1200 \\
		0.2x_{13}  +\,\ 0.2x_{23}  + 0.6x_{33} &\leq 1500 \\
		0.2x_{14}  +   0.25x_{24}  + 0.5x_{34} &\leq 2000 \\
		x_{pm} &\geq 0
	\end{flalign*}

\item Compact Formulation
	\begin{flalign*}
		\operatorname{min}\quad \sum_{p\in P} &\sum_{m\in M} c_{pm}x_{pm}  \\
		\text{s.t. } \hspace{6ex} &\sum_{m\in M} x_{pm} = u_p  \qquad \text{for } p\in P \\
		&\sum_{p\in P} t_{pm}x_{pm}  \leq h_m  \quad \text{for } m\in M \\
		& x_{pm} \geq 0 \hspace{11ex} \text{for } p\in P  \text{ and } m\in M
	\end{flalign*}


\end{enumerate}

\end{document}