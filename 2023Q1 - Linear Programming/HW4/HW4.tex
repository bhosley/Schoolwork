\documentclass[12pt]{amsart}
\usepackage[left=0.5in, right=0.5in, bottom=0.75in, top=0.75in]{geometry}
\usepackage[english]{babel}
\usepackage[utf8x]{inputenc}
\usepackage{amsmath,amssymb,amsthm}
\usepackage{enumerate}
\usepackage{graphicx}

\usepackage{nicefrac}
\usepackage{xcolor,cancel}
\newcommand\hcancel[2][red]{\setbox0=\hbox{$#2$}%
	\rlap{\raisebox{.45\ht0}{\textcolor{#1}{\rule{\wd0}{1pt}}}}#2} 

\begin{document}
	\raggedbottom

\noindent{\large OPER 610 - Linear Programming %
	% Lesson 6 %
	(Due Jan 20 at 10am)}
\hspace{\fill} {\large B. Hosley}
\bigskip


%%%%%%%%%%%%%%%%%%%%%%%
\setcounter{section}{2}
\setcounter{subsection}{23}
\subsection{}
Find all basic solutions of the following system:
\begin{align*}
	-x_1 + 2x_2  + x_3   + 3x_4 -    2x_5 &= 4 \\
	 x_1 - 2x_2 \qquad\, + 2x_4 + \,\ x_5 &= 2.
\end{align*}

% Begin Answer
\hrule \medskip
Solution:

Converting the problem into matrix form we get,
\begin{align*}
	\begin{bmatrix}
		-1 &  2 & 1 & 3 & -2 \\
		 1 & -2 & 0 & 2 &  1
	\end{bmatrix} =
	\begin{bmatrix}
		4 \\ 2
	\end{bmatrix} .
\end{align*}
Then breaking this matrix into sub-problems we get,
\begin{align*}
	% x1 start
	\begin{bmatrix}
		-1 &  2  \\
		 1 & -2 
	\end{bmatrix} (x_1 , x_2 ) = \begin{bmatrix} 4 \\ 2 \end{bmatrix}
		&\rightarrow \text{N/A these segments do not intersect.} \\
	%
	\begin{bmatrix}
		-1 & 1 \\
	 	 1 & 0
	\end{bmatrix}  (x_1 , x_3 ) =  \begin{bmatrix} 4 \\ 2 \end{bmatrix}
		&\rightarrow ( 2 , 0, 6 , 0,0) \\
	%
	\begin{bmatrix}
		-1 & 3 \\
		 1 & 2 
	\end{bmatrix}  (x_1 , x_4 ) =  \begin{bmatrix} 4 \\ 2 \end{bmatrix}
		&\rightarrow ( -\nicefrac{2}{5} ,0,0 \nicefrac{6}{5} ,0 ) \\
	%
	\begin{bmatrix}
		-1 & -2 \\
		 1 &  1
	\end{bmatrix}  (x_1 , x_5 ) =  \begin{bmatrix} 4 \\ 2 \end{bmatrix}
		&\rightarrow ( 8 ,0,0,0, -6 ) \\
	%
	% x2 start
	%
	\begin{bmatrix}
		 2 & 1 \\
		-2 & 0
	\end{bmatrix}  (x_2 , x_3 ) =\begin{bmatrix} 4 \\ 2 \end{bmatrix}
		&\rightarrow (0, -1 ,6 ,0 ,0 ) \\
	%
	\begin{bmatrix}
		2 & 3 \\
		-2 & 2
	\end{bmatrix}  (x_2 , x_4 ) =\begin{bmatrix} 4 \\ 2 \end{bmatrix}
		&\rightarrow (0, \nicefrac{1}{5} ,0, \nicefrac{6}{5} ,0 ) \\
	%
	\begin{bmatrix}
		2 & -2 \\
		-2 &  1
	\end{bmatrix}  (x_2 , x_5 ) =\begin{bmatrix} 4 \\ 2 \end{bmatrix}
		&\rightarrow (0, -4 ,0,0, -6 ) \\
	%
	% x3 start
	%
	\begin{bmatrix}
		1 & 3  \\
		0 & 2 
	\end{bmatrix}  (x_3 , x_4 ) =  \begin{bmatrix} 4 \\ 2 \end{bmatrix}
		&\rightarrow (0,0, 1 , 1 ,0) \\
	%
	\begin{bmatrix}
		1 & -2 \\
		0 &  1
	\end{bmatrix}  (x_3 , x_5 ) =  \begin{bmatrix} 4 \\ 2 \end{bmatrix}
		&\rightarrow (0,0, 8 ,0, 2 ) \\
	%
	% x4 start
	%
	\begin{bmatrix}
		3 & -2 \\
		2 &  1
	\end{bmatrix}  (x_4 , x_5 ) =  \begin{bmatrix} 4 \\ 2 \end{bmatrix}
		&\rightarrow (0,0,0, \nicefrac{8}{7} , -\nicefrac{2}{7} ) .
\end{align*}

\clearpage
%%%%%%%%%%%%%%%%%%%%%%%
\setcounter{subsection}{39}
\subsection{}
Find the extreme points of the region defined by the following
inequalities:
\begin{align*}
	  x_1 + 2x_2 + \,\ x_3 &\leq 5 \\
	- x_1 + \,\ x_2 + 2x_3 &\leq 6 \\
	  x_1, \quad\ x_2, \quad\ x_3 &\geq 0.
\end{align*}
(\textit{Hint}: Consider n = 3 intersecting defining hyperplanes at a time.)

% Begin Answer
\medskip \hrule \medskip
Solution:

The extreme points in this system are located at:

At the origin \((0,0,0)\)

Intersections along the axes and each multi-variable constraint individually:
\begin{align*}
	(x_1=0,x_2=0) &\rightarrow \hcancel{(0,0,5)}, (0,0,3) \\
	(x_1=0,x_3=0) &\rightarrow (0,\nicefrac{5}{2},0), \hcancel{(0,6,0)} \\
	(x_2=0,x_3=0) &\rightarrow (5,0,0), \hcancel{(-6,0,0)} \\
\end{align*}

Along the Cartesian planes and both multi-variable constraints:
\begin{align*}
	(x_1=0) &\rightarrow (0,\nicefrac{4}{3},\nicefrac{7}{3}) \\
	(x_2=0) &\rightarrow (\nicefrac{4}{3},0,\nicefrac{11}{3}) \\
	(x_3=0) &\rightarrow \hcancel{(-\nicefrac{7}{3},\nicefrac{11}{3},0)}.
\end{align*}

Grouped together, the points are,
\[
(0,0,0), \ (0,0,3), \ (0,\nicefrac{5}{2},0), \  (5,0,0), \
(0,\nicefrac{4}{3},\nicefrac{7}{3}), \  (\nicefrac{4}{3},0,\nicefrac{11}{3}).
\]


\clearpage
%%%%%%%%%%%%%%%%%%%%%%%
\setcounter{section}{3}
\setcounter{subsection}{0}
\subsection{}
Consider the following linear programming problem:
\begin{align*}
	\text{Maximize }\qquad  x_1 + 2x_2 & \\
	\text{subject to }\qquad x_1 - 4x_2 &\leq 4 \\
 	-2x_1 +\,\ x_2 &\leq 2  \\ 
 	-3x_1 +   4x_2 &\leq 12 \\
	 2x_1 +\,\ x_2 &\leq 8  \\
	  x_1,\quad\   x_2 &\geq 0	  
\end{align*}

\begin{enumerate}[a.]
	\item 
	Sketch the feasible region in the \((x_1,x_2)\) space and identify the optimal solution.
	\item 
	Identify all the extreme points and reformulate the problem in terms of convex combination of the extreme points. 
	Solve the resulting problem. 
	Is any extreme point degenerate? Explain.
	\item 
	Suppose that the fourth constraint is dropped. Identify the extreme points and directions and reformulate the problem in terms of convex combinations of the extreme points and nonnegative linear combinations of the extreme directions. Solve the resulting problem and interpret the solution.
	\item 
	Is the procedure described in Parts (b) and (c) practical for solving larger problems? Discuss.
\end{enumerate}

% Begin Answer
\medskip \hrule \medskip
Solution:

\begin{enumerate}[a.]
	\item \phantom{text} \\	
	\includegraphics[width=0.5\linewidth]{"Screenshot 2023-01-16 at 2.58.20 PM"} \\
	
	The optimal solution is \(\begin{bmatrix} 1.818 \\ 4.364 \end{bmatrix}\)
	with an objective value of \(10.546\).
	\item 
%	\begin{align*}
%		\text{Minimize} \quad  cx &    & & \\
%		\text{subject to}\quad  Ax &= b & & \\
%		                   x &>0. & \hspace{3em}& \\
%	\end{align*}
	The extreme points are
	\begin{align*}
		x_1 = \begin{bmatrix} 0 \\ 0 \end{bmatrix} \quad
		x_2 = \begin{bmatrix} 0 \\ 2 \end{bmatrix} \quad
		x_3 = \begin{bmatrix} 4 \\ 0 \end{bmatrix} \quad
		x_4 = \begin{bmatrix} \nicefrac{4}{5} \\ \nicefrac{18}{5} \end{bmatrix} \quad
		x_5 = \begin{bmatrix} \nicefrac{20}{11}\\ \nicefrac{48}{11} \end{bmatrix}.
	\end{align*}
	Among these, only \(x_3\) is degenerate, as it is an extreme point with three, or \(n+1\) meeting constraints. \\
	The objective values at each of these points is,
	\begin{align*}
		cx_1 &= (1,2)\begin{bmatrix} 0 \\ 0 \end{bmatrix} \quad  = 0 \\
		cx_2 &= (1,2)\begin{bmatrix} 0 \\ 2 \end{bmatrix} \quad  = 4 \\
		cx_3 &= (1,2)\begin{bmatrix} 4 \\ 0 \end{bmatrix} \quad  = 4 \\
		cx_4 &= (1,2)\begin{bmatrix} \nicefrac{4}{5} \\ \nicefrac{18}{5} \end{bmatrix} \quad = 8 \\
		cx_5 &= (1,2)\begin{bmatrix}\nicefrac{20}{11}\\ \nicefrac{48}{11} \end{bmatrix} = 10 \,\nicefrac{6}{11} \ . 
	\end{align*}

	Reformulated this is:
	\begin{align*}
		\text{max} \quad 0\lambda_1 + 4\lambda_2 + \nicefrac{116}{11}\lambda_3 + 8\lambda_4 + 4\lambda_5 && \\
		\text{s.t.} \hspace{13ex} \lambda_1 + \lambda_2 + \lambda_3 + \lambda_4 + \lambda_5 &= 1 & \\
					\lambda_1, \lambda_2, \lambda_3, \lambda_4, \lambda_5 &\geq 0 & \\
	\end{align*}
	
	\item \phantom{text} \\
	\includegraphics[width=0.7\linewidth]{"Screenshot 2023-01-19 at 8.10.28 PM"} \\
	Dropping the fourth constraint changes the extreme points and directions to:
	\begin{align*}
		x_1 = \begin{bmatrix} 0 \\ 0 \end{bmatrix} \quad
		x_2 = \begin{bmatrix} 0 \\ 2 \end{bmatrix} \quad
		x_3 = \begin{bmatrix} 4 \\ 0 \end{bmatrix} \quad
		x_4 = \begin{bmatrix} \nicefrac{4}{5} \\ \nicefrac{18}{5} \end{bmatrix} \quad
		d_1 = \begin{bmatrix} 4 \\ 1 \end{bmatrix} \quad
		d_2 = \begin{bmatrix} 4 \\ 3 \end{bmatrix}.
	\end{align*}

	\begin{align*}
		cx_1 &= (1,2)\begin{bmatrix} 0 \\ 0 \end{bmatrix}  = 0 \\
		cx_2 &= (1,2)\begin{bmatrix} 0 \\ 2 \end{bmatrix}  = 4 \\
		cx_3 &= (1,2)\begin{bmatrix} 4 \\ 0 \end{bmatrix}  = 4 \\
		cx_4 &= (1,2)\begin{bmatrix} \nicefrac{4}{5} \\ \nicefrac{18}{5} \end{bmatrix} \quad = 8 \\
		cd_1 &= (1,2)\begin{bmatrix} 4 \\ 1 \end{bmatrix}  = 6 \\
		cd_2 &= (1,2)\begin{bmatrix} 4 \\ 3 \end{bmatrix}  = 10 \ . 
	\end{align*}

	And from this we can see that the objective function increases indefinitely along the \(d_2\) direction;
	in this case, the slope of the third constraint (\(-3x+4y\leq12\)).
	
	Reformulated this is:
	\begin{align*}
		\text{max} \quad 0\lambda_1 + 4\lambda_2 + 4\lambda_3 +8\lambda_4 + 6\mu_1 + 10\mu_2 && \\
		\text{s.t.} \hspace{21ex} \lambda_1 + \lambda_2 + \lambda_3 +\lambda_4 &= 1 & \\
		\lambda_1, \lambda_2, \lambda_3, \lambda_4, \mu_1, \mu_2 &\geq 0 & \\
	\end{align*}
	
	\item 
	While this method does not feel practical being done by hand, 
	it is faster than solving the same problem using the tableau technique shown in OPER 510.
	The complexity of the problem grows very fast as constraints are added,
	however, the process should be well adapted to algorithmic implementation.
	
	
\end{enumerate}

\end{document}