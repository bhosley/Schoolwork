\documentclass[12pt]{amsart}
\usepackage[left=0.5in, right=0.5in, bottom=0.75in, top=0.75in]{geometry}
\usepackage[english]{babel}
\usepackage[utf8x]{inputenc}
\usepackage{amsmath,amssymb,amsthm}
\usepackage{enumerate}
\usepackage{graphicx}

\begin{document}

\noindent{\large OPER 610 - Linear Programming %
	% Lesson 6 %
	(Due Jan 20 at 10am)}
\hspace{\fill} {\large B. Hosley}
\bigskip


%%%%%%%%%%%%%%%%%%%%%%%
\setcounter{section}{2}
\setcounter{subsection}{23}
\subsection{}
Find all basic solutions of the following system:
\begin{align*}
	-x_1 + 2x_2  + x_3   + 3x_4 -    2x_5 &= 4 \\
	 x_1 - 2x_2 \qquad\, + 2x_4 + \,\ x_5 &= 2.
\end{align*}

% Begin Answer

\begin{align*}
	\left[
	\begin{array}{ccccc|c}
		-1 &  2 & 1 & 3 & -2 & 4 \\
		 1 & -2 & 0 & 2 &  1 & 2  
	\end{array}
	\right]
	\left[
	\begin{array}{ccccc|c}
		-1 & 2 & 1 & 3 & -2 & 4 \\
		 0 & 0 & 1 & 5 & -1 & 6  
	\end{array}
	\right]
\end{align*}



%%%%%%%%%%%%%%%%%%%%%%%
\setcounter{subsection}{39}
\subsection{}
Find the extreme points of the region defined by the following
inequalities:
\begin{align*}
	  x_1 + 2x_2 + \,\ x_3 &\leq 5 \\
	- x_1 + \,\ x_2 + 2x_3 &\leq 6 \\
	  x_1, \quad\ x_2, \quad\ x_3 &\geq 0.
\end{align*}
(\textit{Hint}: Consider n = 3 intersecting defining hyperplanes at a time.)


%%%%%%%%%%%%%%%%%%%%%%%
\setcounter{section}{3}
\setcounter{subsection}{0}
\subsection{}
Consider the following linear programming problem:
\begin{align*}
	\text{Maximize }\qquad  x_1 + 2x_2 & \\
	\text{subject to }\qquad x_1 - 4x_2 &\leq 4 \\
 	-2x_1 +\,\ x_2 &\leq 2  \\ 
 	-3x_1 +   4x_2 &\leq 12 \\
	 2x_1 +\,\ x_2 &\leq 8  \\
	  x_1,\quad\   x_2 &\geq 0	  
\end{align*}

\begin{enumerate}[a.]
	\item 
	Sketch the feasible region in the \((x_1,x_2)\) space and identify the optimal solution.
	\item 
	Identify all the extreme points and reformulate the problem in terms of convex combination of the extreme points. 
	Solve the resulting problem. 
	Is any extreme point degenerate? Explain.
	\item 
	Suppose that the fourth constraint is dropped. Identify the extreme points and directions and reformulate the problem in terms of convex combinations of the extreme points and nonnegative linear combinations of the extreme directions. Solve the resulting problem and interpret the solution.
	\item 
	Is the procedure described in Parts (b) and (c) practical for solving larger problems? Discuss.
\end{enumerate}

\begin{enumerate}[a.]
	\item 
	\includegraphics[width=0.7\linewidth]{"Screenshot 2023-01-16 at 2.58.20 PM"}
	
	The optimal solution is \(\begin{bmatrix} 1.818 \\ 4.364 \end{bmatrix}\)
	with an objective value of \(10.546\).
	\item 
	\begin{flalign*}
		\text{Minimize} \quad  cx &    && \\
		\text{subject to}\quad  Ax &= b && \\
		                   x &>0. && \\
	\end{flalign*}
	where
	\begin{align*}
		x_1 = \begin{bmatrix} 0 \\ 0 \end{bmatrix} \quad
		x_2 = \begin{bmatrix} 0 \\ 2 \end{bmatrix} \quad
		x_3 = \begin{bmatrix} 4 \\ 0 \end{bmatrix} \quad
		x_4 = \begin{bmatrix} 0.8 \\ 3.6 \end{bmatrix} \quad
		x_5 = \begin{bmatrix} 1.818\\ 4.364 \end{bmatrix}.
	\end{align*}
	Among these, only \(x_3\) is degenerate, as it is an extreme point with three, or \(n+1\) meeting constraints.
	
	\item 
	\includegraphics[width=0.7\linewidth]{"Screenshot 2023-01-17 at 4.13.44 PM"}
	\begin{align*}
		x_1 = \begin{bmatrix} 0 \\ 0 \end{bmatrix} \quad
		x_2 = \begin{bmatrix} 0 \\ 2 \end{bmatrix} \quad
		x_3 = \begin{bmatrix} 4 \\ 0 \end{bmatrix} \quad
		d_1 = \begin{bmatrix} 1 \\ 1 \end{bmatrix} \quad
		d_2 = \begin{bmatrix} 4 \\ 3 \end{bmatrix}.
	\end{align*}
	
	\item 
	
\end{enumerate}

\end{document}