\documentclass[answers]{exam}
\usepackage[english]{babel}
\usepackage[utf8x]{inputenc}
\usepackage{amsmath,amssymb,amsthm}
\usepackage{booktabs}

\title{OPER 610 - Linear Programming%
	\\ Homework 1}
\author{Brandon Hosley}
\date{\today}

\begin{document}
\maketitle
\begin{questions}

%%%%%%%%%%%%%%%%%%%%%%%%%%%
%	\begin{ Question 1}	  %
%%%%%%%%%%%%%%%%%%%%%%%%%%%
\question 
Problem 1.2
\begin{enumerate}
	\item First, write an explicit formulation
	\begin{enumerate}
		\item Define sets, parameters, DVs, assumptions!
	\end{enumerate}
\item Second, use set-based notation and indexing of DVs and constraints to present a compact formulation
\end{enumerate}
\underline{Note}:
\begin{enumerate}
	\item Change the last restriction to read, "... they must exceed 25 percent and 30 percent, respectively..."
\end{enumerate}
\hrule
A manufacturer of plastics is planning to blend a new product from four chemical compounds. These compounds are mainly composed of three elements: A, B, and C. The composition and unit cost of these chemicals are shown in the following table:
\begin{tabular}{lcccc}
	\toprule
	CHEMICAL COMPOUND & 1 & 2 & 3 & 4 \\
	\midrule
	Percentage A  & 35 & 15 & 35 & 25 \\
	Percentage B  & 20 & 65 & 35 & 40 \\
	Percentage C  & 40 & 15 & 25 & 30 \\
	Cost/kilogram & 20 & 30 & 20 & 15 \\
	\bottomrule 
\end{tabular}
The new product consists of 25 percent element A, at least 35 percent element B, and at least 20 percent element C. Owing to side effects of compounds 1 and 2, they must not exceed 25 percent and 30 percent, respectively, of the content of the new product. Formulate the problem of finding the least costly way of blending as a linear program.
\begin{solution}
	S
\end{solution}
%\end{ Question 1}
\end{questions}
\end{document}