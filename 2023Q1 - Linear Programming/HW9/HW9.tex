\documentclass[12pt]{amsart}
\usepackage[left=0.5in, right=0.5in, bottom=0.75in, top=0.75in]{geometry}
\usepackage[english]{babel}
\usepackage[utf8x]{inputenc}
\usepackage{amsmath,amssymb,amsthm}
\usepackage{enumerate}
\usepackage{graphicx}

\begin{document}
\raggedbottom

\noindent{\large OPER 610 - Linear Programming %
	% Lesson X %
	(Due Feb 7 at 10am)}
\hspace{\fill} {\large B. Hosley}
\bigskip


%%%%%%%%%%%%%%%%%%%%%%%
\setcounter{section}{6}
\setcounter{subsection}{1}
\subsection{}
Find the dual of the following problem:
\begin{flalign*}
	\text{Maximize }   -2x_1 + 3x_2 + 5x_3 \qquad\ & \\
	\text{subject to } -2x_1 +\;\ x_2 + 3x_3 + x_4 &\geq\ 5 \\
	2x_1 \qquad\,\ \ +\ x_3 \qquad\ &=\ 4 \\
	    -2x_2 +\,\ x_3 + x_4 &\leq\ 6 \\
	x_1 &\leq\ 0 \\
	x_2, x_3 &\geq\ 0 \\
	x_4 &\quad\text{ unrestricted}  
\end{flalign*}

\textit{Solution:} Dual:
\begin{flalign*}
	\text{Minimize }\quad 5w_1 + 4w_2 + 6w_3 & \\
	\text{subject to } -2w_1 + 2w_2\hspace{6.25ex} &\leq \negthinspace -2 \\
	w_1 \hspace{6.35ex} -\,\ w_3 &\geq\ \ 3 \\
	3w_1 +\,\: w_2 +\,\ w_3 &\geq\ \ 5 \\
	w_1 +\hspace{7.35ex} w_3 &=\ \ 0 \\
	w_1 &\leq\ \ 0 \\
	w_2 &:\quad\text{unrestricted} \\
	w_3 &\geq\ \ 0
\end{flalign*}

\bigskip

\setcounter{section}{6}
\setcounter{subsection}{9}
\subsection{}
Consider the problem: Minimize \(z\) subject to 
\(z - \mathbf{cx} = 0\), 
\(\mathbf{Ax} = \mathbf b, \mathbf x \geq \mathbf 0\).
%
\begin{enumerate}[a.]
	\item State the dual.
	\item At optimality, what will be the value of the first dual variable? Explain.
\end{enumerate} \medskip

\textit{Solution:}
\begin{enumerate}[a.]
	\item Primal:
	\begin{flalign*}
		\text{Max }\qquad z + \mathbf{0x} & \hspace{30em}\\
		\text{st }\qquad z - \mathbf{cx} &= 0 \\
		0z + \mathbf{Ax} &= \mathbf{b} \\
		\mathbf{x} &\geq \mathbf{0} \\
		z &:\quad\text{unrestricted}
	\end{flalign*}
	Dual:
	\begin{flalign*}
		\text{Max }\qquad 0y + \mathbf{wb} & \hspace{30em}\\
		\text{st }\qquad\ y - \mathbf{w0} &= 1 \\
		-\mathbf{c}y + \mathbf{wA} &\leq 0 \\
		y,\ \mathbf{w} &:\quad\text{unrestricted}
	\end{flalign*}
	\item  The value of the first variable is \(1\) as a result of the 
	coefficient of the explicit \(z\) in the objective function. 
	By imputing the \(z\) explicitly into the remaining 
	primal constraints with a \(0\) coefficient, 
	that \(0\) coefficient is reflected in the dual's coefficient
	in a way that, while \(y\) is unrestricted by duality,
	the value is given by constraint.
\end{enumerate}

\end{document}