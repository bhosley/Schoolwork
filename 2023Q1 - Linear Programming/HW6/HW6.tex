\documentclass[12pt]{amsart}
\usepackage[left=0.5in, right=0.5in, bottom=0.75in, top=0.75in]{geometry}
\usepackage[english]{babel}
\usepackage[utf8x]{inputenc}
\usepackage{amsmath,amssymb,amsthm}
\usepackage{enumerate}
\usepackage{graphicx}

\begin{document}
\raggedbottom

\noindent{\large OPER 610 - Linear Programming %
	% Lesson X %
	(Due Jan 27 at 10am)}
\hspace{\fill} {\large B. Hosley}
\bigskip


%%%%%%%%%%%%%%%%%%%%%%%
\setcounter{section}{3}
\setcounter{subsection}{4}
\subsection{}
Solve the following linear programming problem by the simplex method.
At each iteration, identify \(\mathbf{B}\) and \(\mathbf{B^{-1}}\).

\begin{align*}
	\text{Maximize } \quad 3x_1 + 2x_2 +\,\ x_3 & \\
	\text{subject to}\quad 3x_1 - 3x_2 + 2x_3 &\leq 3 \\
	-x_1 + 2x_2 +\,\ x_3 &\leq 6 \\
	x_1,\quad\ x_2,\quad\ x_3 &\geq 0. \\
\end{align*}

\hrule
\medskip
\textit{Solution:}

Step 1:
\begin{align*}
	\mathbf{B} = \begin{bmatrix} 1 & 0 \\ 0 & 1 \end{bmatrix} \quad &
	\mathbf{B}^{-1} = \begin{bmatrix} 1 & 0 \\ 0 & 1 \end{bmatrix} \quad
	\mathbf{N} = \begin{bmatrix} 3 & -3 & 2 \\ -1 & 2 & 1 \end{bmatrix} \quad
	\mathbf{b} = \begin{bmatrix} 3 \\ 6 \end{bmatrix} \\ &
	c_n = (3,2,1) \qquad
	c_B = (0,0)
\end{align*}
Determine which variable enters the basis:
\begin{align*}
	z_1 - c_1 = -3 \\
	z_2 - c_2 = -2 \\
	z_3 - c_3 = -1 \\
\end{align*}
\(x_1\) is the entering variable:
\begin{align*}
	\begin{bmatrix} 1 & 0 \\ 0 & 1 \end{bmatrix}
	\begin{bmatrix} y_{11} \\ y_{21} \end{bmatrix} =
	\begin{bmatrix} 3 \\ -1 \end{bmatrix} \\
	y_{11}=3 \qquad
	y_{21}=-1
\end{align*}
The lower ratio is \(y_{11}\) at 1.

\vspace{4ex}
Step 2:
\begin{align*}
	\mathbf{B} = \begin{bmatrix} 3 & 0 \\ -1 & 1 \end{bmatrix} \quad &
	\mathbf{B}^{-1} = \begin{bmatrix} 1/3 & 0 \\ 1/3 & 1 \end{bmatrix} \quad
	\mathbf{N} = \begin{bmatrix} 1 & -3 & 2 \\ 0 & 2 & 1 \end{bmatrix} \quad
	\mathbf{b} = \begin{bmatrix} 1 \\ 7 \end{bmatrix} \\ &
	c_n = (0,2,1) \qquad
	c_B = (3,0)
\end{align*}
Determine which variable enters the basis:
\begin{align*}
	z_2 - c_2 = -11 \\
	z_3 - c_3 = 5 \\
\end{align*}
\(x_2\) is the entering variable:
\begin{align*}
	\begin{bmatrix} 3 & 0 \\ -1 & 1 \end{bmatrix}
	\begin{bmatrix} y_{12} \\ y_{22} \end{bmatrix} =
	\begin{bmatrix} -3 \\ 2 \end{bmatrix} \\
	y_{12}=-1 \qquad
	y_{22}=1
\end{align*}
Find limit of \(x_2\)
\begin{align*}
	\begin{bmatrix} 1 \\ 7 \end{bmatrix} -
	\begin{bmatrix} -1 \\ 1 \end{bmatrix}x_2
\end{align*}
\(x_5\) drops to \(0\) when \(x_2 = 7\). Update \(\mathbf{b}\) accordingly:
\begin{align*}
	\mathbf{b} = \begin{bmatrix} 8 \\ 7 \end{bmatrix} 
\end{align*}

\vspace{4ex}
Step 3:
\begin{align*}
	\mathbf{B} = \begin{bmatrix} 3 & -3 \\ -1 & 2 \end{bmatrix} \quad &
	\mathbf{B}^{-1} = \begin{bmatrix} 2/3 & 1 \\ 1/3 & 1 \end{bmatrix} \quad
	\mathbf{N} = \begin{bmatrix} 1 & 0 & 2 \\ 0 & 1 & 1 \end{bmatrix} \quad
	\mathbf{b} = \begin{bmatrix} 8 \\ 7 \end{bmatrix} \\ &
	c_n = (0,0,1) \qquad
	c_B = (3,2)
\end{align*}
Determine which variable enters the basis:
\begin{align*}
	z_3 - c_3 = 4 \\
	z_4 - c_4 = 3 \\
\end{align*}
There is no improvement to be made with the remaining variables. 
Thus the current state is optimal, and as a result:
\begin{align*}
	x_1 = 8, \quad
	x_2 = 7, \quad
	z = 38
\end{align*}

\clearpage


\setcounter{subsection}{48}
\subsection{}
The following is the current simplex tableau of a given maximization problem. 
The objective is to maximize \(2x_1 - 4x_2\), and the slack variables are \(x_3\) and \(x_4\). 
The constraints are of the \(\leq\) type.

\begin{center}
	\setlength{\tabcolsep}{1em} % for the horizontal padding
	{\renewcommand{\arraystretch}{1.4}% for the vertical padding
	\begin{tabular}{ccccccc}
		& $z$                    & $x_1$ & $x_2$ & $x_3$ & $x_4$                    & RHS                      \\ \cline{2-7} 
		\multicolumn{1}{c|}{$z$}   & \multicolumn{1}{c|}{1} & $b$   & 1     & $f$   & \multicolumn{1}{c|}{$g$} & \multicolumn{1}{c|}{8}   \\ \cline{2-7} 
		\multicolumn{1}{c|}{$x_3$} & \multicolumn{1}{c|}{0} & $c$   & 0     & 1     & \multicolumn{1}{c|}{1/5} & \multicolumn{1}{c|}{4}   \\
		\multicolumn{1}{c|}{$x_1$} & \multicolumn{1}{c|}{0} & $d$   & $e$   & 0     & \multicolumn{1}{c|}{2}   & \multicolumn{1}{c|}{$a$} \\ \cline{2-7} 
	\end{tabular}} \\
\end{center}

\begin{enumerate}[a.]
	\item Find the unknowns a through g.
	\item Find \(\mathbf B^{-1}\).
	\item Find \(\partial x_3/\partial x_2, \partial z/\partial b_1, \partial z/\partial x_4, \partial x_1/\partial b_2.\)
	\item Is the tableau optimal?
\end{enumerate}

\hrule
\medskip
\textit{Solution:}

\begin{enumerate}[a.]
	\item 
	\begin{align*} &
		a = 4, \quad
		b = 0, \quad
		c = 0, \quad
		d = 1, \\ &
		e = -1.5, \quad
		f = 0, \quad
		g = 4
	\end{align*}
	\begin{center}
		\setlength{\tabcolsep}{1em} % for the horizontal padding
		{\renewcommand{\arraystretch}{1.4}% for the vertical padding
		\begin{tabular}{ccccccc}
			& $z$                    & $x_1$ & $x_2$ & $x_3$ & $x_4$                    & RHS                    \\ \cline{2-7} 
			\multicolumn{1}{c|}{$z$}   & \multicolumn{1}{c|}{1} & 0     & 1     & 0     & \multicolumn{1}{c|}{4}   & \multicolumn{1}{c|}{8} \\ \cline{2-7} 
			\multicolumn{1}{c|}{$x_3$} & \multicolumn{1}{c|}{0} & 0     & 0     & 1     & \multicolumn{1}{c|}{1/5} & \multicolumn{1}{c|}{4} \\
			\multicolumn{1}{c|}{$x_1$} & \multicolumn{1}{c|}{0} & 1     & -1.5  & 0     & \multicolumn{1}{c|}{2}   & \multicolumn{1}{c|}{4} \\ \cline{2-7} 
		\end{tabular}} \\
	\end{center}
	
	\item 
	\begin{align*}
		\mathbf{B^{-1}} = \begin{bmatrix} 1 & 1/5 \\ 0 & 2 \end{bmatrix}
	\end{align*}

	\item 
	\begin{align*}
		\partial x_3/\partial x_2 = 0 \qquad\qquad
		&\partial   z/\partial b_1 = 0 \\
		\partial   z/\partial x_4 = -4 \qquad\qquad
		&\partial x_1/\partial b_2 = -2 \\
	\end{align*}
	
	\item 
	Yes, as a maximization problem with all values in the top row being positive, 
	the current tableau represents optimal.

\end{enumerate}


\end{document}