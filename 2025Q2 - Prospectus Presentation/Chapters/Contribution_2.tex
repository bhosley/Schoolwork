\section{Contribution 2}

% For each Contribution:
%     Introduction
%         Recall Motivation
%         Lit review
%         Contribution
%     Methodology
%     Experimental Procedure
%     Results
%     Discussion

\subsection{Introduction}

\begin{frame}{The Case for Input-Invariant Architectures}
    \begin{itemize}
        \item Heterogeneous agent teams frequently encounter differences in observation 
            structure due to varying sensor suites and mission roles.
        \item Traditional policy networks require fixed, consistent inputs—limiting their 
            ability to scale or adapt.
        \item Input-invariant architectures offer a promising alternative by allowing 
            policies to generalize across agents and configurations without retraining.
        \item This contribution explores how these architectures can improve learning 
            efficiency and robustness in dynamic, real-world team settings.
    \end{itemize}
\end{frame}

\subsubsection{Literature Review}


% #TODO: Pull from Paper Draft


\subsubsection{Research Questions}

\begin{frame}{Contribution 2 - Research Questions}
    \begin{enumerate}
        \item[RQ 1] {
            How does incorporating input-invariant structures \emph{(i.e., networks 
            that are robust to feature permutation and input length differences)}
            into policy networks affect learning efficiency and team robustness 
            when heterogeneous agents have partially overlapping observations?
            }
        \item[RQ 2] {
            Do input-invariant architectures lead to more stable policy performance under 
            team-size changes and partial observation loss during execution?
            }
    \end{enumerate}
\end{frame}

\begin{frame}{RQ 1 - Research Tasks}
    \begin{enumerate}
        \item[RQ 1] \textcolor{gray}{
            How does incorporating input-invariant structures \emph{(i.e., networks 
            that are robust to feature permutation and input length differences)}
            into policy networks affect learning efficiency and team robustness 
            when heterogeneous agents have partially overlapping observations? } \vspace{1em}
    \begin{itemize}
        \item[RT 1.1] {
            Design and implement a policy architecture that exhibits input-invariance through 
            techniques such as pooling layers, attention mechanisms, or permutation-invariant 
            encodings.}
        \item[RT 1.2] {
            Identify benchmark environments where agents possess distinct but partially 
            overlapping observation features (e.g., different sensor arrays).}
        \item[RT 1.3] {
            Train these, input-invariant policy networks in the selected setting(s) and measure 
            learning rate and convergence behavior.}
        \item[RT 1.4] {
            Compare input-invariant shared architectures against baseline (non-invariant) 
            architecture to evaluate performance and learning benefits.}
    \end{itemize}
    \end{enumerate}
\end{frame}

\begin{frame}{RQ 2 - Research Tasks}
    \begin{enumerate}
        \item[RQ 2] \textcolor{gray}{
            Do input-invariant architectures lead to more stable policy performance under 
            team-size changes and partial observation loss during execution? } \vspace{1em}
    \begin{itemize}
        \item[RT 2.1] {
            Simulate runtime degradation by selectively masking subsets of agent observations 
            (e.g., sensor failure) for both the input-invariant and baseline policy models.}
        \item[RT 2.2] {
            Simulate dynamic changes in team size by removing agents during evaluation 
            in both the input-invariant and baseline policy models.}
        \item[RT 2.3] {
            Measure policy stability, reward degradation, and recovery behavior under these 
            perturbations for both the input-invariant and baseline models.}
        \item[RT 2.4] {
            Statistically compare performance between input-invariant and baseline models 
            across all perturbation conditions to assess resilience and stability.}
    \end{itemize}
    \end{enumerate}
\end{frame}


% \subsection{Methodology}



\subsection{Environment Model}


% \subsection{Experimental Procedure}
% \subsection{Results}
% \subsection{Discussion}

