\section{Contribution 2}

% For each Contribution:
%     Introduction
%         Recall Motivation
%         Lit review
%         Contribution
%     Methodology
%     Experimental Procedure
%     Results
%     Discussion

\subsection{Introduction}

\begin{frame}{The Case for Input-Invariant Architectures}
    \begin{itemize}
        \item Heterogeneous agent teams frequently encounter differences in observation 
            structure due to varying sensor suites and mission roles.
        \item Traditional policy networks require fixed, consistent inputs—limiting their 
            ability to scale or adapt.
        \item Input-invariant architectures offer a promising alternative by allowing 
            policies to generalize across agents and configurations without retraining.
        \item This contribution explores how these architectures can improve learning 
            efficiency and robustness in dynamic, real-world team settings.
    \end{itemize}
\end{frame}


\subsubsection{Literature Review}


\begin{frame}{Theoretical Foundations}
    \textbf{Exchangeability and Symmetry}
    \begin{itemize}
        \item \textbf{de Finetti's Theorem:} Exchangeable sequences behave like i.i.d. 
            draws from a latent distribution~\footcite{alvarez-melis2015}.
        \item In MARL, agent observations are often unordered—motivating 
            \textbf{permutation-invariant} architectures.
        \item \textbf{Key Insight:} Architectures that respect symmetry in input 
            ordering can improve generalization and efficiency~\footcite{hartford2018}
            for exchangeable agents.
    \end{itemize}
\end{frame}

\begin{frame}{Permutation Invariance and Equivariance}
    \textbf{Deep Sets} \footcite{zaheer2017}
    \begin{itemize}
        \item Any permutation-invariant function can be expressed in terms of 
            a transformation \(\phi\) over elements and a transformation \(\rho\) of the sum: 
            \[\rho\left(\sum_i \phi(x_i)\right)\]
        \item Zaheer et al. use shared parameters through constrained weight matrices.
        \begin{itemize}
            \item With all off-diagonal elements \(\gamma\)
            \item Diagonal elements \(\lambda+\gamma\)
        \end{itemize}
        \item \textbf{Implication:} Enables parameter sharing across agents with unordered observations.
        \item Improves sample efficiency and scalability in symmetric MARL settings.
    \end{itemize}
\end{frame}

\begin{frame}{Mean-Field Methods and Exchangeability}
    \textbf{From Aggregation to Generalization}
    \begin{itemize}
        \item Replace summation with \textbf{averaging} to make outputs insensitive to agent count.
        \item \textbf{Mean-Field MARL}~\footcite{yang2018}: 
            Use average of neighbors in actor and critic networks.
        \item MF-PPO achieves~\footcite{li2021b}:
        \begin{itemize}
            \item Convergence rates independent of team size
            \item Superior performance and reduced model complexity
        \end{itemize}
        \item \textbf{Limitation:} Subset-specific interaction information may be lost.
    \end{itemize}
\end{frame}

% \begin{frame}{Graph-Based Architectures}
%     \textbf{Relational Inductive Bias}
%     \begin{itemize}
%         \item Graph Neural Networks (GNNs) treat agents as nodes; edges represent interactions.
%         \item \textbf{IHG-MA} \footcite{yang2021a}: Dynamic agent graphs with shared policies.
%         \item \textbf{PIC} \footcite{liu2020b}: Graph critic pooled over agents, scalable to 200 agents.
%         \item Attribute embeddings encode agent roles or types, enabling heterogeneous agent support.
%     \end{itemize}
% \end{frame}

\begin{frame}{Attention for Set-Based Inputs}
    \begin{columns}
        \begin{column}{0.6\textwidth}
            \textbf{Transformers and Self-Attention}
            \begin{itemize}
                \item Attention layers (without positional encodings) are naturally permutation-invariant.
                \item \textbf{Set Transformer} \footcite{lee2019}: 
                    Captures high-order relationships among unordered elements.
                \item \textbf{MAAC} \footcite{iqbal2019}: 
                    Attention-weighted centralized critic for better credit assignment.
                \item Applied to large-scale MARL tasks (e.g., SMAC) with strong results.
            \end{itemize}
        \end{column}
        \begin{column}{0.4\textwidth}
            \begin{figure}
                \includegraphics[width=0.85\linewidth]{C2/lee2019_set_transformer.png}
                \caption{Set Transformer model by Lee et al~\footcite{lee2019}.}
            \end{figure}
        \end{column}
    \end{columns}
\end{frame}

\begin{frame}{Single-Agent Generalization}
    \begin{columns}
        \begin{column}{0.6\textwidth}
            \textbf{Lessons Beyond MARL}
            \begin{itemize}
                \item \textbf{Zambaldi et al.} \footcite{zambaldi2018}: 
                    Graph-based reasoning improves single-agent performance in StarCraft II.
                \item \textbf{Tang and Ha} \footcite{tang2021}: 
                    Sensory-channel invariance enables robustness to occlusion and dropout.
                \item Treating input channels as exchangeable improves perceptual robustness.
            \end{itemize}
        \end{column}
        \begin{column}{0.4\textwidth}
            \begin{figure}
                \includegraphics[width=0.95\linewidth]{C2/tang2021_view.png}
                \caption{Original and permuted image inputs used by Tang and Ha~\footcite{tang2021}.}
            \end{figure}
        \end{column}
    \end{columns}
\end{frame}

\begin{frame}{Toward Heterogeneous MARL}
    \begin{columns}
        \begin{column}{0.6\textwidth}
            \textbf{Challenges in Heterogeneous-Agent Learning}
            \begin{itemize}
                \item Input-invariant methods often assume identical agent observation distributions.
                \item \textbf{Strategy:} Augment observations with role/type embeddings.
                \item \textbf{HPN} \footcite{hao2023}: Hypernetworks generate role-conditioned modules.
                \item \textbf{Open Question:} What are the tradeoffs in cost and complexity vs. performance?
            \end{itemize}
        \end{column}
        \begin{column}{0.4\textwidth}
            \begin{figure}
                \includegraphics[width=\linewidth]{C2/hao2022_network_overview.png}
                \caption{Hao et al's PI and PE policy network overview~\footcite{hao2023}.}
            \end{figure}
        \end{column}
    \end{columns}
\end{frame}

\begin{frame}{Summary of Key Ideas}
\begin{itemize}
    \item Input-invariance enables better generalization and efficiency in symmetric multi-agent settings.
    \item Mean-field, GNNs, and attention each offer scalable strategies with tradeoffs in specificity and complexity.
    % \item Extensions to HARL require mechanisms to encode agent roles/types while preserving symmetry when possible.
    \item Empirical comparisons of cost vs. benefit are still lacking in literature.%—this work aims to address that.
\end{itemize}
\end{frame}

\subsubsection{Research Questions}

\begin{frame}{Contribution 2 - Research Questions}
    \begin{enumerate}
        \item[RQ 1] {
            How does incorporating input-invariant structures \emph{(i.e., networks 
            that are robust to feature permutation and input length differences)}
            into policy networks affect learning efficiency and team robustness 
            when heterogeneous agents have partially overlapping observation spaces?
            }
        \item[RQ 2] {
            Do input-invariant architectures lead to more stable policy performance under 
            team-size changes and partial observation loss during execution?
            }
        \item[RQ 3] {
            What are the computational and implementation costs of mean-field
            architectures relative to their performance benefits?
            }
    \end{enumerate}
\end{frame}

\begin{frame}{RQ 1 - Research Tasks}
    \begin{enumerate}
        \item[RQ 1] \textcolor{gray}{
            How does incorporating input-invariant structures \emph{(i.e., networks 
            that are robust to feature permutation and input length differences)}
            into policy networks effect learning efficiency and team robustness 
            when heterogeneous agents have partially overlapping observations? } \vspace{1em}
    \begin{itemize}
        \item[RT 1.1] {
            Design and implement a policy architecture that exhibits input-invariance through 
            techniques such as pooling layers, attention mechanisms, or permutation-invariant 
            encodings.}
        \item[RT 1.2] {
            Identify benchmark environments where agents possess distinct but partially 
            overlapping observation features (e.g., different sensor arrays).}
        \item[RT 1.3] {
            Train these, input-invariant policy networks in the selected setting(s) and measure 
            learning rate and convergence behavior.}
        \item[RT 1.4] {
            Compare input-invariant shared architectures against baseline (non-invariant) 
            architecture to evaluate performance and learning benefits.}
    \end{itemize}
    \end{enumerate}
\end{frame}

\begin{frame}{RQ 2 - Research Tasks}
    \begin{enumerate}
        \item[RQ 2] \textcolor{gray}{
            Do input-invariant architectures lead to more stable policy performance under 
            team-size changes and partial observation loss during execution? } \vspace{1em}
    \begin{itemize}
        \item[RT 2.1] {
            Simulate runtime degradation by selectively masking subsets of agent observations 
            (e.g., sensor failure) for both the input-invariant and baseline policy models.}
        \item[RT 2.2] {
            Simulate dynamic changes in team size by removing agents during evaluation 
            in both the input-invariant and baseline policy models.}
        \item[RT 2.3] {
            Measure policy stability, reward degradation, and recovery behavior under these 
            perturbations for both the input-invariant and baseline models.}
        \item[RT 2.4] {
            Statistically compare performance between input-invariant and baseline models 
            across all perturbation conditions to assess resilience and stability.}
    \end{itemize}
    \end{enumerate}
\end{frame}

\begin{frame}{RQ 3 - Research Tasks}
    \begin{enumerate}
        \item[RQ 3] \textcolor{gray}{ 
            What are the computational and implementation costs of mean-field
            architectures relative to their performance benefits? }
        \vspace{1em}
        \begin{itemize}
            \item[RT 3.1] {
                Benchmark computational cost during training and 
                inference (e.g., agent-steps, step-costs).}
            \item[RT 3.2] {
                Perform Statistical comparison of relative training cost, rate of convergence, 
                in training and rate of performance degradation in completed models.}
        \end{itemize}
    \end{enumerate}
\end{frame}

\subsection{Methodology}

\subsubsection{Environment Model}

\begin{frame}{Environment Model (RT 1.2)}
    Using Powell's~\footcite{powell2022}
    framework for modeling:
    \begin{columns}
        \begin{column}{0.3\linewidth}
            \begin{enumerate}
                \item State
                \item Decision/Action
                \item Exogeneous Information
                \item Transition Function
                \item Objective Function
            \end{enumerate}
        \end{column}
        \begin{column}{0.7\linewidth}
        \end{column}
    \end{columns}
\end{frame}


\begin{frame}{State: Time and Joint Representation}
    \begin{itemize}
        \item Let \(T\subset\mathbb{N}\) be the set of decision epochs.
        \item Let entities \(E := (I, J, K, L)\), where:
        \begin{itemize}
            \item \(i\in I\): agents
            \item \(j\in J\): objectives
            \item \(k\in K\): hazards
            \item \(l\in L\): obstacles
        \end{itemize}
        \item For time period \(t\in T\) and entities \(e\in E\), the joint state is:
        \[
            s_t := \{s_{te}\ \forall\ e\in E\}, \quad s_t \in S
        \]
    \end{itemize}
\end{frame}

\begin{frame}{State: Marginal Representations}
    Let \(N\in \mathbb{N}\) be the number of spatial dimensions,
    and for each \(n \in \{1, \dots, N\}\),
    let \(D^{(n)} \in \mathbb{F}\) denote the domain of the \(n\)th dimension.
    % Let,
    % \begin{itemize}
    %     \item \(N\in \mathbb{N}\) be the number of spatial dimensions.
    %     \item  and for each \(n \in \{1, \dots, N\}\), 
    %         let \(D^{(n)} \in \mathbb{F}\) denote the domain of the \(n\)th dimension.
    % \end{itemize}
    Then for,
    \begin{itemize}
        \item non-agents \(e\in (J, K, L)\) marginal states are: \vspace{-0.8em}
        \[
            s_{te} := (d_{te})
        \]
        \item agents \(i\in I\) marginal states are: \vspace{-0.8em}
        \[
            s_{ti} := (d_{ti}, \theta_{ti})
        \]
    \end{itemize}
    Where.
    \begin{itemize}
        \item the location vector is: \vspace{-0.8em}
        \[
            d_{te} := \left(d_{te}^{(1)},\ldots, d_{te}^{(N)}\right) \in \prod_{n=1}^{N} D_{e}^{(n)}
        \]
        \item and the orientation vector is: \vspace{-0.8em}
        \[
            \theta_{ti} := \left(\theta_{ti}^{(1,2)},\ldots, \theta_{ti}^{(N-1,N)}\right) \in \prod_{n=1}^{N-1} \Theta_{i}^{(n,n+1)}
        \]
    \end{itemize}
\end{frame}

\begin{frame}{Decision/Action}
    Joint action \(a \in A\) is the combination of agent's actions \(a_i\).
    \[
        a := \left(a_{1}, \ldots, a_{|I|}\right) \ \in A := \prod_{i\in I}^{} A_i
    \]
    Then the marginal action space for agent \(i\in I\) is,
    \[
        a_{ti}^{} := (a_{ti}^\text{interact},a_{ti}^\text{move},\Delta\theta_{ti})
    \]
    where,
    \begin{itemize}
        \item \(a_{ti}^\text{interact} \in \{0,1\}\) indicates whether the agent will attempt 
            to interact with the environment (1 = interact, 0 = no interaction),
        \item \(a_{ti}^\text{move} \in M_i \subseteq \mathbb{F}\) 
            denotes the forward movement magnitude of agent \(i\),
        \item \(\Delta\theta_{ti} := \left(\Delta\theta_{ti}^{(1,2)},\ldots, 
            \theta_{ti}^{(N-1,N)}\right) \in \prod_{n=1}^{N-1} \Delta\Theta_{i}^{(n,n+1)}\) 
            specifies the adjustments to the agent's orientation along each rotational axis.
    \end{itemize}
\end{frame}

\begin{frame}{Heterogeneous Action Spaces}
    Heterogeneous action spaces can be naturally modeled through constraints on action spaces:
    \begin{columns}[T]
        \begin{column}{0.5\textwidth}
            \begin{itemize}
                \item \textbf{Movement:} for \(a_{ti}^\text{move} \in M_i \subseteq \mathbb{F}\)
                \item \textbf{Orientation Adjustment:} in general,
                    \[
                        \Delta\Theta_i := \left\{ k \cdot \delta_{\theta_i} 
                        \mid k \in \mathbb{F},\ |\delta_{\theta_i} k| \leq \pi \right\}
                    \]
                where \(k \in [-1,1] \subset \mathbb{F}\) is the range of allowable adjustment
                and \(\delta\) is the resolution of the adjustment 
            \end{itemize}
        \end{column}
        \begin{column}{0.5\textwidth}
            \textbf{A Gridworld Example:}
                \begin{itemize}
                    \item \(\delta := \frac{\pi}{2}\) for ortholinear constraint
                    \item \(\delta := \frac{\pi}{4}\) allows diagonal movement
                    \item A \textbf{rotary-wing aircraft} 
                    \begin{itemize}
                        \item \(M_i := \{0,1\}\)
                        \item \(k_i \in \mathbb{Z} \equiv \{-4, \ldots, 4\} \)
                    \end{itemize}
                    \item A \textbf{fixed-wing aircraft}
                    \begin{itemize}
                        \item \(M_i := \{1,2\}\)
                        \item \(k_i \in \{-1,0,1\}\)
                    \end{itemize}
                \end{itemize}
        \end{column}
    \end{columns}
\end{frame}

\begin{frame}{Exogeneous Information}
    Currently this representation is
    \begin{itemize}
        \item Complete Information: known reward
        \item Imperfect Information: partially observable environment
    \end{itemize} 
    \textbf{Future-work fodder}: An implementation in which channels convey probabilistic 
    information would change this to a incomplete information (and Bayesian) game.
\end{frame}

\begin{frame}{Transition Function}
    This model is deterministic and progresses with agent decisions executed in parallel.\\[1em]
    Thus, the environment transition from state \(s_t\) under joint action \(a_t\) is:
    \[
        s_{t+1} = \mathcal{T}(s_t, a_t)
    \]
    or equivalently, in conditional form:
    \[
        \mathcal{T}(a_t \mid s_t) = s_{t+1}
    \]
\end{frame}

\begin{frame}{Objective Function}
    The objective is to optimize behavior by:
    \begin{itemize}
        \item Max \(\mathcal{R}(j) = r_{j} \in \mathbb{R}_{>0}\) the reward for collecting objective \(j\),
        \item Min \(\mathcal{R}(i) = r_{i} \in \mathbb{R}_{<0}\) the penalty for interaction use by agent \(i\),
        \item Min \(\mathcal{R}(k) = r_{k} \in \mathbb{R}_{<0}\) the penalty for proximity to hazard \(k\).
    \end{itemize}
    Then, the cumulative objective is:
    % {\footnotesize
    % \begin{align*}
    %     \max \sum_{t\in T} &\left[
    %         \sum_{i\in I, j\in J} r_{j} \cdot \mathbb{I}\left[
    %             \|d_{ti}-d_{tj}\| \leq i^\text{range} 
    %             \land \left| \tan\theta_{ti}^{(n,n+1)} - \frac{d_{tj}^{(n+1)}-d_{ti}^{(n+1)}}{
    %                 d_{tj}^{(n)}-d_{ti}^{(n)}} \right| \leq \epsilon 
    %             \land a_{ti}^\text{interact} = 1 
    %         \right] 
    %     \right. \\
    %     &\left.
    %         \quad - \sum_{i\in I} r_{i} \cdot \mathbb{I}\left[a_{ti}^\text{interact} = 1\right]
    %         - \sum_{i\in I, k\in K} r_{k} \cdot \mathbb{I}\left[\|s_{ti} - s_{tk}\|_\infty 
    %         \leq k^\text{range}\right]
    %     \right]
    % \end{align*}}
    \[ 
        \max \sum_{t\in T} \left[
            \sum_{i\in I, j\in J} r_{j} \cdot \mathbb{I}\left[ \Xi \right]
            - \sum_{i\in I} r_{i} \cdot \mathbb{I}\left[a_{ti}^\text{interact} = 1\right]
            - \sum_{i\in I, k\in K} r_{k} \cdot \mathbb{I}\left[\|s_{ti} - s_{tk}\|_\infty
            \leq k^\text{range}\right]
        \right]
    \]
    Where $\Xi$ represents the conditions of objectives.
\end{frame}

\begin{frame}{Objective Conditions}
    \(\mathbb{I}[\Xi]\) is a function representing the conditions of objectives.
    A basic example:
    \begin{align*}
        \Xi := &\,\ a_{ti}^\text{interact} = 1 && \text{Did agent \(i\) try to interact?} \\ 
        & \land \|d_{ti}-d_{tj}\| \leq i^\text{range} &&\text{Is \(j\) in range of agent \(i\)?}\\
        & \land \left| \tan\theta_{ti}^{(n,n+1)} 
            - \frac{d_{tj}^{(n+1)}-d_{ti}^{(n+1)}}{d_{tj}^{(n)}-d_{ti}^{(n)}} \right| 
            \leq \epsilon && \text{Is \(i\) facing \(j\)?}
    \end{align*}
    Additional conditions likely to consider:
    \begin{itemize}
        \item Min number of agents needed to cooperate
        \item Sum of agent levels exceeds an objective difficulty
        \item Some abstract type matching
    \end{itemize}
\end{frame}

\subsubsection{Observation Model}

\begin{frame}{Observation Model}
    In this subsection:
    \begin{itemize}
        \item Imperfect Information Game
        \item Channels Representing States
        \item Modeling Heterogeneous Observation Space
        \item Intuition for a Heterogeneous Observation Space
    \end{itemize}
\end{frame}

\begin{frame}{Imperfect Information}
    \begin{columns}
        \begin{column}{0.4\linewidth}
            This an imperfect information game resulting from partial observability.
            \\[1em]
            Thus, for agent \(i\), an observation \(o_i\) is a function of the state:
            \[o_{ti} = \mathcal{O}_{i}(s_{t}) .\]
        \end{column}
        \begin{column}{0.5\linewidth}
            \begin{figure}
                \begin{subfigure}[t]{0.45\textwidth}
                    \includegraphics[width=\textwidth]{screen_minigrid}
                    \caption{View-range of agent in Minigrid~\footcite{chevalier-boisvert2023}.}
                \end{subfigure}
                \hfill
                \begin{subfigure}[t]{0.45\textwidth}
                    \includegraphics[width=\textwidth]{screen_lbf}
                    \caption{View-range of agent pair in Level-based Foraging~\footcite{papoudakis2021}.}
                \end{subfigure}
            \end{figure}
        \end{column}
    \end{columns}
\end{frame}

\begin{frame}{Channels for Binary Mapping}
    \begin{columns}
        \begin{column}{0.6\linewidth}
            Specific entities are represented by channels:
                \begin{itemize}
                    \item An observation is a set of one-hot matrices
                    \item One matrix per type of entity
                \end{itemize}
            Pursuit example channels:
                \begin{itemize}
                    \item For \(e\in E := \{\text{Friend, Foe, Barrier}\}\)
                    \item Then state \(s\)
                \end{itemize}
                \[s = \{0,1\}^{E\times D_1\times D_2}\]
        \end{column}
        \begin{column}{0.4\linewidth}
            \begin{figure}
                \includegraphics[width=0.65\linewidth]{screen_pursuit_channels.png}
                \caption{Observation Channels in Pursuit~\footcite{gupta2017}.}
            \end{figure}
            \centering
        \end{column}
    \end{columns}
\end{frame}

\begin{frame}{Heterogeneous Observation Channels}
    For heterogeneous observation channels \(C\),
    visibility of entities \(E\) is, may be considered a
    trait of \(e\):
    \[
        e^{(c)} \in \{\text{True, False}\}.
    \]
    But more flexibly, this may be modeled 
    as a sort of categorical dimension of the state:
    \[
        s_{tec} = 
        \begin{cases}
            1& \text{if \(e\) is visible in \(c\)} \\ 
            0& \text{if \(e\) is not visible in \(c\)}
        \end{cases} 
        \quad\forall t\in T, e\in E, c\in C.
    \]
    regardless of if \(e\)'s visibility in \(c\) is expected to change.
\end{frame}

\begin{frame}{Heterogeneous Observation Channels}
    The addition of the observation channels modifies
    entity state in general to:
    \[s_{te} :(c_{te},d_{te}).\]
    and agent state in particular to:
    \[s_{ti} :(c_{ti},d_{ti},\theta_{ti}).\]

    Similar to Pursuit, our model can be represented as a binary mapping
    and extended with Heterogeneous channels \(c\in C\) as:
    \[
        S = \{0,1\}\exp\left(|C|\times E_\text{types}
        \times \prod_{n}^{N} D_n \times \prod_{n}^{N-1} \theta_n\right),
    \]
\end{frame}

\begin{frame}{Intuition for Heterogeneous Observation}
    Let \(C = \{\text{Red, Blue, Green}\}\).
    Then, the RGB color model 
    provides a convenient shorthand for multiple truths. \\
   \begin{figure}
        \centering
        \begin{tikzpicture}
    \def\di{0.5}
    \def\si{0.75*\di}
    \draw [draw=none, fill=red] (90:\si) circle (\di);
    \draw [draw=none, fill=green] (-30:\si) circle (\di);
    \draw [draw=none, fill=blue] (210:\si) circle (\di);
    \begin{scope} % red + green = yellow
        \clip (90:\si) circle(\di);
        \draw [draw=none, fill=yellow] (-30:\si) circle (\di);
    \end{scope} % blue + red = magenta
    \begin{scope}
        \clip (210:\si) circle(\di);
        \draw [draw=none, fill=magenta] (90:\si) circle (\di);
    \end{scope}
    \begin{scope} % green + blue = cyan
        \clip (-30:\si) circle(\di);
        \draw [draw=none, fill=cyan] (210:\si) circle (\di);
    \end{scope}
    \begin{scope} % red + green + blue = white
        \clip (90:\si) circle(\di);
        \clip (210:\si) circle(\di);
        \draw [draw=none, fill=white] (-30:\si) circle (\di);	
    \end{scope}
\end{tikzpicture}
    \end{figure}
    Then, in the next slide we present a Minigrid-like state made heterogeneous.
\end{frame}

\begin{frame}{An Example State}
    \begin{columns}
        \begin{column}{0.4\linewidth}
            Legend:
            \begin{figure}
                \resizebox{!}{0.5\linewidth}{%
                    \NewDocumentCommand{\drawgrid}{O{black} O{8} O{#2}}{%
    % #1 = color (optional, defaults to black)
    % #2 = size (e.g., 8 for 8x8)
    % #3 = second size (e.g., 4 for 8x4)
    \foreach \i in {0,...,#2} {%
        \draw[thin, color={#1}] (\i,0) -- (\i,#3);}
    \foreach \j in {0,...,#3} {%
        \draw[thin, color={#1}] (0,\j) -- (#2,\j);}
}

\NewDocumentCommand{\dart}{O{} O{(0,0)} O{0}}{%
    % #1 = draw options
    % #2 = center coordinate
    % #3 = degree rotation
    \coordinate (center) at #2;
    \draw[#1]
        ($(center) + ({ 0 +#3}:0.4)$) --
        ($(center) + ({130+#3}:0.4)$) --
        ($(center) + ({180+#3}:0.1)$) --
        ($(center) + ({230+#3}:0.4)$) --
        cycle;
}

\NewDocumentCommand{\isorect}{O{} O{(0,0)} O{0}}{%
    % #1 = draw options
    % #2 = center coordinate
    % #3 = degree rotation
    \coordinate (center) at #2;
    \def\wid{0.4}
    \draw[#1]
        ($(center) + ({-45+#3}:\wid)$) --
        ($(center) + ({ 45+#3}:\wid)$) --
        ($(center) + ({135+#3}:\wid)$) --
        ($(center) + ({225+#3}:\wid)$) --
        cycle;
}

\NewDocumentCommand{\isocirc}{O{} O{(0,0)}}{%
    % #1 = draw options
    % #2 = center coordinate
    \coordinate (center) at #2;
    \def\ra{0.3}
    \draw[#1] ($(center) + (0:\ra)$) 
        arc (0:90:\ra) 
        arc (90:180:\ra)
        arc (180:270:\ra)
        arc (270:360:\ra);
}

\begin{tikzpicture}
    \node[] (lab1) at (0,0) {Agent};
    \dart[draw=cyan,fill=cyan!30][(1,0)][0]
    
    \node[] (lab2) [below of=lab1] {Sight};
    \fill[fill=cyan!15] ($(lab2) + (0.75,-0.5)$) -- +(1,0) -- +(1,1) -- +(0,1) -- cycle;
    
    \node[] (lab3) [below of=lab2] {Objective};
    \isocirc[magenta, fill=magenta!50][($(lab3)+(1.25,0)$)]

    \node[] (lab4) [below of=lab3] {Obstacle};
    \isorect[draw=green, fill=green!50][($(lab4)+(1.25,0)$)]
\end{tikzpicture}
                }
            \end{figure}
        \end{column}
        \begin{column}{0.6\linewidth}
            \begin{figure}
                \resizebox{\linewidth}{!}{%
                    \input{C2/rgb_example_grid.tex}
                }
            \end{figure}
        \end{column}
    \end{columns}
\end{frame}

\begin{frame}{An Example State - By Channels}
    \begin{columns}
        \begin{column}{0.4\linewidth}
            Legend:
            \begin{figure}
                \resizebox{!}{0.5\linewidth}{%
                \NewDocumentCommand{\drawgrid}{O{black} O{8} O{#2}}{%
    % #1 = color (optional, defaults to black)
    % #2 = size (e.g., 8 for 8x8)
    % #3 = second size (e.g., 4 for 8x4)
    \foreach \i in {0,...,#2} {%
        \draw[thin, color={#1}] (\i,0) -- (\i,#3);}
    \foreach \j in {0,...,#3} {%
        \draw[thin, color={#1}] (0,\j) -- (#2,\j);}
}

\NewDocumentCommand{\dart}{O{} O{(0,0)} O{0}}{%
    % #1 = draw options
    % #2 = center coordinate
    % #3 = degree rotation
    \coordinate (center) at #2;
    \draw[#1]
        ($(center) + ({ 0 +#3}:0.4)$) --
        ($(center) + ({130+#3}:0.4)$) --
        ($(center) + ({180+#3}:0.1)$) --
        ($(center) + ({230+#3}:0.4)$) --
        cycle;
}

\NewDocumentCommand{\isorect}{O{} O{(0,0)} O{0}}{%
    % #1 = draw options
    % #2 = center coordinate
    % #3 = degree rotation
    \coordinate (center) at #2;
    \def\wid{0.4}
    \draw[#1]
        ($(center) + ({-45+#3}:\wid)$) --
        ($(center) + ({ 45+#3}:\wid)$) --
        ($(center) + ({135+#3}:\wid)$) --
        ($(center) + ({225+#3}:\wid)$) --
        cycle;
}

\NewDocumentCommand{\isocirc}{O{} O{(0,0)}}{%
    % #1 = draw options
    % #2 = center coordinate
    \coordinate (center) at #2;
    \def\ra{0.3}
    \draw[#1] ($(center) + (0:\ra)$) 
        arc (0:90:\ra) 
        arc (90:180:\ra)
        arc (180:270:\ra)
        arc (270:360:\ra);
}

\begin{tikzpicture}
    \node[] (lab1) at (0,0) {Agent};
    \dart[draw=cyan,fill=cyan!30][(1,0)][0]
    
    \node[] (lab2) [below of=lab1] {Sight};
    \fill[fill=cyan!15] ($(lab2) + (0.75,-0.5)$) -- +(1,0) -- +(1,1) -- +(0,1) -- cycle;
    
    \node[] (lab3) [below of=lab2] {Objective};
    \isocirc[magenta, fill=magenta!50][($(lab3)+(1.25,0)$)]

    \node[] (lab4) [below of=lab3] {Obstacle};
    \isorect[draw=green, fill=green!50][($(lab4)+(1.25,0)$)]
\end{tikzpicture}
                }
            \end{figure}
        \end{column}
        \begin{column}{0.6\linewidth}
            \centering
            \begin{figure}
                \resizebox{!}{0.75\linewidth}{%
                    \NewDocumentCommand{\drawgrid}{O{black} O{8} O{#2}}{%
    % #1 = color (optional, defaults to black)
    % #2 = size (e.g., 8 for 8x8)
    % #3 = second size (e.g., 4 for 8x4)
    \foreach \i in {0,...,#2} {%
        \draw[thin, color={#1}] (\i,0) -- (\i,#3);}
    \foreach \j in {0,...,#3} {%
        \draw[thin, color={#1}] (0,\j) -- (#2,\j);}
}

\NewDocumentCommand{\dart}{O{} O{(0,0)} O{0}}{%
    % #1 = draw options
    % #2 = center coordinate
    % #3 = degree rotation
    \coordinate (center) at #2;
    \draw[#1]
        ($(center) + ({ 0 +#3}:0.4)$) --
        ($(center) + ({130+#3}:0.4)$) --
        ($(center) + ({180+#3}:0.1)$) --
        ($(center) + ({230+#3}:0.4)$) --
        cycle;
}

\NewDocumentCommand{\isorect}{O{} O{(0,0)} O{0}}{%
    % #1 = draw options
    % #2 = center coordinate
    % #3 = degree rotation
    \coordinate (center) at #2;
    \def\wid{0.4}
    \draw[#1]
        ($(center) + ({-45+#3}:\wid)$) --
        ($(center) + ({ 45+#3}:\wid)$) --
        ($(center) + ({135+#3}:\wid)$) --
        ($(center) + ({225+#3}:\wid)$) --
        cycle;
}

\NewDocumentCommand{\isocirc}{O{} O{(0,0)}}{%
    % #1 = draw options
    % #2 = center coordinate
    \coordinate (center) at #2;
    \def\ra{0.3}
    \draw[#1] ($(center) + (0:\ra)$) 
        arc (0:90:\ra) 
        arc (90:180:\ra)
        arc (180:270:\ra)
        arc (270:360:\ra);
}

\begin{tikzpicture}[3d view]
    % Agent:
    \fill[fill=cyan!15] ($(2.5,4.5)+(-0.5,-1.5)$) -- +(3,0) -- +(3,3) -- +(0,3) -- cycle;
    \drawgrid
    \dart[draw=cyan,fill=cyan!30][(2.5, 4.5)]
    % Objectives:
    \isocirc[magenta, fill=magenta!50] [(1.5,1.5)]
    \isocirc[red, fill=red!50] [(6.5,6.5)]
    \isocirc[blue, fill=blue!50] [(4.5,3.5)]
    % Obstacles:
    \isorect[green, fill=green!50][(1.5,6.5)]
    \isorect[yellow, fill=yellow!50][(7.5,3.5)]
    \isorect[black, fill=gray!15][(3.5,2.5)]

    \tikzset{shift={(0,0,-3)}}
    % Agent:
    % \fill[fill=cyan!15] ($(2.5,4.5)+(-0.5,-1.5)$) -- +(3,0) -- +(3,3) -- +(0,3) -- cycle;
    \drawgrid[red!75!black]
    % \dart[draw=cyan,fill=cyan!30][(2.5, 4.5)]
    % Objectives:
    \isocirc[magenta, fill=magenta!50] [(1.5,1.5)]
    \isocirc[red, fill=red!50] [(6.5,6.5)]
    % \isocirc[blue, fill=blue!50] [(4.5,3.5)]
    % Obstacles:
    % \isorect[green, fill=green!50][(1.5,6.5)]
    \isorect[yellow, fill=yellow!50][(7.5,3.5)]
    \isorect[black, fill=gray!15][(3.5,2.5)]


    \tikzset{shift={(0,0,-3)}}
    % Agent:
    \fill[fill=cyan!15] ($(2.5,4.5)+(-0.5,-1.5)$) -- +(3,0) -- +(3,3) -- +(0,3) -- cycle;
    \drawgrid[blue!75!black]
    \dart[draw=cyan,fill=cyan!30][(2.5, 4.5)]
    % Objectives:
    \isocirc[magenta, fill=magenta!50] [(1.5,1.5)]
    % \isocirc[red, fill=red!50] [(6.5,6.5)]
    \isocirc[blue, fill=blue!50] [(4.5,3.5)]
    % Obstacles:
    % \isorect[green, fill=green!50][(1.5,6.5)]
    % \isorect[yellow, fill=yellow!50][(7.5,3.5)]
    \isorect[black, fill=gray!15][(3.5,2.5)]


    \tikzset{shift={(0,0,-3)}}
    % Agent:
    \fill[fill=cyan!15] ($(2.5,4.5)+(-0.5,-1.5)$) -- +(3,0) -- +(3,3) -- +(0,3) -- cycle;
    \drawgrid[green!75!black]
    \dart[draw=cyan,fill=cyan!30][(2.5, 4.5)]
    % Objectives:
    % \isocirc[magenta, fill=magenta!50] [(1.5,1.5)]
    % \isocirc[red, fill=red!50] [(6.5,6.5)]
    % \isocirc[blue, fill=blue!50] [(4.5,3.5)]
    % Obstacles:
    \isorect[green, fill=green!50][(1.5,6.5)]
    \isorect[yellow, fill=yellow!50][(7.5,3.5)]
    \isorect[black, fill=gray!15][(3.5,2.5)]
\end{tikzpicture}
                }
            \end{figure}
        \end{column}
    \end{columns}
\end{frame}

% RT 1.1
    % Architectures: HAPPO, COMA w/ Transformer, Simple-equivariant spanning policy
% RT 1.3
% RT 1.4
    % Baseline, and training types

\subsubsection{Training Architectures}

\begin{frame}{Architectures Compared (RT 1.1,1.4)}
    \textbf{Models Evaluated}
    \begin{enumerate}
        \item \textbf{HAPPO (Baseline)}\footcite{zhong2024}
          \begin{itemize}
            \item Separate PPO-trained policy per agent.
            \item No shared parameters or symmetry constraints.
            \item Chosen for its role as a lightweight and well-established heterogeneous MARL benchmark:
            \begin{itemize}
                \item Based on PPO, a stable and widely adopted algorithm.
                \item Among the most computationally efficient heterogeneous-capable methods available.
                \item Minimal coordination overhead and straightforward implementation.
                \item Serves as a reference point for evaluating cost-effectiveness and generality of more complex policies.
            \end{itemize}
          \end{itemize}
        % \item \textbf{PIC}\footcite{liu2020b}
        %   \begin{itemize}
        %     % \item \emph{COMA + GNN + Transformer}
        %     \item \emph{GNN + MADDPG}
        %     \item Graph neural network for relational encoding.
        %     \item Transformer-based critic for contextual estimation.
        %   \end{itemize}
        \item \textbf{Equivariant Spanning Policy (Proposed)}
          \begin{itemize}
            \item Global observation/action spaces with dynamic masking.
            \item Equivariant layers: \(\lambda I + \gamma \mathbf{11}^\top\).
            \item Shared policy supports heterogeneous agents natively.
          \end{itemize}
    \end{enumerate}
\end{frame}




\subsection{Experimental Procedure}
% RT 2.1
% RT 2.2
% RT 2.3
% RT 2.4

\begin{frame}{Training Efficiency}
\begin{columns}
    \begin{column}{0.5\linewidth}
        \textbf{Training Configurations}
            \begin{itemize}
                \item 3 Algorithms from previous slide.
                \item Observability coverage:
                    \begin{enumerate}
                        \item All agents \(I\) have all \(C\)
                        \item \(\{c(i)\}\) spans \(C\) with overlap
                        \item \(\{c(i)\}\) spans \(C\) without overlap
                        \item \(\{c(i)\} \subset C\) 
                    \end{enumerate}
                \item Independent runs ensure robustness to stochasticity.
            \end{itemize}
    \end{column}
    \begin{column}{0.5\linewidth}
        \textbf{Evaluation metrics:}
            \begin{itemize}
                \item \textbf{Convergence rate} (agent-steps normalized).
                \item \textbf{Computational cost} (training effort).
            \end{itemize}
    \end{column}
\end{columns}
\end{frame}





\begin{frame}{Perturbation Tests}
\textbf{Runtime Robustness Evaluation}
\begin{itemize}
    \item \textbf{Sensor Degradation}
    \begin{itemize}
        \item Random dropout of observation channels.
        \item Masking updated during inference to simulate occlusion.
    \end{itemize}
    \item \textbf{Team-Size Changes}
    \begin{itemize}
        \item Agents added/removed post-training.
        \item Evaluate for reward degradation and adaptation.
    \end{itemize}
    \item \textbf{Novel Team Composition}
    \begin{itemize}
        \item For our spanning policy
        \item Trained \(\{c(i)\}\neq\) tested \(\{c(i)\}\)
    \end{itemize}
    % \item Metrics:
    % \begin{itemize}
    %     \item Reward stability under perturbation.
    %     \item Policy robustness without retraining.
    % \end{itemize}
\end{itemize}
\end{frame}

% \subsection{Results}
% \subsection{Discussion}
