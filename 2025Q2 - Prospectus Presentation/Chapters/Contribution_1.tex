\section{Contribution 1}

% For each Contribution:
% Re-motiviate
%     Introduction
%         Lit review
%         Contribution
%     Methodology
%     Experimental Procedure
%     Results
%     Discussion

\subsection{Introduction}

\begin{frame}
\end{frame}

\subsubsection{Literature Review}

\subsubsection{Research Questions}

\begin{frame}{Research Questions}
    \begin{enumerate}
        \item[RQ 1] {
            Can pretraining smaller teams of agents and then scaling to the target 
            team size via policy duplication and retraining improve training efficiency 
            without sacrificing final policy performance in MARL?}
        \item[RQ 2] {
            How does the effectiveness of this direct scaling strategy vary across 
            environments with different forms of agent heterogeneity 
            (e.g., behavioral vs. intrinsic)?}
    \end{enumerate}
\end{frame}

\begin{frame}{RQ 1 - Research Tasks}
    \begin{enumerate}
        \item[RQ 1] \textcolor{gray}{
            Can pretraining smaller teams of agents and then scaling to the target 
            team size via policy duplication and retraining improve training efficiency 
            without sacrificing final policy performance in HARL? } \vspace{1em}
    \begin{itemize}
        \item[RT 1.1] {
            Design an upsampling-based curriculum using policy duplication and retraining.}
        \item[RT 1.2] {
            Define a metric (agent-steps) accounting for agent count and training time.}
        \item[RT 1.3] {
            Train tabula rasa agents each target environment and team size as baselines.}
        \item[RT 1.4] {
            Evaluate training performance across various pretraining length and target team sizes.}
    \end{itemize}
    \end{enumerate}
\end{frame}

\begin{frame}{RQ 2 - Research Tasks}
    \begin{enumerate}
        \item[RQ 2] \textcolor{gray}{
            How does the effectiveness of this direct scaling strategy vary across 
            environments with different forms of agent heterogeneity 
            (e.g., behavioral vs. intrinsic)? } \vspace{1em}
    \begin{itemize}
        \item[RT 2.1] {
            Select environments that represent distinct forms of agent heterogeneity. \\
            Behavioral, Intrinsic.}
        \item[RT 2.2] {
            Adapt observation structures to enable fixed policy architectures across team sizes.}
        \item[RT 2.3] {
            Evaluate the effect of heterogeneity type on the scalability and retraining benefit.}
    \end{itemize}
    \end{enumerate}
\end{frame}

% \subsection{Methodology}
% \subsection{Experimental Procedure}
% \subsection{Results}
% \subsection{Discussion}