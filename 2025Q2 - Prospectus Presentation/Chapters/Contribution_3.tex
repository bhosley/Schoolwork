\section{Contribution 3}

% For each Contribution:
%     Introduction
%         Recall Motivation
%         Lit review
%         Contribution
%     Methdology
%     Experimental Procedure
%     Results
%     Discussion
\subsection{Introduction}

\begin{frame}{The Case for Progressive Network Growth}
    \begin{itemize}
        \item In reinforcement learning, larger networks often yield better performance—but 
            training large models from scratch can be inefficient and unstable.
        \item Many real-world systems require policies to grow over time in response to 
            increasing task complexity or mission demands.
        \item Projection-based methods offer a way to expand policy networks mid-training 
            without discarding previously learned behavior.
        \item This contribution investigates the feasibility and benefits of progressive 
            architectural growth as an efficient training strategy.
    \end{itemize}
\end{frame}

\subsubsection{Literature Review}

\subsubsection{Research Questions}

\begin{frame}{Contribution 3 - Research Questions}
    \begin{enumerate}
        \item[RQ 1] {
            Can tensor-based projections be used to grow a policy network's capacity during 
            training while preserving the functional behavior of the original network?}
        \item[RQ 2] {
            Can we identify appropriate transition points during training when projecting 
            to a larger policy network yields the greatest benefit?}
        \item[RQ 3] {
            Does this progressive architectural growth strategy reduce total training cost 
            or improve final policy performance compared to fixed-size architectures?}
    \end{enumerate}
\end{frame}

\begin{frame}{RQ 1 - Research Tasks}
    \begin{enumerate}
        \item[RQ 1] \textcolor{gray}{
            Can tensor-based projections be used to grow a policy network's capacity during 
            training while preserving the functional behavior of the original network? }
            \vspace{1em}
    \begin{itemize}
        \item[RT 1.1] {
            Select a method to project policy networks into higher-dimensional tensor 
            representations at scheduled points during training.}
        \item[RT 1.2] {
            Validate that projected networks approximate the original network's behavior 
            by measuring divergence in policy outputs } 
            %(e.g., bounded error on inference)(e.g., action distributions or logits).}
        \item[RT 1.3] {
            Integrate projection-based expansion into a curriculum training pipeline 
            and assess training stability.}
    \end{itemize}
    \end{enumerate}
\end{frame}

\begin{frame}{RQ 2 - Research Tasks}
    \begin{enumerate}
        \item[RQ 2] \textcolor{gray}{
            Can we identify appropriate transition points during training when projecting 
            to a larger policy network yields the greatest benefit? } \vspace{1em}
    \begin{itemize}
        \item[RT 2.1] {
            Define a set of projection schedules using fixed step counts or adaptive triggers 
            (e.g., plateau detection) to determine when network expansion occurs during training.}
        \item[RT 2.2] {
            Treat projection timing as a tunable training parameter and evaluate its impact on 
            learning dynamics across the defined schedule conditions.}
        \item[RT 2.3] {
            Compare total training cost, convergence rate, and final policy quality across 
            projection schedules and fixed-size network baselines.}
        \item[RT 2.4] {
            Analyze projection timing effects by measuring short-term learning disruption 
            and long-term policy improvement.}
    \end{itemize}
    \end{enumerate}
\end{frame}

\begin{frame}{RQ 3 - Research Tasks}
    \begin{enumerate}
        \item[RQ 3] \textcolor{gray}{
            Does this progressive architectural growth strategy reduce total training cost 
            or improve final policy performance compared to fixed-size architectures? }
            \vspace{1em}
    \begin{itemize}
        \item[RT 3.1] {
            Construct fixed-size policy network baselines that match the initial, 
            intermediate, and final sizes used in the progressive curriculum.}
        \item[RT 3.2] {
            Evaluate relative training efficiency by comparing convergence speed, 
            sample efficiency, and performance trajectories across all conditions.}
    \end{itemize}
    \end{enumerate}
\end{frame}

% \subsection{Methodology}
% \subsection{Experimental Procedure}
% \subsection{Results}
% \subsection{Discussion}