\section{Introduction}

\begin{frame}{The Vision for Autonomy in Defense}
    \begin{columns}
        \begin{column}{0.6\textwidth}
            \begin{itemize}
                \item \textbf{Department of Defense initiatives are actively advancing 
                    autonomous system deployment.}
                \begin{itemize}
                    \item \textbf{Replicator Initiative:} {Calls for the rapid integration of 
                        commercial-off-the-shelf (COTS) autonomous systems for scalable 
                        deployment. \footcite{robertson2023} }
                    \item \textbf{DARPA's OFFSET:} {Demonstrated feasibility of swarm-enabled 
                        ground and aerial assets in contested environments. 
                        \footcite{zotero-2835} }
                \end{itemize}
                \item \textbf{Limitation:} {Reliable autonomy in unpredictable environments 
                    remains an unsolved challenge.}
            \end{itemize}
        \end{column}
        \begin{column}{0.4\textwidth}
            \begin{figure}[!h]
                \centering
                \includegraphics[width=0.95\textwidth]{replicator_collage.png}
                \caption{\textbf{Replicator Initiative} aimed at multiple platforms. 
                    \footcite{robertson2023} }
                \label{fig:replicator_collage}
            \end{figure}
        \end{column}
    \end{columns}
\end{frame}


\begin{frame}{Rules-Based Control}
    \begin{columns}
        \begin{column}{0.6\textwidth}
            \begin{itemize}
                \item The earliest autonomous systems operate on hand-coded rules.
                \item Effective in constrained environments with predictable dynamics.
                \item Limitations:
                \begin{itemize}
                    \item Brittle under changing conditions.
                    \item Poor generalization to novel scenarios.
                \end{itemize}
                % \item Led to interest in adaptive learning systems.
            \end{itemize}
        \end{column}
        \begin{column}{0.4\textwidth}
            % \item Example is also the majority of video game "AI"
            \textbf{Example:}
            MYCIN, an expert System for diagnosis of bacterial infections \footcite{buchanan1984}.
            \begin{figure}[!h]
                \centering
                \includegraphics[width=0.65\textwidth]{mycin_logic.png}
                \caption{\textbf{MYCIN}'s logical flow.}
                \label{fig:mycin_logic}
            \end{figure}
        \end{column}
    \end{columns}
\end{frame}

\begin{frame}{Reinforcement Learning (RL)}
    \begin{itemize}
        \item RL enables agents to learn behaviors through trial and 
            error in dynamic environments.
        % \item Successes in single-agent settings with clear reward signals and isolated tasks.
    \end{itemize}
    \begin{columns}
        \begin{column}{0.65\textwidth}
            \begin{itemize}
                \item Generally follows the Agent-Environment Cycle (AEC) paradigm. 
                    \footcite{sutton2018}
                \begin{itemize}
                    \item Modeled well by Markov Decision Processes (MDPs).
                \end{itemize}
                \item Limitations:
                \begin{itemize}
                    \item Multi-agent scenarios are modeled by 
                        composed action and observation spaces.
                    \item Composed spaces are larger and encode more rigid agent interactions.
                \end{itemize}
            \end{itemize}
        \end{column}
        \begin{column}{0.3\textwidth}
            \begin{figure}[!h]
                \centering
                % \begin{tikzpicture}[>=stealth, node distance=1cm, on grid, auto,
    entry/.style = {draw, rectangle, inner sep=8pt, rounded corners=3pt,
                    minimum width=3cm},
    arrow/.style = {thick,-stealth}]

    % States
    \node [] (center) {};
    \node [entry, fill= Blue!15] (agent) [above=of center] {Agent};
    \node [entry, fill=Green!15] (env) [below=of center] {Environment};
    \node [] (time) [below left= 0.6cm and 2.3cm of env]{\scriptsize\(t:=t+1\)};
    \draw [dashed] (time) -- +(0,1);

    % Transitions
    \draw [arrow] (agent.east) -- +(1,0) --coordinate(A) +(1,-2) -- (env.east);
    \draw [arrow] ([yshift=-0.2cm]env.west) -- +(-1.8,0) --coordinate(S) 
                    +(-1.8,2.4) -- ([yshift=+0.2cm]agent.west);
    \draw [arrow] ([yshift=+0.2cm]env.west) -- +(-1.4,0) --coordinate(R) 
                    +(-1.4,1.6) -- ([yshift=-0.2cm]agent.west);

    % Labels
    \node [] () [right= 1pt of A] {\(a_t\in\) \Gls{a}};
    \node [] () [left=  1pt of S] {\(s_t\in\) \Gls{s}};
    \node [] () [right= 1pt of R] {\(r_t\in\) \Gls{r}};
\end{tikzpicture}
                \includegraphics[width=0.95\textwidth]{ae_cycle.tex}
                \caption{\textbf{AEC}: Agent-Environment Cycle}
                \label{fig:ae_cycle}
            \end{figure}
        \end{column}
    \end{columns}
    % Extending Markov Chains (MDPs) to dynamic decision making
\end{frame}

\begin{frame}{Multi-Agent RL (MARL)}
    \begin{itemize}
        \item MARL extends RL to scenarios involving multiple learning agents.
        \item Used for both cooperative and competitive environments (e.g., games, simulations).
        \item Limitations:
        \begin{itemize}
            \item Often assumes agents are homogeneous or interchangeable.
            \item High coordination cost and non-stationarity challenges. \footcite{albrecht2024}
        \end{itemize}
    \end{itemize}
    \begin{figure}[!h]
        \centering
        \includegraphics[width=0.45\textwidth]{mae_cycle.tex}
        \caption{Multi-Agent AEC}
        \label{fig:marl_aec}
    \end{figure}
\end{frame}

\begin{frame}{Heterogeneous-Agent RL (HARL)}
    \begin{itemize}
        \item HARL addresses coordination among agents that may differ structurally in
            observation and/or action spaces.
    \end{itemize}
    \begin{columns}
        \begin{column}{.35\textwidth}
            \begin{itemize}
                \item Promising for real-world applications with diverse platforms 
                    (e.g., COTS drones, mixed teams\footcite{guo2024}).
                \item Limitations:
                \begin{itemize}
                    \item Limited literature.
                    \item Higher training cost than MARL.
                \end{itemize}
            \end{itemize}
            \hfil
        \end{column}
        \begin{column}{.6\textwidth}
            \begin{figure}[!h]
                \centering
                \includegraphics[width=0.75\textwidth]{hae_cycle.tex}
                \caption{Heterogeneous-Agent AEC}
                \label{fig:harl_aec}
            \end{figure}
        \end{column}
    \end{columns}
\end{frame}

\begin{frame}{Open Challenges in Scalable Multi-Agent Autonomy}
    \begin{itemize}
        \item Integration across heterogeneous platforms is constrained by inflexible model assumptions.
        \item Retraining cost grows rapidly with team size and complexity.
        \item Teams struggle to adapt dynamically to changing sensors or agent availability.
        \item Existing strategies for curriculum learning and network scaling are underdeveloped.
    \end{itemize}
\end{frame}

\begin{frame}{Why Heterogeneous Autonomy Matters}
\begin{columns}
    \begin{column}{0.6\textwidth}
    \begin{itemize}
        \item \textbf{Operational autonomy must account for platform diversity.}
        \begin{itemize}
            \item Teams may integrate UAVs and UGVs with different sensors, dynamics, and connectivity constraints.
            \item Heterogeneous-agent reinforcement learning (HARL) enables coordination across dissimilar platforms.
        \end{itemize}
        \item \textbf{Policy transfer and input-invariant networks} can support:
        \begin{itemize}
            \item Adaptability to new team compositions.
            \item Resilience to partial observation loss or sensor degradation.
        \end{itemize}
    \end{itemize}
    \end{column}
    \begin{column}{0.4\textwidth}
        \begin{figure}[!h]
            \centering
            \includegraphics[width=0.95\textwidth]{offset_collage.jpg}
            \caption{\textbf{OFFSET} test platforms.
                \footnote[frame]{\cite{zotero-2835}}}
            \label{fig:offset_collage}
        \end{figure}
    \end{column}
\end{columns}
\end{frame}

% Research Contributions
\begin{frame}{Research Contributions}
    \begin{itemize}
        \item {% Through-line:
            Advance the study of heterogeneous-agent reinforcement learning by exploring 
            scalable training architectures and policy designs that support adaptability 
            to agent diversity and team variation.}
            \begin{enumerate}
                \item {% Contribution 1:
                    Evaluate a policy upsampling strategy for training larger multi-agent teams 
                    more efficiently using pretrained smaller-team policies.}
                \item {% {Contribution 2:
                    Investigate input-invariant policy architectures for shared learning and 
                    robustness in teams of heterogeneous agents.}
                \item {% Contribution 3: 
                    Design and evaluate a curriculum that progressively expands network capacity 
                    via tensor projection to improve training efficiency.}
            \end{enumerate}
    \end{itemize}
\end{frame}

\begin{frame}{Research Objectives}
    \textbf{Contribution 1}
    % \textbf{Contribution 1 - Direct Scaling of Teams}
    \begin{itemize}
        \item {Improve training efficiency by scaling teams from smaller pretrained groups 
            using policy duplication instead of retraining from scratch.}
        \item {Demonstrate that policy reuse across increasing agent counts can reduce 
            training cost while maintaining performance.}
    \end{itemize}
    \vspace{1em}
    \textbf{Contribution 2}
    % \textbf{Contribution 2 - Input-Invariant Policy Architectures}
    \begin{itemize}
        \item {Construct input-invariant policy architectures to enable shared training 
            updates across agents with heterogeneous observation structures.}
        \item {Evaluate the effectiveness of input-invariant models in improving learning 
            efficiency in teams with overlapping observations.}
        \item {Test whether these models maintain stable performance under dynamic 
            observation space changes during execution.}
    \end{itemize}
\end{frame}

\begin{frame}{Research Objectives}
    \textbf{Contribution 3}
    % \textbf{Contribution 3 - Progressive Network Growth via Tensor Projections}
    \begin{itemize}
        \item {Demonstrate the feasibility of expanding a policy network mid-training 
            using tensor projection without discarding prior knowledge.}
        \item {Identify when during training such growth provides learning advantages 
            over fixed-size architectures.}
        \item {Compare performance and efficiency of progressive-growth networks to 
            fixed-size baselines in terms of convergence and cost.}
    \end{itemize}
\end{frame}