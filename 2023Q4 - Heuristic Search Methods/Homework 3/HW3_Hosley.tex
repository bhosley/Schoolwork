\documentclass[12pt,letterpaper]{exam}
\usepackage[utf8]{inputenc}
\usepackage[T1]{fontenc}
\usepackage[width=8.50in, height=11.00in, left=0.50in, right=0.50in, top=0.50in, bottom=0.50in]{geometry}

\usepackage{libertine}
\usepackage{multicol}
\usepackage[shortlabels]{enumitem}

\usepackage{booktabs}
\usepackage[table]{xcolor}

\usepackage{amssymb}
\usepackage{amsthm}
\usepackage{mathtools}
\usepackage{bbm}
\usepackage{comment}

\usepackage{hyperref}
\usepackage{graphicx}
%\usepackage{wrapfig}
%\usepackage{capt-of}
%\usepackage{tikz}
%\usepackage{pgfplots}
%\usetikzlibrary{shapes,arrows,positioning,patterns}
%\usepackage{pythonhighlight}

\newcommand\chapter{ 3 }
%\renewcommand{\thequestion}{\textbf{\chapter.\arabic{question}}}
%\renewcommand{\questionlabel}{\thequestion}

%%%%%%%%%%%%%%%%%%%%%%%%%%%%%%%%%%%%%%%%%%%%%%%%%%%%%%%%%%%%%%%%%%
\newcommand{\class}{ OPER 623 - Heuristic Search Methods } % This is the name of the course 
\newcommand{\assignmentname}{Homework \# \chapter} % 
\newcommand{\authorname}{Hosley, Brandon} % 
\newcommand{\workdate}{\today} % 
\printanswers % this includes the solutions sections
%%%%%%%%%%%%%%%%%%%%%%%%%%%%%%%%%%%%%%%%%%%%%%%%%%%%%%%%%%%%%%%%%%



\begin{document}
	\pagestyle{plain}
	\thispagestyle{empty}
	\noindent
	
	%%%%%%%%%%%%%%%%%%%%%%%%%%%%%%%%%%%%%%%%%%%%%%%%%%%%%%%%%%%%%%%%%%%%%%%%%%%%%%%%%%%
	\noindent
	\begin{tabular*}{\textwidth}{l @{\extracolsep{\fill}} r @{\extracolsep{10pt}} l}
		\textbf{\class} & \textbf{\authorname}  &\\ %Your name here instead, obviously 
		\textbf{\assignmentname } & \textbf{\workdate} & \\
	\end{tabular*}\\ 
	\rule{\textwidth}{2pt}
	%%%%%%%%%%%%%%%%%%%%%%%%%%%%%%%% HEADER %%%%%%%%%%%%%%%%%%%%%%%%%%%%%%%%%%%%%%%%%%%
	
	\begin{questions}
		
		\setcounter{question}{0}
		\question 
		(6 pts) Given that Vertex Cover is an NP Complete problem, prove that Set Cover is an NP Complete problem.
		
		\begin{parts}
			\part
			(1 pt) Explain the optimization versions of Set Cover and Vertex Cover.
			\part
			(1 pt) Explain the decision versions of both problems. 
			\part
			(3 pts) Create an appropriate transformation to show that Set Cover is at least as difficult as Vertex Cover 
			(Warning: You should reduce the known NP-complete problem to the problem you are interested in) Steps you should follow:
			
			\begin{subparts}
				\subpart
				Generate an illustrative example to show your transformation from one instance to another.
				\subpart
				Generalize your example to cover any arbitrary instances.
				\subpart
				Prove your transformation meets all of Tovey’s 3 Criteria
			\end{subparts}
			
			\part
			(1 pt) Prove Set Cover is in NP
		\end{parts}
		
		\begin{solution}
			\begin{parts}
				\part
				
				In set cover the optimization objective is to select the smallest number of subsets such that each element in the (Universe)
				set is included in the union of the subsets.
				
				Similarly, the optimization object in the case of the vertex cover problem is to select the smallest number of nodes (vertices)
				on an undirected graph such that ever edge is connected to at least one of the selected nodes.
				
				\part
				
				In the case of set coverage, the decision version of the problem asks if there is a \(k\) sized group of subsets \(s_i \in S \)
				such that \(S \equiv \bigcup_{i=0}^k s_i \) .
				
				And for the vertex cover problem we may ask, is there a set of \(k\) nodes such that every edge in a graph 
				is connected to at least one of the selected nodes.
				
				\part
				
				For any graph, let \(v\) be a vertex, \(e\) an edge, \(V\) be the set of all vertices, and \(E\) be the set of all edges in the graph.
				A vertex \(v\) is meaningfully a set of the edges that join to create it.
						
				\begin{center}
					\includegraphics[width=0.2\linewidth]{6n-graf.svg}
				\end{center}
				
				Using this graph as an example we can illustrate 
				\(v_1 = \{e_{12},e_{15}\}\) where the subscript represent the label for the node, and the nodes at either end for an edge.
				
				In this manner we transform the vertex covering problem into a set cover problem.
				If it is true for a subset \(S\subset V\) that \(E\subset \bigcup_{i\in S} v_i \), in the vertex covering problem, 
				it is true in the set covering problem.
				
				The first two of Tovey's criteria are confirmed by the fact that in this representation, 
				all possible vertex covering problems are a specific type of set covering problem in which each element in the universe set
				is a member of exactly two subsets, and each subset can have any number (\(>0\)) of elements as members.
				
				This transformation is likely linear as it requires only one pass when translating vertices to sets and when collecting the set of edges.
				
				\part
				In the previous part we described the nature of a function that is injective, invertable, and could be executed in linear time
				that maps every possible vertex covering problem into a type of set cover problem, 
				and is thus set cover is at least as hard as vertex cover.
				
				Because we have been given that the vertex cover problem is NP complete, we can conclude that the set cover problem is also in NP.
				
				As a result  set covering problem 
				
			\end{parts}
			
		\end{solution}
		%%%%%%%%%%%%%%%%%%%%%%%%%%%%%%%%%%%%%%%%%%%%%%%%%%%%%%%%%%%%%
		
		\question 
		(4 pts) Propose a constructive heuristic to solve the 0-1 Multi-Constraint Knapsack Problem (MKP).  
		
		\begin{parts}
			\part
			The MKP is a version of the Knapsack problem where instead of maximizing the utility 
			of a knapsack subject to a single constraint (e.g., weight) you are maximizing the utility of the knapsack subject to multiple constraints (e.g., weight, volume, radioactivity, etc).  Thus, each item will have a utility score \(u_i\), and parameter values for each of these constraints (e.g., \(w_i, v_i, r_i\)). 
			
			(Please note this is not the multiple constraint, multiple knapsack problem from a prior homework.  There is only a single knapsack.)
			
			\part
			The 0-1 MKP focuses on the specific situation where you can either bring an item or not (i.e no duplicates and no partial items.)  
		\end{parts}
		
		\begin{solution}
			A simple, linear time, option is to proceed through the items, adding them to the sack until one triggers one of the constraints.
			Then for each subsequent item one will examine if adding this item represents an improvement over another,
			this means that the \(u_{\text{new}}> u_{\text{old}}\) and \(\{w, v, r\}_{\text{new}} < \{w, v, r\}_{\text{old}} + \{w, v, r\}_{\text{slack}} \).
			If both of these are true, replace the item and move on. 
			
			If one is willing to spend a little extra time on the problem they might consider every subset of items in the sack for replacement instead of
			single items. \\
			
			\hrule
			
			To propose a destructive heuristic as an answer to a constructive heuristic question I would propose the following.
			\begin{enumerate}
				\item Add every item to the knapsack.
				\item Replace each item's attributes by that attribute's value divided by the item's utility \(\frac{\{w_i, v_i, r_i\}}{u_i}\)
				\item Identify which of the sack's constraints are most violated; if all are feasible, then we are done packed!
				\item Remove the item with the highest adjusted value in the previously identified constraint.
				\item Return to step 3.
			\end{enumerate}		
			
		\end{solution}
		%%%%%%%%%%%%%%%%%%%%%%%%%%%%%%%%%%%%%%%%%%%%%%%%%%%%%%%%%%%%%
		
	\end{questions}
\end{document}  


\begin{comment}
			Assume 0-1 means item is or is not
			
			Let 
			\(I\) = set of all items
			\(A\) be a matrix (vector?) of all item's constraints \(\{w_i, v_i, r_i\}_{i\in I}\)
			
			
			Assume all items are in the pack, then we will remove
			
			A = I x C matrix of \(\{w_i, v_i, r_i\}/ u_i\)
			B = I x 1
			
			A @ B  = vector of con values C, 
			C subtracted by bag capacity, select the highest value constraint c
			c row of A @ B (t?) ... a 1xI or Ix1 result
			Select the highest value
			change the corresponding item's element in B from 1 to 0
			
			repeat until C subtract BagCap is <0 for each element
			
			Resulting B is the matrix of items not included, 
			\(\not B\) is the affirmative version
\end{comment}			