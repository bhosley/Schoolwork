\documentclass[12pt,letterpaper]{exam}
\usepackage[utf8]{inputenc}
\usepackage[T1]{fontenc}
\usepackage[width=8.50in, height=11.00in, left=0.50in, right=0.50in, top=0.50in, bottom=0.50in]{geometry}

\usepackage{libertine}
\usepackage{multicol}
\usepackage[shortlabels]{enumitem}

\usepackage{booktabs}
\usepackage[table]{xcolor}

\usepackage{amssymb}
\usepackage{amsthm}
\usepackage{mathtools}
\usepackage{bbm}

\usepackage{hyperref}
\usepackage{graphicx}
%\usepackage{wrapfig}
%\usepackage{capt-of}
%\usepackage{tikz}
%\usepackage{pgfplots}
%\usetikzlibrary{shapes,arrows,positioning,patterns}
%\usepackage{pythonhighlight}

\newcommand\chapter{ 3 }
%\renewcommand{\thequestion}{\textbf{\chapter.\arabic{question}}}
%\renewcommand{\questionlabel}{\thequestion}

%%%%%%%%%%%%%%%%%%%%%%%%%%%%%%%%%%%%%%%%%%%%%%%%%%%%%%%%%%%%%%%%%%
\newcommand{\class}{ OPER 623 - Heuristic Search Methods } % This is the name of the course 
\newcommand{\assignmentname}{Homework \# \chapter} % 
\newcommand{\authorname}{Hosley, Brandon} % 
\newcommand{\workdate}{\today} % 
\printanswers % this includes the solutions sections
%%%%%%%%%%%%%%%%%%%%%%%%%%%%%%%%%%%%%%%%%%%%%%%%%%%%%%%%%%%%%%%%%%



\begin{document}
	\pagestyle{plain}
	\thispagestyle{empty}
	\noindent
	
	%%%%%%%%%%%%%%%%%%%%%%%%%%%%%%%%%%%%%%%%%%%%%%%%%%%%%%%%%%%%%%%%%%%%%%%%%%%%%%%%%%%
	\noindent
	\begin{tabular*}{\textwidth}{l @{\extracolsep{\fill}} r @{\extracolsep{10pt}} l}
		\textbf{\class} & \textbf{\authorname}  &\\ %Your name here instead, obviously 
		\textbf{\assignmentname } & \textbf{\workdate} & \\
	\end{tabular*}\\ 
	\rule{\textwidth}{2pt}
	%%%%%%%%%%%%%%%%%%%%%%%%%%%%%%%% HEADER %%%%%%%%%%%%%%%%%%%%%%%%%%%%%%%%%%%%%%%%%%%
	
	\begin{questions}
		
		\setcounter{question}{0}
		\question 
		(6 pts) Given that Vertex Cover is an NP Complete problem, prove that Set Cover is an NP Complete problem.
		
		\begin{parts}
			\part
			(1 pt) Explain the optimization versions of Set Cover and Vertex Cover.
			\part
			(1 pt) Explain the decision versions of both problems. 
			\part
			(3 pts) Create an appropriate transformation to show that Set Cover is at least as difficult as Vertex Cover 
			(Warning: You should reduce the known NP-complete problem to the problem you are interested in) Steps you should follow:
			
			\begin{subparts}
				\subpart
				Generate an illustrative example to show your transformation from one instance to another.
				\subpart
				Generalize your example to cover any arbitrary instances.
				\subpart
				Prove your transformation meets all of Tovey’s 3 Criteria
			\end{subparts}
			
			\part
			(1 pt) Prove Set Cover is in NP
		\end{parts}
		
		\begin{solution}
			
		\end{solution}
		%%%%%%%%%%%%%%%%%%%%%%%%%%%%%%%%%%%%%%%%%%%%%%%%%%%%%%%%%%%%%
		
		\question 
		(4 pts) Propose a constructive heuristic to solve the 0-1 Multi-Constraint Knapsack Problem (MKP).  
		
		\begin{parts}
			\part
			The MKP is a version of the Knapsack problem where instead of maximizing the utility 
			of a knapsack subject to a single constraint (e.g., weight) you are maximizing the utility of the knapsack subject to multiple constraints (e.g., weight, volume, radioactivity, etc).  Thus, each item will have a utility score ui, and parameter values for each of these constraints (e.g., \(w_i, v_i, r_i\)). 
			
			(Please note this is not the multiple constraint, multiple knapsack problem from a prior homework.  There is only a single knapsack.)
			
			\part
			The 0-1 MKP focuses on the specific situation where you can either bring an item or not (i.e no duplicates and no partial items.)  
		\end{parts}
		
		\begin{solution}
			
		\end{solution}
		%%%%%%%%%%%%%%%%%%%%%%%%%%%%%%%%%%%%%%%%%%%%%%%%%%%%%%%%%%%%%
		
	\end{questions}
\end{document}  
