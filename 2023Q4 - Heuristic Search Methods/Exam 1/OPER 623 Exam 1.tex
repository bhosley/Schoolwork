\documentclass[12pt,letterpaper]{exam}
\usepackage[utf8]{inputenc}
\usepackage[T1]{fontenc}
\usepackage[width=8.50in, height=11.00in, left=0.50in, right=0.50in, top=0.50in, bottom=0.50in]{geometry}

\usepackage{libertine}
\usepackage{multicol}
\usepackage[shortlabels]{enumitem}

\usepackage{booktabs}
\usepackage[table]{xcolor}

\usepackage{amssymb}
\usepackage{amsthm}
\usepackage{mathtools}
\usepackage{bbm}

\usepackage{hyperref}
\usepackage{graphicx}
%\usepackage{wrapfig}
%\usepackage{capt-of}
%\usepackage{tikz}
%\usepackage{pgfplots}
%\usetikzlibrary{shapes,arrows,positioning,patterns}
%\usepackage{pythonhighlight}

%\newcommand\chapter{5}
\renewcommand{\thequestion}{\textbf{\arabic{question}}}
\renewcommand{\thepartno}{\textbf{\arabic{partno}}}
\renewcommand{\thesubpart}{\textbf{\alph{subpart}}}

\renewcommand{\questionlabel}{\thequestion.}
\renewcommand{\partlabel}{\thequestion.\thepartno)}
\renewcommand{\subpartlabel}{\thesubpart.}

%%%%%%%%%%%%%%%%%%%%%%%%%%%%%%%%%%%%%%%%%%%%%%%%%%%%%%%%%%%%%%%%%%
\newcommand{\class}{OPER 623} % This is the name of the course 
\newcommand{\assignmentname}{Exam 1} % 
\newcommand{\authorname}{Hosley, Brandon} % 
\newcommand{\workdate}{\today} % 
\printanswers % this includes the solutions sections
%%%%%%%%%%%%%%%%%%%%%%%%%%%%%%%%%%%%%%%%%%%%%%%%%%%%%%%%%%%%%%%%%%



\begin{document}
\pagestyle{plain}
\thispagestyle{empty}
\noindent
 
%%%%%%%%%%%%%%%%%%%%%%%%%%%%%%%%%%%%%%%%%%%%%%%%%%%%%%%%%%%%%%%%%%%%%%%%%%%%%%%%%%%
\noindent
\begin{tabular*}{\textwidth}{l @{\extracolsep{\fill}} r @{\extracolsep{10pt}} l}
	\textbf{\class} & \textbf{\authorname}  &\\ %Your name here instead, obviously 
	\textbf{\assignmentname } & \textbf{\workdate} & \\
\end{tabular*}\\ 
\rule{\textwidth}{2pt}
%%%%%%%%%%%%%%%%%%%%%%%%%%%%%%%% HEADER %%%%%%%%%%%%%%%%%%%%%%%%%%%%%%%%%%%%%%%%%%%

\begin{questions}
	\setcounter{question}{0}
	
	\question % 1
	\textbf{\large Computational Complexity and Transformations (30 Pts)}
	
	\begin{parts}
		% 1.1
		\part
		(5 pts) Given the following function: \(f(n) = 6n^5 + 5n^4 + 4n^3 + 3n^2 + 2n + 1\)?
		\begin{subparts}
			\subpart
			What is the Big \textbf{O} runtime for \(f(n)\)?
			\subpart
			 Prove your assertion.
		\end{subparts}
		
		\begin{solution}
			\begin{subparts}
				\subpart
				\(f(n) = \mathbf{O}(n^5)\)
				\subpart
				\begin{proof}
					This assertion is definitionally true iff there exists \(c\in\mathbb R\)
					such that \(f(n) \leq c\,n^5\).
					This \(c\) can be calculated as
					\[c \geq ( 6^5 + 5^4 + 4^3 + 3^2 + 2 + 1 )^{1/6} \approx 6.1045\].
					Thus, \[f(n) \leq c\,n^5 \quad \forall c\geq 6.1045.\]
				\end{proof}
			\end{subparts}
		\end{solution}
		
		
		% 1.2
		\part
		(5 pts) You are given a graph \(G = (V, E)\) where \(|V| = n,\text{ and }|E| = m\). 
		You apply the following constructive heuristic to find an Independent Set: 
		Calculate the degree for all vertices \(v \in V\) and pick the \(v_i\) 
		with the lowest degree to be a member of your IS. 
		Recalculate the degree for all vertices \(v \in V\) to all vertices \(v \in V \cap v \notin\) IS
		(i.e., the degree of edges to vertices not in IS) 
		and pick the \(v_i\) with the lowest degree to be a member of your IS.
		Repeat until no new vertices can be added to your IS.
		\begin{subparts}
			\subpart
			What is the Big \textbf{O} notation run time for this Heuristic? 
			(Remember Big \textbf{O} is worst case analysis).
			\subpart
			Prove your assertion.
		\end{subparts}
		
		\begin{solution}
			
		\end{solution}
		
		
		% 1.3
		\part
		(10 pts) Given that Hamiltonian Cycle is in NP-Complete prove that the TSP problem is in NP-Complete. \\
		
		\textbf{Hamiltonian Cycle}: \textit{Given a, \underline{not necessarily complete}, 
			graph G = (V, E), does there exist a cycle that visits all vertices exactly once?}
		\textbf{Traveling Salesman Problem}: \textit{Given a \underline{complete} Graph G=(V,E), 
			does there exist a tour of length k or shorter?}
		\begin{subparts}
			\subpart
			Create a small but interesting instance of Hamiltonian Cycle 
			and show the transformation on that instance.
			\subpart
			Provide a general description of your valid transformation showing that TSP is in NP-Hard
			\subpart
			Show your transformation meets Tovey’s 3 criteria.
			\subpart
			Prove that TSP is in NP
		\end{subparts}
		
		\begin{solution}
			
		\end{solution}
		
		
		% 1.4
		\part
		(10 pts) Prove that the problem of determining if a, not necessarily complete, 
		graph \textbf{G =(V,E)} yields a Spanning Tree where all vertices in the 
		spanning tree have degree 2 or less, is in NP-Complete.
		\begin{solution}
		
		\end{solution}
	\end{parts}		
%%%%%%%%%%%%%%%%%%%%%%%%%%%%%%%%%%%%%%%%%%%%%%%%%%%%%%%%%%%%%
	
	
	\question % 2
	\textbf{\large Constructive Heuristics (30 Pts)}
	Consider the 8 queens problem:
	\begin{parts}
		\part 
		(15 Pts) Propose a constructive heuristic to place all 8 queens 
		which aims to minimize infeasibility of placement.
		\part 
		(5 Pts) Provide a step by step example to illustrate your heuristic.
		\part 
		(5 pts) Critique your heuristic based on the criteria presented in Lesson 4 
		(Features of a Good Heuristic). Provide pros and cons.
		\part 
		(5 Pts) What are the initial steps to code your heuristic, specifically: 
		In a computer code how would you encode the queen locations and track growth in different queens attack paths?
	\end{parts}
	
	\begin{solution}
		
	\end{solution}
%%%%%%%%%%%%%%%%%%%%%%%%%%%%%%%%%%%%%%%%%%%%%%%%%%%%%%%%%%%%%
	
	
	\question % 3
	\textbf{\large Local Search Heuristics (30 Pts)}
	Consider the classic children’s shuffle puzzle, where you are given a picture and 
	need to shuffle tiles to make the picture look correct. An example is shown below. 
	(Note that by physical construction the only pieces that can move are those adjacent to the “missing” tile.)
	\includegraphics[width=0.27\linewidth]{"Screenshot 2023-11-09 at 7.55.50 PM"}
	\begin{parts}
		\part
		(20 Pts) Propose a local search heuristic to solve the puzzle. 
		(Recall local search heuristics do not have memory, and cannot use global information). 
		Ensure you address the following in your description.
		\begin{subparts}
			\subpart What is your “objective function”?
			\subpart What is your definition of a move?
			\subpart How do you select a neighbor?
		\end{subparts}
		\begin{solution}
			
		\end{solution}
		
		\part
		(10 Pts) Provide a concrete 3 x 3 example to illustrate your heuristic. 
		Show all necessary iterations to achieve local optima. 
		Note: Your 3 x 3 example can just use the numbers 1-8 for ease of reference.
		\begin{solution}
			
		\end{solution}
		
	\end{parts}
%%%%%%%%%%%%%%%%%%%%%%%%%%%%%%%%%%%%%%%%%%%%%%%%%%%%%%%%%%%%%
	
	
	\question % 4
	\textbf{\large Coding (10 Pts)}
	\begin{parts}
		\part
		Using the software of your choice1 write an implementation of the GRASP 
		Heuristic for the Traveling Salesman Problem (TSP).
		\begin{subparts}
			\subpart
			Ensure that your code runs the lab data example, and is robust enough that if the TSP instance within
			that file were replaced by a larger/different TSP instance your code still runs correctly.
			\subpart
			Ensure you comment your code \underline{thoroughly}.
			\subpart
			You are free to recycle sections of the Nearest Neighbor and/or 2OPT code provided in labs as desired.
		\end{subparts}
		\begin{solution}
			
		\end{solution}
	\end{parts}
%%%%%%%%%%%%%%%%%%%%%%%%%%%%%%%%%%%%%%%%%%%%%%%%%%%%%%%%%%%%%
	
	
	\question % 5
	\textbf{\large Extra Credit (5 Pts)}
	\begin{parts}
		\part
		\begin{subparts}
			\subpart
			Create a transformation, associated explanation, and proof of validity, 
			that enables the transformation of any instance of Sudoku to an 
			instance of Graph Coloring (Aka Chromatic Number).
		\end{subparts}
	\end{parts}
	\begin{solution}
		
	\end{solution}
	%%%%%%%%%%%%%%%%%%%%%%%%%%%%%%%%%%%%%%%%%%%%%%%%%%%%%%%%%%%%%
	
\end{questions}
\end{document}