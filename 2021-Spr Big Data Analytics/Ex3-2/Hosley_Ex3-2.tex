\documentclass[]{article}
\usepackage{graphicx}
\usepackage{caption}
\graphicspath{ {./images/} }

% Minted
\usepackage[cache=false]{minted}
	\usemintedstyle{vs}
	\usepackage{xcolor}
		\definecolor{light-gray}{gray}{0.97}

% Subsubsection header run along with text.	
\usepackage{titlesec}
\titleformat{\subsubsection}[runin]% runin puts it in the same paragraph
	{\normalfont\bfseries}% formatting commands to apply to the whole heading
	{\thesubsection}% the label and number
	{0.5em}% space between label/number and subsection title
	{}% formatting commands applied just to subsection title
	[]% punctuation or other commands following subsection title

\usepackage{enumitem}
\usepackage{hyperref}

\usepackage[normalem]{ulem}

\title{Big Data Analytics: Exercise 3-2}
\author{Brandon Hosley}
\date{\today}

\begin{document}
\maketitle

\section*{Assignment 1}
\emph{We exercised GROUPING SETS. There are some additional ways to perform these aggregations, using CUBE or ROLLUP. These are almost like shortcuts. While CUBE returns all possible aggregation combinations, ROLLUP does it in a more hierarchical fashion. You will replace GROUPING SETS with ROLLUP, and then similarly replace GROUPING SETS with CUBE.}

\subsection*{Q1.1: Rollup} 
\emph{Modify/rewrite the grouping set query in the Section 2.2 using ROLLUP. Run and show the results. Explain the presentation of the result data by comparing the grouping set query’s presentation
(First 10 lines of query results only)} \\
\noindent
\includegraphics[width=\linewidth]{image1.1.1} %\vspace{-1.5em}
\begin{minted}[breaklines,bgcolor=light-gray]{shell-session}
SELECT driverId, eventType, count(*) 
	AS occurance FROM csc534.bhosl2_truck_event 
	GROUP BY driverId, eventType 
	WITH ROLLUP
	LIMIT 10;
\end{minted}
\includegraphics[width=\linewidth]{image1.1.2} %\vspace{-1.5em}

The hierarchical grouping of the Rollup method produces results that are almost identical to the group-by presented in the exercise, the lone exception being the data-set total result from the all-NULL group.

\subsection*{Q1.2: Cube} 
\emph{Modify/rewrite the grouping set query in the Section 2.2 using CUBE. Run and show the results. Explain the presentation of the result data by comparing both the grouping set query and and ROLLUP query’s presentation
(First 10 lines of query results only)} \\
\noindent
\includegraphics[width=\linewidth]{image1.2.1} %\vspace{-1.5em}
\begin{minted}[breaklines,bgcolor=light-gray]{shell-session}
SELECT driverId, eventType, count(*) 
	AS occurance FROM csc534.bhosl2_truck_event 
	GROUP BY driverId, eventType 
	WITH CUBE
	LIMIT 10;
\end{minted}
\includegraphics[width=\linewidth]{image1.2.2} %\vspace{-1.5em}

The Cube results differ significantly more from the group-by because it includes datasets that aggregate based on the second variable irrespective of the first; in this case: (NULL, $<$eventType$>$)

\section*{Assignment 2}

\subsection*{Step 1: Load data into user space in HDFS} 

\begin{minted}[breaklines,bgcolor=light-gray]{shell-session}
hadoop fs -mkdir Titanic
hadoop fs - ls

hadoop fs -put /home/data/CSC534BDA/datasets/Titanic/titanic.csv Titanic
\end{minted}
\includegraphics[width=\linewidth]{image2.1.1} %\vspace{-1.5em}

\subsection*{Step 2: Create a Hive table}

\begin{minted}[breaklines,bgcolor=light-gray]{shell-session}
CREATE TABLE csc534.bhosl2_titanic (
	PassengerId INT,
	Survived INT,
	Pclass INT,
	Name STRING,
	Sex STRING,
	Age INT,
	SibSp INT,
	Parch INT,
	Ticket STRING,
	Fare DOUBLE,
	Cabin STRING,
	Embarked STRING
)
COMMENT 'Titanic Passenger Data'
ROW FORMAT SERDE 'org.apache.hadoop.hive.serde2.OpenCSVSerde'
WITH SERDEPROPERTIES (
	"separatorChar" = ",",
	"quoteChar"     = "\""
)
STORED AS TEXTFILE
TBLPROPERTIES('skip.header.line.count'='1');
\end{minted}
\includegraphics[width=\linewidth]{image2.2.1} %\vspace{-1.5em}

\subsection*{Step 3: Load data into Hive}
\emph{verify that the loading was successful}

\begin{minted}[breaklines,bgcolor=light-gray]{shell-session}
LOAD DATA INPATH 'Titanic/titanic.csv' INTO TABLE csc534.bhosl2_titanic;
\end{minted}
\includegraphics[width=\linewidth]{image2.3.1} %\vspace{-1.5em}

\begin{minted}[breaklines,bgcolor=light-gray]{shell-session}
select count(*) from csc534.bhosl2_titanic;
exit;
\end{minted}
\includegraphics[width=\linewidth]{image2.3.2} %\vspace{-1.5em}

\begin{minted}[breaklines,bgcolor=light-gray]{shell-session}
cat /home/data/CSC534BDA/datasets/Titanic/titanic.csv  | wc -l
\end{minted}
\includegraphics[width=\linewidth]{image2.3.3} %\vspace{-1.5em}

891 entries + 1 row of column headers.

\subsection*{Step 4: Calculate survivor counts and survival rates}

\begin{minted}[breaklines,bgcolor=light-gray]{shell-session}
set hive.cli.print.header=true;

SELECT survived AS survived,
	count(*) AS total,
	count(*) / 8.91 AS survival_rate
FROM csc534.bhosl2_titanic
GROUP BY survived;
\end{minted}
\includegraphics[width=\linewidth]{image2.4.1} %\vspace{-1.5em}
\includegraphics[width=\linewidth]{image2.4.2} %\vspace{-1.5em}

\subsection*{Step 5: Calculate survivor counts and survival rates by sex}

\begin{minted}[breaklines,bgcolor=light-gray]{shell-session}
SELECT survived AS survived,
	sex AS sex,
	count(*) AS total,
	count(*) / 8.91 AS survival_rate
FROM csc534.bhosl2_titanic
GROUP BY survived, sex;
\end{minted}
\includegraphics[width=\linewidth]{image2.5.1} %\vspace{-1.5em}
\includegraphics[width=\linewidth]{image2.5.2} %\vspace{-1.5em}

\subsection*{Step 6: Write a query to provide the results of steps 4 and 5}

\begin{minted}[breaklines,bgcolor=light-gray]{shell-session}
SELECT survived AS survived,
	sex AS sex,
	count(*) AS total,
	count(*) / 8.91 AS survival_rate
FROM csc534.bhosl2_titanic
GROUP BY survived, sex
WITH CUBE;
\end{minted}
\includegraphics[width=\linewidth]{image2.6.1} %\vspace{-1.5em}
\includegraphics[width=\linewidth]{image2.6.2} %\vspace{-1.5em}

\end{document}

\includegraphics[width=\linewidth]{image1} %\vspace{-1.5em}
\begin{minted}[breaklines,bgcolor=light-gray,fontsize=\scriptsize]{shell-session}
\end{minted}
\mintinline[bgcolor=light-gray]{bash}{} \begin{enumerate}[before=\itshape,font=\normalfont,label=\alph*.]
\end{enumerate}