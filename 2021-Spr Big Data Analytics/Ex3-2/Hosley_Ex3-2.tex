\documentclass[]{article}
\usepackage{graphicx}
\usepackage{caption}
\graphicspath{ {./images/} }

% Minted
\usepackage[cache=false]{minted}
	\usemintedstyle{vs}
	\usepackage{xcolor}
		\definecolor{light-gray}{gray}{0.97}

% Subsubsection header run along with text.	
\usepackage{titlesec}
\titleformat{\subsubsection}[runin]% runin puts it in the same paragraph
	{\normalfont\bfseries}% formatting commands to apply to the whole heading
	{\thesubsection}% the label and number
	{0.5em}% space between label/number and subsection title
	{}% formatting commands applied just to subsection title
	[]% punctuation or other commands following subsection title

\usepackage{enumitem}
\usepackage{hyperref}

\usepackage[normalem]{ulem}
\usepackage[T1]{fontenc}

\title{Big Data Analytics: Exercise 3-2}
\author{Brandon Hosley}
\date{\today}

\begin{document}
\maketitle

\section*{Assignment 1}
\emph{We exercised GROUPING SETS. There are some additional ways to perform these aggregations, using CUBE or ROLLUP. These are almost like shortcuts. While CUBE returns all possible aggregation combinations, ROLLUP does it in a more hierarchical fashion. You will replace GROUPING SETS with ROLLUP, and then similarly replace GROUPING SETS with CUBE.}

\subsection*{Q1.1: Rollup} 
\emph{Modify/rewrite the grouping set query in the Section 2.2 using ROLLUP. Run and show the results. Explain the presentation of the result data by comparing the grouping set query’s presentation
(First 10 lines of query results only)} \\
\noindent
\includegraphics[width=\linewidth]{image1.1.1} %\vspace{-1.5em}
\begin{minted}[breaklines,bgcolor=light-gray]{shell-session}
SELECT driverId, eventType, count(*) 
	AS occurance FROM csc534.bhosl2_truck_event 
	GROUP BY driverId, eventType 
	WITH ROLLUP
	LIMIT 10;
\end{minted}
\includegraphics[width=\linewidth]{image1.1.2} %\vspace{-1.5em}

The hierarchical grouping of the Rollup method produces results that are almost identical to the group-by presented in the exercise, the lone exception being the data-set total result from the all-NULL group.

\subsection*{Q1.2: Cube} 
\emph{Modify/rewrite the grouping set query in the Section 2.2 using CUBE. Run and show the results. Explain the presentation of the result data by comparing both the grouping set query and and ROLLUP query’s presentation
(First 10 lines of query results only)} \\
\noindent
\includegraphics[width=\linewidth]{image1.2.1} %\vspace{-1.5em}
\begin{minted}[breaklines,bgcolor=light-gray]{shell-session}
SELECT driverId, eventType, count(*) 
	AS occurance FROM csc534.bhosl2_truck_event 
	GROUP BY driverId, eventType 
	WITH CUBE
	LIMIT 10;
\end{minted}
\includegraphics[width=\linewidth]{image1.2.2} %\vspace{-1.5em}

The Cube results differ significantly more from the group-by because it includes datasets that aggregate based on the second variable irrespective of the first; in this case: (NULL, $<$eventType$>$)

\section*{Assignment 2}

\subsection*{Step 1: Load data into user space in HDFS} 
\subsection*{Step 2: Create a Hive table}
\subsection*{Step 3: Load data into Hive}
\emph{verify that the loading was successful}
\subsection*{Step 4: Calculate survivor counts and survival rates}
\subsection*{Step 5: Calculate survivor counts and survival rates by sex}
\subsection*{Step 6: Write a query to provide the results of steps 4 and 5}


\end{document}

\includegraphics[width=\linewidth]{image1} %\vspace{-1.5em}
\begin{minted}[breaklines,bgcolor=light-gray,fontsize=\scriptsize]{shell-session}
\end{minted}
\mintinline[bgcolor=light-gray]{bash}{} \begin{enumerate}[before=\itshape,font=\normalfont,label=\alph*.]
\end{enumerate}