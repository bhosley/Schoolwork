\documentclass[]{article}
\usepackage{graphicx}
\usepackage{caption}
\graphicspath{ {./images/} }

% Minted
\usepackage[cache=false]{minted}
	\usemintedstyle{vs}
	\usepackage{xcolor}
		\definecolor{light-gray}{gray}{0.97}

% Subsubsection header run along with text.	
\usepackage{titlesec}
\titleformat{\subsubsection}[runin]% runin puts it in the same paragraph
	{\normalfont\bfseries}% formatting commands to apply to the whole heading
	{\thesubsection}% the label and number
	{0.5em}% space between label/number and subsection title
	{}% formatting commands applied just to subsection title
	[]% punctuation or other commands following subsection title

\usepackage{enumitem}
\usepackage{hyperref}

\usepackage[normalem]{ulem}
\usepackage[T1]{fontenc}

\title{Big Data Analytics: Exercise 2-2}
\author{Brandon Hosley}
\date{\today}

\begin{document}
\maketitle

\subsection*{Q1: Write and run 11 HBase commands to insert a new row into the table.} 
\begin{enumerate}[before=\itshape,font=\normalfont,label=\alph*.]
	\item Table name: $<$your-namespace$>$:truck\_event
	\item Rowkey: 20000
	\item Column family name: events
	\item Columns: values
	\begin{enumerate}[label=\roman*.]
		\item driverId: $<$your-login or UIS NetID$>$
		\item truckId: 999
		\item eventTime
		\item eventType: $<$Pick one from Normal, Overspeed, and Lane Departure$>$
		\item longitude: -94.58
		\item latitude: 37.03
		\item \sout{ eventKey (This is a RowKey) }
		\item CorrelationId: 1000
		\item driverName: $<$Your name$>$
		\item routeId: 888
		\item routeName: UIS to Chicago
		\item eventDate: $<$current date$>$ e.g. 2030-01-01-01, year-month-day-hour (24)
	\end{enumerate}
\end{enumerate}

\begin{minted}[breaklines,bgcolor=light-gray,fontsize=\footnotesize]{shell-session}
put 'bhosl2:truck_event', '20000', 'events:driverId', 'bhosl2'
put 'bhosl2:truck_event', '20000', 'events:truckId', '999'
put 'bhosl2:truck_event', '20000', 'events:eventTime', '1200'
put 'bhosl2:truck_event', '20000', 'events:eventType', 'Overspeed'
put 'bhosl2:truck_event', '20000', 'events:longitude', '-94.58'
put 'bhosl2:truck_event', '20000', 'events:latitude', '37.03'
put 'bhosl2:truck_event', '20000', 'events:CorrelationId', '1000'
put 'bhosl2:truck_event', '20000', 'events:driverName', 'Brandon Hosley'
put 'bhosl2:truck_event', '20000', 'events:routeId', '888'
put 'bhosl2:truck_event', '20000', 'events:routeName', 'UIS to Chicago'
put 'bhosl2:truck_event', '20000', 'events:eventDate', '2021-02-19-12'
\end{minted}

\subsection*{Q2: Write and run a HBase command to retrieve the row only you just inserted.} 
\begin{enumerate}[before=\itshape,font=\normalfont,label=\alph*.]
	\item Table name: $<$your-namespace$>$:truck\_event
	\item Rowkey: 20000
	\item Column family name: events
	\item Note: the row only. No other rows.
	\item You may want to check the help page ‘help $<$command$>$’
\end{enumerate}

\begin{minted}[breaklines,bgcolor=light-gray]{shell-session}
get 'bhosl2:truck_event', '20000'
\end{minted}

\subsection*{Q3: Write and run two HBase commands to update the value of the row you inserted/retrieved and show the changes.} 
\begin{enumerate}[before=\itshape,font=\normalfont,label=\alph*.]
	\item Table name: $<$your-namespace$>$:truck\_event
	\item Rowkey: 20000
	\item Column family name: events
	\item Column name(qualifier): routeName
	\item NEW value: Chicago to UIS
	\item Write and run your second command to show the changes
	\begin{enumerate}[label=\roman*.]
		\item Expected changes: ‘UIS to Chicago’ $\rightarrow$ ‘Chicago to UIS’
	\end{enumerate}
\end{enumerate}

\begin{minted}[breaklines,bgcolor=light-gray,fontsize=\footnotesize]{shell-session}
put 'bhosl2:truck_event', '20000', 'events:routeName', 'Chicago to UIS'
get 'bhosl2:truck_event', '20000', 'events:routeName'
\end{minted}

\subsection*{Q4: Write and run a HBase command to retrieve two columns of the row you inserted/retrieved.} 
\begin{enumerate}[before=\itshape,font=\normalfont,label=\alph*.]
	\item Table name: $<$your-namespace$>$:truck\_event
	\item Rowkey: 20000
	\item Column family name: events
	\item Two columns we want to see: driverName and routeName ONLY
	\item Note: Two columns (qualifiers) only. No other columns.
	\item You may want to check the help page ‘help $<$command$>$’
\end{enumerate}

\begin{minted}[breaklines,bgcolor=light-gray]{shell-session}
get 'bhosl2:truck_event', '20000', ['driverName','routeName']
\end{minted}

\subsection*{Q5: Write and run a HBase command in non-interactive mode.} 
\begin{enumerate}[before=\itshape,font=\normalfont,label=\alph*.]
	\item Retrieve meta-data of your table in Linux shell.
	\item Table name: $<$your-namespace$>$:truck\_event
	\item Use Both Linux commands and HBase shell commands
	\begin{enumerate}[label=\roman*.]
		\item Linux command: echo
		\item HBase shell command: describe (retrieve meta-data of a table)
	\end{enumerate}
	\item This code should run outside of the HBase shell.
	\begin{enumerate}[label=\roman*.]
		\item Don’t login to the HBase shell.
		\item Should be run in Linux shell
		\item e.g.%
		{\normalfont 
			\begin{minted}[breaklines,bgcolor=light-gray]{shell-session}
[sslee777@node00 ~]$ <your command here....>
			\end{minted} 
		}
	\end{enumerate}
	\item This scripting is quite useful in some cases, e.g. You want to run HBase commands outside HBase shell. 
	\begin{enumerate}[label=\roman*.]
		\item Batch processing
		\item Administration
		\item Automation – Cron jobs
		\item Etc.
	\end{enumerate}
	\item See \href{https://hbase.apache.org/book.html#_running_the_shell_in_non_interactive_mode}{link} and 
	\href{https://hbase.apache.org/book.html#hbase.shell.noninteractive}{link.}
\end{enumerate}

\begin{minted}[breaklines,bgcolor=light-gray]{shell-session}
$ echo "describe 'bhosl2:truck_event'" | ./hbase shell -n
\end{minted}

\end{document}

\includegraphics[width=\linewidth]{image1} %\vspace{-1.5em}
\begin{minted}[breaklines,bgcolor=light-gray]{shell-session}
\end{minted}
\mintinline[bgcolor=light-gray]{bash}{} 