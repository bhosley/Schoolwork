\documentclass[]{article}
\usepackage{graphicx}
\usepackage{caption}
\graphicspath{ {./images/} }

% Minted
\usepackage[cache=false]{minted}
	\usemintedstyle{vs}
	\usepackage{xcolor}
		\definecolor{light-gray}{gray}{0.97}

% Subsubsection header run along with text.	
\usepackage{titlesec}
\titleformat{\subsubsection}[runin]% runin puts it in the same paragraph
	{\normalfont\bfseries}% formatting commands to apply to the whole heading
	{\thesubsection}% the label and number
	{0.5em}% space between label/number and subsection title
	{}% formatting commands applied just to subsection title
	[]% punctuation or other commands following subsection title

\usepackage{enumitem}
\usepackage{hyperref}

\usepackage[normalem]{ulem}

\title{Big Data Analytics: Exercise 4-2}
\author{Brandon Hosley}
\date{\today}

\begin{document}
\maketitle

\section*{Pre-Assignment}
\begin{minted}[breaklines,bgcolor=light-gray]{shell-session}
from pyspark.ml.linalg import Vectors
from pyspark.ml.classification import LogisticRegression
training = spark.createDataFrame([
	(1.0, Vectors.dense([0.0, 1.1, 0.1])),
	(0.0, Vectors.dense([2.0, 1.0, -1.0])),
	(0.0, Vectors.dense([2.0, 1.3, 1.0])),
	(1.0, Vectors.dense([0.0, 1.2, -0.5]))], ["label", "features"])
\end{minted}

\section*{Assignment 1}
\emph{ Do the exercise in section 2.2.1.2 and 2.2.1.3 }

\begin{minted}[breaklines,bgcolor=light-gray]{shell-session}
lr = LogisticRegression(maxIter=10, regParam=0.01)
print("LogisticRegression parameters:\n" + lr.explainParams() + "\n")
model1 = lr.fit(training)

print("Model 1 was fit using parameters: ")
print(model1.extractParamMap())
\end{minted}
\includegraphics[width=\linewidth]{image1} %\vspace{-1.5em}


\section*{Assignment 2}
\emph{ Do the exercise in section 2.2.1.4 and 2.2.1.5 }

\begin{minted}[breaklines,bgcolor=light-gray]{shell-session}
paramMap = {lr.maxIter: 20}
paramMap[lr.maxIter] = 30 # Specify 1 Param, overwriting the original maxIter.
paramMap.update({lr.regParam: 0.1, lr.threshold: 0.55}) # Specify multiple Params.

paramMap2 = {lr.probabilityCol: "myProbability"} # Change output column name
paramMapCombined = paramMap.copy()
paramMapCombined.update(paramMap2)

model2 = lr.fit(training, paramMapCombined)
print("Model 2 was fit using parameters: ")
print(model2.extractParamMap())
\end{minted}
\includegraphics[width=\linewidth]{image2} %\vspace{-1.5em}


\section*{Assignment 3}
\emph{ Do the exercise in section 2.2.1.7 and 2.2.1.8 }

\begin{minted}[breaklines,bgcolor=light-gray]{shell-session}
test = spark.createDataFrame([
	(1.0, Vectors.dense([-1.0, 1.5, 1.3])),
	(0.0, Vectors.dense([3.0, 2.0, -0.1])),
	(1.0, Vectors.dense([0.0, 2.2, -1.5]))], ["label", "features"])

prediction = model2.transform(test)
result = prediction.select("features", "label", "myProbability", "prediction").collect()

for row in result:
	print("features=%s, label=%s -> prob=%s, prediction=%s"
		% (row.features, row.label, row.myProbability, row.prediction))
\end{minted}
\includegraphics[width=\linewidth]{image3} %\vspace{-1.5em}


\section*{Assignment 4}
\emph{ Do the exercise to learn ML pipeline in
	\hyperref{https://spark.apache.org/docs/2.4.0/ml-pipeline.html#example-pipeline}{
		Appache Docs}.
 }

\begin{minted}[breaklines,bgcolor=light-gray]{shell-session}
\end{minted}
\includegraphics[width=\linewidth]{image1} %\vspace{-1.5em}


\section*{Assignment 5}
\emph{  }

\begin{minted}[breaklines,bgcolor=light-gray]{shell-session}
\end{minted}
\includegraphics[width=\linewidth]{image1} %\vspace{-1.5em}


\end{document}

\includegraphics[width=\linewidth]{image1} %\vspace{-1.5em}
\begin{minted}[breaklines,bgcolor=light-gray,fontsize=\scriptsize]{shell-session}
\end{minted}
\mintinline[bgcolor=light-gray]{bash}{} \begin{enumerate}[before=\itshape,font=\normalfont,label=\alph*.]
\end{enumerate}