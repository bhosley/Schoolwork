\documentclass[]{article}
\usepackage{graphicx}
\usepackage{caption}
\graphicspath{ {./images/} }

% Minted
\usepackage[cache=false]{minted}
	\usemintedstyle{vs}
	\usepackage{xcolor}
		\definecolor{light-gray}{gray}{0.97}

% Subsubsection header run along with text.	
\usepackage{titlesec}
\titleformat{\subsubsection}[runin]% runin puts it in the same paragraph
	{\normalfont\bfseries}% formatting commands to apply to the whole heading
	{\thesubsection}% the label and number
	{0.5em}% space between label/number and subsection title
	{}% formatting commands applied just to subsection title
	[]% punctuation or other commands following subsection title

\usepackage{enumitem}
\usepackage{hyperref}

\usepackage[normalem]{ulem}

\title{Big Data Analytics: Exercise 5-2}
\author{Brandon Hosley}
\date{\today}

\begin{document}
\maketitle

\vspace{-1.5em}

Colab Notebook located
\href{https://colab.research.google.com/drive/1rxCV1SSGTpGZSGphk5AH8Iuuz3TFCNxB?usp=sharing}{here.}

\vspace{-1.5em}

\section*{Assignment 1}
\vspace{-1em}
\emph{ Do the exercise in section 3.6 }
\vspace{-0.5em}

\begin{minted}[breaklines,bgcolor=light-gray,fontsize=\footnotesize]{shell-session}
document_assembler = DocumentAssembler() \
	.setInputCol("headline_text") \
	.setOutputCol("document") \
	.setCleanupMode("shrink")

tokenizer = Tokenizer() \
	.setInputCols(["document"]) \
	.setOutputCol("token")

normalizer = Normalizer() \
	.setInputCols(["token"]) \
	.setOutputCol("normalized")

stopwords_cleaner = StopWordsCleaner()\
	.setInputCols("normalized")\
	.setOutputCol("cleanTokens")\
	.setCaseSensitive(False)

stemmer = Stemmer() \
	.setInputCols(["cleanTokens"]) \
	.setOutputCol("stem")

finisher = Finisher() \
	.setInputCols(["stem"]) \
	.setOutputCols(["tokens"]) \
	.setOutputAsArray(True) \
	.setCleanAnnotations(False)

nlp_pipeline = Pipeline(
	stages=[document_assembler, 
		tokenizer,
		normalizer,
		stopwords_cleaner, 
		stemmer, 
		finisher])

nlp_model = nlp_pipeline.fit(df)
processed_df = nlp_model.transform(df)
tokens_df = processed_df.select('publish_date','tokens').limit(10000)
tokens_df.show()
\end{minted}
\includegraphics[width=0.8\linewidth]{image1.1} \\
\includegraphics[width=0.8\linewidth]{image1.2}

\section*{Assignment 2}
\emph{ Do the exercise in section 3.7 }

\begin{minted}[breaklines,bgcolor=light-gray]{shell-session}
from pyspark.ml.feature import CountVectorizer
cv = CountVectorizer(inputCol="tokens", outputCol="features", 
vocabSize=500, minDF=3.0)
# train the model
cv_model = cv.fit(tokens_df)
# transform the data. Output column name will be features.
vectorized_tokens = cv_model.transform(tokens_df)
\end{minted}
\includegraphics[width=\linewidth]{image2} 


\section*{Assignment 3}
\emph{ Do the exercise in section 3.8 }

\begin{minted}[breaklines,bgcolor=light-gray]{shell-session}
from pyspark.ml.clustering import LDA
num_topics = 3
lda = LDA(k=num_topics, maxIter=10)
model = lda.fit(vectorized_tokens)
ll = model.logLikelihood(vectorized_tokens)
lp = model.logPerplexity(vectorized_tokens)
print("The lower bound on the log likelihood of the entire corpus: " +  str(ll))
print("The upper bound on perplexity: " + str(lp))
\end{minted}
\includegraphics[width=\linewidth]{image3} 


\section*{Assignment 4}
\emph{ Do the exercise in section 3.9 }

\begin{minted}[breaklines,bgcolor=light-gray]{shell-session}
# extract vocabulary from CountVectorizer
vocab = cv_model.vocabulary
topics = model.describeTopics() 
topics_rdd = topics.rdd
topics_words = topics_rdd\
		.map(lambda row: row['termIndices'])\
		.map(lambda idx_list: [vocab[idx] for idx in idx_list])\
		.collect()
for idx, topic in enumerate(topics_words):
	print("topic: {}".format(idx))
	print("*"*25)
	for word in topic:
		print(word)
	print("*"*25)
\end{minted}
\includegraphics[width=\linewidth]{image4.1} %\vspace{-1.5em}
\includegraphics[width=\linewidth]{image4.2} %\vspace{-1.5em}


\section*{Assignment 5}
\emph{ Try different values of k and maxIter to see which combination best suits your data in Section 3.8. Show at least five different combinations, show their results, and explain why it’s best.
}

\begin{minted}[breaklines,bgcolor=light-gray]{shell-session}
testVals = [[3,20],[5,10],[5,20],[10,10],[10,20],[15,10]]
for i, iters in testVals:
	from pyspark.ml.clustering import LDA
	num_topics = i
	lda = LDA(k=num_topics, maxIter=iters)
	model = lda.fit(vectorized_tokens)
	ll = model.logLikelihood(vectorized_tokens)
	lp = model.logPerplexity(vectorized_tokens)
	print("Results for {0} topics with {1} iterations:".format(num_topics,iters))
	print("The lower bound on the log likelihood of the entire corpus: " +  str(ll))
	print("The upper bound on perplexity: " + str(lp))
\end{minted}
\includegraphics[width=\linewidth]{image5} %\vspace{-1.5em}

3 topics with 10 iterations performed the best as they had the highest log-likelihood and the lowest perplexity.

\clearpage

\section*{Assignment 6}
\emph{ Rewrite the codes for finding topics in tweets coronavirus dataset.  Also try different values of k and maxIter to see which combination best suits the data. Show at least five different combinations, show their results, and explain why it’s best.
 }

\begin{minted}[breaklines,bgcolor=light-gray]{shell-session}
file_location = r'./coronavirus-text-only-1000.txt'
file_type = "csv"
infer_schema = "true"
first_row_is_header = "true"
df = spark.read.format(file_type) \
	.option("inferSchema", infer_schema) \
	.option("header", first_row_is_header) \
	.load(file_location)
df.count()
df.show(4)


document_assembler = DocumentAssembler() \
	.setInputCol("text") \
	.setOutputCol("document") \
	.setCleanupMode("shrink")

tokenizer = Tokenizer() \
	.setInputCols(["document"]) \
	.setOutputCol("token")

normalizer = Normalizer() \
	.setInputCols(["token"]) \
	.setOutputCol("normalized")

wordlist = StopWordsCleaner().getStopWords()
wordlist.append("coronavirus")

stopwords_cleaner = StopWordsCleaner() \
	.setInputCols("normalized") \
	.setOutputCol("cleanTokens") \
	.setStopWords(wordlist) \
	.setCaseSensitive(False)

stemmer = Stemmer() \
	.setInputCols(["cleanTokens"]) \
	.setOutputCol("stem")

finisher = Finisher() \
	.setInputCols(["stem"]) \
	.setOutputCols(["tokens"]) \
	.setOutputAsArray(True) \
	.setCleanAnnotations(False)

nlp_pipeline = Pipeline(
	stages=[document_assembler, 
		tokenizer,
		normalizer,
		stopwords_cleaner,
		stemmer, 
		finisher])

nlp_model = nlp_pipeline.fit(df)
processed_df = nlp_model.transform(df)
tokens_df = processed_df.select('tokens').limit(1000)
tokens_df.show()


from pyspark.ml.feature import CountVectorizer
cv = CountVectorizer(inputCol="tokens", outputCol="features", 
vocabSize=500, minDF=3.0)
cv_model = cv.fit(tokens_df)
vectorized_tokens = cv_model.transform(tokens_df)

from pyspark.ml.clustering import LDA
testVals = [[3,10],[3,20],[5,10],[5,20],[10,10],[10,20],[20,10]]
for i, iters in testVals:
	num_topics = i
	lda = LDA(k=num_topics, maxIter=iters)
	model = lda.fit(vectorized_tokens)
	ll = model.logLikelihood(vectorized_tokens)
	lp = model.logPerplexity(vectorized_tokens)
	print("Results for {0} topics with {1} iterations:".format(num_topics,iters))
	print("The lower bound on the log likelihood of the entire corpus: " +  str(ll))
	print("The upper bound on perplexity: " + str(lp))
	print()
\end{minted}
\includegraphics[width=0.8\linewidth]{image6.1} \\
\includegraphics[width=0.8\linewidth]{image6.2} \\
\includegraphics[width=\linewidth]{image6.3} %\vspace{-1.5em}
\includegraphics[width=\linewidth]{image6.4} %\vspace{-1.5em}
\includegraphics[width=\linewidth]{image6.5} %\vspace{-1.5em}

\end{document}