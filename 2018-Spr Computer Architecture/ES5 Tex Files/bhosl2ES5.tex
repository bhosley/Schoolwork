\documentclass[a4paper,man,natbib]{apa6}

\usepackage[english]{babel}
\usepackage[utf8x]{inputenc}

% Common Packages - Delete Unused Ones %
\usepackage{setspace}
\usepackage{amsmath}


\usepackage[cache=false]{minted}
\usepackage{graphicx}
\usepackage{caption}
\graphicspath{ {./images/} }
% End Packages %

\title{Exercise Assignment 5}
\shorttitle{ES5}
\author{Brandon Hosley}
\date{2018 09 25}
\affiliation{Mike Davis}

%\abstract{}

\begin{document}
\maketitle
\singlespacing
\subsection{25a}
\emph{Assuming nine-bit two's complement representation, convert each of the following decimal numbers to binary, show the effect of the ASL operation on it, and
	then convert the result back to decimal. Repeat with the ASR operation:} \\
94 \\
BIN $\rightarrow$ 0 0101 1110 \\
ASL $\rightarrow$ 0 1011 1100 \\
DEC $\rightarrow$ 188 \\
ASR $\rightarrow$ 0 0101 1110 $\rightarrow$ 94 \\
\subsection{25c}
\emph{Assuming nine-bit two's complement representation, convert each of the following decimal numbers to binary, show the effect of the ASL operation on it, and
	then convert the result back to decimal. Repeat with the ASR operation:} \\
-62 \\
BIN $\rightarrow$ 1 1100 0010 \\
ASL $\rightarrow$ 1 1000 0100 \\
DEC $\rightarrow$ -124
ASR $\rightarrow$ 0 1100 0010$_{2}$ $\rightarrow$ 194$_{10}$
\subsection{26a}
\emph{Write the RTL specification for an arithmetic shift right on a six-bit cell.} \\
$c\rightarrow r\langle 0\rangle ,r \langle 0...5\rangle\rightarrow r\langle 1...6\rangle ,r\langle 6\rangle\rightarrow C$ \\
\subsection{26b}
\emph{Write the RTL specification for an arithmetic shift left on a 16-bit cell.} \\
$C\leftarrow r\langle 0\rangle ,r \langle 0...15\rangle\leftarrow rlangle 1...16\rangle ,r\langle 6\rangle\leftarrow c$
\subsection{28a}
\emph{C = 1, ROL 0 0110 1101} \\
C=0, 0 1101 1011 \\
\subsection{28b}
\emph{C = 0, ROL 0 0110 1101} \\
C=0, 0 1101 1010 \\
\subsection{35b}
\emph{Assuming nine-bit two's complement binary representation, convert the following numbers from hexadecimal to decimal. Remember to check the sign bit: 0F5} \\
0F5$_{16}$ $\rightarrow$ 0 1111 0101$_{2}$ $\rightarrow$ 245$_{10}$
\subsection{35c}
\emph{Assuming nine-bit two's complement binary representation, convert the following numbers from hexadecimal to decimal. Remember to check the sign bit: 100} \\
100$_{16}$ $\rightarrow$ 1 0000 0000$_{2}$ $\rightarrow$ -255$_{10}$
\subsection{53b}
\emph{For IEEE 754 single precision floating point, write the hexadecimal representation for the following decimal values: -1.0} \\
-1.0$_{10}$ $\rightarrow$ 1 0111 1111 000 0000 0000 0000 0000 0000$_{2}$ $\rightarrow$ BF800000$_{16}$
\subsection{53c}
\emph{For IEEE 754 single precision floating point, write the hexadecimal representation for the following decimal values: -0.0} \\
-0.0$_{10}$ $\rightarrow$ 1 0000 0000 000 0000 0000 0000 0000 0000$_{2}$ $\rightarrow$ 80000000$_{16}$

\nocite{warford10}
\bibliographystyle{apacite}
\bibliography{CS} %link to relevant .bib file

\end{document}
