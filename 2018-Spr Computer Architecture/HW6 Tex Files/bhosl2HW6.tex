\documentclass[a4paper,man,natbib]{apa6}
\usepackage[english]{babel}
\usepackage[utf8x]{inputenc}

% Common Packages - Delete Unused Ones %
\usepackage{setspace}
\usepackage{amsmath}
\usepackage{graphicx}
\usepackage{caption}
\graphicspath{ {./images/} }
\input{kvmacros}
% End Packages %

\title{Homework 6}
\shorttitle{HW6}
\author{Brandon Hosley}
\date{2018 11 13}
\affiliation{Mike Davis}
%\abstract{}

\begin{document}
\maketitle
\raggedbottom
\singlespacing
\subsection{Problem 1}
\emph{Draw the Design Table, the K-maps and the circuit for:}
\subsubsection{a.}
\emph{Construct an SR flip-flop from a D flip-flop}
\includegraphics[width=0.7\linewidth]{HW6p1a.png}
\subsubsection{b.}
\emph{Construct a T flip-flop from a JK flip-flop}
\includegraphics[width=0.7\linewidth]{HW6p1b.png}
\subsection{Problem 2}
\includegraphics[width=0.7\linewidth]{HW6p2.png}
\begin{tabular}{ c c c c | c | c c c c | c c }
	A$(t)$ & B$(t)$ & X$_1 (t)$ & X$_2 (t)$ & Z$(t)$ & JA$(t)$ & KA$(t)$ & JB$(t)$ & KB$(t)$ & A$(t+1)$ & B$(t+1)$ \\
	\hline
	  0    &   0    &     0     &     0     &   0    &    0    &    0    &    0    &    1    &    0     &  0 \\
	  0    &   0    &     0     &     1     &   1    &    1    &    0    &    0    &    1    &    1     &  0 \\
	  0    &   0    &     1     &     0     &   1    &    0    &    1    &    0    &    1    &    0     &  0 \\
	  0    &   0    &     1     &     1     &   1    &    1    &    1    &    0    &    0    &    1     &  0 \\
	&&&&&&&&&&\\	      
	  0    &   1    &     0     &     0     &   0    &    0    &    0    &    0    &    1    &    0     &  0 \\
	  0    &   1    &     0     &     1     &   0    &    1    &    0    &    0    &    1    &    1     &  0 \\
	  0    &   1    &     1     &     0     &   1    &    0    &    0    &    0    &    1    &    0     &  0 \\
	  0    &   1    &     1     &     1     &   1    &    1    &    0    &    0    &    0    &    1     &  1 \\
	&&&&&&&&&&\\     
	  1    &   0    &     0     &     0     &   0    &    0    &    0    &    0    &    1    &    1     &  0 \\
	  1    &   0    &     0     &     1     &   1    &    1    &    0    &    0    &    1    &    1     &  0 \\
	  1    &   0    &     1     &     0     &   0    &    0    &    1    &    0    &    1    &    0     &  0 \\
	  1    &   0    &     1     &     1     &   1    &    1    &    1    &    1    &    0    &    0     &  1 \\
	&&&&&&&&&&\\
	  1    &   1    &     0     &     0     &   0    &    0    &    0    &    0    &    1    &    1     &  0 \\
	  1	   &   1    &     0     &     1     &   0    &    1    &    0    &    0    &    1    &    1     &  0 \\
	  1	   &   1    &     1     &     0     &   0    &    0    &    0    &    0    &    1    &    1     &  0 \\
	  1    &   1    &     1     &     1     &   0    &    1    &    0    &    1    &    0    &    1     &  1 \\
\end{tabular}
\clearpage
\subsection{Problem 3}
\emph{ Textbook page 593, \#16 (d) Show the Truth Table, the K-maps (There will be two – one for each Flip-Flop) and then finally draw the circuit.}
\singlespacing
\begin{center}
\begin{tabular}{ c c c c | c c }
	A$(t)$ & B$(t)$ & X$_1$ & X$_2$ & A$(t+1)$ & B$(t+1)$ \\
	\hline
       0   &    0   &   0   &   0   &     0    &  0 \\
       0   &    0   &   0   &   1   &     0    &  1 \\
       0   &    0   &   1   &   0   &     1    &  1 \\
       0   &    0   &   1   &   1   &     0    &  0 \\
&&&&&\\
       0   &    1   &   0   &   0   &     0    &  1 \\
       0   &    1   &   0   &   1   &     1    &  0 \\
       0   &    1   &   1   &   0   &     0    &  0 \\
       0   &    1   &   1   &   1   &     0    &  1 \\
&&&&&\\
       1   &    0   &   0   &   0   &     1    &  0 \\
       1   &    0   &   0   &   1   &     1    &  1 \\
       1   &    0   &   1   &   0   &     0    &  1 \\
       1   &    0   &   1   &   1   &     1    &  0 \\
&&&&&\\
       1   &    1   &   0   &   0   &     1    &  1 \\
       1   &    1   &   0   &   1   &     0    &  0 \\
       1   &    1   &   1   &   0   &     1    &  0 \\
       1   &    1   &   1   &   1   &     1    &  1 \\
\end{tabular}

\kvnoindex
\begin{minipage}{2.5in}
	\karnaughmap{4}{A$(t+1)$}{A{X$_1$}B{X$_2$}}{01011x1x10100x0x}{}
\end{minipage}
\begin{minipage}{2.5in}
	\karnaughmap{4}{B$(t+1)$}{A{X$_1$}B{X$_2$}}{01101x0x01101x0x}{}
\end{minipage}

\includegraphics[width=0.6\linewidth]{HW6p3.png}
\end{center}
\doublespacing
\clearpage
\subsection{Problem 4}
\emph{A sequential circuit has three state registers (or flip-flops) and 3 inputs}
\\
\subsubsection{(a) How many states does it have?}~\\
$2^3=8$ Possible Flip-Flop states;
\subsubsection{(b) How many transitions from each state does it have?}~\\
$2^3=8$ Possible switch states;
\subsubsection{(c) How many total transitions does it have?}~\\
$8\times8=64$ \\
Number of Switch States $\times$ Number of Flip-Flop States. \\

\clearpage

\subsection{Problem 5}
\emph{Textbook page 594, \#21(a)}~\\
\subsubsection{(a) Draw the Full Address decoding – Similar to Fig 11.48 }~\\
\includegraphics[width=0.8\linewidth]{HW6p5a.png}
\subsubsection{(b) Include a table similar to Fig 11.47 }~\\
~\\
\singlespacing
\begin{tabular}{l l l l l}
	\hline
	Device      &  64 x 8 PROM &  32 x 8 RAM  &  4-Port I/O  & 4-Port I/O   \\
Minimum Address & 00 0000 0000 & 00 0100 0000 & 00 0110 0000 & 00 0110 0100 \\
Maximum Address & 00 0011 1111 & 00 0101 1111 & 00 0110 0011 & 00 0110 0111 \\
General Address & 00 00xx xxxx & 00 010x xxxx & 00 0110 00xx & 00 0110 01xx \\
	\hline
\end{tabular}
~\\
\doublespacing
\subsubsection{(c) Draw a Memory Map like in Fig 11.50 and show which addresses each chip is enabled }~\\
\includegraphics[width=0.8\linewidth]{HW6p5d.png}
\subsubsection{(d) Can you add additional chips on this bus? Explain }~\\
Yes, this bus has plenty of seating left. This chipset arrangement would be able to fit into an 7-bit addressed bus, this 10-bit has 8-times the capacity currently used. 920 addresses remain available.

\nocite{warford10}
\bibliographystyle{apacite}
\bibliography{CS} %link to relevant .bib file
\end{document}