\documentclass[a4paper,man,natbib]{apa6}
\usepackage[english]{babel}
\usepackage[utf8x]{inputenc}

% Common Packages - Delete Unused Ones %
\usepackage{setspace}
\usepackage{amsmath}
\usepackage{graphicx}
\usepackage[linguistics]{forest}

\graphicspath{ {./images/} }

\usepackage[export]{adjustbox}

\usepackage[cache=false]{minted}

\usepackage{caption}

% End Packages %

\title{Exercise 8}
\shorttitle{ES8}
\author{Brandon Hosley}
\date{2018 10 24}
\affiliation{Mike Davis}
%\abstract{}
\begin{document}
\maketitle
\raggedbottom

\subsection{7.3}
\singlespacing
\emph{Derive the following strings with the grammar of Figure 7.1 and draw the corresponding syntax tree:}
\begin{center} \includegraphics[width=0.8\linewidth]{7_1grammar.png} \\ \end{center}
\doublespacing
\subsubsection{b.}
\textbf{a1b2c3} \\
\begin{forest}
	[<identifier> 
		[<identifier> 
			[<identifier> 
				[<identifier> 
					[<identifier> 
						[<identifier> 
							[<letter> [a ]]
						]
						[<digit> [1 ]]
					]
					[<letter> [b ]]
				]
				[<digit> [2 ]]
			]
			[<letter> [c ]]
		] 
		[<digit> [3 ]] 
	]
\end{forest}
\subsubsection{c.}
\textbf{a321bc} \\
\begin{forest}
[<identifier> 
	[<identifier> 
		[<identifier> 
			[<identifier> 
				[<identifier> 
					[<identifier> 
						[<letter> [a ]]
					]
					[<digit> [3 ]]
				]
				[<digit> [2 ]]
			]
			[<digit> [1 ]]
		]
		[<letter> [b ]]
	] 
	[<letter> [c ]] 
]
\end{forest}
\clearpage
\subsection{7.4}
\singlespacing
\emph{Derive the following strings with the grammar of Figure 7.2 and draw the corresponding syntax tree:}
\begin{center} \includegraphics[width=0.45\linewidth]{7_2grammar.png} \\ \end{center}
\doublespacing
\subsubsection{b.}
\textbf{+ddd} \\
\begin{center}\begin{forest}
[<I> 
	[F [+ ]]
	[M 
		[d ]
		[M 
			[d ]
			[M [d ]]
		]
	]
]
\end{forest}\end{center}
\subsubsection{c.}
\textbf{d} \\
\begin{center}\begin{forest}
[<I> 
	[F [$\epsilon$ ]]
	[M [d ]]
]
\end{forest}\end{center}
\clearpage

\subsection{7.11}
\emph{For each of the machines shown in Figure 7.48, \textbf{(a)} state whether the FSM is deterministic or nondeterministic and \textbf{(b)} identify any states that are inaccessible.}

\subsubsection{b.}
\begin{center} \includegraphics[]{7_11_b.png} \\ \end{center}
\noindent
\textbf{(a)} 
This FSM is deterministic. \\
\noindent
\textbf{(b)} 
All states are accessible in this FSM. \\
\subsubsection{d.}
\begin{center} \includegraphics[]{7_11_d.png} \\ \end{center}
\noindent
\textbf{(a)} 
This FSM is deterministic. \\
\noindent
\textbf{(b)} 
State (Y) is inaccessible in this FSM.\\
\clearpage

\subsection{7.12}
\emph{Remove the empty transitions to produce the equivalent machine for each of the finite state machines in Figure 7.49.}

\subsubsection{a.}

\begin{minipage}[c]{\linewidth}
	\centering
	\includegraphics[valign=c, width=0.34\linewidth ]{7_12_a.png}
	$\rightarrow$
	\includegraphics[valign=c, width=0.34\linewidth]{7_12_a_solution.png}
\end{minipage}
	
\subsubsection{b.}
\begin{minipage}[c]{\linewidth}
	\centering
	\includegraphics[valign=c, width=0.34\linewidth]{7_12_b.png}
	$\rightarrow$
	\includegraphics[valign=c, width=0.34\linewidth]{7_12_b_solution.png}
\end{minipage}
\clearpage

\subsection{7.13}
\emph{Draw a deterministic FSM that recognizes strings of 1's and 0's specified by each of the following criteria. Each FSM should reject any characters that are not 0 or 1.}

\subsubsection{a.}
\emph{The string of three characters, 101.} \\
\begin{center} \includegraphics[width=0.5\linewidth]{7_13_a_solution.png} \\ \end{center}

\subsubsection{d.}
\emph{All strings of arbitrary length that contain a 101 at least once anywhere. For example, the FSM should accept all the strings mentioned in parts (a), (b), and (c) as well as strings such as 11100001011111100111.} \\
\begin{center} \includegraphics[width=0.5\linewidth]{7_13_d_solution.png} \\ \end{center}

\nocite{warford10}
\bibliographystyle{apacite}
\bibliography{CS}
\end{document}