\documentclass[a4paper,man,natbib]{apa6}
\usepackage[english]{babel}
\usepackage[utf8x]{inputenc}

% Common Packages - Delete Unused Ones %
\usepackage{setspace}
\usepackage{amsmath}
\usepackage[cache=false]{minted}
\usepackage{algorithm}
\usepackage[noend]{algpseudocode}
\usepackage{graphicx}
\usepackage{caption}
\usepackage{float}
\usepackage{adjustbox}
\graphicspath{ {./Images/} }
% End Packages %
\algdef{SE}[DOWHILE]{Do}{doWhile}{\algorithmicdo}[1]{\algorithmicwhile\ #1}% make Do-While construct for my psuedocode

\title{Homework 2}
\shorttitle{HW2}
\author{Brandon Hosley}
\date{2018 09 17}
\affiliation{Mike Davis}

%\abstract{}

\begin{document}
\maketitle
\singlespacing

\section{Problem 1}
\subsection{A.} 82 \\
1. Signed Magnitude: 0101 0010 \\
2. One's Complement: 0101 0010 \\
3. Two's Complement: 0101 0010
\subsection{B.} -17 \\
1. Signed Magnitude: 1001 0001 \\
2. One's Complement: 1110 1110 \\
3. Two's Complement: 1110 1111
\subsection{C.} The sum of the Two's Compliment in A and B. \\
0101 0010 + \\
\underline{1110 1111 =} \\
0100 0001 \\
1. Signed Magnitude: 65 \\
2. One's Complement: 65 \\
3. Two's Complement: 65

\section{Problem 2}
Using 14-bit floating point model represent [-100.375]
\subsection{A.}
The number is negative, the sign bit is 1.
\subsection{B.}
$100_{10}$ = 0110 0100$_{2}$ => $01.10 0100*2^{6}$
giving a biased exponent of 0 0110 or corrected (+15) of 1 0101
\subsection{C.}
$100.375_{10} = 0110 0100.0110_{2}$ \\
=> 01.10 0100 0110
The significand = 10 0100 0110 (the final two bits will end up being dropped.)
\subsection{D.}
In 14-bit floating point notation:
$-100.375_{10}$ = 1 10101 1001 0001$_{2}$
\section{Problem 3}
\begin{center}
	\begin{adjustbox}{width=\textwidth}
		\begin{tabular}{|c|c|c|c|c|c|c|c|}
			
\cline{1-8}    & Do & Register & Mode     & A    & X    & Mem[0A3C] & Mem[2A42] \\ \cline{1-8}
Original Content &    &          &          & 10B6 & FE25 & 0A41      & 0A3F \\	\cline{1-1}
Instruction  &&&&&&& \\	\cline{1-8}
79 2A42  & Add to register & Index & Direct & 0A3F & 2A42 & 0A41 & 0A3F \\	
E1 2A42  & Store reg to mem&Accumulator&Direct&10B6 &2A42 & 0A41 & 10B6 \\	
A9 0A3C  & Bitwise or      & Index & Direct & 1ABF & FE25 & 0A41 & 0A3F \\	
C2 2A42  &Load reg from mem&Accumulator&Indirect&xxx &0A3F& 0A41 & 0A3F \\
\cline{1-8}	

		\end{tabular}
	\end{adjustbox}
\end{center}
\section{Problem 4}
\subsection{A.}
\begin{tabular}{l l l}
	Address & \multicolumn{2}{l}{Machine Language (Hex)} \\ \cline{1-3}
	0000 & C1000C & :Load Register from 000C \\
	0003 & 18 & :Bitwise inversion \\
	0004 & F1000B & :Store Byte register to memory too 000B \\
	0007 & 51000B & :Character Output from 000B\\
	000A & 00 & :Stop \\ 
	000B & 00 & :Empty \\
	000C & FFDA & :1111 1111 1101 1010 \\
\end{tabular}
\subsection{B.}
This program outputs the character \%
\subsection{C.}
The unary instruction at address 0003 is an operation that works on the current value without need of any operands.
\subsection{D.}
This operation inverts the binary in the accumulator: \\
1111 1111 1101 1010 becomes: \\
0000 0000 0010 0101 = ASCII \%
\section{Problem 5}
\subsection{A.}
\begin{algorithm}[H]
	\begin{algorithmic}
		\State\emph{Load the machine language program into memory starting at address 0000.}
		\State PC $\gets$ 0000
		\State SP $\gets$ Mem [FFF8]
		\Do
			\State\emph{Fetch the instruction specifier at address in} PC
			\State PC $\gets$ PC + 1
			\State\emph{Decode the instruction specifier}
			\If {(\emph{the instruction is not unary})}
				\State\emph{Fetch the operand specifier at address in} PC
				\State PC $\gets$ PS + 2
			\EndIf
			\State\emph{Execute the instruction fetched}
		\doWhile{((\emph{the stop instruction does not execute}) \&\& (\emph{the instruction is legal})}
	\end{algorithmic}
	\caption{\cite{warford10}'s algorithm detailed in \emph{figure 4.31}}
\end{algorithm}
For this implementation each instruction is to print an ASCII letter from a specific address.
\subsection{B.}
	\begin{tabular}{l l l}
		Address & \multicolumn{2}{l}{Machine Language (Hex)} \\ \cline{1-3}
		0000 & 510016 & :Character Output \\
		0003 & 510017 & :Character Output \\
		0006 & 510018 & :Character Output \\
		0009 & 510019 & :Character Output \\ 
		000C & 51001A & :Character Output \\
		000F & 51001B & :Character Output \\
		0010 & 51001C & :Character Output \\
		0013 & 00 & :Stop \\
		0016 & 42 & :ASCII B Character \\
		0017 & 52 & :ASCII R Character \\
		0018 & 41 & :ASCII A Character \\
		0019 & 4E & :ASCII N Character \\
		001A & 44 & :ASCII D Character \\
		001B & 4F & :ASCII O Character \\
		001C & 4E & :ASCII N Character \\
	\end{tabular}
\subsection{C.}
Plain Hex of the above program: \\
51 00 16 51 00 17 51 00 18 51 00 19 51 00 1A 51 00 1B 51 00 1C \\
00 42 52 41 4E 44 4F 4E 00 zz
\subsection{D.}
\begin{minipage}{\linewidth}
	\includegraphics[width=\linewidth]{pep8NameScreen.png}
	\captionof{figure}{Pep/8 simulator with name output.}
\end{minipage}

\section{Problem 6}
\subsection{A.}
\begin{algorithm}
	\begin{algorithmic}
		\State\emph{Numbers are hardcoded into data at end of program.}
		\State Acc $\gets$ first number$_{bin}$
		\State Acc += second number$_{bin}$ \Comment{in two's compliment form.}
		\State Acc += third number$_{bin}$
		\State Acc $\rightarrow$ \emph{Memory}
		\State\emph{Print from Memory}
	\end{algorithmic}
\end{algorithm}
\subsection{B.}
	\begin{tabular}{l l l}
		Address & \multicolumn{2}{l}{Machine Language (Hex)} \\ \cline{1-3}
		0000 & C10015 & :Load First Number \\
		0003 & 710017 & :Add Second Number \\
		0006 & 710019 & :Add Third number \\
		0009 & A1001B & :Convert to ASCII \\
		000C & F10013 & :Save result \\ 
		000F & 510013 & :Number Output \\
		0012 & 00 & :Stop \\
		0013 & 0000 & :Result \emph{initially empty} \\
		0015 & 0006 & :First Number (6)\\
		0017 & 00F9 & :Second Number (-7) \\
		0019 & 0004 & :Third Number (4) \\
		001B & 0030 & :ASCII Mask
	\end{tabular}
\subsection{C.}
Plain hex for above program:
C1 00 15 71 00 17 71 00 19 A1 00 1B F1 00 13 51 00 13 00 00 00 00 06 00 F9 00 04 00 30 zz
\subsection{D.}
	\begin{minipage}{\linewidth}
		\includegraphics[width=\linewidth]{pep8AdditionScreen.png}
		\captionof{figure}{Pep/8 simulator with result of (6)+(-7)+(4) output.}
	\end{minipage}
\subsection{E.}
The pep8 would be able to use numbers in the following ranges. \\
Signed Int: -128 to 127 \\
Unsigned Int: 0 to 255 \\
It should be noted that while all of the operands and the result may fall into this range, if any intermediate results are capable of producing overflow errors.
\newpage
\bibliographystyle{apacite}
\bibliography{ComputerArchitecture} %link to relevant .bib file
\end{document}