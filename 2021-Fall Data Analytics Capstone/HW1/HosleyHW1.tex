\documentclass[]{article}
\usepackage[english]{babel}
\usepackage{amsmath}
\usepackage[hypcap=false]{caption}
\usepackage{graphicx}
\usepackage{hyperref}
\hypersetup{
	hidelinks
	}

\title{Data Analytics Capstone: Reading 1}
\author{Brandon Hosley}
\date{\today}

\begin{document}
	\maketitle
	
\section{Summary} 

In this paper \cite{Lakkaraju2015} the research team seeks to increase the efficacy of machine learning models applied to certain problems by improving the way in which the usefulness of the models is evaluated. 
The framework compares the predictive results of five models produced using common machine learning algorithms: Random Forest, AdaBoost, Linear Regression, Decision Tree, SVM.
The first measurement used is the ROC curve, a traditional metric used to evaluate the predictive effectiveness of the models.
In this case all models are performing quite similarly.

In the scenario used by the research team the final predictive accuracy of the model is not the sole measurement of value.
The team's framework provides visualization techniques for the high-use features in each model;
which allows the end user to eliminate models that may over-emphasize features that are problematic for any reason.


\section{Analysis}




\clearpage
\bibliographystyle{acm}
\bibliography{\jobname}
\end{document}