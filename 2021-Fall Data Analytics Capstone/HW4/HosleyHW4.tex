\documentclass[]{article}
\usepackage[english]{babel}
\usepackage{amsmath}
\usepackage[hypcap=false]{caption}
\usepackage{graphicx}
\usepackage{hyperref}
\hypersetup{
	hidelinks
	}

\title{Data Analytics Capstone: Reading 4}
\author{Brandon Hosley}
\date{\today}

\begin{document}
	\maketitle
	
\section{Purpose} 

This paper will summarize two proposals for a capstone project for the Data Analytics program.

\section{Computer Vision Project 1}

This proposal seeks to address a problem set associated with visual image target scoring. Traditionally in ISSF (International Spot Shooting Federation) and Olympic style shooting
targets are scored using acoustic or laser triangulation 
\cite{Anderson2018} \cite{SUIS}.
Due to the high cost associated with commercial products that use these methods, 
there are numerous open source \cite{etarg} \cite{freetarget}
and research projects \cite{Rudzinski2012} \cite{Stenhager2021}
dedicated to producing cheaper alternatives.
A promising alternative is using visual image target scoring.

While the visual method has shown great success in scoring hit-miss events \cite{Ye2011}
limitations to producing accurate scores in a practical setting exist.
Some methods \cite{Rudzinski2012} produce accurate results 
but require strong controls on the setting.
Others \cite{Stenhager2021} are more setting robust, 
with a lower detection rate or accuracy in its product.

The first project proposal will be to extend on previous work done in this line and to  attempt to produce a more robust visual scoring framework.

\section{Computer Vision Project 2}

The research and open source projects referenced in the previous section seek to solve the problem of a lack of affordability in commercial solutions to shot scoring and recording.
Generally, the use cases described are for shooting clubs or individual practitioners without the means to acquire professional grade equipment, not the type of club or organization sizable enough to host a tournament.
As such it seems that the underlying motivation is to make opportunities for higher quality training more attainable for all people.
Shot scoring is only one aspect of training and dry-fire training is often \cite{}
considered to be a more important for developing a effective results.

As with shot scoring solutions, commercial options for dry-fire training can be expensive, or very limited in scope.
The second project proposal will be to use computer vision, a standard USB camera, and a laser to emulate point of aim feedback available in more expensive 
\cite{} commercial products.
% ShootOff is a similar laser spot tracker, but only tracks the shot


\section{Web Scraping Project}

\clearpage
\bibliographystyle{acm}
\bibliography{\jobname}
\end{document}