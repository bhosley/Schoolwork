\documentclass[]{article}
\usepackage[english]{babel}
\usepackage{amsmath}
\usepackage[hypcap=false]{caption}
\usepackage{graphicx}
\usepackage{hyperref}
\hypersetup{
	hidelinks
	}

\title{Data Analytics Capstone: Reading 2-A}
\author{Brandon Hosley}
\date{\today}

\begin{document}
	\maketitle
	
\section{Summary} 

Wing \cite{Wing2020} writes about 10 areas of study for the future of data science.
Wing prefaces this list with a question regarding the definition of academic disciplines,
they characterize this definition with examples of the largest questions facing well established, traditional disciplines.
Wing does not provide a singular root question for the field of data science but instead uses the proposal of 10 areas as a potential foundation.
Additionally, several questions concerning the relation between data science and it related disciplines of computer science, mathematics, and statistics.


\section{Analysis}

Wing's assessment of ten areas of interest within the field of data science is backed by several modern papers that are cited within their own paper, though they could easily have included many more on the same topics.
The ten proposed areas are a strong basis for inquiry in the field, though, to call them an enumeration may be a bit bold.
Potential boldness aside, if one accepts data science as a field, the strength of each of these support the claim that the field has a wide breadth of purpose.

However, a problem that Wing calls attention to is supported by these examples in its own right;
that is that one may struggle to call data science its own distinct field due to its inextricable closeness to its related fields.
For example, proposal seven discusses computation systems designed specifically to process large amounts of data. 
While this problem would likely be better faced by a computer engineer than a data scientist, it may be considered a cross-domain problem.
This trend follows for most of the ten proposals.

Wing calls attention to the difficulty in defining academic domains, and very effectively shows data science to be a domain at the cross-sections of the computer science, mathematics, and statistics domains.

\clearpage
\bibliographystyle{acm}
\bibliography{\jobname}
\end{document}