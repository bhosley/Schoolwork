\documentclass[conference]{IEEEtran}
\usepackage[english]{babel}
\usepackage{amsmath}
\usepackage{graphicx}
\usepackage[hypcap=false]{caption}
\usepackage{subcaption}
\graphicspath{ {./images/} }
\usepackage{hyperref}
\hypersetup{
	hidelinks
	}

\title{DAT Midterm Project: Working Title}
\author{Brandon Hosley}
\date{\today}

\begin{document}
	\maketitle
	
\begin{abstract}
	%
	%	content...
	%
\end{abstract}

\section{Introduction}

Shooting sports are a type of sport inextricably linked to technology, 
both in terms of the equipment used by the athletes and tools used in their training.
There is substantial interest in maintaining the cutting edge on both fronts as athletes and teams work to remain on top of their competition.
Access to this type of technology comes with a substantial price, which may be suitable to professional athletes or clubs and individuals of greater means.
Due to the high cost, some highly talented athletes may never get access to high quality equipment or training.

In this paper we propose to develop a two-part open source framework for training that my accomplish the following:

\begin{enumerate}
	\item Recording the Shot Cycle
	\begin{enumerate}
		\item Identify the bull's-eye as a frame of reference.
		\item Identify the point of aim in real-time.
		\item Record the point of aim over the period of an entire shot cycle.
		\item Record actual shot position
	\end{enumerate}
	\item Reporting the Shot Cycle
	\begin{enumerate}
		\item Report the above information in an intuitive manner, that may be interpreted by the athlete and to allow for immediate feedback.
		\item Use the shot position to provide the score and direction.
	\end{enumerate}
\end{enumerate}

With these functions we desire to advance the field of open source options for shooting athletes to improve their craft in a manner functionally similar to the commercial offerings \cite{scatt, noptel} that may otherwise be out of reach.
% ShootOff is a similar laser spot tracker, but only tracks the shot

\section{Related Work}

\subsection{Similar Open-Source Tools}

First we examine work that similarly looks to address the problem of high cost associated with competitive equipment.
At the time of this writing the available open-source options all address the problem of shot scoring.
Traditionally, target scoring is performed using Piezo-electronic sensor to achieve acoustic triangulation \cite{Anderson2018}.
However, some manufacturers have had success with laser triangulation \cite{SUIS}.
In response to the price of this equipment some developers have produced open source projects featuring acoustic scoring \cite{etarg, freetarget}.
Still more researchers have sought to use computer vision to perform the scoring \cite{Rudzinski2012, Stenhager2021}.

Accurate scoring with computer vision and commonly available parts continues to progress rapidly in recent years.
\cite{Ye2011} reliably produced  hit-miss results while \cite{Rudzinski2012} were able to produce competition level accuracy in scoring static targets with the trade-off of extremely controlled reading conditions.
More recently \cite{Stenhager2021} was able to achieve a fairly robust model that could perform in a realistic setting and gave results usable at an amateur level; their system unable to accurately score a round at the necessary resolution.

\subsection{Aim tracing}

Shot scoring is only one aspect of training and dry-fire training is often 
considered to be the most important for developing a effective results.
\cite{Potter2017}
This importance is reflected in shot trainers that sometimes cost as much as the top of the line equipment itself.
\cite{scatt, noptel}

Both the Noptel and SCATT systems utilize computer vision for their tracking.
They utilize the relative positioning of the bullseye within the camera's view to determine the position of the aim.
Actual point of aim is calibrated by having the weapon fire and recording where in the frame the resulting hole is observed.
The cameras use an uncommon lens configuration for the optical zoom which produce effective results for their intended purpose, but are not likely useful in other applications.
Conversely, cameras of this type are not easily sourced from other applications.

The Mantis X \cite{mantisx} product utilizes an inertial measurement unit that has been ruggedized against small arm recoil. 
This product does not utilize a camera and is not able to produce a prediction of a point of aim.
Instead the output is a trace of relative position and will correlate the relative positions to the position when the weapon is fired.
In this way it can still produce data valuable to the shot cycle without actually being able to score.
The advantage is that it may eschew any visual obstacles or shortcomings.
This product also utilizes very niche components that are not ideal for open source.

The commercial trainers share a set of common features and outputs,
in addition to collecting the point of aim these software break up the trace into several parts:
\begin{enumerate}
	\item Entry onto target and sighting
	\item Triggering (approximated as 1 second before impact)
	\item Shot (0.3 seconds before impact)  and Impact
	\item Follow-through and leaving the target
\end{enumerate}
For ease of interpretation each of these stages are delineated by color; the schema presented by SCATT can be seen in figures 
\ref{fig:SCATT_Example} and
\ref{fig:MantisX_Example}.

\begin{figure}[h]
	\centering
	\includegraphics[width=\linewidth]{perfect_shot}
	\caption{SCATT \cite{scatt} feedback.}
	\label{fig:SCATT_Example}
\end{figure}

\begin{figure}[h]
	\centering
	\includegraphics[width=0.6\linewidth]{mantisx_output}
	\caption{MantisX \cite{mantisx} output.}
	\label{fig:MantisX_Example}
\end{figure}

\section{Dataset}

All of the data sets that we encountered during the preliminary research for this project utilized proprietary storage methods.
While one site was able to use a framework that retrieved the actual shot results from SCATT traces \cite{scatt-db}, that particular framework was not able to retrieve the trace data itself.

To address this we use an easier to ingest data storage method, trading potential gains in efficiency when using binaries for ease of access.

\section{Methods}

The hardware used for this experiment was 
a Microsoft Lifecam HD-5000,
a Steyr evo 10 E,
and a red laser pointer of unknown origin.
All recording, processing, and analysis was performed on 
a 2020 Apple M1 Macbook Pro.

Video capture and processing was performed using the 
OpenCV \cite{itseez2015opencv} framework as a base.
The bullseye area is used as a fiducial for recording and 
the necessary data is determined by applying a series of OpenCV functions.

\begin{figure}[h]
	\centering
	\includegraphics[width=0.75\linewidth]{example-image}
	\caption{Recording Algorithm.}
	\label{fig:capture_algorithm}
\end{figure}

The first step is to import the frame as a hue, saturation, value image.
Then a mask is applied to filter for only the near-black area.
A contour is applied to the outer edge of the mask.
The moments of the outer contour are calculated.
The first order moment of each dimension is divided by the zeroth moment to determine the center coordinate for the corresponding dimension.
($M_{10}$ and $M_{01}$ each divided by $M_{10}$ using eq. \eqref{Moments})
Finally, a minimum enclosing circle is applied around the contour and used to provide an approximate radius.

\begin{equation}\label{Moments}
	\text{Moments} = M_{ij} = \sum_{x} \sum_{y} x^i y^j I(x,y)
\end{equation}

A Hough Circle Transform solution \cite{Hough1964} was considered for producing both of these data.
However, the results were extremely unstable when applied to the conditions used for this experiment and the video capture itself.
Additionally, while this method may have been more accurate for determining the true center of the target, it would not work well for an ellipsoid reference.
This problem is discussed further in the conclusion of this paper.

The surface of standard competition targets causes the laser to reflect diffusely.
The diffusion causes the marker to appear as a highly irregular ellipsoid,
however, when the angle of incident is low the moment is still an effective measure of center for the marker.
This was not a problem with the dark area of the bulls-eye as it was far less reflective.



\section{Analysis}

\noindent
\textlangle Output of replay graph \textrangle \\
\textlangle Heatmap \textrangle \\
\textlangle Time-series scalar distance \textrangle \\
\textlangle Score report \textrangle \\
\textlangle Ref Zeljko for analysis to shot cycle correlation \textrangle

\section{Results}

\noindent
\textlangle Include example of common mistakes \textrangle \\
\textlangle Visualizations should be intuitive \textrangle

\section{Conclusion}

The diffuse reflection mentioned in the method section did cause some problems, especially when the marker was crossing from the light areas of the outer target to the dark area of the bullseye which experienced a lot less diffusion.

\subsection{Areas for Improvement}
The method used to evaluate the bullseye reference frame does not compensate for the viewing angle. Instead of using a circular-bound it will need to use an ellipsoid and additional transformations will need to be applied to more accurately place the trace on the target.

\subsection{Future Work}

\noindent
\textlangle Incorporate other open source projects \textrangle \\
\textlangle Expand this project + Areas of Improvement \textrangle

% 		Expand on open sourced technologies
%			Polycarbonate impact surface and Piezo-electronic vibration sensing
%			Screen Behind for target variation, immediate shot recording and track playback, even with dry-fire training
%		Another Commercial option: https://www.traceshooting.com/?lang=en
%	Helpful for script generation: https://techvidvan.com/tutorials/create-air-canvas-using-opencv-python/
\clearpage
\bibliographystyle{IEEEtran}
\bibliography{\jobname}
\end{document}