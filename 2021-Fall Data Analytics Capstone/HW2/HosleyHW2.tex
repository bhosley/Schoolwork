\documentclass[]{article}
\usepackage[english]{babel}
\usepackage{amsmath}
\usepackage[hypcap=false]{caption}
\usepackage{graphicx}
\usepackage{hyperref}
\hypersetup{
	hidelinks
	}

\title{Data Analytics Capstone: Reading 1}
\author{Brandon Hosley}
\date{\today}

\begin{document}
	\maketitle
	
\section{Summary} 

In this paper \cite{Lakkaraju2015} the research team seeks to increase the efficacy of machine learning models applied to certain problems by improving the way in which the usefulness of the models is evaluated. 
The framework compares the predictive results of five models produced using common machine learning algorithms: Random Forest, AdaBoost, Linear Regression, Decision Tree, SVM.
The scenario approached by the research team is using school data to predict at risk students.
The first measurement used is the ROC curve, a traditional metric used to evaluate the predictive effectiveness of the models.
In this case all models are performing quite similarly.

In the scenario used by the research team the final predictive accuracy of the model is not the sole measurement of value.
The team's framework provides visualization techniques for the high-use features in each model;
which allows the end user to eliminate models that may over-emphasize features that are problematic for any reason.
Next, the team introduces a features of their framework in which early prediction accuracy is visualized for each mode.
In this scenario the earlier a model can predict an at-risk student, the earlier intervention efforts may begin.

\section{Analysis}

The framework proposed by the team seems to be a good approach to solving two potential problems with traditional model evaluation methods.
With respect to eliminating models that over-emphasize objectionable features, it seems that reserving the elimination to be performed post-facto allows the feature to be used up to a certain threshold before being subject to elimination.
While the model will likely perform better with all features included, if one is to respect the user's desire to eliminate problematic features it would be more in-line with that respect to eliminate it from the models altogether.

Adding the temporal consideration to models in a manner digestible to the average user will offer tremendous benefits to any purpose that values intervention, such as health and social services.

\clearpage
\bibliographystyle{acm}
\bibliography{\jobname}
\end{document}