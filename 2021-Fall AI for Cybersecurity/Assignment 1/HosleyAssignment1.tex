\documentclass[]{article}
\usepackage[english]{babel}
\usepackage{amsmath}
\usepackage[hypcap=false]{caption}
\usepackage{graphicx}
\usepackage{hyperref}
\hypersetup{
	hidelinks
	}

\title{Artificial Intelligence in Cybersecurity: Assignment 1}
\author{Brandon Hosley}
\date{\today}

\begin{document}
	\maketitle
	
\section{Introduction}

\section{Data Sources}

% Describe the data used in this paper including source, sample, attribute, etc. (10 points)
The research in this article utilizes data from three datasets; 
the OmniDroid \cite{Martin2019} dataset,  
the CIC2019 \cite{Taheri2019} dataset, and
the CIC2020 \cite{Rahali2020} dataset.
Together, these datasets provide over 80,000 samples of both benignware and malware.
Each dataset provides a large number of different features gathered under different methods of analysis.
For the OmniDroid dataset the features are gathered from several types of API states, permissions, and flows;
and a dynamic analysis that developed a Markov chain of states with state change probabilities representing the features.
The CIC features are developed as permission and intent features,extracted from the sample's manifest file and apk file.

\section{Algorithm}

% Explain the algorithm/method for visualization in detail from your understanding (20 points)
% Draw a flowchart of the algorithm for visualization (10 points)

\section{Results}

% Explain the experimental results in detail from your understanding (10 points)

\section{Advantages and Disadvantages}

% Discuss the advantages from your understanding (10 points)
% Discuss the disadvantages from your understanding (15 points)

\section{Improvements}

% Provide the specific ideas to improve the algorithm. General ideas are not allowed. (15 points)

\clearpage
\bibliographystyle{acm}
\bibliography{\jobname}
\end{document}