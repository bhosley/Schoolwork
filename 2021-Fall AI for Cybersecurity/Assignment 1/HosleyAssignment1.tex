\documentclass[]{article}
\usepackage[english]{babel}
\usepackage{amsmath}
\usepackage{graphicx}
\usepackage[hypcap=false]{caption}
\usepackage{subcaption}
\graphicspath{ {./images/} }
\usepackage{hyperref}
\hypersetup{
	hidelinks
	}

\title{Artificial Intelligence in Cybersecurity: Assignment 1}
\author{Brandon Hosley}
\date{\today}

\begin{document}
	\maketitle
	
\section{Introduction}

\section{Data Sources}

% Describe the data used in this paper including source, sample, attribute, etc. (10 points)
The research in this article utilizes data from three datasets; 
the OmniDroid \cite{Martin2019} dataset,  
the CIC2019 \cite{Taheri2019} dataset, and
the CIC2020 \cite{Rahali2020} dataset.
Together, these datasets provide over 80,000 samples of both benignware and malware.
Each dataset provides a large number of different features gathered under different methods of analysis.
For the OmniDroid dataset the features are gathered from several types of API states, permissions, and flows;
and a dynamic analysis that developed a Markov chain of states with state change probabilities representing the features.
The CIC features are developed as permission and intent features,extracted from the sample's manifest file and apk file.

\section{Algorithm}

% Explain the algorithm/method for visualization in detail from your understanding (20 points)
The algorithm presented is based on several techniques that have been shown to independently improve unsupervised model generation.
The first step is based on the idea that random subspaces can improve the results of an ensemble RBM model, however, instead of using random subspaces the team applies a $k$-means algorithm.
The resulting clusters are used as the subspaces.


The constituent RBM models are trained with traditional gradient ascent, and utilize a sigmoid activation function as the output layer.

% Draw a flowchart of the algorithm for visualization (10 points)
\begin{figure}[h]
	\centering
	\includegraphics[width=0.5\linewidth]{SRBM_Flowchart}
	\caption{SRBM Flow Chart \cite{Liu2021}}
\end{figure}


\section{Results}

% Explain the experimental results in detail from your understanding (10 points)
The team compares their model to a Restricted Bolztman Machine (RBM), Stacked Auto-Encoder (SAE), Principle Component Analysis (PCA), and Agglomeration techniques.
RBM and SAE are unsupervised feature learning methods. 
PCA and Agglomeration are unsupervised feature extraction methods.

\section{Advantages and Disadvantages}

% Discuss the advantages from your understanding (10 points)
The research team attributes the primary advantage of this algorithm to the feature-set subspaces.
By using the clustering algorithm to generate the subspaces the team is able to reduce the overall computational time significantly as the resulting subspaces do not need feature reduction.

% Discuss the disadvantages from your understanding (15 points)

\section{Improvements}

% Provide the specific ideas to improve the algorithm. General ideas are not allowed. (15 points)

\clearpage
\bibliographystyle{acm}
\bibliography{\jobname}
\end{document}