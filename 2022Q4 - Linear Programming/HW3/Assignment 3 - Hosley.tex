\documentclass[answers]{exam}
\usepackage[english]{babel}
\usepackage[utf8x]{inputenc}
\usepackage{amsmath,amssymb,amsthm}

\title{OPER 510 - Introduction to Mathematical Programming%
	\\ Assignment 3}
\author{Brandon Hosley}
\date{\today}

\setcounter{secnumdepth}{0}
\usepackage{soul}
\usepackage{xcolor}
\usepackage{graphicx}
\usepackage{nicefrac}

\begin{document}
\maketitle
\unframedsolutions

\begin{questions}
\question


\question
\begin{parts}
	\part
	Find the dual of the following problem:
	\begin{flalign*}
		\operatorname{Max }z =4x_1 +6x_2 +10x_3 +12x_4 & &\\
		\text{s.t.} 1x_1+3x_2+ 2x_3+ 4x_4 &\leq 5 &\\
		1x_1+1x_2+ 5x_3+ 3x_4 &\leq 15 &\\
		x_1, x_2, x_3, x_4 &\geq 0 &\\
	\end{flalign*}
	
	
	\part
	Graph the dual and use the Complementary Slackness Theorem to find the solution of the primal. Show that your solution to the primal is an extreme point to that problem.
\end{parts}


\question
Consider the solution to the following linear programming problem:
\begin{flalign*}
	\operatorname{Max } z = 10x_1 + 8x_2 + 17x_3 & & &\\
	\text{s.t.} (\nicefrac{1}{2}) x_1 + (\nicefrac{1}{4})x_2 + (\nicefrac{7}{8})x_3 + x_4 &= 25 &\text{(resource 1)} &\\
	(\nicefrac{1}{2}) x_1 + (\nicefrac{3}{4})x_2 + (\nicefrac{9}{8})x_3 + x_5 &=45 &\text{(resource 2)} &\\
	(\nicefrac{1}{2}) x_1 + (\nicefrac{11}{4})x_2 + (\nicefrac{25}{8})x_3 + x_6 &=145 &\text{(resource 3)} &\\
	x_j &\geq 0 &\text{for all }j &
\end{flalign*}
\begin{parts}
	\part
\end{parts}

Optimal Tableau (Classic)

(You may use either the classic or the text tableau as you prefer.)
Using the final tableau in either form answer the following questions showing your work:
a. What are the current marginal values (shadow prices) of resources 1, 2 and 3 in the optimal tableau?
b. Find the sensitivity range for c1, c2, and c3?
c. Find the sensitivity range for b1, b2, and b3?
d. By how much would profit be increased if 10 additional units of resource 1 could be obtained? What would be the new solution?
e. Suppose a new product can be produced at a cost of \$8.00 and would use 1 unit of resource 1, 2 units of resource 2 and 2 units of resource 3. Based on the current optimal solution, what is the minimum selling price that could be set if the new product is to be profitable to produce?


\question
\begin{parts}
	\part
	Solve the following problem via the dual simplex method.
	\begin{flalign*}
		\operatorname{Min } z=2x_1 +1x_2 & &\\
		\text{s.t.} 1x_1 + 1x_2 &\geq 15 &\\
		1x_1 - 1x2 &\leq 1 &\\
		x_1, x_2 &\geq 0 &\\
	\end{flalign*}
	
	\part
	Add the constraint \(2x_1 + x_2 \leq 7\) to the optimal tableau in (a). State the new solution with this additional constraint.
	
\end{parts}


\end{questions}
\end{document}