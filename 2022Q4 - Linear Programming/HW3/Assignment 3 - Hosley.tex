\documentclass[answers]{exam}
\usepackage[english]{babel}
\usepackage[utf8x]{inputenc}
\usepackage{amsmath,amssymb,amsthm}

\title{OPER 510 - Introduction to Mathematical Programming%
	\\ Assignment 3}
\author{Brandon Hosley}
\date{\today}

\usepackage[table,dvipsnames]{xcolor}
\usepackage{graphicx}
\usepackage{nicefrac}
\usepackage{enumitem}
\renewcommand{\subpartlabel}{}

\begin{document}
\maketitle
\unframedsolutions

\begin{questions}
\question
A company produces two products. Relevant information for each product is shown in Table 58. The company has a goal of \$48 in profits and incurs a \$1 penalty for each dollar it falls short of this goal. A total of 32 hours of labor are available. A \$2 penalty is incurred for each hour of overtime (labor over 32 hours) used, and a \$1 penalty is incurred for each hour of available labor that is unused. Marketing considerations require that at least 10 units of product 2 be produced. For each unit (of either product) by which production falls short of demand, a penalty of \$5 is assessed.

\begin{parts}
	\part Formulate an LP that can be used to minimize the penalty incurred by the company.
	\part Suppose the company sets (in order of importance) the following goals:
	\begin{subparts}
		\subpart
		\begin{enumerate}
			\item[Goal 1] Avoid underutilization of labor.
			\item[Goal 2] Meet demand for product 1.  
			\item[Goal 3] Meet demand for product 2.
			\item[Goal 4] Do not use any overtime.
		\end{enumerate}
	\end{subparts}
	\begin{tabular}[]{ccc} 
		\arrayrulecolor{MidnightBlue} \noalign{\global\arrayrulewidth=1pt} \hline 
		\rowcolor{gray!25} & \textcolor{MidnightBlue}{\textbf{Product 1}} & \textcolor{MidnightBlue}{\textbf{Product 2}} \\
		\hline
		Labor required & 4 hours & 2 hours \\
		Contribution to profit & \$4 & \$2 \\
		\noalign{\global\arrayrulewidth=2pt}
		\hline
		\noalign{\global\arrayrulewidth=0.4pt}
	\end{tabular}
\end{parts}

Formulate and solve a preemptive goal programming model for this situation.

\begin{solution}
	% Demand: 7 10 
\end{solution}


\question
\begin{parts}
	\part
	Find the dual of the following problem:
	\begin{flalign*}
		\operatorname{Max }z =4x_1 +6x_2 +10x_3 +12x_4 & &&\\
		\text{s.t.}\hspace{3em} 1x_1+3x_2+ \hspace{1ex} 2x_3+ \hspace{1ex} 4x_4 &\leq 5 &&\\
		1x_1+1x_2+ \hspace{1ex} 5x_3+ \hspace{1ex} 3x_4 &\leq 15 &&\\
		x_1, x_2, x_3, x_4 &\geq 0 &&\\
	\end{flalign*}
	
	\part
	Graph the dual and use the Complementary Slackness Theorem to find the solution of the primal. Show that your solution to the primal is an extreme point to that problem.
\end{parts}
\begin{solution}
	\begin{parts}
		\part
		Dual:
		\begin{flalign*}
			\operatorname{Min } z_2 = 5y_1 + 15y_2 \\
			\text{s.t.}\hspace{2em} 
			y_1 + y_2 \geq 4 \\
			3y_1 + y_2 \geq 6 \\
			2y_1 + 5y_2 \geq 10 \\
			4y_1 + 3y_2 \geq 12 \\
			y_1, y_2 \geq 0 \\
		\end{flalign*}
		Refactored as a maximization problem with slack variables added.
		\begin{flalign*}
			\operatorname{Max } -z_2 = -5y_1 -15y_2 \\
			\text{s.t.}\hspace{2em} 
			y_1 + y_2 + e_1 = 4 \\
			3y_1 + y_2 + e_1 = 6 \\
			2y_1 + 5y_2 + e_1 = 10 \\
			4y_1 + 3y_2 + e_1 = 12 \\
			y_1, y_2, e_1, e_2, e_3, e_4 \geq 0 \\
		\end{flalign*}
		
		\part
		%need labeled axes
		Graph of the Dual: \\
		\includegraphics[width=0.8\linewidth]{"2b"} \bigskip \\
		\begin{tabular}{cc}
			Point & Value ($z_2=5y_1+15y_2$) \\
			\hline
			$(0,6)$ & $90$\\
			$\mathbf{(5,0)}$ & $\mathbf{25}$\\
			$(1,3)$ & $50$ \\
			$(10/3,2/3)$ & $26.33\bar{3}$ \\
		\end{tabular} \bigskip \\
		Solving for values of slack values in the dual when $y_1=5$ and $y_2=0$,
		then applying complementary slackness: \bigskip \\
		\begin{tabular}{cc}
			Dual Slack & Primal Variable \\
			\hline
			$e_1=1$ & $x_1=0$ \\
			$e_2=9$ & $x_2=0$ \\
			$e_3=0$ & $x_3\geq0$ \\
			$e_4=8$ & $x_4=0$ \\
		\end{tabular} \bigskip \\
		Placing these values back into the primal system, we see $x_3\leq2.5$ and $x_3\leq3$.
		Evaluating the primal objective function with $x_3=2.5$ returns $z=25$, matching the value of the dual objective.
		
	\end{parts}
\end{solution}


\question
Consider the solution to the following linear programming problem:
\begin{flalign*}
	\operatorname{Maximize } z = 10x_1 + 8x_2 + 17x_3 \hspace{15ex} & & \\
	\text{s.t.}\hspace{2em} (\nicefrac{1}{2}) x_1 + (\nicefrac{1}{4})x_2 + (\nicefrac{7}{8})x_3 + x_4 \hspace{9ex} &= 25 &\text{(resource 1)} &&&&\hspace{10em}\\
	(\nicefrac{1}{2}) x_1 + (\nicefrac{3}{4})x_2 + (\nicefrac{9}{8})x_3 \hspace{3ex} +x_5 \hspace{6ex} &=45 &\text{(resource 2)} &&&&\hspace{10em}\\
	(\nicefrac{1}{2}) x_1 + (\nicefrac{11}{4})x_2 + (\nicefrac{25}{8})x_3 \hspace{6ex} +x_6 \hspace{3ex} &=145 &\text{(resource 3)} &&&&\hspace{10em}\\
	x_j &\geq 0 &\text{for all }j &&&&\hspace{10em}
\end{flalign*}
Optimal Tableau (Classic)

\begin{tabular}{ccccccccc}
	$c_j$                   &                            &                                & 10    & 8     & 17    & 0     & 0     & 0     \\
	\multicolumn{1}{c|}{}   & \multicolumn{1}{c|}{bv}    & \multicolumn{1}{c|}{RHS}       & $x_1$ & $x_2$ & $x_3$ & $x_4$ & $x_5$ & $x_6$ \\ \cline{1-9}
	\multicolumn{1}{c|}{8}  & \multicolumn{1}{c|}{$x_2$} & \multicolumn{1}{c|}{40}        & 0     & 1     & 1/2   & -2    & 2     & 0     \\
	\multicolumn{1}{c|}{10} & \multicolumn{1}{c|}{$x_1$} & \multicolumn{1}{c|}{30}        & 1     & 0     & 3/2   & 3     & -1    & 0     \\
	\multicolumn{1}{c|}{0}  & \multicolumn{1}{c|}{$x_6$} & \multicolumn{1}{c|}{20}        & 0     & 0     & 1     & 4     & -5    & 1     \\ \cline{1-9}
	& \multicolumn{1}{c|}{$z_j$} & \multicolumn{1}{c|}{620}       & 10    & 8     & 19    & 14    & 6     & 0            \\ \cline{3-9}
	&                            & \multicolumn{1}{c|}{$c_j-z_j$} & 0     & 0     & -2    & -14   & -6    & 0          
\end{tabular}

(You may use either the classic or the text tableau as you prefer.)
Using the final tableau in either form answer the following questions showing your work:
\begin{parts}
	\part
	What are the current marginal values (shadow prices) of resources 1, 2 and 3 in the optimal tableau?
	\part
	Find the sensitivity range for c1, c2, and c3?
	\part
	Find the sensitivity range for b1, b2, and b3?
	\part
	By how much would profit be increased if 10 additional units of resource 1 could be obtained? What would be the new solution?
	\part
	Suppose a new product can be produced at a cost of \$8.00 and would use 1 unit of resource 1, 2 units of resource 2 and 2 units of resource 3. Based on the current optimal solution, what is the minimum selling price that could be set if the new product is to be profitable to produce?
\end{parts}
\begin{solution}
	\begin{parts}
		\part
		Shadow Prices shown by $z_j$ of the corresponding slack variable ($x_4,x_5,x_6$ respectively): \bigskip \\
		\begin{tabular}{c|c}
			Resource: & Unit of Currency: \\
			\hline
			1 & 14 \\
			2 & 6 \\
			3 & 0 \\
		\end{tabular} \bigskip \\
		
		
		\part
		\begin{align}
			c_1 = 8 \\
			\operatorname{max}(\frac{-19}{1/2},\frac{-2}{6})\leq\delta&c_1\leq\operatorname{min}(\frac{-14}{-2}) \\
			-\frac{1}{3}\leq\delta&c_1\leq7 \\
			7\nicefrac{2}{3}\leq\delta&c_1\leq15 \\
		\end{align}
		\begin{align}
			c_2 = 10 \\
			\operatorname{max}(\frac{-19}{3/2},\frac{-14}{3})\leq\delta&c_2\leq\operatorname{min}(\frac{-6}{-1}) \\
			-4\frac{1}{3}\leq\delta&c_2\leq7 \\
			5\nicefrac{1}{3}\leq\delta&c_1\leq16 \\
		\end{align}
		\begin{align}
			c_3 = 0 \\
			\operatorname{max}(\frac{-19}{1},\frac{-14}{4})\leq\delta&c_3\leq\operatorname{min}(\frac{-6}{-5}) \\
			-3\frac{1}{3}\leq\delta&c_3\leq\frac{6}{5} \\
			-3\nicefrac{1}{3}\leq\delta&c_3\leq1\nicefrac{1}{5} \\
		\end{align}
	
	
		\part
		\begin{align}
			b_1: x_4 = 25 \\
			\operatorname{max}(\frac{-40}{2})\leq\delta&b_1\leq\operatorname{min}(\frac{-20}{-5},\frac{-30}{-1}) \\
			-20\leq\delta&b_1\leq4 \\
			5\leq\delta&b_1\leq29 \\
		\end{align}
		\begin{align}
			b_2: x_5 = 45 \\
			\operatorname{max}(\frac{-30}{3},\frac{-20}{4})\leq\delta&b_2\leq\operatorname{min}(\frac{-40}{-2}) \\
			-5\leq\delta&b_2\leq20 \\
			40\leq\delta&b_2\leq65 \\
		\end{align}
		\begin{align}
			b_3: x_6 = 145 \\
			\operatorname{max}(\frac{-30}{0},\frac{-20}{1},\frac{-40}{0})\leq\delta&b_3\leq\operatorname{min}() \\
			-20\leq\delta&b_3 \\
			125\leq\delta&b_3 \\
		\end{align}
		
		
		$BV_j$ Matrix
		\(\begin{bmatrix}
			% x_2 x_1 x_3
			\nicefrac{1}{4}  & \nicefrac{1}{2} & \nicefrac{7}{8}  \\
			\nicefrac{3}{4}  & \nicefrac{1}{2} & \nicefrac{9}{8}  \\
			\nicefrac{11}{4} & \nicefrac{1}{2} & \nicefrac{25}{8} \\
		\end{bmatrix}\)
		
	\end{parts}
\end{solution}

\question
\begin{parts}
	\part
	Solve the following problem via the dual simplex method.
	\begin{flalign*}
		\operatorname{Min } z=2x_1 +1x_2 & &\\
		\text{s.t.} \hspace{2em} 1x_1 + 1x_2 &\geq 15 &\\
		1x_1 - 1x2 &\leq 1 &\\
		x_1, x_2 &\geq 0 &\\
	\end{flalign*}
	
	\part
	Add the constraint \(2x_1 + x_2 \leq 7\) to the optimal tableau in (a). State the new solution with this additional constraint.
	
\end{parts}


\end{questions}
\end{document}