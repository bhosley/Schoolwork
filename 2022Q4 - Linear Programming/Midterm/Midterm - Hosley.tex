\documentclass[answers]{exam}
\usepackage[english]{babel}
\usepackage[utf8x]{inputenc}
\usepackage{amsmath,amssymb,amsthm,mathtools}

\title{OPER 510 - Introduction to Mathematical Programming%
	\\ Midterm}
\author{Brandon Hosley}
\date{\today}

\usepackage[table,dvipsnames]{xcolor}
\usepackage{graphicx}
\usepackage{nicefrac}
\usepackage{enumitem}
\renewcommand{\subpartlabel}{}
\usepackage{multicol}
\usepackage{setspace}
\usepackage[final]{pdfpages}

\begin{document}
\includepdf[pages=-]{Midterm Agreement Signed.pdf}
\maketitle
\unframedsolutions

\begin{questions}
	
%%%%%%%%%%%%%%%%%%%%%%%%%%%
% 	\begin{ Question 1}	  %
%%%%%%%%%%%%%%%%%%%%%%%%%%%
\question
\begin{parts}
	\part Via the simplex method, find the solution to the following problem:
	\begin{flalign*}
		\text{Max } z=90x_1 +40x_2 +10x_3 +30x_4            &         &  \\
		\text{s.t.}\hspace{2.5em} 15x_1 +10x_2 +10x_3 + \,\ 5x_4 & \leq 45 &  \\
		15x_1+ \,\ 5x_2+ \,\ 5x_3+ \,\ 5x_4                             & \leq 35 &  \\
		x_1 \hspace{22.5ex}                                    & \geq 1  &  \\
		x_3 \hspace{7ex}                                    & \geq 2  &  \\
		x_1,x_2,x_3,x_4                                     & \geq 0  &
	\end{flalign*}

	\part The General Service Organization, GSO, has contracted to sell certain quantities of a particular product over the next four months. Because of variations in the size of the labor force, the production capacity and materials, costs vary from month to month. Storage costs are incurred on any items carried over to the next month, but not on items sold in the same month that they are produced. The costs and requirements are summarized below: \bigskip \\
	\begin{tabular}{ccccc}
		& Contracted & Production & Production & Storage \\
		\underline{Month} & \underline{Sales} & \underline{Capacity} & \underline{Cost/Unit (\$)} & \underline{Cost/Unit/Month (\$)} \\
		1 & 60 & 90 & 70 & 2 \\
		2 & 70 & 60 & 72 & 1 \\
		3 & 90 & 80 & 70 & 1 \\
		4 & 70 & 100 & 65 & 3 \\
	\end{tabular} \bigskip \\
	Initial inventory is 20 units. For service reasons, GSO wants to maintain at least 5 units of inventory at the end of each month. \textit{Formulate} a linear programming model to determine GSO’s production and inventory schedule.
	
	\part Give the rank of the following system of equations: 
	\begin{flalign*}
		2x_1 +\ \ 3x_2 +4x_3 &=3 &\\
		4x_1 +\ \ 5x_2 +1x_3 &=2 &\\
		 x_1 +1.5x_2 +2x_3 &=3/2 &\\
	\end{flalign*}
	
	\part Solve the following linear program.
	\begin{flalign*}
		\text{Max } z=10x_1 +3x_2 +12x_3 & & \\
		\text{s.t.}\hspace{3em} 2x_1 +1x_2+ \,\ 5x_3 &\leq 20 & \\
		x_1, x_2, x_3 &\geq 0 & \\
	\end{flalign*}
\end{parts}

\begin{solution}
\begin{parts}
	\part % Q1-A
	\noindent \\
	\begin{tabular}{cccccccccccccc}
		$c_j$                   &                            &                                & 90    & 40    & 10    & 30    & 0     & 0     & 0     & 0     & -M    & -M    &           \\
		\multicolumn{1}{c|}{}   & \multicolumn{1}{c|}{bv}    & \multicolumn{1}{c|}{RHS}       & $x_1$ & $x_2$ & $x_3$ & $x_4$ & $s_1$ & $s_2$ & $s_3$ & $s_4$ & $A_1$ & $A_2$ & ratio     \\ \cline{1-13}
		\multicolumn{1}{c|}{0}  & \multicolumn{1}{c|}{$s_1$} & \multicolumn{1}{c|}{45}        & 15    & 10    & 10    & 5     & 1     & 0     & 0     & 0     & 0     & 0     & 45/15=3   \\
		\multicolumn{1}{c|}{0}  & \multicolumn{1}{c|}{$s_2$} & \multicolumn{1}{c|}{35}        & 15    & 5     & 5     & 5     & 0     & 1     & 0     & 0     & 0     & 0     & 35/15=2.3 \\
		\multicolumn{1}{c|}{-M} & \multicolumn{1}{c|}{$A_1$} & \multicolumn{1}{c|}{1}         & 1     & 0     & 0     & 0     & 0     & 0     & -1    & 0     & 1     & 0     & 1/1=1     \\
		\multicolumn{1}{c|}{-M} & \multicolumn{1}{c|}{$A_2$} & \multicolumn{1}{c|}{2}         & 0     & 0     & 1     & 0     & 0     & 0     & 0     & -1    & 0     & 1     & 0         \\ \cline{1-13}
		& \multicolumn{1}{c|}{$z_j$} & \multicolumn{1}{c|}{-2M}       & -M    & 0     & -M    & 0     & 0     & 0     & M     & M     & -M    & -M    &           \\ \cline{3-13}
		&                            & \multicolumn{1}{c|}{$c_j-z_j$} & 90+M  & 0     & 10+M  & 0     & 0     & 0     & -M    & -M    & 0     & 0     &          
	\end{tabular} \\
	Enter on $x_1$, exit $A_1$ \\
	R1 - 15R3 \\
	R2 - 15R3 \\
	\begin{tabular}{cccccccccccccc}
		$c_j$                   &                            &                                & 90    & 40    & 10    & 30    & 0     & 0     & 0     & 0     & -M    & -M    &         \\
		\multicolumn{1}{c|}{}   & \multicolumn{1}{c|}{bv}    & \multicolumn{1}{c|}{RHS}       & $x_1$ & $x_2$ & $x_3$ & $x_4$ & $s_1$ & $s_2$ & $s_3$ & $s_4$ & $A_1$ & $A_2$ & ratio   \\ \cline{1-13}
		\multicolumn{1}{c|}{0}  & \multicolumn{1}{c|}{$s_1$} & \multicolumn{1}{c|}{30}        & 0     & 10    & 10    & 5     & 1     & 0     & 15    & 0     & 0     & 0     & 30/10=3 \\
		\multicolumn{1}{c|}{0}  & \multicolumn{1}{c|}{$s_2$} & \multicolumn{1}{c|}{20}        & 0     & 5     & 5     & 5     & 0     & 1     & 15    & 0     & 0     & 0     & 20/5=4  \\
		\multicolumn{1}{c|}{90} & \multicolumn{1}{c|}{$x_1$} & \multicolumn{1}{c|}{1}         & 1     & 0     & 0     & 0     & 0     & 0     & -1    & 0     & 1     & 0     & 0       \\
		\multicolumn{1}{c|}{-M} & \multicolumn{1}{c|}{$A_2$} & \multicolumn{1}{c|}{2}         & 0     & 0     & 1     & 0     & 0     & 0     & 0     & -1    & 0     & 1     & 2/1=2   \\ \cline{1-13}
		& \multicolumn{1}{c|}{$z_j$} & \multicolumn{1}{c|}{90-2M}     & 90    & 0     & -M    & 0     & 0     & 0     & -90   & M     & 90    & -M    &         \\ \cline{3-13}
		&                            & \multicolumn{1}{c|}{$c_j-z_j$} & 0     & 40    & 10+M  & 0     & 0     & 0     & 90    & -M    & -M-90 & 0     &        
	\end{tabular} \\
	Enter $x_3$, exit $A_2$ \\
	R1 - 10R4 \\
	R2 - 5R4 \\
	\begin{tabular}{cccccccccccccc}
		$c_j$                   &                            &                                & 90    & 40    & 10    & 30    & 0     & 0     & 0     & 0     & -M    & -M    &           \\
		\multicolumn{1}{c|}{}   & \multicolumn{1}{c|}{bv}    & \multicolumn{1}{c|}{RHS}       & $x_1$ & $x_2$ & $x_3$ & $x_4$ & $s_1$ & $s_2$ & $s_3$ & $s_4$ & $A_1$ & $A_2$ & ratio     \\ \cline{1-13}
		\multicolumn{1}{c|}{0}  & \multicolumn{1}{c|}{$s_1$} & \multicolumn{1}{c|}{10}        & 0     & 10    & 0     & 5     & 1     & 0     & 15    & 10    & -15   & -10   & 10/15=0.6 \\
		\multicolumn{1}{c|}{0}  & \multicolumn{1}{c|}{$s_2$} & \multicolumn{1}{c|}{10}        & 0     & 5     & 0     & 5     & 0     & 1     & 15    & 5     & 0     & -5    & 10/15=0.6 \\
		\multicolumn{1}{c|}{90} & \multicolumn{1}{c|}{$x_1$} & \multicolumn{1}{c|}{1}         & 1     & 0     & 0     & 0     & 0     & 0     & -1    & 0     & 1     & 0     & 1/-1=-1   \\
		\multicolumn{1}{c|}{10} & \multicolumn{1}{c|}{$x_3$} & \multicolumn{1}{c|}{2}         & 0     & 0     & 1     & 0     & 0     & 0     & 0     & -1    & 0     & 1     & 0         \\ \cline{1-13}
		& \multicolumn{1}{c|}{$z_j$} & \multicolumn{1}{c|}{110}       & 90    & 0     & 10    & 0     & 0     & 0     & -90   & -10   & 90    & 10    &           \\ \cline{3-13}
		&                            & \multicolumn{1}{c|}{$c_j-z_j$} & 0     & 40    & 0     & 30    & 0     & 0     & 90    & 10    & -M-90 & -M-10 &          
	\end{tabular} \\
	Enter $s_3$, exit $s_2$ \\
	R2 / 15 \\
	R1 - 15R2 \\
	R3 + R2 \\
	\begin{tabular}{cccccccccccccc}
		$c_j$                   &                            &                                & 90    & 40    & 10    & 30    & 0     & 0     & 0     & 0     & -M    & -M    &               \\
		\multicolumn{1}{c|}{}   & \multicolumn{1}{c|}{bv}    & \multicolumn{1}{c|}{RHS}       & $x_1$ & $x_2$ & $x_3$ & $x_4$ & $s_1$ & $s_2$ & $s_3$ & $s_4$ & $A_1$ & $A_2$ & ratio         \\ \cline{1-13}
		\multicolumn{1}{c|}{0}  & \multicolumn{1}{c|}{$s_1$} & \multicolumn{1}{c|}{0}         & 0     & 5     & 0     & 0     & 1     & -1    & 0     & 5     & -15   & -5    &               \\
		\multicolumn{1}{c|}{0}  & \multicolumn{1}{c|}{$s_3$} & \multicolumn{1}{c|}{2/3}       & 0     & 1/3   & 0     & 1/3   & 0     & 1/15  & 1     & 1/3   & 0     & -1/3  & (2/3)/(1/3)=2 \\
		\multicolumn{1}{c|}{90} & \multicolumn{1}{c|}{$x_1$} & \multicolumn{1}{c|}{5/3}       & 1     & 1/3   & 0     & 1/3   & 0     & 1/15  & 0     & 1/3   & 1     & -1/3  & (5/3)/(1/3)=5 \\
		\multicolumn{1}{c|}{10} & \multicolumn{1}{c|}{$x_3$} & \multicolumn{1}{c|}{2}         & 0     & 0     & 1     & 0     & 0     & 0     & 0     & -1    & 0     & 1     &               \\ \cline{1-13}
		& \multicolumn{1}{c|}{$z_j$} & \multicolumn{1}{c|}{170}       & 90    & 30    & 10    & 30    & 0     & 6     & 0     & 20    & 90    & -20   &               \\ \cline{3-13}
		&                            & \multicolumn{1}{c|}{$c_j-z_j$} & 0     & 10    & 0     & 0     & 0     & -6    & 0     & -20   & 90-M  & -20-M &              
	\end{tabular} \\
	Enter $x_2$ exit $s_3$ \\
	R2 x 3 \\
	R1 - 5R2 \\
	R3 - 1/3R2 \\
	\begin{tabular}{cccccccccccccc}
		$c_j$                   &                            &                                & 90    & 40    & 10    & 30    & 0     & 0     & 0     & 0     & -M    & -M    &       \\
		\multicolumn{1}{c|}{}   & \multicolumn{1}{c|}{bv}    & \multicolumn{1}{c|}{RHS}       & $x_1$ & $x_2$ & $x_3$ & $x_4$ & $s_1$ & $s_2$ & $s_3$ & $s_4$ & $A_1$ & $A_2$ & ratio \\ \cline{1-13}
		\multicolumn{1}{c|}{0}  & \multicolumn{1}{c|}{$s_1$} & \multicolumn{1}{c|}{-10}       & 0     & 0     & 0     & -5    & 1     & -2    & 0     & 0     & -15   & -10   &       \\
		\multicolumn{1}{c|}{40} & \multicolumn{1}{c|}{$x_2$} & \multicolumn{1}{c|}{2}         & 0     & 1     & 0     & 1     & 0     & 1/5   & 3     & 1     & 0     & -1    &       \\
		\multicolumn{1}{c|}{90} & \multicolumn{1}{c|}{$x_1$} & \multicolumn{1}{c|}{1}         & 1     & 0     & 0     & 0     & 0     & 0     & -1    & 0     & 1     & 0     &       \\
		\multicolumn{1}{c|}{10} & \multicolumn{1}{c|}{$x_3$} & \multicolumn{1}{c|}{2}         & 0     & 0     & 1     & 0     & 0     & 0     & 0     & -1    & 0     & 1     &       \\ \cline{1-13}
		& \multicolumn{1}{c|}{$z_j$} & \multicolumn{1}{c|}{190}       & 90    & 40    & 30    & 40    & 0     & 8     & 30    & 30    & 90    & -30   &       \\ \cline{3-13}
		&                            & \multicolumn{1}{c|}{$c_j-z_j$} & 0     & 0     & 0     & -10   & 0     & -8    & -30   & -30   & 90-M  & -30-M &      
	\end{tabular} \\

	\textbf{Summary: } The optimal values will be $x_1=1, x_2=2, x_3=2, x_4=0$ for a maximum $z=190$.
	
	\part % Q1-B 
	The situation described may look like this chart. \\
	\includegraphics[width=.5\linewidth]{TransportGraph} \\
	The system of linear equations representing this situation 
	will be presented below and will use the following variables: \\
	$z:$ Production costs, \\
	$x_{ps}:$ Products produced in month $p$ and sold in month $s$, \\
	$x_0:$ Is the initial inventory, \\
	$s_p:$ Slack in production capacity for month $p$. \\
	\textbf{Linear System Formulated: }
	\begin{flalign*}
		&\text{Min } z = 70x_{11} + 72x_{12} + 74x_{13} + 76x_{14} +72x_{22} + 73x_{23} + 74x_{24} + 70x_{33} + 71x_{34} + 65x_{44}  & \\
		&\text{s.t.}  & 
	\end{flalign*} \vspace{-3em} % x_{} 
	\begin{flalign*}
		\intertext{Contracted Sales + GSO Inventory Constraints}
		x_0 + x_{11} \hspace{18ex} &\geq 65 & \\
		x_0 + x_{12} + x_{22} \hspace{12ex} &\geq 75 & \\
		x_0 + x_{13} + x_{23} + x_{33} \hspace{6ex} &\geq 95 & \\
		x_0 + x_{14} + x_{24} + x_{34} + x_{44} &\geq 75 & \\
		\intertext{Production Capacity Constraints}
		x_0 + s_0 &= 20 & \\
		x_{11} + x_{12} + x_{13} + x_{14} + s_1 &= 90 & \\
		x_{22} + x_{23} + x_{24} + s_2 &= 60 & \\
		x_{33} + x_{34} + s_3 &= 80 & \\
		x_{44} + s_4 &= 100 & 
		\intertext{Physics Constraints}
		x_{11}, x_{12}, x_{13}, x_{14}, x_{22}, x_{23}, x_{24},x_{33}, x_{34}, x_{44}, s_1, s_2, s_3, s_4 &\geq 0 \\
		x_{21}, x_{31}, x_{32}, x_{41}, x_{42}, x_{43} &= 0
	\end{flalign*}
	
	\part % Q1-C
	\begin{align*} % \xRightarrow[below]{above}
		&\begin{bmatrix}
			2 & 3 & 4 & 3 \\
			4 & 5 & 1 & 2 \\
			1 & 1.5 & 2 & 3/2 \\
		\end{bmatrix}
		\xRightarrow[R1=R1/2]{R3=R3-2R1}
		\begin{bmatrix}
			1 & 1.5 & 2 & 1.5 \\
			4 & 5 & 1 & 2 \\
			0 & 0 & 0 & 0 \\
		\end{bmatrix} 
		\xRightarrow[R1=R1+1.5R2]{R2=R2-4R1} \\
		&\begin{bmatrix}
			1 & 0 & -8.5 & -4.5 \\
			0 & -1 & -7 & -4 \\
			0 & 0 & 0 & 0 \\
		\end{bmatrix}
	\xRightarrow[]{R2=-R2}
	\begin{bmatrix}
		1 & 0 & -8.5 & -4.5 \\
		0 & 1 & 7 & 4 \\
		0 & 0 & 0 & 0 \\
	\end{bmatrix}
	\end{align*}
	
	\textbf{Summary: } The rank of the provided system of equations is 2. As a result, $x_3$ is a free-variable, and there are infinitely many solutions.
	
	\part % Q1-D
	Contribution to $z$: $x_1 = 10/2, x_2=3/1, x_3=12/5$ \\
	Since $x_1$ is the largest, we will maximize it.
	
	\textbf{Summary: } The optimal arrangement will be 
	$x_1=10, x_2=0, x_3=0$ for a maximum $z=100$.
	
\end{parts}
\end{solution}
%\end{ Question 1}

%%%%%%%%%%%%%%%%%%%%%%%%%%%
% 	\begin{ Question 2}	  %
%%%%%%%%%%%%%%%%%%%%%%%%%%%
\question
\begin{parts}
	\part Show that for $a \geq$ constraint $i$ in a maximization problem 
	that $y_i = −(\bar{c}_{BV} B^{−1}\bar{a}_{n+i} )$ 
	
	\part State the dual of the following linear program.
	\begin{flalign*}
		\text{Min } z=3x_1 +7x_2 −15x_3 +43x_4 & & \\
		\text{s.t.} \hspace{2.5em} 
		10x_1 +6x_2 +4x_3 +13x_4 &\leq 100 & \\
		−2x_1 + 3x_2 −5x_3 \hspace{7ex}\, &\geq 200 & \\
		12x_1 −3x_2 +2x_3 +22x_4 &\geq 225 & \\
		x_1,x_3,x_4 \geq 0 x_2 \text{urs} & & \\ 
	\end{flalign*}
	
	\part Given the following final tableau for a maximization problem with all less than or equal to constraints, state the original model. \\
	\begin{tabular}{lllllllll}
		$z$ & $x_1$ & $x_2$ & $x_3$ & $x_4$ & $x_5$ & RHS & BV    & Ratio \\ \hline
		1   & 0     & 0     & 0     & 3/2   & 1     & 36  & Z     &       \\
		0   & 0     & 0     & 1     & 1/3   & -1/3  & 2   & $x_3$ &       \\
		0   & 0     & 1     & 0     & 1/2   & 0     & 6   & $x_2$ &       \\
		0   & 1     & 0     & 0     & -1/3  & 1/3   & 2   & $x_1$ &      
	\end{tabular}
	
\end{parts}

\begin{solution}
	\begin{parts}
		\part % Q2-A
		
		
		\part % Q2-B
		\textbf{Summary: } The dual of the provided system is:
		\begin{flalign*}
			\text{Max } z = -100y_1 +200y_2 +225y_3 & & \\
			\text{s.t. }\hspace{2.5em}
			-10y_1 -\quad\, 2y_2 +\ 12y_3 &\leq 3 & \\
			-6y_1 +\quad\ 3y_2 -\ \ 3y_3 &\leq 7 & \\
			-4y_1 -\quad\ 5y_2 +\ \ 2y_3 &\leq -15 & \\
			-13y_1 +\quad \qquad + \ 22y_3 &\leq 43 &
		\end{flalign*}
		
		\part % Q2-C
		\noindent \\
		Exit $x_1$, Re-add $x_5$ \\
		R3 x 3 \\
		R1 + 1/3R3 \\
		R0 - R3 \\
		\begin{tabular}{lllllllll}
			$z$ & $x_1$ & $x_2$ & $x_3$ & $x_4$ & $x_5$ & RHS & BV    & Ratio \\ \hline
			1   & -3    & 0     & 0     & 5/2   & 0     & 30  & Z     &       \\
			0   & 1     & 0     & 1     & 0     & 0     & 4   & $x_3$ &       \\
			0   & 0     & 1     & 0     & 1/2   & 0     & 6   & $x_2$ &       \\
			0   & 3     & 0     & 0     & -1    & 1     & 6   & $x_5$ &      
		\end{tabular} \\
		Exit $x_2$, through $x_4$ \\
		R2 x 2 \\
		R3 + R2 \\
		R0 - 5/2R2 \\
		\begin{tabular}{lllllllll}
			$z$ & $x_1$ & $x_2$ & $x_3$ & $x_4$ & $x_5$ & RHS & BV    & Ratio \\ \hline
			1   & -3    & -5    & 0     & 0     & 0     & 0   & Z     &       \\
			0   & 1     & 0     & 1     & 0     & 0     & 4   & $x_3$ &       \\
			0   & 0     & 2     & 0     & 1     & 0     & 12  & $x_4$ &       \\
			0   & 3     & 2     & 0     & 0     & 1     & 18  & $x_5$ &      
		\end{tabular} \bigskip \\
		\textbf{Summary: } The original model was such that the least common multiple of the coefficients would have have appeared as follows:
		\begin{flalign*}
			\text{Max } z = 3x_1 + 5x_2 & & \\
			\text{s.t.}\hspace{3em}
			x_1 \hspace{6ex} \, &\leq 4 & \\
			2x_2 & \leq 12 & \\
			3x_1 + 2x_2 & \leq 18 & \\
		\end{flalign*}
	\end{parts}
\end{solution}
%\end{ Question 2}

%%%%%%%%%%%%%%%%%%%%%%%%%%%
% 	\begin{ Question 3}	  %
%%%%%%%%%%%%%%%%%%%%%%%%%%%
\question
\begin{parts}
	\part % Q3-A
	\part % Q3-B
	\part % Q3-C
	\part % Q3-D
	\part % Q3-E
	\part % Q3-F
	\part % Q3-G
\end{parts}

\begin{solution}
	\begin{parts}
		\part % Q3-A
		\part % Q3-B
		\part % Q3-C
		\part % Q3-D
		\part % Q3-E
		\part % Q3-F
		\part % Q3-G
	\end{parts}
\end{solution}
%\end{ Question 3}

%%%%%%%%%%%%%%%%%%%%%%%%%%%
% 	\begin{ Question 4}	  %
%%%%%%%%%%%%%%%%%%%%%%%%%%%
\question
\begin{parts}
	\part 
	\part
	\begin{subparts}
		\subpart
		\subpart
	\end{subparts}
\end{parts}

\begin{solution}
	\begin{parts}
		\part % Q4-A
		\part % Q4-B
		\begin{subparts}
			\subpart % Q4-B-1
			\subpart % Q4-B-2
		\end{subparts}
	\end{parts}
\end{solution}
%\end{ Question 4}

%%%%%%%%%%%%%%%%%%%%%%%%%%%
% 	\begin{ Question 5}	  %
%%%%%%%%%%%%%%%%%%%%%%%%%%%
\question
\begin{parts}
	\part % 5-A
	\begin{subparts}
		\subpart % 5-A-i
		\subpart % 5-A-ii
		\subpart % 5-A-iii
	\end{subparts}
	\part % 5-B 
\end{parts}

\begin{solution}
	\begin{parts}
		\part % 5-A
		\begin{subparts}
			\subpart % 5-A-i
			\subpart % 5-A-ii
			\subpart % 5-A-iii
		\end{subparts}
		\part % 5-B
	\end{parts}
\end{solution}
%\end{ Question 5}

\end{questions}
\end{document}