\documentclass[answers]{exam}
\usepackage[english]{babel}
\usepackage[utf8x]{inputenc}
\usepackage{amsmath,amssymb,amsthm}

\begin{document}

\noindent{\large OPER 561 - Discrete-Event Simulation - %
	Lab 2 }
\hspace{\fill} {\large B. Hosley}
\bigskip

\begin{questions}

%%%%%%%%%%%%%%%%%%%%%%%%%%%
%	\begin{ Question 1}	  %
%%%%%%%%%%%%%%%%%%%%%%%%%%%
\question 
Record the mean computer inspection utilization and 95\% confidence interval half-widths for each scenario. Do any of these values surprise you?
\begin{solution}
	
	
	
	
\end{solution}
%\end{ Question 1}

%%%%%%%%%%%%%%%%%%%%%%%%%%%
%	\begin{ Question 2}	  %
%%%%%%%%%%%%%%%%%%%%%%%%%%%
\question 
Which scenario is the best? How do you define best?
\begin{solution}
	If minimal time in system is viable as a benchmark for success, then scenario D performed the best.
\end{solution}
%\end{ Question 2}

%%%%%%%%%%%%%%%%%%%%%%%%%%%
%	\begin{ Question 3}	  %
%%%%%%%%%%%%%%%%%%%%%%%%%%%
\question 
Suppose our goal is to minimize total time in system for deploying airmen. Which scenario do you recommend?
\begin{solution}
	Echoing the answer from question 2.
\end{solution}
%\end{ Question 3}

%%%%%%%%%%%%%%%%%%%%%%%%%%%
%	\begin{ Question 4}	  %
%%%%%%%%%%%%%%%%%%%%%%%%%%%
\question 
Suppose our goal is the same, but we are constrained by money. Make reasonable assumptions about the relative costs associated with computer and human inspections, then recommend a scenario. Be prepared to justify your recommendation and assumptions.
\begin{solution}
	Timescale and current resources will make the determination on this.
	Assuming that the machines are already available and fully operational, 
	then diverting all tasks that can be automated will save the most money.
\end{solution}
%\end{ Question 4}

\end{questions}
\end{document}