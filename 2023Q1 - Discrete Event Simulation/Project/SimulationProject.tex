%\documentclass[12pt]{journal}
\documentclass[sigplan,screen]{acmart}
\usepackage[english]{babel}
\usepackage[utf8x]{inputenc}
%\usepackage{amsmath,amssymb,amsthm}
\usepackage{cite}
\bibliographystyle{IEEE}
%\usepackage{enumerate}
%\usepackage{graphicx}


\begin{document}
\title{Title}
%\author{Brandon Hosley,Student,Air Force Institute of Technology}
\author{Brandon Hosley}


	\begin{abstract}
		The purpose of this paper is to demonstrate competency in the learning 
		objectives of OPER 561-Discrete Event Simulation.
		Simultaneously, it will be used to practice writing in a format
		common within the field.
		Lastly it will be used to explore a topic that has been interesting
		to the author for some time, but had not yet been feasible to explore.
		The subject of the simulations in this paper will be the well-studied
		Schelling models. The simulations performed will not necessarily be
		novel, but are intended to take advantage of improvements in computation
		to explore some of Schelling original analysis in a simulated, experimental manner.
	\end{abstract}
	
	% Note that keywords are not normally used for peerreview papers.

% make the title area
\maketitle


% TODO

Modify Schelling Game Set-up

Build Output methods

Analysis

Write report

Presentation

\section{Introduction}

 1971, while working for the RAND Corporation, Thomas C. Schelling began investigating
the manner in which segregation occurs in the absence of 
%\cite{Schelling_1971}

\subsection{Background and Motivation}

\section{Related Work}

\section{System Models and Problem Formulation}
\subsection{Optimization}

Original model was too slow, clocking an average of 18.1 seconds.

replacing indexed iterations with enumerations and 
removing redundant addition in the check happy routine shaved this to 6.8

Removing gridsize+1 had no improvement, likely that the interpreter handled that.

Changed the way that new zones are selected for movement of unhappy agents got to 5.8 seconds
discovered that new agents moved randomly to zones empty in the first grid, but there was no accounting
of where agents moved this time, as such multiple agents could move to the same zone

removed random int just before figure (purpose uknown possibly to advance the rng, but only once...)
removed the accounting on the old grid for moving agents


did the same for placing agents, shortened to 3.8 sec


\section{Simulation Results and Discussion}

different numbers of teams

different distros, two teams



\section{Conclusion}









\bibliography{./bibliography}

%\begin{IEEEbiography}[{\includegraphics[width=1in,height=1.25in,clip,keepaspectratio]{mshell}}]{Michael Shell}
%\end{IEEEbiography}

\end{document}