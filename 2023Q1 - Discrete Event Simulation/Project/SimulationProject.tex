%\documentclass[12pt]{journal}
\documentclass[sigplan,nonacm]{acmart}
\usepackage[english]{babel}
\usepackage[utf8x]{inputenc}

%\usepackage{amsmath,amssymb,amsthm}
\usepackage{tcolorbox}
\newtcbox{\inlinecode}{on line, boxrule=0pt, boxsep=0pt, top=2pt, left=2pt, bottom=2pt, right=2pt, colback=gray!15, colframe=white, fontupper={\ttfamily \footnotesize}}

%\usepackage{enumerate}
\usepackage{graphicx}
\usepackage{booktabs,tabularx}
\usepackage{caption}


\begin{document}
\title{Revisiting Schelling's Games}

\author{Brandon Hosley}
\orcid{0000-0002-2152-8192}
%\authornotemark[1]
%\authornote{text}
\email{brandon.hosley.1@us.af.mil}
\affiliation{%
	\institution{Air Force Institute of Technology}
	\streetaddress{1751 11th St.}
	\city{Wright-Patterson Air Force Base}
	\state{Ohio}
	\country{USA}
	\postcode{45433}
}


\begin{abstract}
	The purpose of this paper is to demonstrate competency in the learning 
	objectives of OPER 561-Discrete Event Simulation.
	Simultaneously, it will be used to practice writing in a format
	common within the field.
	Lastly it will be used to explore a topic that has been interesting
	to the author for some time, but had not yet been feasible to explore.
	The subject of the simulations in this paper will be the well-studied
	Schelling models. The simulations performed will not necessarily be
	novel, but are intended to take advantage of improvements in computation
	to explore some of Schelling original analysis on a larger scale.
\end{abstract}

% Note that keywords are not normally used for peerreview papers.

\received{10 March 2023}

% make the title area
\maketitle

\section{Introduction}

In this paper we will revisit the well-known Schelling Models\cite{Schelling1971}.
As an exercise, first we will explore the available tools to emulate his original
simulation but at a larger scale.
Afterward we will simulate the conditions that he investigated analytically.
Lastly we will extend the investigation to encompass scenarios that he had not
written about.

\subsection{Background and Motivation}

In 1971, while working for the RAND Corporation, Thomas C. Schelling began investigating
the manner in which segregation occurs in the absence of external enforcement.
To this end, Schelling proposed that groups of agents with similar objectives in their relations
with other agents would arrange themselves in a manner that resembled a directing hand.
While acknowledging that there are a multitude of contributing and compounding factors that can amplify these effects, 
for simplicity's sake Schelling chose to abstract the net prejudices, biases, and presences into a simple number.


In direct relevance to the civil rights movement in full swing at the time of his writing, Schelling chose to 
tie the intuition of his model to one of the most obvious traits that could classify differences between agents,
which was race. He did write about situations in which age could have a similar effect; as when groups of young people
begin spending a lot of time in certain areas displacing older persons that may find the teenage presence objectionable.

Schelling concluded his paper describing what he called tipping points, wherein fragile equilibrium breaks and send a local
demographic towards an extreme or even homogeneous outcome.

\section{Related Work}

Schelling's model offers a lot of opportunity to extend to a large variety of different conditions
and is very simple to understand, as such it has been the subject of numerous studies and experiments.
Here we will examine just a few of them, first several with interesting conditions that will not be considered
followed by several that have conditions similar to those that we wish to implement for this paper.


Bullinger et al. \cite{Bullinger2021} describe an analytical method to seek a pareto optimal
condition within a single assignment on arbitrary networks. 
This can become a fairly difficult task very rapidly,
if a feasible solution exists at all; but most importantly for this paper,
it does not make for compelling simulation.

Elkind et al.\cite{Elkind2021} utilize an approach in which agents occupy nodes on a graph rather than sectors on a grid.
In this way, many nodes have oppotunities to be less volatile as they have fewer neighbors, whereas
some nodes may have more than the eight connections associated with grid spacing.
While this dimension may prove interesting in that paper they never examine a node with greater than five connections.

Kreisel et al. \cite{Kreisel2022} also examined methods of analytically calculating equilibrium.
In their work they focus on the computational complexity of evaluating equilibrium,
and stick to the agents-on-graph type of models.

Gretha et al.\cite{Gretha2018} applied the Schelling model to social networks. The observed clustering similar,
though not as well defined delineations as most models. This type of model offers the opportunity to generate
far more connections than most implementations, and greatly changes the manner in which these connections 
affect the utility of the agents.

Bilo et al.\cite{Bilo2022} added a cost associated with moving, that scaled with distance.
While this did not change the probability of achieving equilibrium it did increase the amount
of time for the system to find equilibrium.
Additionally, it had the effect of increasing the probability of reaching an equilibrium with
smaller neighborhoods, which was not unusual in other implementations, just less common.

Kanellopoulos et al.\cite{Kanellopoulos2021} implement a model in which there are more than the typical two classes of agent.
Additionally, they differing levels of tolerance between different groups. T
his work is highly related to Liu et al.\cite{Liu2019} who also add a larger number of classifications of agents,
however, rather than using different levels of tolerance defined specifically between each pair of classes,
they provide multiple traits for each class with a utility associated positively with agents that share
like traits. They observed that agents still formed neighborhoods with their own classification and were far
more likely to share borders with classes of agents that shared traits.
They also examined changes in density and increasing the levels of tolerance; where increased tolerance
and decreased system density both (unsurprisingly) had the effect of dissolving neighborhoods such that
the graphic representation resembled a multicolored Perlin noise.


\section{System Models and Problem Formulation}
\subsection{Optimization}

The simulation model was based on a previous implementation provided by Dr. Lance Champagne.
However, the original model was not optimized.
The run time for a \(100\times100\) generating \(50\) frames executed in a visual studio code python notebook run
on a 2021 M1 macbook pro was consistently \(18.1\pm1\) seconds.
While this is not necessarily problematic for running this simulation a small number of times,
we determined that the run time would only get worse as we adding complexity by modeling more complex behaviors,
before long, this would have a profound negative effect for iterative development and gathering data from a large number of runs
and this was determined to be unacceptable, and thus the base model would need to be optimized.

First, we changed all of the functions using indexed iterations by passing an iterative object instead,
we changed the manner in which each agent checks its neighbors.
Rather than calculating the combination of the agent's coordinates by subtracting one, adding zero, and adding one,
we calculated a range based on the agent's coordinate once and iterated over that instead.
Finally, there was an extra checking happiness routine prior to passing all the functions to matplotlib's
animation generator, determining this to be redundant we removed it.
These actions reduced the run time to \(6.8\pm1\) seconds.
This was a good improvement; but inspired by the Icarus statue in the courtyard of the AFIT campus we decided to continue.


We changed all of the occurrences in which the \inlinecode{gridsize+1} was called with a variable calculated once.
But this did not improve the time, likely because that simple of a call was likely already simplified by the python interpreter.
Next, focus was given to the manner in which agent chose their next destination, first, agents were randomly picking 
a set of coordinates and checking the prior grid to see if that sector was occupied and then moving there.
The problem was that there was no accounting on the original grid so multiple agents could choose to move to the same empty zone.
We adding an update to the old grid so that this behavior wouldn't occur. Additionally, at the movement
step we chose to make a list of all unoccupied zones at the beginning, then unhappy agents would randomly pick an opening from
this list rather than using an acceptance-rejection technique.
The chosen coordinates were then removed from that choice list so that there would not be multiple agents moving into the same zone.
These changes reduced the run time to \(5.8\pm1\) seconds.

Next, we discovered a call for a random integer that did not seem to serve a purpose.
While it may have been used to advance the random number generator to increase the entropy of the system,
the python random package has a cycle large enough that this should have a negligible benefit.
We also removed the system for updating the old grid to account for agents moving,
since choosing from the openings list had this effect anyway.
We implemented the same accounting system for placing the agents the first time,
wherein at the first iteration the openings list comprised the coordinates of for the entire grid.
While the initial usage of this is relatively inefficient, as the grid gets larger and more agents must be placed,
this drastically improves the time it takes to place the agents by eliminating one more acceptance-rejection scenario.
All of this brought the baseline scenario's run time to \(3.8\pm1\) seconds.

\subsection{Conditions to be Examined}

Schelling's initial model had an equal homophily for all agents across both demographics.
This is also the case for the baseline model, so we will examine this condition as a 
control by which to compare the other conditions.

We are interested in examining effects of asymmetric utility, differences in population density,
and larger numbers of group types. Additionally, we will investigate effects of changes in
the size of what agents consider their neighbors and add integrationist utility.

The preceding are all scenarios that Schelling modeled physically, we will be able to compare
results to those originally described; what Schelling was unable to do was model what he described
as tolerance schedules or distributions of differences in tolerance among the agents.
He did provide an analytic claim to what behavior would occur, 
we wish to examine how this behavior manifests in simulation.

One aspect that we will depart with from Schelling's conceptualization is the openness of the system,
in his analysis he focused on a limited area with access to an arbitrary external population
and where the entire area is treated as a singular neighborhood. Without an open system with 
a smaller segment of interest it may tipping will not likely happen in the manner that he described.
However, we wish to investigate how the agents of his analytical models will behave in a closed system.

Finally, we will extend our model to include some of the traits shown by other researchers,
with emphasis on different numbers of groups, and nonstandard utility functions.

\subsection{Assumptions}

Assumptions used in this set of experiments are the same as those used in most implementations.
Specifically, we eschew the cost of moving, we assume that the agents play the game with perfect information,
that geographic limitations do not exist beyond occupation, and that all neighborhoods are arranged ortholinearly.

\subsection{Building the Model}

The process of actually building the model occurred iteratively.
Each additional feature or condition added ot the scenario added some complexity to building the model.
The intent being that once the model is build that the one would be able to simulate all of the above.

The easiest and first add was to abstract the the number of agents.
This led to the restructuring of the initiation template, now a dictionary will be used to pass
features into the model.
Initiation consists of iterating over the dictionary when building all of the agents.
Within this data structure we add a population definition to let us control the population of each type of agent.
This definition is the way for a user to control the types of agents, the populations of each type, 
and by extension can control the population density.

To vary the levels of tolerance each agent type was assigned a tolerance value within the dictionary,
this allowed different agents to express different levels of tolerance.
The remaining conditions involved more complex implementations of tolerance.
In order to achieve this it was decided to replace the tolerance trait with a utility function instead.
The change to utility function allowed an easier definition of differing tolerance levels of other out-groups.

The last set of conditions involved agents within groups having different levels of tolerance than their peers.
To achieve this, anonymous function casting was leveraged so that instantiation of each agent
unwrapped the first function call which would assign a variable parameter to the utility function assigned to the agent.

\subsection{Experiment Design}

The scenarios we simulate will be a 100 by 100 grid, with 6000 agents, except in the cases where population density
is decreased. The agent will consider the eight squares immediately surrounding it's position when evaluating its happiness.
Unless otherwise stated, the default utility function will be 0.3; an agent is happy when the ratio of like agents
to adjacent zones is 0.3.

For each of the scenarios examined we will run 10 simulations with 100 rounds of movement and collect the metric
of aggregate dissatisfaction among agents, these parameters were chosen because they had reasonable run time
and were stable in the kernel on the hardware that we were using. 
Running additional simulations frequently caused the kernel to seize.
It was decided that these number should be sufficient for an exploratory analysis.

We will then to one additional run to output an animated visualization of that scenario.
Schelling put particular emphasis on the qualitative patterning of the visual representation of his models,
we will also primarily use qualitative to seek generalize insight into the scenarios, with the goal of 
determining which conditions suggest more in depth analysis.
In an effort to add some objective measurements as well, we will leverage that aggregate dissatisfaction over time
to determine if the system is able to reach a Pareto efficiency, where the value of the convergence represents
continuous movement, and stable equilibrium will occur if dissatisfaction converges to zero.

Dissatisfaction curves will also provide a tool with which to compare scenarios in a quantitative manner.

\section{Simulation Results and Discussion}

Enough testing was performed that it would be somewhat tedious for the reader to place every result here,
instead we will highlight results that produced interesting or unexpected results.
First we will examine the qualitative results to examine the emergent behavior and patterning of the systems,
then we will examine the time series utilities of the scenarios to compare the dynamic behavior of the systems as they progress.

All of the images presented are after 100 move cycles, in some cases the systems have achieved equilibrium or pareto efficiency,
but some still have a trajectory of improvement at that time the simulation concluded and are depicted in that state.

\subsection{Qualitative Results}

The baseline (figure \ref{fig:baseline}) followed the same specifications as the first model Schelling proposes and exhibits the same type of results,
and converged relatively quickly.

\begin{figure}
	\centering
	\includegraphics[width=0.8\linewidth]{images/Baseline}
	\caption{Baseline}
	\label{fig:baseline}
\end{figure}

\subsubsection{Single Variable Analysis}

Asymmetric populations (figure \ref{fig:largemajority}) produced patterning similar to baseline for the majority, 
with the minority group forming smaller, discontinuous communities. 
The same occurred when the population density was 20\% lower, and likely due to the random movement of the agents,
the rate at which the system converged on equilibrium was very similar.

\begin{figure}
	\centering
	\includegraphics[width=0.8\linewidth]{images/LargeMajority}
	\caption{}
	\label{fig:largemajority}
\end{figure}

In the case of asymmetric utility (figure \ref{fig:lessasymmetricutility}) the group with a lower homophilic utility
would achieve happiness earlier and then cease moving. For the agents with a higher threshold, many would continue searching for a long time,
but when groups did form they would become large and with less empty space within their 'territory'. 
If the gap of asymmetry is made larger, the more tolerant agents settle in earlier and in smaller, sparser groups with the secondary effect
of preventing the less tolerant group from settling in an area, with most stay nomadic.
Lower population density had a similar effect to reduced asymmetry in the utility functions.

\begin{figure}
	\centering
	\includegraphics[width=0.8\linewidth]{images/LessAsymmetricUtility}
	\caption{}
	\label{fig:lessasymmetricutility}
\end{figure}

Schelling's original observation of equilibrium always being achievable when the utility was 0.3 or lower continued to hold 
with three (figure \ref{fig:threegroups}) and six (figure \ref{fig:sixgroups}) groups. 
It appears that the increasing number of groups increases the convergence time, at least under the random movement
scenario by way of increasing the entropy in the system.
We suspect the convergence rate would be much closer regardless of the number of groups when the agents have intentional movement.

\begin{figure}
	\centering
	\includegraphics[width=0.8\linewidth]{images/ThreeGroups}
	\caption{}
	\label{fig:threegroups}
\end{figure}
\begin{figure}
	\centering
	\includegraphics[width=0.8\linewidth]{images/SixGroups}
	\caption{}
	\label{fig:sixgroups}
\end{figure}

When Schelling examined agents with integration desire within their utility, 
e.g. having a minimum of the other group as neighbors in order to achieve happiness.,
he observed that the agents would form lines where his 'in-demand' minority group would be thinly laid out to maximize the 
surface area that the majority party could share with them.

This is a pattern that requires intentional movement, however, 
even under random movements a different type of linear pattern does emerge as seen in figure \ref{fig:integrationist}.

If, however, the system is made sparse a pattern similar to what Schelling described does form.
The necessary factor to observe this behavior is time, unlike the other images in this section 
figure \ref{fig:integrationistsparse} was taken after 200 rounds of moves. 
This is further discussed in the next section.

\begin{figure}
	\centering
	\includegraphics[width=0.8\linewidth]{images/Integrationist}
	\caption{}
	\label{fig:integrationist}
\end{figure}
\begin{figure}
	\centering
	\includegraphics[width=0.8\linewidth]{images/IntegrationistSparse}
	\caption{}
	\label{fig:integrationistsparse}
\end{figure}


Increasing the size of what is considered a neighborhood to \(5\times5\) behaves in a manner very similar to the baseline,
but with each group's 'territory' being thicker, seen in figure \ref{fig:fivebyneighboorhoods}, a change that continues
as the neighborhood size increases; figure \ref{fig:sevenbyneighboorhoods} depicts \(7\times7\) neighborhoods.

\begin{figure}
	\centering
	\includegraphics[width=0.8\linewidth]{images/FiveByNeighboorhoods}
	\caption{}
	\label{fig:fivebyneighboorhoods}
\end{figure}
\begin{figure}
	\centering
	\includegraphics[width=0.8\linewidth]{images/SevenByNeighboorhoods}
	\caption{}
	\label{fig:sevenbyneighboorhoods}
\end{figure}

\subsubsection{Distribution Utility Functions}

In this section we examine the effects of assigning utility functions by utilizing random variate generation.
Each variate used is provided by the \inlinecode{scipy} statistics library.
Where a normal distribution was desired we instead used a binomial distribution with a \(p=0.5\),
this method served the need that the distribution could be truncated so that outliers would not occur
outside of the possible range, and had the added benefit of providing an integer, which could be compared 
to the neighbor count without needing any mathematical operations to determine ratios.

A symmetric binomial distribution with a mean of 3, and a range of 0 to 6 (inclusive), resulted in an equilibrium indistinguishable from the baseline.
Suggesting that the aggregate behavior is the same between a uniform utility ratio of a certain value and a distribution 
with an equivalent mean.

When the mean value was shifted to 4 with range of 0 to 8, results in a slow convergence with large, filled territories forming,
we believe that agents with lower thresholds settle in a space and members with higher homophilic needs coalescing around
them, in a manner similar to crystal formation. Figure \ref{fig:binomsym4} shows this, of note, at this point the system 
still appeared to be improving. Additionally, it was observed that this condition was highly probabilistic, where some replications did not
produce any notable patterns, resembling noise for the duration.
In at least one replication, several territories formed, then grew, then shrunk, and eventually dissolved. 

\begin{figure}
	\centering
	\includegraphics[width=0.8\linewidth]{images/BinomSym4}
	\caption{}
	\label{fig:binomsym4}
\end{figure}

A uniform distribution formed some clusters, but further growth was blocked by agents with very low or non-existent needs 
that were immediately satisfied and never moved. Green agents of this type can be seen on the periphery of the yellow 
clusters in figure \ref{fig:uniform0to6}.

\begin{figure}
	\centering
	\includegraphics[width=0.8\linewidth]{images/uniform0to6}
	\caption{}
	\label{fig:uniform0to6}
\end{figure}

With a bimodal distribution, agents close to the lower mode settle early, and agents possessing the utility functions
closer to the higher mode moved into the areas filling in gaps or occupying edges. 
The end result is similar to baseline but with slightly thinner clustering.

\subsection{Quantitative Results}



\subsubsection{Consistency of Behavior}

We are interested in determining which conditions results in consistent emergent behavior
and which exhibit very different behavior depending on the initial state of the system.
Similarly, we wish to consider trend lines that are translations of other to be considered similar,
For example, two constant lines with different intersections would be considered to have the same behavior.

To accomplish this we have decided to take the arithmetic mean of all of the pairwise 
Pearson correlation coefficient, represented as \(\rho\) in

\begin{equation}
	\frac{\sum_{i,j}\rho(i,j)}{\frac 1 2 (n^2-n)}.
\end{equation}

We also considered using Kullback-Leibler divergence and pairwise mutual information calculations,
but discovered that the information returned was not sufficiently better than the one we used to 
justify the added computational complexity.

The results of the average cross-correlation can be seen in table \ref{tab:corr}.

\begin{table*}
	\centering
	\caption{}
	\label{tab:corr}
	\begin{tabular}{c|l}
		\toprule
		Scenario & Mean Sample Cross Correlation \\
		\midrule
		Baseline &   0.9993653593508636  \\
		Large Majority &   0.9981457559336203  \\
		Large Majority and Low Density System &   0.9973101095131449  \\
		Asymmetric Utility (0.5 and 0.2) &   0.9793917826311105  \\
		Asymmetric Utility (0.5 and 0.3) &   0.9959356131499515  \\
		Three Groups &   0.9998012895522678  \\
		Six Groups &   0.9995279887239704  \\
		Integrationist &   0.9041819060924364  \\
		Integrationist and Low Density System &   0.9552039638356898  \\
		Integrationist but okay with empty spaces & 0.8227803567267047 \\
		One Integrationist, One Baseline  &   0.9987941147621255  \\
		5x5 Neighboorhoods &   0.9992804607818385  \\
		7x7 Neighboorhoods &   0.9987265936770766  \\
		Symmetric Binomial Utility Distribution \((\mu=3)\) &   0.9996786194708911  \\
		Symmetric Binomial Utility Distribution \((\mu=4)\) &   0.9984692796627254  \\
		Uniform Utility Distribution (0 to 6) &   0.9924352614933506  \\		
		Bimodal Distribution with modes at 3 and 6 &   0.8929659057777041  \\
		\bottomrule
	\end{tabular}
\end{table*}

While the results from this analysis validated some anecdotal observations,
it provided some surprised by contradicting some anecdotal observations from some
runs in which the behavior appeared to be very different between different replications.

In particular, the integrationist utility with a sparse system produced visually very different behavior during early testing. 
While it is one of the lower results in table \ref{tab:corr}; \(\rho=0.90\) is notably higher than we would had predicted.

\subsubsection{Contradictory Behavior}

When observing the behavior over time we have chosen to examine the total utility, specifically,
we measure the unhappiness, and convergence at 0 represents equilibrium in which every agent is happy.
Figure \ref{fig:baseline-graph} shows the baseline behavior for reference.

\begin{figure}
	\centering
	\includegraphics[width=\linewidth]{"images/Baseline Graph"}
	\caption{}
	\label{fig:baseline-graph}
\end{figure}


Under the integrationist scenario we obtained results that were quite surprising in that their approximate
equilibrium was a higher unhappiness than the starting state;
where the individual agents seeking to improve their own utility reduce the overall utility.
This trend can be seen in figure \ref{fig:integrationist-graph}.

\begin{figure}
	\centering
	\includegraphics[width=\linewidth]{"images/integrationist graph"}
	\caption{}
	\label{fig:integrationist-graph}
\end{figure}

In most cases, the systems reach their convergent state within the first 50 moves, however, some systems took much longer.
In particular the six group scenario did eventually reach a full happiness equilibrium, but it consistently takes
close to 200 rounds (figure \ref{fig:long-run-6}).

\begin{figure}
	\centering
	\includegraphics[width=\linewidth]{"images/Long Run 6"}
	\caption{}
	\label{fig:long-run-6}
\end{figure}

The longest run was the sparse system with integrationist sentiments, the complex pattern from figure \ref{fig:integrationistsparse} 


\section{Conclusion and Future Work}

This project proved to be a optimization challenge and a great exercise in agent based modeling.

Additionally, it proved to be an excellent opportunity to sate the curiosity we had regarding how Schelling's models behave
under the different conditions described in the 1971 paper and with regard to distribution based 'tolerance schedules'.

\subsection{Lessons Learned}

We discovered that the open space accounting system that we introduced had a weakness in that it was possible 
for the number of unhappy agents to exceed the number of open spaces,
in this scenario agents searching for new places to move would not be able to locate the recently vacated cells and would
attempt to get a random coordinate from an empty list, throwing an error. Interestingly, this behavior did not manifest until testing the three groups condition.

We observed that, in general, higher demand had the effect of small numbers of agents clustering, 
and forming insular bubbles in which even like agent had a difficult time attaching, further complicated by the movement of the other agents.

\subsection{Conclusion}

In our exploratory analysis there were not enough replications performed to definitively make conclusions, however, from our results
we have observed that Schelling's original models hold at a larger scale; that his predictions, when he made them, were accurate;
and that many of the scenarios that he described but did not experiment with himself, behaved in the manner often resembling
the baseline behavior but occasionally with a nonzero convergence.

It appears that symmetric distributions with a mean at the 0.3 ratio utility behaved nearly identically to baseline if
the distribution had a central tendency, but not when uniform. 
What range of kurtosis this holds for may be an interesting question for future work.

\subsection{Future Work}

There are several other metrics that may provide some insight into the system state, in particular we believe that 
counts of contiguous groups and a metric evaluating the irregularity of borders between clusters may provide some interesting insight
and better quantify the qualitative observations made here.



\bibliographystyle{ACM-Reference-Format}
\bibliography{SimulationProject}

\end{document}