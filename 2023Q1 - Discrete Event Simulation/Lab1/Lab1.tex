\documentclass[answers]{exam}
\usepackage[english]{babel}
\usepackage[utf8x]{inputenc}
\usepackage{amsmath,amssymb,amsthm}

\begin{document}

\noindent{\large OPER 561 - Discrete-Event Simulation - %
	Lab 1 }
\hspace{\fill} {\large B. Hosley}
\bigskip

\begin{questions}

%%%%%%%%%%%%%%%%%%%%%%%%%%%
%	\begin{ Question 1}	  %
%%%%%%%%%%%%%%%%%%%%%%%%%%%
\question 
Record the mean queue length for inspection and its 95\% confidence interval half-width. \\
\hspace{3em} \textit{Half-widths can be found in the Pivot Grid or SMORE plot.}
\begin{solution}
	S
\end{solution}
%\end{ Question 1}

%%%%%%%%%%%%%%%%%%%%%%%%%%%
%	\begin{ Question 2}	  %
%%%%%%%%%%%%%%%%%%%%%%%%%%%
\question 
In our conceptual model, we left out the \textit{Events}. What are the events for this scenario?
\begin{solution}
	S
\end{solution}
%\end{ Question 2}

%%%%%%%%%%%%%%%%%%%%%%%%%%%
%	\begin{ Question 3}	  %
%%%%%%%%%%%%%%%%%%%%%%%%%%%
\question 
For each measure of interest, is it time-weighted or observational?
\begin{solution}
	S
\end{solution}
%\end{ Question 3}

%%%%%%%%%%%%%%%%%%%%%%%%%%%
%	\begin{ Question 4}	  %
%%%%%%%%%%%%%%%%%%%%%%%%%%%
\question 
Come up with two additional assumptions that don’t disrupt the model you have built.
\begin{solution}
	S
\end{solution}
%\end{ Question 4}

%%%%%%%%%%%%%%%%%%%%%%%%%%%
%	\begin{ Question 5}	  %
%%%%%%%%%%%%%%%%%%%%%%%%%%%
\question 
Suppose we installed a second check-out kiosk. Describe how you would change 
the model.
\begin{solution}
	S
\end{solution}
%\end{ Question 5}

%%%%%%%%%%%%%%%%%%%%%%%%%%%
%	\begin{ Question 6}	  %
%%%%%%%%%%%%%%%%%%%%%%%%%%%
\question 
Simulation is often used to study alternative designs. What is one other change 
you could imagine making to this system?
\begin{solution}
	S
\end{solution}
%\end{ Question 6}

\end{questions}
\end{document}