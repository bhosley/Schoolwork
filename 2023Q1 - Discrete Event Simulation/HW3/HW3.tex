\documentclass[12pt]{amsart}
\usepackage[left=0.5in, right=0.5in, bottom=0.75in, top=0.75in]{geometry}
\usepackage[english]{babel}
\usepackage[utf8x]{inputenc}
\usepackage{amsmath,amssymb,amsthm}
\usepackage{enumerate}
\usepackage{graphicx}
\usepackage{booktabs}

\begin{document}
\raggedbottom

\noindent{\large OPER 561 - Discrete-Event Simulation - %
	Homework X }
\hspace{\fill} {\large B. Hosley}
\bigskip


%%%%%%%%%%%%%%%%%%%%%%%
\setcounter{section}{9}
\setcounter{subsection}{5}
\subsection{} % BCNN: 9.6;
	\textit{Prepare four theoretical normal density functions, 
		all on the same figure, each distribution having mean zero,
		 but let the standard deviations be 1/4, 1/2, 1, and 2.}	
		(ensure you label/indicate which curve is for which standard deviation.)
		
		\bigskip
		\textbf{Solution:}
	
	
\setcounter{subsection}{15}
\subsection{} % BCNN: 9.16
	\textit{Records pertaining to the monthly number of job-related injuries 
		at an underground coal mine were being studied by a federal agency. 
		The values for the past 100 months were as follows:}
	\begin{tabular}{cc}
		\toprule
		\textit{Injuries per Month} & \textit{Frequency of Occurrence} \\
		\midrule
		0 & 35 \\
		1 & 40 \\
		2 & 13 \\
		3 & 6 \\
		4 & 4 \\
		5 & 1 \\
		6 & 1 \\
		\bottomrule
	\end{tabular}
	\begin{enumerate}[(a)]
		\item Apply the chi-square test to these data to test the hypothesis that the underlying distribution is Poisson. Use the level of significance α = 0.05.
		\item Apply the chi-square test to these data to test the hypothesis that the distribution is Poisson with mean 1.0. Again let α = 0.05.
		\item What are the differences between parts (a) and (b), and when might each case arise?
	\end{enumerate}

	\bigskip
	\textbf{Solution:}
	
	
\end{document}