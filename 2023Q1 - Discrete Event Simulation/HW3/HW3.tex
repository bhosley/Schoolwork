\documentclass[12pt]{amsart}
\usepackage[left=0.5in, right=0.5in, bottom=0.75in, top=0.75in]{geometry}
\usepackage[english]{babel}
\usepackage[utf8x]{inputenc}
\usepackage{amsmath,amssymb,amsthm}
\usepackage{enumerate}
\usepackage{graphicx}
\usepackage{booktabs}

\usepackage{multirow}

\begin{document}
\raggedbottom

\noindent{\large OPER 561 - Discrete-Event Simulation - %
	Homework 3 }
\hspace{\fill} {\large B. Hosley}
\bigskip


%%%%%%%%%%%%%%%%%%%%%%%
\setcounter{section}{9}
\setcounter{subsection}{5}
\subsection{} % BCNN: 9.6;
\textit{Prepare four theoretical normal density functions, 
		all on the same figure, each distribution having mean zero, 
		but let the standard deviations be 1/4, 1/2, 1, and 2.}
		(ensure you label/indicate which curve is for which standard deviation.)
		
		\bigskip
		\textbf{Solution:}
	
	\includegraphics[width=0.85\linewidth]{"Screenshot 2023-02-13 at 5.43.00 PM"}\\
	
	\noindent
	Red: \(\sigma=1/4\) \\
	Blue: \(\sigma=1/2\) \\
	Green: \(\sigma=1\) \\
	Orange: \(\sigma=2\) \\
	
\setcounter{subsection}{15}
\subsection{} % BCNN: 9.16
	\textit{Records pertaining to the monthly number of job-related injuries 
		at an underground coal mine were being studied by a federal agency. 
		The values for the past 100 months were as follows:} \\
	\begin{center}
		\begin{tabular}{cc}
			\toprule
			\textit{Injuries per Month} & \textit{Frequency of Occurrence} \\
			\midrule
			0 & 35 \\
			1 & 40 \\
			2 & 13 \\
			3 & 6 \\
			4 & 4 \\
			5 & 1 \\
			6 & 1 \\
			\bottomrule
		\end{tabular}
	\end{center}
	\begin{enumerate}[(a)]
		\item \textit{Apply the chi-square test to these data to test the hypothesis that the underlying distribution is Poisson. Use the level of significance \(\alpha\) = 0.05.}
		\item \textit{Apply the chi-square test to these data to test the hypothesis that the distribution is Poisson with mean 1.0. Again let \(\alpha\) = 0.05.}
		\item \textit{What are the differences between parts (a) and (b), and when might each case arise?}
	\end{enumerate}

	\clearpage
	
	\textbf{Solution:} \\

	\begin{center}
	\begin{tabular}{cccccccc}
		\toprule
	 & Obs & \multicolumn{2}{c}{\(E_1(A)\)} & \(\frac{(O_i-E_i)^2}{E_i}\) & \multicolumn{2}{c}{\(E_1(B)\)} & \(\frac{(O_i-E_i)^2}{E_i}\) \\
	    \midrule
	 0  &  35 &  32.9558 & &  0.12678  &  36.7879 & &  0.08689 \\
	 1  &  40 &  36.5810 & &  0.31954  &  36.7879 & &  0.28045 \\
	 2  &  13 &  20.3024 & &  2.62658  &  18.3939 & &  1.58176 \\
	 3  &  6  &  7.51191 &\multirow{4}*{\(\left.\rule{0cm}{5.5ex}\right\}\)10.1448} &  \multirow{4}*{0.33924}  &  6.13132 &  \multirow{4}*{\(\left.\rule{0cm}{5.5ex}\right\}\)8.0218} &  \multirow{4}*{1.97286} \\
	 4  &  4  &  2.08455 & & &  1.53283 & &  \\
	 5  &  1  &  0.46277 & & &  0.30656 & &  \\
	 6  &  1  &  0.08561 & & &  0.05109 & &  \\
	 \midrule
	Sum & 100 &  99.9842 & &  3.41216  &  99.9916 & &  3.92198 \\
	\bottomrule
	\end{tabular} 
	\end{center} 
	\bigskip
	
	\begin{enumerate}[(a)]
		\item In this case, with a \(\chi^2_0=3.41216\) we fail reject the null hypothesis, 
		and conclude that based on the available data that the underlying distribution may be accurately
		represented by a Poisson distribution. \\ % 9.48773
		\item In this case, with a \(\chi^2_0=3.92198\) we fail reject the null hypothesis.,
		and conclude that based on the available sample a Poisson distribution with a mean of 1
		may accurately represent this data. \\ % 7.81473
		\item The degrees of freedom are different in each case, with the first problem 
		requiring an expected mean derived from the sample data, where the second part 
		provided a suspected mean, thus having one greater degree of freedom.
		The former situation is more likely to represent a situation in which the sample is the only
		dataset and there is a suspected distribution.
		The latter case may represent a situation in which there is previous information,
		such as mean and distribution from previous samples, and we wish to examine another sample to 
		evaluate if it fits or is representative of the previous observations.
	\end{enumerate}

\end{document}