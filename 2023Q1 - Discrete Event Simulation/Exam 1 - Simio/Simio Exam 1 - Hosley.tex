\documentclass[answers]{exam}
\usepackage[english]{babel}
\usepackage[utf8x]{inputenc}
\usepackage{amsmath,amssymb,amsthm}
\usepackage{graphicx}
\usepackage{booktabs}

\begin{document}

\noindent{\large OPER 561 - Discrete-Event Simulation - %
	Simio Exam 1 }
\hspace{\fill} {\large B. Hosley}
\bigskip

\begin{questions}
\unframedsolutions

%%%%%%%%%%%%%%%%%%%%%%%%%%%
%	\begin{ Question 1}	  %
%%%%%%%%%%%%%%%%%%%%%%%%%%%
\question 
(40\%) Working Simio model file saved with experiment results. \\
- There is not only one correct model for this problem. 
You have room for creativity while still accurately modeling the described system.
%\end{ Question 1}

%%%%%%%%%%%%%%%%%%%%%%%%%%%
%	\begin{ Question 2}	  %
%%%%%%%%%%%%%%%%%%%%%%%%%%%
\question 
Typed summary report (no more than 4 pages) containing:
\begin{enumerate}
	\item[2a.] (30\%) Conceptual modeling description:
	\begin{solution}
		\begin{enumerate}
			\item[2ai.] Entities/Components
				Entities needed:
				\begin{enumerate}
					\item Workers: Crew Chief, Maintenance Crew, Service Crew 
					\item Model Entities: Jet, Flight (Parent Object for 4 fighters)
					\item Resources: Runway
				\end{enumerate}
				Components needed:
				\begin{enumerate}
					\item Servers: Mission Prep, Preflight, Takeoff, Sortie, Landing, Servicing, Unscheduled Maintenance
					\item Combiner: To form flight of 4
					\item Separator: To dissolve the flight
				\end{enumerate}
			
			\item[2aii.] Assumptions
				\begin{enumerate}
					\item Travel time is either included in activity distributions or ignored
					\item Workers travel instantaneously
					\item We do not care about pilots.
					\item Any breaks are accounted for within the times of the other tasks
					\item Supplies, tools, fuel, etc. are always readily available in necessary amounts
					\item 10 minutes covers the take-off of all 4 jets, with no variation
					\item Same for landing
				\end{enumerate}
			
			\item[2aiii.] Inputs required \\
				The inputs required to construct this model are the
				statistics provided by the table in the assignment.
				The value of the model is fundamentally reliant
				on the quality of the data (or in this case, the fit
				of the distributions) provided.
				
			\item[2aiv.] Measures of performance \\
				The most natural measure of performance for this model 
				will be the number of sorties flown.
				A contributing aspect of this performance are also
				the statistics related to times spent at each stage
				of the process.
				Rather than gathering this information individually,
				measuring the utilization of the resources/workers
				required to perform the task could give a more nuanced look
				at bottle necks that can occur in a very irregular manner as
				the shared resources are deficit from different tasks
				at different times.
				
			\item[2av.] Experimental controls required \\
				The controls explicitly required by the assignment outline are
				whether or not the maintenance crew can do service crew work 
				and how many service crews are available.
				Because failure rate is referenced in more than one logic
				expression this model stores it as a referenced variable.
				As an unintended result, the Simio environment makes it trivial
				to execute addition experiments with variations on this failure rate.
		\end{enumerate}
	\end{solution}
	\item[2b.] (10\%) Results table summarizing response variables requested
	\begin{solution} \\ \phantom{text} \\
		\begin{tabular}{c|cccc}
			\footnotesize InstanceName & \footnotesize NumServiceTeams & \footnotesize bool\_MnxrFlexUtil & 
			\footnotesize AvgDailySorties & \footnotesize AvgDailyUnschedMnx \\
			\midrule
			Baseline   & 3 & TRUE  & 21.665 & 1.4233 \\
			Scenario A & 2 & TRUE  & 18.995 & 1.2333 \\
			Scenario B & 3 & FALSE & 19.92  & 1.2983 \\
		\end{tabular} \phantom{text}
		\bigskip \\
		\begin{tabular}{c|cccc}
			\footnotesize InstanceName & \footnotesize AvgMnxQueueTime & \footnotesize CrewChiefUtilization & 
			\footnotesize MaintainerUtilization & \footnotesize ServiceUtilization  \\
			\midrule
			Baseline   & 5.7636 & 101.9728 & 96.7755  & 84.5981  \\
			Scenario A & 5.3157 & 89.3185  & 104.5866 & 101.1766 \\
			Scenario B & 1.6274 & 93.6741  & 23.2738  & 99.5249  \\
		\end{tabular} \bigskip
		
	\end{solution}
	\item[2c.] (20\%) Analysis of scenarios
	\begin{solution} 
	\begin{enumerate}
		\item[2ci.] Individual scenario analysis (what do the numbers mean?)\\
			Average Daily Sorties, Average Daily Unscheduled Maintenance, and Average Maintenance Queue Time
			are exactly as their names describe.
			The various crew utilizations are not as self-explanatory; 
			these numbers represent the ratio of time worked over
			time scheduled to work.
			Utilization over 100\% is the result of overtime causing the numerator
			to exceed their scheduled time.
		
		\item[2cii.] Comparative analysis of scenarios using SMORE plots of response variables.\\
			The baseline shows strong utilization of all personnel, 
			with a slight overtaxing of the crew chiefs.
			Scenario B shows us that dropping one of the four teams
			changes the likely limiting factor to the teams, and reduces the overall throughput.
			Scenario B surprisingly performs better than A in terms of sortie average.
			Having the maintenance team stand-by for only unplanned maintenance drastically 
			reduces the maintenance queue time and drastically reduced the team's utilization.
			This hints to a possible strategy that could retain this low maintenance
			queue time and improve the sortie throughput.
			
			Examination of the SMORE plots further shows that Scenario B
			offers the benefit of a lower variance.
			If leadership values consistency more than a higher number of
			sorties this may be a significant factor.
			
			
			
			\begin{center}
				\includegraphics[width=0.6\linewidth]{AvgDailySorties} \bigskip
				\includegraphics[width=0.6\linewidth]{AvgDailyUnschedMnx} \bigskip
				\includegraphics[width=0.6\linewidth]{AvgMnxQueueTime}
			\end{center}
		
		\item[2ciii.] Recommendations based on statistical and practical differences between scenario responses. \\
			Based on the tested scenarios, the baseline provides the greatest sortie capabilities.
			From this it appears that Crew Chiefs are the most taxed part of the system.
			If their roles in servicing/debriefing and unscheduled maintenance can be abridged
			it may increase the overall throughput of the system, and should increase the morale
			of the crew chiefs.
			If more acquisitions can be made, it should be an additional crew chief if their tasks
			cannot be abridged, but if they can, then the additional acquisition should be a maintenance
			team, and it may be beneficial if only one of those teams were
			able to perform a dual role as a service team at a time.
			This may capture both benefits.
		
	\end{enumerate}	
	\end{solution}
\end{enumerate}

%\end{ Question 2}

\end{questions}
\end{document}