\documentclass[answers]{exam}
\usepackage[english]{babel}
\usepackage[utf8x]{inputenc}
\usepackage{amsmath,amssymb,amsthm}

\begin{document}

\noindent{\large OPER 561 - Discrete-Event Simulation - %
	Homework X }
\hspace{\fill} {\large B. Hosley}
\bigskip

\begin{questions}

%%%%%%%%%%%%%%%%%%%%%%%%%%%
%	\begin{ Question 1}	  %
%%%%%%%%%%%%%%%%%%%%%%%%%%%
\question 
Now that you have raw arrival and interarrival times to the Adjust server, you can do more math than just the min/mean/max that Simio otherwise provides. Suggest two measures or analyses you might perform.
\begin{solution}
	Standard Deviation and Variance are classics,
	
	but we may also fit a distribution to the dataset.
\end{solution}
%\end{ Question 1}

%%%%%%%%%%%%%%%%%%%%%%%%%%%
%	\begin{ Question 2}	  %
%%%%%%%%%%%%%%%%%%%%%%%%%%%
\question 
The Adjust server is the first place new entities go. Why does the mean time between arrivals to Input@Adjust differ from the interarrival time at the entity Source block?
\begin{solution}
	The source only instantiates new entities (radar components)
	whereas, Input@Adjust receives both the new entities and the repeat adjustment entities.
\end{solution}
%\end{ Question 2}

%%%%%%%%%%%%%%%%%%%%%%%%%%%
%	\begin{ Question 3}	  %
%%%%%%%%%%%%%%%%%%%%%%%%%%%
\question 
Try to analytically estimate the mean interarrival time at Input@Adjust. (As in, using just your model inputs, logic, and rules of probability, but no simulation output data.)
\begin{solution}
	\begin{align*}
		&3   \text{minutes} \\
		&- 2 \text{minutes} \times 0.2   \text{(first round rejects)}  \\
		&- 2 \text{minutes} \times 0.2^2 \text{(second round rejects)} \\
		&- 2 \text{minutes} \times 0.2^3 \text{(third round rejects)}  \\
		&= 2.504
	\end{align*}
	
\end{solution}
%\end{ Question 3}

\end{questions}
\end{document}