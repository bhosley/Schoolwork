\documentclass[12pt]{amsart}
\usepackage[left=0.5in, right=0.5in, bottom=0.75in, top=0.75in]{geometry}
\usepackage[english]{babel}
\usepackage[utf8x]{inputenc}
\usepackage{amsmath,amssymb,amsthm}
\usepackage{enumerate}
\usepackage{graphicx}
\usepackage{booktabs}


\begin{document}
\raggedbottom

\noindent{\large OPER 561 - Discrete-Event Simulation - %
	Homework 6 }
\hspace{\fill} {\large B. Hosley}
\bigskip


%%%%%%%%%%%%%%%%%%%%%%%
\begin{enumerate}[1.]
	\item Go to the “Response Results” tab for the experiment and select “Export Details” at far
	right of the top ribbon. From the created Excel spreadsheet construct the following six
	separate two population confidence intervals for each response: \\
	
	\textbf{Solution:} Using excel's built in t-test:\bigskip\\
	\begin{tabular}{c|cc|cc|cc} 
		\toprule 
		& \multicolumn{2}{|c}{Independent} & \multicolumn{2}{|c}{CRN} & \multicolumn{2}{|c}{\(\Delta\)} \\
		Pair & Response Time & Max Hold & Response Time & Max Hold & RT & MH \\
		\midrule
		Cur/CT6 & \(0.12885\pm0.09541\) & \(0.70\pm2.08015\) & \(0.11488\pm0.04208\) & \(0.2\pm0.87937\) & \(0.55892\) & \(0.55773\) \\ [3pt]
		Cur/CT7 & \(0.24648\pm0.08452\) & \(2.10\pm1.83000\) & \(0.23662\pm0.06071\) & \(1.8\pm1.15840\) & \(0.28170\) & \(0.36701\) \\ [3pt]
		CT6/CT7 & \(0.11763\pm0.06246\) & \(1.40\pm1.39855\) & \(0.12173\pm0.02801\) & \(1.6\pm0.50018\) & \(0.57207\) & \(0.64236\) \\ [3pt]
		\bottomrule
	\end{tabular}\bigskip\\
	These results suggest that both new methods are statistically better than the current method with regard to response time,
	because the confidence interval on the the first pairing does intersect with 0 there is room within that level of confidence
	that the two scenarios are not different. This doubt is alleviated in the 7 member cross-training model. \\

	\item Reference BCNN 11.6 and 12.2. You have been provided a Simio model of the system
	described in 11.6 using common random numbers with an experiment set up using
	Max\_Inventory (M) and Reorder\_Level (L) controls.
	\begin{enumerate}[a.]
		\item Set up 4 scenarios matching the 4 (M, L) systems listed in 12.2 [(50,30); (50,40);
		(100,30); (100,40)].
		
		\item Run 10 replications for each scenario. \textbf{Look at your response results and see if you
		can pick out the “best” (lowest average total cost) system.}
	
	\bigskip
	\textbf{Solution:}
	Scenario (50,40) has the lowest total cost at 207.127.
	\bigskip
	
		\item Now choose the “Select the Best using KN” add-in procedure with following settings
		under properties: Primary Response: AvgTC (NOTE: must select Minimize as Objective for 
		AvgTC response); Confidence Level: .90; Indifference Zone: 5; and Replication
		Limit: 500. Open up properties for the AvgTC response and set Objective to Minimize.
		
		\item Now select run from the Experiment Design tab and \textbf{report the number of total
		replications Simio ran for each of the scenarios and which scenario was selected as
		best} (that scenario will be checked)
		
	\bigskip
	\textbf{Solution:}
	Interestingly, this run contradicted the previous simulation in that (50,30) appears to have the best results.\\
	\bigskip
	\begin{figure}[h]
		\centering
		\includegraphics[width=0.7\linewidth]{ScreenCap}
		\caption{}
		\label{fig:screencap}
	\end{figure}
		
	\end{enumerate}
\end{enumerate}
	
\end{document}