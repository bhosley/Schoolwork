\documentclass[12pt]{amsart}
\usepackage[left=0.5in, right=0.5in, bottom=0.75in, top=0.75in]{geometry}
\usepackage[english]{babel}
\usepackage[utf8x]{inputenc}
\usepackage{amsmath,amssymb,amsthm}
\usepackage{enumerate}
\usepackage{graphicx}

\usepackage{booktabs}

\begin{document}
\raggedbottom

\noindent{\large OPER 561 - Discrete-Event Simulation - %
	Homework 2 }
\hspace{\fill} {\large B. Hosley}
\bigskip


%%%%%%%%%%%%%%%%%%%%%%%
\setcounter{section}{7}
\setcounter{subsection}{3}
\subsection{} 
	\textit{Make sure you provide random integers and corresponding random numbers.} \\
	Use the linear congruential method to generate a sequence of three two-digit random integers and corresponding random numbers. Let \(X_0 = 27, a = 8, c = 47,\) and \(m = 100\).

\bigskip
	\textit{Solution:} \\ 
	\phantom{text}\hspace{10ex}\phantom{text}
	\begin{tabular}{c|c|c}
		$\quad i\quad$ & $\quad X_i\quad$ & $\quad R_i\quad$ \\
		\midrule
		0 & 27 & - \\
		1 & 63 & 0.63 \\
		2 & 51 & 0.51 \\
		3 & 55 & 0.55 \\
	\end{tabular}
	
\bigskip
	
\setcounter{section}{8}
\setcounter{subsection}{0}
\subsection{} 
	\textit{Develop a random-variate generator for the specified pdf. 
		Given a uniform random draw of U=0.374, what is the value of 
		the RV generated with your algorithm for this problem.} \\
	Develop a random-variate generator for a random variable X with the pdf
	\begin{align*}
		f(x)= \begin{cases}
			e^{2x}, & -\infty<x\leq0 \\
			e^{-2x}, & 0<x<\infty
		\end{cases}
	\end{align*}
	
\bigskip
	\textit{Solution:} \\
	From the PDF, on \(-\infty<x\leq0\),
	\begin{align*}
		F(x)	=\ 	\int_{-\infty}^{x} e^{2x}\, dx  \
			 =\		\left.\frac{e^{2x}}{2} \right|_{-\infty}^x \
			 =\		\frac{e^{2x}}{2} - 0,
	\end{align*}
	and on the \(0<x<\infty\) domain,
	\begin{align*}
		F(x)&=\  \int_{-\infty}^{0} e^{2x}\, dx + \int_{0}^{x} e^{-2x}\, dx \\
		&=\  \left.\frac{e^{2x}}{2} \right|_{-\infty}^0 + \left.\frac{e^{-2x}}{2} \right|_{0}^x \\
		&=\  \frac{1}{2} - 0 - \frac{e^{-2x}}{2} + \frac{1}{2} \\
		&=\  1 - \frac{e^{-2x}}{2},
	\end{align*}
	The resulting CDF is 
	\begin{align*}
	F(x)= \begin{cases}
		\frac{e^{2x}}{2}, & -\infty<x\leq0 \\
		1- \frac{e^{-2x}}{2}, & 0<x<\infty.
	\end{cases}
	\end{align*}
	Then we invert the piece-wise functions
	\begin{align*}
		R &= F_1(X) & R &= F_2(X) \\
		R &= \frac{e^{2X}}{2} & R &= 1-\frac{e^{-2X}}{2} \\
		2R &= e^{2X} & 2(1-R) &= e^{-2X} \\
		\ln (2R) &= 2X & \ln (2(1-R)) &= -2X \\
		\frac{\ln (2R)}{2} &= X & -\frac{\ln(2(1-R))}{2} &= X, \\
	\end{align*}
	and combine them back together for the RVG of
		\begin{align*}
		F^{-1}(R)= \begin{cases}
			\frac{\ln (2R_i)}{2}, & 0\leq R_i \leq \frac{1}{2} \\
			-\frac{\ln(2(1-R_i))}{2}, & \frac{1}{2}<R_i\leq1.
		\end{cases}
	\end{align*}

	When \(U=0.374\) then \(F^{-1}(U)=-0.14518\).
\end{document}