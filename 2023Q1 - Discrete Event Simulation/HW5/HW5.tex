\documentclass[12pt]{amsart}
\usepackage[left=0.5in, right=0.5in, bottom=0.75in, top=0.75in]{geometry}
\usepackage[english]{babel}
\usepackage[utf8x]{inputenc}
\usepackage{amsmath,amssymb,amsthm}
\usepackage{enumerate}
\usepackage{graphicx}

\begin{document}
\raggedbottom

\noindent{\large OPER 561 - Discrete-Event Simulation - %
	Homework 5 }
\hspace{\fill} {\large B. Hosley}
\bigskip


%%%%%%%%%%%%%%%%%%%%%%%
\begin{enumerate}[1.]
\item (Law 9.7) For the following systems, state whether you think a terminating or nonterminating
simulation would be more appropriate. In the terminating cases, state the terminating event, \(E\). \\

\begin{enumerate}[a.]
	\item Consider a telephone system for which an arriving call may experience a delay before
	obtaining a line. Suppose that the goal is to estimate the mean delay of the 100th arriving
	call. \\
	
	\textit{Many replications of a terminating simulation, 
		with the simulations terminating after collecting 
		the weight time of the 100\textsuperscript{th} call.} \\
	
	\item Consider a military inventory system during peacetime, which is assumed to have a long
	duration. Assume that system parameters (e.g., the inter-demand time distribution) do not
	change over time and we are interested in the output process \(C_1 , C_2 ,\ldots,\) where \(C_i\) is the
	total cost in the \(i\)th month. Suppose further that we want a measure of mean cost. \\
	
	\textit{Non-terminating simulation; as worded, it seems that the months need not be repeated simulated.}\\
	
	\item Consider a manufacturing system for food products. A production schedule is issued, the
	system produces product for 13 days, and then the system is completely cleaned out on the
	fourteenth day. Then a new production schedule is issued, and the 2-week cycle is
	repeated, etc. The goal is to estimate the mean throughput over a cycle. \\
	
	\textit{Terminating simulation; consistent with the 14\textsuperscript{th} 
		day complete clean-out, each simulation may represent a two-week period
		of the cyclic.}\\
	
	\item Consider an air freight company that provides overnight delivery of packages. Aircraft
	loaded with packages start arriving at the hub operations at approximately 11 P.M. The
	packages are unloaded and then sorted in a warehouse according to the destination zip code.
	Packages with similar zip codes are placed on one aircraft, and the last plane departs at
	approximately 5 A.M. It is desired to estimate the mean (across departing planes) amount
	of time that planes are late in departing. \\
	
	\textit{Terminating simulation; terminating with the departure of the last
		airplane entity.}\\
	
\end{enumerate}
\clearpage

\item You have made four independent runs of 2-hour duration for the Able-Baker carhop problem.
The purpose of the simulation is to estimate Able’s utilization, \(\rho\), and the mean time spent in
the system per customer, \(\omega\), over the first 2 hours of the workday. Each run of the model is for
a 2-hour period, with the system being empty and idle at time 0. Four statistically independent
runs were made by using four distinct streams of random numbers to generate the interarrival
and service times. Abe’s utilization results for each run are: \(0.78, 0.86, 0.79\), and \(0.81\). The
results for the average system time for each run (in minutes) are: \(3.74, 4.53, 3.84,\) and \(3.98\). \\

\begin{enumerate}
	\item Suppose that management desires a 95\% confidence interval on the average system time
	with an absolute error less than or equal to 0.4 minutes. Estimate the required number of
	additional runs needed from the sample data given. DO NOT use BCNN Simulation Tools
	for this question. Use a \(Z\) for initial estimate like done in the text, but your final answer
	should be based on a t-statistic. \\
	
	\textbf{Solution}:\\
	
	First, we find an initial estimate $R$,
	\[R \geq \left(\frac{z_{\alpha/2}S_0}{\epsilon}\right)^2 = \left(\frac{(1.96)(0.35236)}{0.4}\right)^2 = 2.981. \]
	
	Next, because that was not terribly helpful, we will try with an incrementing t-statistic. 
	In effort to pick a good starting point we will round the previous answer and add it to the initial sample size.
	Testing with \(R=7\) we see,
	\[R \geq \left(\frac{t_{\alpha/2,R-1}S_0}{\epsilon}\right)^2 
	= \left(\frac{(t_{0.025,6})(0.35236)}{0.4}\right)^2
	= \left(\frac{(2.96865)(0.35236)}{0.4}\right)^2 = 6.83869.  \]
	This value is feasible. For verification, testing with \(R=6\) gives a right hand side of \(7.7634413\)
	which shows that \(R=7\) is the lowest value where this inequality holds. \\
	
	\item Suppose that management wants a confidence interval on Able’s utilization in addition to a
	confidence interval on the average system time. Construct these confidence intervals to
	ensure both are met at the 95\% confidence level. \\
	
	Using the information from part a we can construct a confidence interval using the half-length criterion,
	\[H = t_{\alpha/2,R-1}\frac{S_0}{\sqrt R}.\]
	For Abe's utilization this can be calculated by
	\[H = t_{0.025,6}\frac{0.03559}{\sqrt 7} = 0.03822.\]
	For the average system time the half-length is
	\[H = t_{0.025,6}\frac{0.35236}{\sqrt 7} = 0.37839.\]
	Combining these half-widths with the respective means, we can get a 95\% confidence interval;
	for Abe's utilization
	\[0.81 \pm 0.03822 \quad\text{ or }\quad (0.77178<\rho<0.84822)\]
	and the average system utilization
	\[4.0225 \pm 0.37839 \quad\text{ or }\quad (3.64411<\omega<4.40089) .\]
	
	
\end{enumerate}
\end{enumerate}
\end{document}