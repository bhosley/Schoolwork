\documentclass{beamer}
\usetheme{CambridgeUS}

\title{Federated Optimization - Distributed Optimization Beyond the Datacenter}
\subtitle{}
\author{Brandon Hosley}
\institute{}
\date{\today}

\begin{document}

\begin{frame}
	\titlepage
\end{frame}

%\begin{frame}
%	\frametitle{Outline}
%	\tableofcontents
%\end{frame}


\begin{frame}
	\frametitle{Problem Addressed/Identified}
	\begin{itemize}
		\item Want: Centralized model trained with all available data
		\item Have: Very decentralized data
	\end{itemize}
\end{frame}

\begin{frame}
	\frametitle{Research Contribution}
	Motivated by a connection between:
	\begin{itemize}
		\item SVRG (Stochastic Variance Reduced Gradient)
		\item Used as the single node solver
		\item High compute use, lower storage need
		\item DANE (Distributed Approximate NEwton)
		\item Solves sub-problems on nodes
	\end{itemize}
\end{frame}

\begin{frame}
	\frametitle{Research Contribution}
	Algorithm 1. Distributed version of SVRG
	\begin{enumerate}
		\item Parameterization
		\item Compute objective gradient as arithmetic mean of sub-function gradients
		\item Init weights on each node
		\item Update weights on node using SVRG
		\item Aggregate updates as original weights + Arithmetic Mean of node weight-server weights
	\end{enumerate}
\end{frame}

\begin{frame}
	\frametitle{Novelty/Rationale/Significance}
	\begin{enumerate}
		\item Stepsize on each node is inversely proportional to size of local data
		\item Node updates are scaled by a diagonal matrix - elements defined as the ratios of the features' appearance globally and locally (on node).
		\item Aggregation elements are scaled by another diagonal matrix - Number of nodes / number with the feature.
	\end{enumerate}
\end{frame}

\begin{frame}
	\frametitle{Experimental Methodology}
	Binary classification of if a social media post would receive at least one comment. \\
	3:1 Training, Testing Split
\end{frame}

\begin{frame}
	\frametitle{Limitations and Weaknesses}
	Non-existence of a large, naturally user-clustered dataset
\end{frame}

\begin{frame}
	\frametitle{Findings and Conclusions}
	The referenced prior attempts at communication-efficient algorithms were limiting in their implementation.\\
	"Inefficient" - Did not converge in a reasonable time on disparate data.
\end{frame}

\begin{frame}
	\frametitle{Areas for Improvement/Future Directions}
	"Currently, there is no proper theoretical justification of the method, which would surely drive further improvements."\\
	Concerns with privacy, and how that fits within ML pipelines
\end{frame}

\end{document}