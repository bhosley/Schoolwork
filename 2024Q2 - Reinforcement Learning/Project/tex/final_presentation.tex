\documentclass{beamer}
\usepackage[english]{babel}
\usepackage{lmodern}
\usepackage{libertinus}
\usepackage{csquotes}
\usepackage{graphicx}
\usepackage{multicol}
\usepackage[backend=biber,style=ieee]{biblatex}
\addbibresource{project.bib}

\usepackage{algorithm}
\usepackage{algpseudocode}
\usepackage{comment}
\usepackage{minted}


\title{DSOR646 - Reinforcement Learning Final Project}
\author{Capt. Brandon Hosley\inst{1}}
\institute[ENS]{
    \inst{1}
    Department of Operational Sciences\\
    Air Force Institute of Technology}
\date{\today}

\AtBeginSection[]{
  \begin{frame}
    \tableofcontents[currentsection]
  \end{frame}
}


\usetheme{Madrid}
\usecolortheme{beaver}
\mode<presentation>

\begin{document}
\frame{\titlepage}
\begin{frame}
    \tableofcontents
\end{frame}

\section{Problem Description}

\begin{frame}
    \frametitle{}
    \begin{block}{Objective}
        Develop an autonomous agent that can control the descent and landing of a lunar module 
        on a specified landing pad, simulating the challenges faced during space missions.
    \end{block}
    \begin{figure}
        \includegraphics[width=0.6\linewidth]{images/lander_screen.png}
        \caption{An un-animated frame of Lunar Lander}
    \end{figure}
\end{frame}

\section{MDP Model}

\begin{frame}
    \frametitle{MDP Model}
    \begin{description}
        \item[Agent] The Lander
        \item[State]
            \[s_t = \begin{cases}
                x_t \in [-1.5,1.5]   & \text{Position in } x \\
                y_t \in [-1.5,1.5]   & \text{Position in } y \\
                \vec{x}_t \in [-5,5] & \text{Velocity in } x \\
                \vec{y}_t \in [-5,5] & \text{Velocity in } y \\
                \omega_t \in [-\pi,\pi] & \text{Angle}\\
                \vec{\omega}_t \in [-5,5] & \text{Angular Velocity}\\
                \mathbb{I}_{t}(\text{leg 1}) \in\{0,1\}  & \text{Leg on ground}\\
                \mathbb{I}_{t}(\text{leg 2}) \in\{0,1\}  & \text{Leg on ground}
            \end{cases}\]
        \item[Action] 
        \item[Probability] 
        \item[Reward] 
    \end{description}
\end{frame}

\begin{frame}
    \frametitle{MDP Model}
    \begin{description}
        \item[Agent] The Lander
        \item[State] \(s_t\)
        \item[Action] 
            \[a_t \in A = 
                \begin{cases}
                    0 & \text{No-Op} \\
                    1 & \text{Fire Left Engine} \\
                    2 & \text{Fire Main Engine} \\
                    3 & \text{Fire Right Engine}
                \end{cases}
            \]
        \item[Probability] 
        \item[Reward] 
    \end{description}
\end{frame}

\begin{frame}
    \frametitle{MDP Model}
    \begin{description}
        \item[Agent] The Lander
        \item[State] \(s_t\)
        \item[Action] \(a_t\)
        \item[Probability] 
            \[P(s_{t+1}|s_t,a_t) \cong \begin{cases}
                \text{Dispersion} \sim U(-1,1) \\
                \text{Wind(Linear)} = \tanh(\sin(2 k x) + \sin(\pi k x)) \\
                \text{Wind(Rotate)} = \tanh(\sin(2 k x) + \sin(\pi k x)) \\
            \end{cases}\]
        \item[Reward] 
    \end{description}
\end{frame}

\begin{frame}
    \frametitle{MDP Model}
    \begin{description}
        \item[Agent] The Lander
        \item[State] \(s_t\)
        \item[Action] \(a_t\)
        \item[Probability] \(P(s_{t+1}|s_t,a_t)\)
        \item[Reward] 
    \end{description}
    \begin{align*}
        r_t = \\
        & \pm 100                   & \text{End-state, crash or land} \\
        & +10 (s_{t,6} + s_{t,7})   & \text{leg(s) on ground} \\
        & -a_t\cdot [0,0.03,0.3,0.03] & \text{thruster cost} \\
        & - 100\sqrt{a_{t,0}^2+a_{t,1}^2} & \text{Distance} \\
        & - 100\sqrt{a_{t,2}^2+a_{t,3}^2} & \text{Velocity} \\
        & - 100|\omega_t| & \text{Tilt}
    \end{align*}
\end{frame}

\section{Solution Approach}

\begin{frame}
    \frametitle{Vanilla Algorithms}
    \begin{figure}
        \includegraphics[width=0.75\linewidth]{images/semigrad_sarsa_algo.png}
    \end{figure}
\end{frame}

\begin{frame}
    \frametitle{Vanilla Algorithms}
    \begin{figure}
        \includegraphics[width=0.75\linewidth]{images/lam_sarsa_algo.png}
    \end{figure}
\end{frame}

\begin{frame}
    \frametitle{Fourier Cosine Basis}
    \[ \phi_i(s)=\cos(\pi\mathbf{c}_i\cdot\mathbf{s}) \]
    \begin{itemize}
        \item \(\mathbf{c}_i\) is a vector of coefficients (frequencies)
        \item \(\mathbf{s}\) is the state vector
        \item \(i\) is the index of the basis function
    \end{itemize}
\end{frame}

\begin{frame}
    \frametitle{Fourier Cosine Basis}
    \begin{figure}
        \includegraphics[width=0.75\linewidth]{images/fourierclass.png}
    \end{figure}
\end{frame}

\begin{frame}
    \frametitle{Fourier Cosine Basis}
    \begin{figure}
        \includegraphics[width=0.9\linewidth]{images/fourier_imple.png}
    \end{figure}
\end{frame}

\begin{frame}
    \frametitle{Boltzmann Exploration}
    \begin{itemize}
        \item Replace \(\epsilon\)-greedy with: \\[2em]
        \[ P(a|s) = \frac{e^{Q(s,a)/\tau}}{\sum_{\alpha\in A} e^{Q(s,\alpha)/\tau}} \] \\[2em]
        \item Where \(\tau\) is a 'temperature' parameter (not unlike \(\epsilon\))
    \end{itemize}
\end{frame}

\begin{frame}
    \frametitle{Boltzmann Exploration}
    \begin{figure}
        \includegraphics[width=0.9\linewidth]{images/boltzmann_code.png}
    \end{figure}
\end{frame}

\section{Results and Analysis}

\section{Conclusion}

%%%%%%%%%%%%%%%%%%%%%%%%%%%%%%%%%%%%%%%%%%%%%%%%%%%%%%%%%%%%%%%%%%%%%%%%%%%%%%%%%%%%%%%%%%%%%%%%%%%

%\section{References}

%\renewcommand*{\bibfont}{\tiny}
%\frame[allowframebreaks]{\printbibliography}

\begin{frame}
    Questions?
\end{frame}

\end{document}