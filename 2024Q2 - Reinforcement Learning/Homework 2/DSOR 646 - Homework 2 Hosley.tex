\documentclass[12pt,letterpaper]{exam}

\usepackage[utf8]{inputenc}
\usepackage[T1]{fontenc}
\usepackage[left=0.50in, right=0.50in, top=0.50in, bottom=0.50in]{geometry}

\usepackage{libertine}
\usepackage{multicol}
\usepackage[shortlabels]{enumitem}

\usepackage{booktabs}
\usepackage[table]{xcolor}

\usepackage{amssymb}
\usepackage{amsthm}
\usepackage{mathtools}
\usepackage{bbm}

\usepackage{hyperref}
\usepackage{graphicx}
%\usepackage{wrapfig}
%\usepackage{capt-of}
%\usepackage{tikz}
%\usepackage{pgfplots}
%\usetikzlibrary{shapes,arrows,positioning,patterns}
%\usepackage{pythonhighlight}

\newcommand\chapter{2}
\renewcommand{\thequestion}{\textbf{\chapter.\arabic{question}}}
\renewcommand{\questionlabel}{\thequestion}

%%%%%%%%%%%%%%%%%%%%%%%%%%%%%%%%%%%%%%%%%%%%%%%%%%%%%%%%%%%%%%%%%%
\newcommand{\class}{ DSOR 646 $-$ Reinforcement Learning } % This is the name of the course 
\newcommand{\assignmentname}{Homework \# \chapter} % 
\newcommand{\authorname}{Hosley, Brandon} % 
\newcommand{\workdate}{\today} % 
\printanswers% this includes the solutions sections
%%%%%%%%%%%%%%%%%%%%%%%%%%%%%%%%%%%%%%%%%%%%%%%%%%%%%%%%%%%%%%%%%%


\begin{document}
\pagestyle{plain}
\thispagestyle{empty}
\noindent
 
%%%%%%%%%%%%%%%%%%%%%%%%%%%%%%%%%%%%%%%%%%%%%%%%%%%%%%%%%%%%%%%%%%%%%%%%%%%%%%%%%%%
\noindent
\begin{tabular*}{\textwidth}{l @{\extracolsep{\fill}} r @{\extracolsep{10pt}} l}
	\textbf{\class} & \textbf{\authorname}  &\\ %Your name here instead, obviously 
	\textbf{\assignmentname} & \textbf{\workdate} & \\
\end{tabular*}\\ 
\rule{\textwidth}{2pt}
%%%%%%%%%%%%%%%%%%%%%%%%%%%%%%%% HEADER %%%%%%%%%%%%%%%%%%%%%%%%%%%%%%%%%%%%%%%%%%%

\begin{questions}

	\renewcommand\chapter{3}
	\question%
	Exercise 3.1: MDP examples (p.51)
	\emph{Question text.}

	\begin{solution}
		Solution.
	\end{solution}
	%%%%%%%%%%%%%%%%%%%%%%%%%%%%%%%%%%%%%%%%%%%%%%%%%%%%%%%%%%%%%

2. Exercise 3.3: Scoping (p.51)

\question%
\emph{}
\begin{solution}
\end{solution}
%%%%%%%%%%%%%%%%%%%%%%%%%%%%%%%%%%%%%%%%%%%%%%%%%%%%%%%%%%%%%

3. Exercise 3.7: Selecting rewards (p.56)

\question%
\emph{}
\begin{solution}
\end{solution}
%%%%%%%%%%%%%%%%%%%%%%%%%%%%%%%%%%%%%%%%%%%%%%%%%%%%%%%%%%%%%

4. Exercise 3.8: Returns, episodic (p.56)

\question%
\emph{}
\begin{solution}
\end{solution}
%%%%%%%%%%%%%%%%%%%%%%%%%%%%%%%%%%%%%%%%%%%%%%%%%%%%%%%%%%%%%

5. Exercise 3.9: Returns, continuing (p.56)

\question%
\emph{}
\begin{solution}
\end{solution}
%%%%%%%%%%%%%%%%%%%%%%%%%%%%%%%%%%%%%%%%%%%%%%%%%%%%%%%%%%%%%

6. Exercise 3.18: Value of a state (p.62)

\question%
\emph{}
\begin{solution}
\end{solution}
%%%%%%%%%%%%%%%%%%%%%%%%%%%%%%%%%%%%%%%%%%%%%%%%%%%%%%%%%%%%%

7. Exercise 3.19: Value of a state-action (p.62)

\question%
\emph{}
\begin{solution}
\end{solution}
%%%%%%%%%%%%%%%%%%%%%%%%%%%%%%%%%%%%%%%%%%%%%%%%%%%%%%%%%%%%%

8. Exercise 3.25: Optimal state-value function (p.67)

\question%
\emph{}
\begin{solution}
\end{solution}
%%%%%%%%%%%%%%%%%%%%%%%%%%%%%%%%%%%%%%%%%%%%%%%%%%%%%%%%%%%%%

9. Exercise 3.26: Optimal action-value function (p.67)

\question%
\emph{}
\begin{solution}
\end{solution}
%%%%%%%%%%%%%%%%%%%%%%%%%%%%%%%%%%%%%%%%%%%%%%%%%%%%%%%%%%%%%

10. Exercise 3.27: Optimal policy (p.67)

\question%
\emph{}
\begin{solution}
\end{solution}
%%%%%%%%%%%%%%%%%%%%%%%%%%%%%%%%%%%%%%%%%%%%%%%%%%%%%%%%%%%%%


\renewcommand\chapter{4}
11. Exercise 4.1: Policy evaluation (p.76)

\question%
\emph{}
\begin{solution}
\end{solution}
%%%%%%%%%%%%%%%%%%%%%%%%%%%%%%%%%%%%%%%%%%%%%%%%%%%%%%%%%%%%%

12. Exercise 4.5: Policy iteration using action-values (p.82)

\question%
\emph{}
\begin{solution}
\end{solution}
%%%%%%%%%%%%%%%%%%%%%%%%%%%%%%%%%%%%%%%%%%%%%%%%%%%%%%%%%%%%%

\setcounter{question}{6}
\question%
Jack's Car Rental Problem, Extended (p.82)
\begin{enumerate}[label= (\alph*)]
	\item 
	MDP model formulation. 
	Use the following \textbf{mathematical notation} to fully define the MDP model components.
	\begin{itemize}
		\item set of decision epochs: \(\mathcal{T}\)
		\item set of states (i.e., state space): \(\mathcal{S}\), 
		requiring definition of \emph{state variable} \(s \in \mathcal{S}\)
		\item set of actions (i.e., state-dependent action space): \(\mathcal{A}_s\), 
		requiring definition of \emph{decision variable} \(a\in\mathcal{A}_s\) for all \(s\in\mathcal{S}\)
		\item transition probability function: \(p(s^\prime|s, a)\), 
		for all \(s, s^\prime \in \mathcal{S}, a \in \mathcal{A}_s\). 
		\textbf{It is acceptable to express the state transition function} \(S^M(S_t, A_t)\), 
		which returns the next state \(S_{t+1}\), in terms of the four primary random variables driving the stochastic process forward, 
		rather than the transition probability function. Other problem parameters \- e.g., capacity of each car lot
		and maximum number of cars that may be transferred \- should be incorporated as well.
		\item reward function: \(r(s, a, s^\prime)\), for all \(s, s^\prime \in S, a \in A_s\)
	\end{itemize}
	\item 
	Dynamic programming. Solve the extended problem using \emph{Policy Iteration}. 
	\textbf{You must encode your own implementation of the algorithm on page 80 in Python.} 
	Upload your Python file when you submit your assignment. 
	Report results by creating a policy chart and a final value function graph, 
	similar to those seen in the bottom right panels of Figure 4.2 on page 81.
	\item 
	Policy insights. How did the changes impact Jack's Rental Car business? That is, compare
	your policy and value results with those of the original problem. Your answer should reference
	the optimal policy and optimal value function graphs you created for part (b). Describe any
	differences in structure observed between the policies.
	\item 
	Dynamic programming. Solve the extended problem using \emph{Value Iteration}. 
	\textbf{You must encode your own implementation of the algorithm on page 83 in Python.} 
	Upload your Python file when you submit your assignment. 
	Report results by creating a policy chart for the optimal policy and a final value function graph, 
	similar to the last two figures in Figure 4.2 on page 81.
	\item 
	Algorithm comparison. Compare your algorithm results with respect to \(\pi_*,\upsilon_*\),
	and computational time.
\end{enumerate}
\begin{solution}
	\begin{enumerate}
		\item 
	\end{enumerate}
\end{solution}
%%%%%%%%%%%%%%%%%%%%%%%%%%%%%%%%%%%%%%%%%%%%%%%%%%%%%%%%%%%%%

\end{questions}
\end{document}