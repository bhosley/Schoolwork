\documentclass[12pt,letterpaper]{exam}

\usepackage[utf8]{inputenc}
\usepackage[T1]{fontenc}
%\usepackage{libertine}
\usepackage[left=0.50in, right=0.50in, top=0.50in, bottom=0.50in]{geometry}

\usepackage{amssymb}
\usepackage{amsthm}
\usepackage{mathtools}
\usepackage{bbm}

\usepackage{hyperref}
\usepackage[table]{xcolor}
\usepackage{minted}
\setminted[python3]{
    fontsize=\footnotesize,
    %bgcolor=lightgray!10,
    linenos}
\usepackage{graphicx}
\graphicspath{{./images/}}
\usepackage{fancyvrb}

\usepackage{booktabs}
%\usepackage{multicol}
%\usepackage[shortlabels]{enumitem}
%\usepackage{wrapfig}
%\usepackage{tikz}
%\usepackage{pgfplots}
%\usetikzlibrary{shapes,arrows,positioning,patterns}

\usepackage{tcolorbox}
\newtcbox{\inlinecode}{on line, boxrule=0pt, boxsep=0pt, top=2pt, left=2pt, bottom=2pt, right=2pt,
    colback=gray!15, colframe=white, fontupper={\ttfamily \footnotesize}}

\newcommand\chapter{4}
\renewcommand{\thequestion}{\textbf{\chapter.\arabic{question}}}
\renewcommand{\questionlabel}{\thequestion}

%%%%%%%%%%%%%%%%%%%%%%%%%%%%%%%%%%%%%%%%%%%%%%%%%%%%%%%%%%%%%%%%%%
\newcommand{\class}{ DSOR 646 $-$ Reinforcement Learning } % This is the name of the course 
\newcommand{\assignmentname}{Homework \# \chapter} % 
\newcommand{\authorname}{Hosley, Brandon} % 
\newcommand{\workdate}{\today} % 
\printanswers% this includes the solutions sections
%%%%%%%%%%%%%%%%%%%%%%%%%%%%%%%%%%%%%%%%%%%%%%%%%%%%%%%%%%%%%%%%%%

\begin{document}
\pagestyle{plain}
\thispagestyle{empty}

%%%%%%%%%%%%%%%%%%%%%%%%%%%%%%%%%%%%%%%%%%%%%%%%%%%%%%%%%%%%%%%%%%%%%%%%%%%%%%%%%%%
\noindent
\begin{tabular*}{\textwidth}{l @{\extracolsep{\fill}} r @{\extracolsep{10pt}} l}
	\textbf{\class} & \textbf{\authorname}  &\\ %Your name here instead, obviously 
	\textbf{\assignmentname} & \textbf{\workdate} & \\
\end{tabular*}\\ 
\rule{\textwidth}{2pt}
%%%%%%%%%%%%%%%%%%%%%%%%%%%%%%%% HEADER %%%%%%%%%%%%%%%%%%%%%%%%%%%%%%%%%%%%%%%%%%%

\begin{questions}
\question%
\textbf{SARSA Temporal Difference Control:} \\
\emph{
Consider the Farama OpenAI \inlinecode{Gymnasium MountainCar-v0 environment}.
See the following link for a description of the environment: 
\url{https://gymnasium.farama.org/environments/classic_control/mountain_car/}.
Solve the agent's sequential decision problem using an implementation of the SARSA algorithm. 
You must encode your own implementation of the algorithm in Python. See page 130 of the course 
textbook for pseudocode and Lesson 10 Handout for an example implementation of SARSA using Python.
\\[0.25em]
Tune your algorithm via experimentation to attain best results. 
Utilize a parallelized Latin Hyper-cube Sampling (LHS) design. 
See Lesson 9 for an example implementation. 
What algorithm design run parameter values attain the best performance? 
Which parameters appear to most impact performance? Utilize estimated expected total discounted 
reward (EETDR) of generated target policies at milestone episodes to evaluate results.
\\[0.25em]
Test and report the best performing algorithm parameter values for your algorithm. 
Use tables and figures to convey your results. Execute at least 50 design runs. 
Execute at least 10 replications for each run. Use 1000 episodes per replication. 
Report Mean Time-Avg EETDR and Mean Max EETDR to convey algorithm performance in terms quality of 
solution and reliability. Report the best performing, superlative policy found by your SARSA 
implementation across all parameter tuning design runs performed. 
Record and report the average times required to execute a single design run replication. When 
milestone testing a policy, use at least 30 episodes. Report 95\% confidence interval of EETDR. 
Use the lower bound of the confidence interval, i.e., 95CILB, to identify a superlative policy. 
The 95CILB seeks to measure reliable performance. Show a scatter plot of the observed 95CILB 
values for average and maximum performance for each design run.
\\[0.25em]
Use the following stepsize TD learning rate. The learning rate at episode n is defined as
\[ \alpha_n = \alpha_a \left(\frac{1}{(n+1)^{\alpha_b}}\right). \]
Use state-action visit counts instead of the episode number to improve performance.
Use the following exploration-exploitation \(\epsilon\)-greedy rule. 
The probability of selecting a random action at episode \(n\) is
\[ \epsilon_n = \epsilon_a \left(\frac{1}{(n+1)^{\epsilon_b}}\right). \]
Use state visit counts instead of the episode number to improve performance.
Test the following algorithm parameters
\vspace*{-0.75em}
\begin{itemize}
    \setlength\itemsep{-0.25em}
    \item \(\alpha_a\): initial learning rate
    \item \(\alpha_b\): learning rate decay (higher indicates faster decay)
    \item \(\epsilon_a\): initial exploration probability
    \item \(\epsilon_b\): exploration probability decay (higher indicates faster decay)
    \item \(Q^{init}\): initial Q-values across all state-action pairs
\end{itemize}
}
\inputminted{python3}{scripts/SARSA.py}
\begin{figure}[H]
    \includegraphics*[height=0.33\linewidth]{one_step_sarsa_doe_performance.png}
    \includegraphics*[height=0.33\linewidth]{one_step_sarsa_doe_params.png}
\end{figure}
{\footnotesize\VerbatimInput{anova_tables/one_step_sarsa.txt}}

%%%%%%%%%%%%%%%%%%%%%
%%    PROBLEM 2    %%
%%%%%%%%%%%%%%%%%%%%%

\question%
\textbf{Q-learning Temporal Difference Control:} \\
\emph{ 
Consider the Farama OpenAI \inlinecode{Gymnasium MountainCar-v0 environment}.
Solve the agent’s sequential decision problem using the Q-learning algorithm, 
an off-policy temporal difference (TD) control solution procedure. 
You must encode your own implementation of the algorithm in Python. 
See page 131 of the course textbook for pseudocode. 
Perform the same analysis as described for SARSA above.
}
\inputminted{python3}{scripts/QLearning.py}
\begin{figure}[H]
    \includegraphics*[height=0.33\linewidth]{q_learning_doe_performance.png}
    \includegraphics*[height=0.33\linewidth]{q_learning_doe_params.png}
\end{figure}
{\footnotesize\VerbatimInput{anova_tables/q_learning.txt}}

%%%%%%%%%%%%%%%%%%%%%
%%    PROBLEM 3    %%
%%%%%%%%%%%%%%%%%%%%%

\question%
\textbf{Expected SARSA:} \\
\emph{ 
Consider the Farama OpenAI \inlinecode{Gymnasium MountainCar-v0 environment}.
Solve the agent's sequential decision problem using either the 
Expected SARSA Algorithm (discussed in Section 6.6) or 
the Double Q-learning Algorithm (discussed in Section 6.7 with pseudocode on p.136). 
You must encode your own implementation of the algorithm in Python. 
Perform the same analysis as described for SARSA above.
}
\inputminted{python3}{scripts/ExpectedSARSA.py}
\begin{figure}[H]
    \includegraphics*[height=0.33\linewidth]{expected_sarsa_doe_performance.png}
    \includegraphics*[height=0.33\linewidth]{expected_sarsa_doe_params.png}
\end{figure}
{\footnotesize\VerbatimInput{anova_tables/expected_sarsa.txt}}

%%%%%%%%%%%%%%%%%%%%%
%%    PROBLEM 4    %%
%%%%%%%%%%%%%%%%%%%%%

\question%
\textbf{\(n\)-step SARSA:} \\
\emph{ 
Consider the Farama OpenAI \inlinecode{Gymnasium MountainCar-v0 environment}.
Solve the agent's sequential decision problem using the n-step SARSA algorithm. 
You must encode your own implementation of the algorithm in Python. 
See page 147 of the course textbook for pseudocode. See Lesson 12 Handout for an example 
implementation of n-step SARSA using Python. 
Perform the same analysis as described for SARSA above. 
You must test different values of step parameter n.
}

\begin{solution}
    Because the default Scipy Latin Hypercube Samples on a range \([0,1)\) without
    an immediately obvious method for sampling from a set of integers, we write
    the \(steps\) variable to accept a float value.
    Given a float \(f\) then \(n=2^{\lfloor cf \rfloor}\).
    The constant \(c\) determines the range of values,
    in this implementation \(c=6\) means that \(n\) is selected from \(\{1,2,4,8,16,32\}\)
    with equal probability.
\end{solution}

\begin{minted}[gobble=4]{python3}
    env = gym.make('MountainCar-v0')
    features = ["alpha_a", "alpha_b", "eps_a", "eps_b", "steps"]
    sarsa_experiment = LHS_Experiment(SARSA,env,features)
    sarsa_experiment.parallel_lhs()
    sarsa_experiment.export_results()
    sarsa_experiment.plot_results()
    sarsa_experiment.plot_param_comparison()
    sarsa_experiment.anova()
\end{minted}

\begin{figure}[H]
    \includegraphics*[height=0.33\linewidth]{n_step_sarsa_doe_performance.png}
    \includegraphics*[height=0.33\linewidth]{n_step_sarsa_doe_params.png}
\end{figure}
{\footnotesize\VerbatimInput{anova_tables/n_step_sarsa.txt}}

%%%%%%%%%%%%%%%%%%%%%
%%    PROBLEM 5    %%
%%%%%%%%%%%%%%%%%%%%%

\question%
\textbf{Evaluate the Benchmark Momentum Policy:} \\
\emph{ 
In this fixed benchmark policy, the agent accelerates in the direction it is currently moving. 
This policy is expressed as follows.
\[\pi(s) = 
\begin{cases}
    0 & \text{if } s^{velocity} < 0 \\
    2 & \text{otherwise}
\end{cases}\]
Evaluate the policy by simulating its execution for 100 trials then taking the average. 
This is one replication. Perform 30 replications, reporting the average and 95\% 
confidence interval half-width from the 30 test replication averages.
}
\inputminted{python3}{scripts/benchmark_momentum.py}
\begin{figure}[H]
    \includegraphics*[height=0.45\linewidth]{momentum_performance.png}
\end{figure}

%%%%%%%%%%%%%%%%%%%%%
%%    PROBLEM 6    %%
%%%%%%%%%%%%%%%%%%%%%

\question%
\textbf{Compare the performance of the four algorithms with respect to quality of solution, 
reliability, and computational expense:} \\
\emph{ 
Which algorithm is best for this problem? Justify your claim. 
Create a single color-coded and/or icon-differentiated scatter plot that depicts the algorithms' 
results with respect to mean and max 95CILB measures of reliable average and maximum performance. 
Also create a single learning curve figure showing the best learning curve 
(associated with the best parameter design run) for each algorithm. 
Create a summary statistics table to compactly represent your results. 
Provide an accompanying narrative to support your claims. 
Make sure you explain your key observations. Can the algorithms determine superlative policies 
that outperform the benchmark Momentum Policy? (I hope so.)
\\[0.25em]
What was the most challenging aspect of solving the MountainCar-v0 environment using these RL
approaches? What novel algorithm design mechanisms would you suggest to improve performance?
}

\begin{solution}
    As with all questions regarding the best algorithm, the answer depends on multiple factors
    or the allowance of certain assumptions.
    In the case of this problem, being trained in simulation, there is no exceptional cost 
    associated with failure, riskier algorithms that explore more are not a problem.

    The parameters of the superlative policies trained during this experiment were
    \begin{center}
        \begin{tabular}{c cccc c}
            Algorithm & \(\alpha_a\) & \(\alpha_b\) & \(\epsilon_a\) & \(\epsilon_b\) & n \\
            \midrule
            One-step SARSA & 0.71 & 0.39 & 0.33 & 0.13 &  \\
            n-step SARSA   & 0.71 & 0.39 & 0.33 & 0.13 & 2 \\
            Expected SARSA & 0.99 & 0.31 & 0.07 & 0.73 &  \\
            Q-learning     & 0.87 & 0.21 & 0.40 & 0.54 &  \\
            \bottomrule
        \end{tabular}
    \end{center}

    Given that the performance results of my implementations are all very similar,
    the primary factor that would influence which algorithm should be considered the best
    is probably the resource expense required to train the policy.
    In this experiment that would indicate Q-learning as a clear winner, given that its training
    time is approximately 16\% shorter than the next algorithm on average.

    Below is a table of summary statistics.
    \begin{center}
        \begin{tabular}{ccccc}
            Algorithm & Avg Execution Time & Avg Score & Avg Sup EETDR & Max Score \\
            \midrule
            One-step SARSA & 17.500773 & -180.382835 & -151.354667 & -155.071442 \\
            n-step SARSA   & 17.830613 & -182.281669 & -154.208000 & -154.190447 \\
            Expected SARSA & 17.881559 & -182.136424 & -152.470000 & -155.199330 \\
            Q-learning     & 15.102565 & -181.978577 & -154.812667 & -153.492831 \\
            \bottomrule
        \end{tabular}
    \end{center}

    One interesting note is that while all of these implementations failed to achieve the 
    desired performance (I believe due to ineffective exploration) when left to run for 
    significantly longer Expected SARSA and Q-learning do continue to improve at a faster
    rate than 1-step and 2-step SARSA. This can be seen in the comparison of the figures below.

    \begin{figure}[H]
        \includegraphics*[height=0.36\linewidth]{comparison.png}
        \includegraphics*[height=0.36\linewidth]{long comparison.png}
    \end{figure}

    Unfortunately, it seems the expert system (momentum baseline) may technically be the best 
    option of the given options. However, it seems that Q-learning has the greatest promise.
\end{solution}
\end{questions}

\clearpage
\textbf{Support Functions} 
\inlinecode{HW4\_functions.py}
\inputminted{python3}{../HW4_functions.py}

\end{document}