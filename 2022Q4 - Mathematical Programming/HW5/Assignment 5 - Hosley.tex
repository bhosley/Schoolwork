\documentclass[answers]{exam}
\usepackage[english]{babel}
\usepackage[utf8x]{inputenc}
\usepackage{amsmath,amssymb,amsthm}

\title{OPER 510 - Introduction to Mathematical Programming%
	\\ Assignment 5}
\author{Brandon Hosley}
\date{\today}

\usepackage[table,dvipsnames]{xcolor}
\usepackage{graphicx}
\usepackage{enumitem}

\usepackage{booktabs}

\usepackage{tabularx,makecell,diagbox}
\newcolumntype{Y}{>{\centering\arraybackslash}X}

\begin{document}
\maketitle
\unframedsolutions

\begin{questions}
\question
Find the shortest path from node 1 to node 6 in Figure 3.

\includegraphics[width=0.9\linewidth]{"Screenshot 2022-11-24 at 12.10.09 PM"}

\begin{solution}

\end{solution}


\question
Figures 18–20 show the networks for Problems 1–3. Find the maximum flow from source to sink in each network. Find a cut in the network whose capacity equals the
maximum flow in the network. Also, set up an LP that could be used to determine the maximum flow in the network.

\includegraphics[width=\linewidth]{"Screenshot 2022-11-24 at 12.23.30 PM"}



\begin{solution}

\end{solution}


\question
The promoter of a rock concert in Indianapolis must perform the tasks shown in Table 19 before the concert can be held (all durations are in days).
\begin{parts}
	\part Draw the project network.
	\part Determine the critical path.
	\part If the advance promoter wants to have a 99% chance
of completing all preparations by June 30, when should work begin on finding a concert site?
	\part Set up the LP that could be used to find the project’s critical path.
\end{parts}

\begin{solution}

\end{solution}


\question
Consider the following travel distance data: \\
\makebox[\linewidth]{Distance in Miles}\\
\setcellgapes{1pt}
\makegapedcells
\begin{tabularx}{\textwidth}{|c|Y|Y|Y|Y|Y|Y|}
	\hline
	\diagbox{From}{To} & 1  & 2  & 3  & 4  & 5  & 6  \\ \hline
	        1          & -  & 42 & 20 & 31 & 29 & 33 \\ \hline
	        2          & 42 & -  & 58 & 45 & 68 & 47 \\ \hline
	        3          & 20 & 58 & -  & 28 & 32 & 51 \\ \hline
	        4          & 31 & 45 & 28 & -  & 57 & 61 \\ \hline
	        5          & 29 & 68 & 32 & 57 & -  & 34 \\ \hline
	        6          & 33 & 47 & 51 & 61 & 34 & -  \\ \hline
\end{tabularx} \\

\begin{parts}
	\part
	Use the nearest neighbor heuristic (it is just what it sounds like), starting with a depot at 1 to provide a sequence for the traveling salesman problem where you must visit each city only once and return to the depot. Be sure to give the total distance with your sequence.
	
	\part
	Discuss the pros and cons of this approach
	
\end{parts}

\begin{solution}
	
\end{solution}


\end{questions}
\end{document}