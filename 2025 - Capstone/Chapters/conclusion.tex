This dissertation addressed the challenge of training robust multi-agent 
policies in heterogeneous settings, where agents may differ in their sensing capabilities, 
actuation constraints, or behavioral roles. 
The three contributions examined complementary strategies for improving 
training efficiency, enabling parameter sharing across structurally distinct agents, 
and evaluating the relative merits of architectural versus representational 
approaches to heterogeneity. Together, these investigations advance both the 
theoretical understanding and practical application of \gls{harl}.

\section{Summary of Contributions}

The first contribution established that curriculum-based team scaling, 
where policies are pretrained on smaller teams before upsampling to target 
configurations, can meaningfully reduce training costs in cooperative \gls{marl}. 
Results across Waterworld, Multiwalker, and \gls{lbf} demonstrated that the 
effectiveness of this strategy depends critically on task structure. 
In environments with low coordination demands and symmetric agent roles, 
pretraining accelerated convergence without compromising final performance. 
In settings requiring tight coordination under sparse rewards, pretraining 
served as a stabilizing mechanism that enabled learning in configurations 
where training from scratch frequently failed. However, when dynamic role 
complexity was high, the benefits diminished, indicating that small-team 
pretraining offers limited foresight into the diverse interactions encountered 
by larger, more heterogeneous groups.

The second contribution introduced implicit indication as a representational 
framework for enabling shared policies across heterogeneous agents. 
By constructing homogenized observation spaces that span all agent-specific 
subspaces and allowing policies to condition implicitly on the pattern of 
populated observation elements, this approach eliminates the need for 
explicit agent identifiers or per-type policy networks. 
Empirical evaluation demonstrated that implicit indication matches the 
performance of heterogeneous baselines while achieving a storage footprint
of $1/|\gls{i}|$ relative to methods maintaining separate policies per agent type. 
The disjoint-span training condition yielded particularly strong relative improvements, 
suggesting that non-overlapping observation structures may reduce gradient interference 
during shared-parameter learning~\cite{zhong2024}.

The third contribution provided a direct comparison between architectural 
and representational paradigms for handling observation heterogeneity. 
The \gls{pic}~\cite{liu2020b}, which employs \glspl{gnn} to achieve 
permutation-invariant value estimation, was evaluated against implicit 
indication under controlled experimental conditions. Results demonstrated 
that the representational approach substantially outperformed the 
architectural alternative across all sensor configurations and evaluation 
conditions, including scenarios involving sensor loss, team composition 
changes, and zero-shot generalization to novel observation patterns. 
These findings indicate that when heterogeneity is structural and 
semantically decomposable, explicit observation masking provides 
more effective conditioning for policy learning than learned graph-based aggregation.

\section{Unifying Insights}

Across all three contributions, a consistent theme emerged: the manner 
in which the learning problem is represented matters as much or more 
than algorithmic sophistication in enabling effective \gls{harl}. 
Observation schema design in the first contribution, homogenized 
spaces in the second, and the comparison of masking-based versus graph-based 
approaches in the third all point toward the primacy of representational 
choices in determining learning outcomes. This finding aligns with broader 
trends in machine learning, where feature engineering and input encoding have 
historically played decisive roles even as model architectures have grown more powerful.

A second insight concerns the relationship between heterogeneity type and method 
selection. The distinction between behavioral heterogeneity, where structurally 
identical agents develop divergent behaviors through independent learning, and 
intrinsic heterogeneity, where agents differ in their observation or action spaces, 
proves consequential for determining which strategies are most effective. 
Curriculum-based scaling addresses behavioral heterogeneity by providing 
foundational coordination skills that transfer to larger teams, while 
homogenization addresses intrinsic heterogeneity by enabling a single 
policy to operate across structurally distinct agents. Practitioners 
confronting heterogeneous multi-agent systems should first characterize 
the nature of the heterogeneity present before selecting an appropriate training methodology.

A third insight concerns robustness. The implicit indication framework exhibited 
inherent tolerance to sensor dropout, team-size changes, and novel agent compositions 
without requiring explicit robustness training. This suggests that well-designed 
representations can yield deployment flexibility as a natural consequence rather 
than as an additional objective to be optimized. For autonomous systems operating 
in dynamic environments where configurations may change during operation, such 
inherent robustness properties offer practical advantages over methods requiring 
retraining or architectural modification.

\section{Significance}

The barriers to deploying autonomous multi-agent systems remain substantial~\cite{jin2025}. 
Training costs scale with team size and interaction complexity, 
heterogeneous agent configurations complicate policy design, 
and deployed systems must operate reliably under conditions that 
may differ from training. This dissertation contributes to addressing each of these barriers.

By demonstrating that reduced-size pretraining can lower computational 
costs in settings where role differentiation is limited, the first 
contribution offers a practical tool for improving training efficiency. 
By establishing implicit indication as a viable framework for shared 
learning across structurally distinct agents, the second contribution 
reduces the architectural overhead traditionally associated with heterogeneous teams. 
By showing that representational solutions can outperform more complex architectural 
alternatives, the third contribution simplifies the design space confronting 
practitioners building heterogeneous multi-agent systems.

These findings have particular relevance for domains where heterogeneity is 
common and computational resources are constrained. 
Multi-robot systems deployed for disaster response, environmental monitoring, 
or defense applications often involve agents with varying sensor suites and 
must be trained efficiently to meet operational 
timelines~\cite{mohddaud2022, kouzeghar2023}. The methods 
developed here offer paths toward more tractable training regimes 
and more flexible deployment configurations.

More broadly, this work underscores that the challenges of 
heterogeneous multi-agent learning are fundamentally representational 
before they are algorithmic. Advances in optimization and architecture 
remain valuable, but their impact is mediated by how the learning 
problem is structured. As multi-agent systems continue to grow in 
scale and diversity, attending carefully to representational design 
will be essential for realizing their potential in real-world applications.
