\documentclass[12 pt,letterpaper]{article}
\usepackage{amsmath}
\usepackage{amssymb}
\usepackage{amsthm}
\usepackage[left=0.5in, right=0.5in, bottom=0.75in, top=0.75in]{geometry}

\usepackage{comment}

\usepackage{mathptmx}
\SetSymbolFont{letters}{bold}{OML}{cmm}{b}{it}
\SetSymbolFont{operators}{bold}{OT1}{cmr}{bx}{n}
\DeclareMathAlphabet{\mathcal}{OMS}{cmsy}{m}{n}

\allowdisplaybreaks

\usepackage{lastpage}
\usepackage{fancyhdr}
\pagestyle{fancy}
\fancyhead[L]{Homework 2 Solutions}
\fancyhead[C]{Due: October 18, 2024}
\fancyhead[R]{Page \thepage\ of \pageref{LastPage}}
\fancyfoot[L]{Air Force Institute of Technology}
\fancyfoot[R]{Fall 2024 (Hosley)}
\fancyfoot[C]{MATH 600: Mathematical Analysis}
\renewcommand{\headrulewidth}{0.25pt}
\renewcommand{\footrulewidth}{0.25pt}

\newcommand{\la}{\lambda}

\newcommand{\bbC}{\mathbb{C}}
\newcommand{\bbF}{\mathbb{F}}
\newcommand{\bbN}{\mathbb{N}}
\newcommand{\bbQ}{\mathbb{Q}}
\newcommand{\bbR}{\mathbb{R}}
\newcommand{\bbT}{\mathbb{T}}
\newcommand{\bbZ}{\mathbb{Z}}

\newcommand{\brI}{\mathbf{I}}
\newcommand{\brJ}{\mathbf{J}}
\newcommand{\brM}{\mathbf{M}}
\newcommand{\brR}{\mathbf{R}}
\newcommand{\brT}{\mathbf{T}}

\newcommand{\bfa}{\boldsymbol{a}}
\newcommand{\bfA}{\boldsymbol{A}}
\newcommand{\bfb}{\boldsymbol{b}}
\newcommand{\bfB}{\boldsymbol{B}}
\newcommand{\bfc}{\boldsymbol{c}}
\newcommand{\bfC}{\boldsymbol{C}}
\newcommand{\bfD}{\boldsymbol{D}}
\newcommand{\bfe}{\boldsymbol{e}}
\newcommand{\bfE}{\boldsymbol{E}}
\newcommand{\bff}{\boldsymbol{f}}
\newcommand{\bfF}{\boldsymbol{F}}
\newcommand{\bfG}{\boldsymbol{G}}
\newcommand{\bfH}{\boldsymbol{H}}
\newcommand{\bfI}{\boldsymbol{I}}
\newcommand{\bfJ}{\boldsymbol{J}}
\newcommand{\bfK}{\boldsymbol{K}}
\newcommand{\bfL}{\boldsymbol{L}}
\newcommand{\bfM}{\boldsymbol{M}}
\newcommand{\bfp}{\boldsymbol{p}}
\newcommand{\bfP}{\boldsymbol{P}}
\newcommand{\bfR}{\boldsymbol{R}}
\newcommand{\bfT}{\boldsymbol{T}}
\newcommand{\bfu}{\boldsymbol{u}}
\newcommand{\bfU}{\boldsymbol{U}}
\newcommand{\bfv}{\boldsymbol{v}}
\newcommand{\bfV}{\boldsymbol{V}}
\newcommand{\bfw}{\boldsymbol{w}}
\newcommand{\bfW}{\boldsymbol{W}}
\newcommand{\bfx}{\boldsymbol{x}}
\newcommand{\bfX}{\boldsymbol{X}}
\newcommand{\bfy}{\boldsymbol{y}}
\newcommand{\bfY}{\boldsymbol{Y}}
\newcommand{\bfz}{\boldsymbol{z}}
\newcommand{\bfZ}{\boldsymbol{Z}}

\newcommand{\bfone}{\boldsymbol{1}}
\newcommand{\bfzero}{\boldsymbol{0}}
\newcommand{\bfxi}{\boldsymbol{\xi}}
\newcommand{\bfXi}{\boldsymbol{\Xi}}
\newcommand{\bfphi}{\boldsymbol{\varphi}}
\newcommand{\bfPhi}{\boldsymbol{\varPhi}}
\newcommand{\bfPi}{\boldsymbol{\varPi}}
\newcommand{\bfpsi}{\boldsymbol{\psi}}
\newcommand{\bfPsi}{\boldsymbol{\varPsi}}
\newcommand{\bfchi}{\boldsymbol{\chi}}
\newcommand{\bfzeta}{\boldsymbol{\zeta}}
\newcommand{\bfdelta}{\boldsymbol{\delta}}
\newcommand{\bfDelta}{\boldsymbol{\varDelta}}
\newcommand{\bfGamma}{\boldsymbol{\varGamma}}
\newcommand{\bfOmega}{\boldsymbol{\varOmega}}
\newcommand{\bfSigma}{\boldsymbol{\varSigma}}
\newcommand{\bfLambda}{\boldsymbol{\varLambda}}

\newcommand{\calA}{\mathcal{A}}
\newcommand{\calB}{\mathcal{B}}
\newcommand{\calC}{\mathcal{C}}
\newcommand{\calD}{\mathcal{D}}
\newcommand{\calE}{\mathcal{E}}
\newcommand{\calF}{\mathcal{F}}
\newcommand{\calG}{\mathcal{G}}
\newcommand{\calH}{\mathcal{H}}
\newcommand{\calI}{\mathcal{I}}
\newcommand{\calK}{\mathcal{K}}
\newcommand{\calM}{\mathcal{M}}
\newcommand{\calN}{\mathcal{N}}
\newcommand{\calS}{\mathcal{S}}
\newcommand{\calT}{\mathcal{T}}
\newcommand{\calU}{\mathcal{U}}
\newcommand{\calV}{\mathcal{V}}
\newcommand{\calX}{\mathcal{X}}
\newcommand{\calY}{\mathcal{Y}}

\newcommand{\rmA}{\mathrm{A}}
\newcommand{\rmB}{\mathrm{B}}
\newcommand{\rmc}{\mathrm{c}}
\newcommand{\rmC}{\mathrm{C}}
\newcommand{\rmd}{\mathrm{d}}
\newcommand{\rme}{\mathrm{e}}
\newcommand{\rmi}{\mathrm{i}}
\newcommand{\rmj}{\mathrm{j}}
\newcommand{\rmk}{\mathrm{k}}
\newcommand{\rms}{\mathrm{s}}
\newcommand{\rmS}{\mathrm{S}}
\newcommand{\rmT}{\mathrm{T}}

\newcommand{\Tr}{\mathrm{Tr}}
\newcommand{\Part}{\mathrm{part}}
\newcommand{\Null}{\mathrm{null}}
\newcommand{\Span}{\operatorname{span}}
\newcommand{\rank}{\operatorname{rank}}
\newcommand{\real}{\operatorname{Re}}
\newcommand{\imag}{\operatorname{Im}}
\newcommand{\cond}{\operatorname{cond}}

\newcommand{\im}{\operatorname{im}}

\newcommand{\conv}[2]{{#1}\ast{#2}}
\newcommand{\argmin}[1]{\underset{#1}{\mathrm{argmin}}}
\newcommand{\argmax}[1]{\underset{#1}{\mathrm{argmax}}}

\newcommand{\mat}[2]{\left[\begin{array}{#1}#2\end{array}\right]}
\newcommand{\dmat}[2]{\left|\begin{array}{#1}#2\end{array}\right|}
\newcommand{\arbset}[1]{\left\{{#1}\right\}}
\newcommand{\arbparen}[1]{\left({#1}\right)}
\newcommand{\sign}{\operatorname{sign}}

\newcommand{\abs}[1]{|{#1}|}
\newcommand{\bigabs}[1]{\bigl|{#1}\bigr|}
\newcommand{\Bigabs}[1]{\Bigl|{#1}\Bigr|}
\newcommand{\biggabs}[1]{\biggl|{#1}\biggr|}
\newcommand{\Biggabs}[1]{\Biggl|{#1}\Biggr|}

\newcommand{\paren}[1]{({#1})}
\newcommand{\bigparen}[1]{\bigl({#1}\bigr)}
\newcommand{\Bigparen}[1]{\Bigl({#1}\Bigr)}
\newcommand{\biggparen}[1]{\biggl({#1}\biggr)}
\newcommand{\Biggparen}[1]{\Biggl({#1}\Biggr)}

\newcommand{\bracket}[1]{[{#1}]}
\newcommand{\bigbracket}[1]{\bigl[{#1}\bigr]}
\newcommand{\Bigbracket}[1]{\Bigl[{#1}\Bigr]}
\newcommand{\biggbracket}[1]{\biggl[{#1}\biggr]}
\newcommand{\Biggbracket}[1]{\Biggl[{#1}\Biggr]}

\newcommand{\set}[1]{\{{#1}\}}
\newcommand{\bigset}[1]{\bigl\{{#1}\bigr\}}
\newcommand{\Bigset}[1]{\Bigl\{{#1}\Bigr\}}
\newcommand{\biggset}[1]{\biggl\{{#1}\biggr\}}
\newcommand{\Biggset}[1]{\Biggl\{{#1}\Biggr\}}

\newcommand{\norm}[1]{\|{#1}\|}
\newcommand{\bignorm}[1]{\bigl\|{#1}\bigr\|}
\newcommand{\Bignorm}[1]{\Bigl\|{#1}\Bigr\|}
\newcommand{\biggnorm}[1]{\biggl\|{#1}\biggr\|}
\newcommand{\Biggnorm}[1]{\Biggl\|{#1}\Biggr\|}

\newcommand{\ip}[2]{\langle{#1},{#2}\rangle}
\newcommand{\bigip}[2]{\bigl\langle{#1},{#2}\bigr\rangle}
\newcommand{\Bigip}[2]{\Bigl\langle{#1},{#2}\Bigr\rangle}
\newcommand{\biggip}[2]{\biggl\langle{#1},{#2}\biggr\rangle}
\newcommand{\Biggip}[2]{\Biggl\langle{#1},{#2}\Biggr\rangle}

\newcommand{\alphi}{\renewcommand{\labelenumi}{(\alph{enumi})}}
\newcommand{\alphii}{\renewcommand{\labelenumii}{(\alph{enumii})}}
\newcommand{\alphiii}{\renewcommand{\labelenumiii}{(\alph{enumiii})}}
\newcommand{\romani}{\renewcommand{\labelenumi}{(\roman{enumi})}}
\newcommand{\romanii}{\renewcommand{\labelenumii}{(\roman{enumii})}}
\newcommand{\romaniii}{\renewcommand{\labelenumiii}{(\roman{enumiii})}}
\newcommand{\arabici}{\renewcommand{\labelenumi}{(\arabic{enumi})}}
\newcommand{\arabicii}{\renewcommand{\labelenumii}{(\arabic{enumii})}}
\newcommand{\arabiciii}{\renewcommand{\labelenumiii}{(\arabic{enumiii})}}

\begin{document}

\noindent
On this homework we discuss a proof technique known as \textit{(mathematical) induction}:
in essence,
to prove that a certain statement holds for all natural numbers,
it suffices to (i) show that it holds for the number $1$, and (ii) show that if it holds for a given number $n$ then it also holds for $n+1$.

(As a technical aside,
in this class we formally define the set of natural numbers as a subset of the set of real numbers as follows.
Let $\bbR$ be the set of real numbers, that is, an ordered field that has the supremum property.
We say that a subset $\calS$ of $\bbR$ is an \textit{inductive set} if it has both the property that $1\in\calS$ and the property that if $x\in\calS$ then $x+1\in\calS$.
For example, the set $\bbR$ of all real numbers is an inductive set,
as is the set of all positive real numbers.
Note that by definition, any inductive set contains $1$, $2:=1+1$, $3:=2+1$, etc., namely every number obtained by adding $1$ to itself any finite number of times.
It is straightforward to show that the intersection of (any number of) inductive sets is itself an inductive set.
Formally, we define the set $\bbN$ of natural numbers to be the intersection of all inductive sets.
Thus, $\bbN$ is both a subset of any inductive set and is itself an inductive set.
The former implies that the aforementioned proof technique of induction is valid.
The latter implies that $\bbN$ contains $1$, $2$, $3$, etc., that is, every number that can be obtained by adding $1$ to itself any finite number of times.
Since the set $\set{1,2,3,\dotsc}$ of all such numbers is itself an inductive set,
we have that $\bbN$ equals this set, i.e., $\bbN=\set{1,2,3,\dotsc}$.)

For example,
we now use induction to prove that the following is true for every $n\in\bbN$:
\begin{equation}
\label{eq.sum}
\sum_{k=1}^{n}k
=\tfrac{n(n+1)}{2}.
\end{equation}
To see that~\eqref{eq.sum} holds when $n=1$ note $\sum_{k=1}^{1}k=1=\tfrac{1(1+1)}{2}$.
Next, if~\eqref{eq.sum} holds for a particular $n$ then it also holds when ``$n$" is replaced with $n+1$:
\begin{equation*}
\sum_{k=1}^{n+1}k
=\sum_{k=1}^{n}k+(n+1)
=\tfrac{n(n+1)}{2}+(n+1)
=(n+1)\bigparen{\tfrac n2+1}
=\tfrac{(n+1)(n+2)}{2}
=\tfrac{(n+1)[(n+1)+1]}{2}.
\end{equation*}
(In such proofs, the assumption that a given property holds for a given $n$ is called the \textit{inductive hypothesis}.)
To be clear, results proven with induction can sometimes be proven directly.
For example, \eqref{eq.sum} can alternatively be proven by writing two copies of this sum, one with its terms in increasing order and the other with its terms in decreasing order,
and then dividing the result by $2$:
\begin{align*}
\label{eq.sum}
2\sum_{k=1}^{n}k
&=\sum_{k=1}^{n}k+\sum_{k=1}^{n}k\\
&=[1+2+\dotsb+(n-1)+n]+[n+(n-1)+\dotsb+2+1]\\
&=(1+n)+[2+(n-1)]+\dotsb+[(n-1)+2]+(n+1)\\
&=(n+1)+(n+1)+\dotsb+(n+1)+(n+1)\\
&=n(n+1).
\end{align*}

\begin{enumerate}
\newpage
%%%%%%%%%%%%%%%%%%%%%%%%%%%%%%%%%%%%%%%%%%%%%%%%%%%%%%%%%%%%%%%%
\item
Use induction to prove that
$\displaystyle\sum_{k=1}^{n}k^2=\tfrac{n(n+1)(2n+1)}{6}$ for all $n\in\bbN$.

%%%%%%%%%%%%%%%%%%%%%%%%%%%%%%%%%%%%%%%%%%%%%%%%%%%%%%%%%%%%%%%%
\item
For any $x\geq -1$,
use induction to prove \textit{Bernoulli's inequality},
namely that for all $n\in\bbN$,
\begin{equation*}
(1+x)^n\geq 1+nx.
\end{equation*}

%%%%%%%%%%%%%%%%%%%%%%%%%%%%%%%%%%%%%%%%%%%%%%%%%%%%%%%%%%%%%%%%
\item
For all $x\neq 1$,
use induction to prove the \textit{geometric sum formula},
namely that $\displaystyle\sum_{k=0}^{n}x^k=\tfrac{1-x^{n+1}}{1-x}$ for all $n\in\bbN$.

%%%%%%%%%%%%%%%%%%%%%%%%%%%%%%%%%%%%%%%%%%%%%%%%%%%%%%%%%%%%%%%%
\item
Give a direct (noninductive) proof of the previous result,
namely that $\displaystyle\sum_{k=0}^{n}x^k=\tfrac{1-x^{n+1}}{1-x}$ for all $x\neq1$, $n\in\bbN$.

%%%%%%%%%%%%%%%%%%%%%%%%%%%%%%%%%%%%%%%%%%%%%%%%%%%%%%%%%%%%%%%%
\item
For any $n\in\bbN$, define the \textit{factorial} of $n$ as
\begin{equation*}
n!:=\prod_{k=1}^{n}k=(1)(2)\cdots(n).
\end{equation*}
That is, $1!=1$, $2!=(1)(2)=2$, $3!=(1)(2)(3)=6$, $4!=(1)(2)(3)(4)=24$, etc.
By convention, we also define the factorial of $0$ to be $0!=1$.
For any integers $n\geq k\geq0$,
the corresponding binomial coefficient is:
\begin{equation*}
\binom{n}{k}
:=\frac{n!}{k!(n-k)!}.
\end{equation*}
Though we do not prove this,
this number is the number of distinct $k$-element subsets of a set of cardinality $n$,
and so is often read aloud as ``$n$ choose $k$".

For any integers $n\geq k\geq 1$,
show that $\displaystyle\binom{n+1}{0}=\binom{n+1}{n+1}=1$ and that $\displaystyle\binom{n}{k-1}+\binom{n}{k}=\binom{n+1}{k}$.\medskip

\textit{Note:
You are not required to use induction to prove these facts.
Direct proofs are preferable.
These results mean that binomial coefficients are natural numbers, and are moreover given by Pascal's triangle:}
\begin{equation*}
\begin{array}{ccccccccc}
 & & & &1& & & & \\
 & & &1& &1& & & \\
 & &1& &2& &1& & \\
 &1& &3& &3& &1& \\
1& &4& &6& &4& &1\\
 & & & &\vdots
\end{array}
\end{equation*}

%%%%%%%%%%%%%%%%%%%%%%%%%%%%%%%%%%%%%%%%%%%%%%%%%%%%%%%%%%%%%%%%
\item
For all $x,y\in\bbR$,
use induction and the previous problem to prove the \textit{binomial theorem}:
for any $n\in\bbN$,
\begin{equation*}
(x+y)^n=\sum_{k=0}^{n}\binom{n}{k}x^ky^{n-k}.
\end{equation*}

%%%%%%%%%%%%%%%%%%%%%%%%%%%%%%%%%%%%%%%%%%%%%%%%%%%%%%%%%%%%%%%%
\item
Find the flaw in the following ``proof" by induction of the ``fact" that all horses are of the same color,
specifically the claim that for every $n\in\bbN$,
any $n$ horses are of the same color.
This is true when $n=1$, since any single horse is of a single color.
Next, for any $n\in\bbN$,
assume that any $n$ horses are of the same color,
and take any $n+1$ horses.
Number these horses as $1$, $2$, up to $n+1$.
By the inductive hypothesis, the horses numbered $\set{1,2,\dotsc,n}$ are of the same color, and the horses numbered $\set{2,\dotsc,n,n+1}$ are of the same color.
In particular, all $n+1$ horses are of the same color,
being that of horse $2$.
\end{enumerate}

\clearpage
\section*{Question 1.}
Let \(n:=1\) then,
\[\sum_{k=1}^{n} k^2 = (1)^2 = 1 = \frac{1(2)(3)}{6} = \frac{n(n+1)(2n+1)}{6}.\]
Next, let \(m:=n+1\), then,
\begin{align*}
    \sum_{k=1}^{m} k^2
    &= \frac{m(m+1)(2m+1)}{6} \\
    &= \frac{(n+1)((n+1)+1)(2(n+1)+1)}{6} \\
    &= \frac{(n+1)(n+2)(2n+3)}{6} \\
    &= \frac{2n^3+9n^2+13n+6}{6} \\
    &= \frac{6n^2+12n+6}{6} + \frac{2n^3+3n^2+n}{6} \\
    &= (n^2+2n+1) + \frac{n(2n^2+3n+1)}{6} \\
    &= (n+1)^2 + \frac{n(n+1)(2n+1)}{6} \\
    &= (n+1)^2 + \sum_{k=1}^{n} k^2 \\
    &= \sum_{k=1}^{n+1} k^2.
\end{align*}

\clearpage

\section*{Question 2.}
First, we evaluate the base case starting with, 
\(n:=1\),
\[1+x = (1+x)^1 = (1+x)^n \geq  1 + nx = 1 + (1)x = 1+x.\]
%
Next we evaluate the base case at \(n:=0\),
resolving the potential ambiguity of \(\mathbb{N}\),
\[1 = (1+x)^0 = (1+x)^n \geq 1 + nx = 1 + (0)x = 1 + 0 = 1.\]
%
Then to show the inductive case,
\(n+1\),
we start with the base case multiplied by \((1+x)\),
which recalls the assumption \(x\geq-1\) to maintain the inequality,
\begin{flalign*} &&
    (1+x)^n(1+x) &\geq (1 + nx)(1 + x) \\ &&
    &= 1 + nx + x + nx^2 \\ &&
    &= 1 + (n+1)x + nx^2 \hspace{5em} (\text{since }nx^2\geq0, n\in\mathbb{N},x\in\mathbb{R})\\ &&
    &\geq 1 + (n + 1)x.
\end{flalign*}
Thus via transitivity we have,
\[(1+x)^n(1+x) \geq 1 + (n+1)x\]
which is
\[(1+x)^{n+1} \geq 1 + (n+1)x.\]


\clearpage

\section*{Question 3.}

First, we consider the base cases \(n:=0\)
\[
    \sum_{k=0}^{0}x^k = x^0 = 1 
    = \frac{1}{1} = \frac{1-x}{1-x} = \frac{1-x^{0+1}}{1-x},
\]
and \(n:=1\)
\[
\sum_{k=0}^{1}x^k = x^0 + x^1 = 1+x 
= (1+x)\frac{1+x}{1+x} = \frac{(1-x)^2}{1-x} = \frac{1-x^{1+1}}{1-x}.
\]
Next, given the base case, we consider the inductive case \(m:=n+1\)
\begin{align*}
    \sum_{k=0}^{m}x^k
    = \sum_{k=0}^{n+1}x^k
    &= x^{n+1} + \sum_{k=0}^{n}x^k \\
    &= x^{n+1} + \frac{1-x^{n+1}}{1-x} \\
    &= \frac{x^{n+1}(1-x)}{1-x} + \frac{1-x^{n+1}}{1-x} \\
    &= \frac{x^{n+1}-x^{n+2}}{1-x} + \frac{1-x^{n+1}}{1-x} \\
    &= \frac{1-x^{n+2}}{1-x} \\
    &= \frac{1-x^{(n+1)+1}}{1-x} \\
    &= \frac{1-x^{m+1}}{1-x}.
\end{align*}

\clearpage
\section*{Question 4.}
One direct proof can be shown by first multiplying by \((1-x)\).
\begin{align*}
    (1-x) \sum_{k=0}^{n}x^k
    &= \sum_{k=0}^{n}x^k - x \sum_{k=0}^{n}x^k \\
    &= \sum_{k=0}^{n}x^k - \sum_{k=0}^{n}x^{k+1} \\
    &= \sum_{k=0}^{n}x^k - \sum_{k=1}^{n+1}x^k \\
    &= x^0 + \sum_{k=1}^{n}x^k - \sum_{k=1}^{n}x^k - x^{n+1} \\
    &= x^0 - x^{n+1} \\
    &= 1 - x^{n+1}.
\end{align*}
Dividing the last expression by \((1-x)\) completes the proof.

\section*{Question 5.}

For the first identity, we can see equality with each side and \(1\) as follows,
\begin{align*}
    \binom{n+1}{0}
    &= \frac{(n+1)!}{(0)!((n+1)-(0))!}
    = \frac{(n+1)!}{(1)(n+1)!}
    = 1 \\ \\ 1
    &= \frac{(n+1)!}{(n+1)!}
    = \frac{(n+1)!}{(n+1)!(1)}
    = \frac{(n+1)!}{(n+1)!(0)!}
    = \frac{(n+1)!}{(n+1)!((n+1)-(n+1))!}
    = \binom{n+1}{n+1}.
\end{align*}
The second identity can be shown as,
\begin{align*}
    \binom{n}{k-1} + \binom{n}{k}
    &= \frac{n!}{(k-1)!(n-k+1)!} + \frac{n!}{(k)!(n-k)!} \\
    &= \frac{n!(k)}{k!(n-k+1)!} + \frac{n!(n-k+1)}{(k)!(n-k+1)!} \\
    &= \frac{n!(k)+ n!(n-k+1)}{k!(n-k+1)!} \\
    &= \frac{n!(n-k+1+k)}{k!(n-k+1)!} \\
    &= \frac{n!(n+1)}{k!(n-k+1)!} \\
    &= \frac{(n+1)!}{k!((n+1)-k)!} \\
    &= \binom{n+1}{k}.
\end{align*}

\clearpage
\section*{Question 6.}

For base case \(n=1\) we have,
\[
    (x+y)^1
    = x + y
    = \binom{1}{1}x + \binom{1}{0}y
    = \sum_{k=0}^{1} \binom{1}{k} x^{k} y^{n-k}.
\]
Next, we evaluate the inductive case \(n+1\),
\begin{align*}
    (x+y)^{n+1}
    &= (x+y)(x+y)^{n} \\
    &= (x+y)\sum_{k=0}^{n} \binom{1}{k} x^{k} y^{n-k} & \text{(assuming base-case)}\\
    &= x\sum_{k=0}^{n} \binom{n}{k} x^{k} y^{n-k} + y\sum_{k=0}^{n} \binom{n}{k} x^{k} y^{n-k} \\
    &= \sum_{k=0}^{n} \binom{n}{k} x^{k+1} y^{n-k} + \sum_{k=0}^{n} \binom{n}{k} x^{k} y^{n-k+1} \\
    &= x^{n+1} + \sum_{k=0}^{n-1} \binom{n}{k} x^{k+1} y^{n-k}
        + y^{n+1} + \sum_{k=1}^{n} \binom{n}{k} x^{k} y^{n-k+1}
        & \text{(\(k=n\) and \(k=0\) respectively)} \\
    &= x^{n+1} + y^{n+1} + \sum_{k=1}^{n} \binom{n}{k-1} x^{(k-1)+1} y^{n-(k-1)}
        + \sum_{k=1}^{n} \binom{n}{k} x^{k} y^{n-k+1} & \text{(shift index on first sum)} \\
    &= x^{n+1} + y^{n+1} + \sum_{k=1}^{n} \binom{n}{k-1} x^{k} y^{n-k+1}
        + \sum_{k=1}^{n} \binom{n}{k} x^{k} y^{n-k+1} \\
    &= x^{n+1} + y^{n+1} + \sum_{k=1}^{n} \binom{n}{k-1} + \binom{n}{k} x^{k} y^{n-k+1} \\
    &= x^{n+1} + y^{n+1} + \sum_{k=1}^{n} \binom{n+1}{k} x^{k} y^{(n+1)-k} &\text{(Problem 5.2)} \\
    &= \binom{n+1}{n+1}x^{n+1} + \binom{n+1}{0}y^{n+1} 
        + \sum_{k=1}^{n} \binom{n+1}{k} x^{k} y^{(n+1)-k} & \text{(Problem 5.1)} \\
    &= \sum_{k=0}^{n+1} \binom{n+1}{k} x^{k} y^{(n+1)-k} & \text{(\(k=0\) and \(k=n+1\) terms)} \\
\end{align*}
Concluding the proof.

\clearpage
\section*{Question 7.}

One possible critique of this example proof
is a problem with the ambiguity in the definition and application of the function.

Adding some precision to the function is necessary;
similarities between two things are generally only interesting if they are distinct.
That is, if the color of horse \(x\) is \(f(x)\), then we might interpret all horses 
being the same color as \(f(x_1)=f(x_2) \forall\ x_1,x_2\in \{x_i\}_{i\in n}\).
This is vacuously true for the base case, 
which might be considered pathological for the assumed intent of the problem.
The base case is indistinct from the assertion,
\(f(x_1)=f(x_1) \forall\ x_1\in \{x_i\}_{i\in n}\).
This version may be used to (correctly) arrive at the wholly uninteresting conclusion
that all horses are the same color as themselves.

By explicitly omitting comparison of the same horse we might model the question instead as,
\(f(x_1)=f(x_2), x_1\neq x_2 \forall\ x_1,x_2\in \{x_i\}_{i\in n}\).
Under this representation the pathology of the \(n=1\) case should become more clear.
Thus it becomes necessary to prove the base case for \(n\geq 2\);
that every two horses share the same color.


\end{document} 