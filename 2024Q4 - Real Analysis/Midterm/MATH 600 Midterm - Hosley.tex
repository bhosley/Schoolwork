\documentclass[12pt,letterpaper]{exam}
\usepackage[utf8]{inputenc}
\usepackage[T1]{fontenc}
\usepackage[width=8.50in, height=11.00in, left=0.50in, right=0.50in, top=0.50in, bottom=0.50in]{geometry}

\usepackage{libertine}
\usepackage{multicol}
\usepackage[shortlabels]{enumitem}

\usepackage{booktabs}
\usepackage[table]{xcolor}

\usepackage{amssymb}
\usepackage{amsthm}
\usepackage{mathtools}
\usepackage{bbm}

\usepackage{hyperref}
\usepackage{graphicx}
%\usepackage{wrapfig}
%\usepackage{capt-of}
%\usepackage{tikz}
%\usepackage{pgfplots}
%\usetikzlibrary{shapes,arrows,positioning,patterns}
%\usepackage{pythonhighlight}

%\renewcommand{\thequestion}{\textbf{\chapter.\arabic{question}}}
%\renewcommand{\questionlabel}{\thequestion}

%%%%%%%%%%%%%%%%%%%%%%%%%%%%%%%%%%%%%%%%%%%%%%%%%%%%%%%%%%%%%%%%%%
\newcommand{\class}{MATH 600} % This is the name of the course 
\newcommand{\assignmentname}{Midterm} % 
\newcommand{\authorname}{Hosley, Brandon} % 
\newcommand{\workdate}{\today} % 
\printanswers % this includes the solutions sections
%%%%%%%%%%%%%%%%%%%%%%%%%%%%%%%%%%%%%%%%%%%%%%%%%%%%%%%%%%%%%%%%%%

\usepackage{mathptmx}
\SetSymbolFont{letters}{bold}{OML}{cmm}{b}{it}
\SetSymbolFont{operators}{bold}{OT1}{cmr}{bx}{n}
\DeclareMathAlphabet{\mathcal}{OMS}{cmsy}{m}{n}

\begin{document}
\pagestyle{plain}
\thispagestyle{empty}
\noindent

%%%%%%%%%%%%%%%%%%%%%%%%%%%%%%%%%%%%%%%%%%%%%%%%%%%%%%%%%%%%%%%%%%%%%%%%%%%%%%%%%%%
\noindent
\begin{tabular*}{\textwidth}{l @{\extracolsep{\fill}} r @{\extracolsep{10pt}} l}
	\textbf{\class} & \textbf{\authorname}  &\\ %Your name here instead, obviously 
	\textbf{\assignmentname } & \textbf{\workdate} & \\
\end{tabular*}\\ 
\rule{\textwidth}{2pt}
%%%%%%%%%%%%%%%%%%%%%%%%%%%%%%%% HEADER %%%%%%%%%%%%%%%%%%%%%%%%%%%%%%%%%%%%%%%%%%%

Instructions: Answers all five problems. Each is worth four points,
for a total of twenty points. You have three hours to take the exam.
You can use any results from Sections 1-4 of our notes and 
Homework Solutions 1-4 without further justification. 
You are moreover permitted to consult any books and notes that you wish,
but if your arguments rely on results from them,
you should cite those results and moreover explain (prove) them
from basic principles and/or results that we have already covered.
Interactive resources are prohibited (including
conversations with anyone besides the instructor, search engines, AIs, etc.).

\begin{questions}
\question
Use mathematical induction to prove that
\(\overset{n}{\underset{k=1}{\sum}}\frac{1}{k(k+1)} = \frac{n}{n+1}\)
for any natural number \(n\).

\begin{solution}
    First, let us evaluate a base case in which \(n=1\).
    \begin{align*}
        \sum_{k=1}^{n} \frac{1}{k(k+1)}
        &= \frac{1}{1(1+1)}
        &= \frac{1}{(1+1)}
        &= \frac{n}{(n+1)}.
    \end{align*}
    Next, we evaluate the inductive case of \(n+1\) to conclude the proof,
    \begin{align*}
        \sum_{k=1}^{n+1} \frac{1}{k(k+1)}
        &= \frac{1}{(n+1)((n+1)+1)} + \sum_{k=1}^{n} \frac{1}{k(k+1)} \\
        &= \frac{1}{(n+1)(n+2)} + \frac{n}{(n+1)} \hspace{4em}\text{(base case)}\\
        &= \frac{n(n+2) + 1}{(n+1)(n+2)} \\
        &= \frac{n^2 + 2n + 1}{(n+1)(n+2)} \\
        &= \frac{(n+1)^2}{(n+1)(n+2)} \\
        &= \frac{(n+1)}{(n+1)+1}.
    \end{align*}
\end{solution}
%%%%%%%%%%%%%%%%%%%%%%%%%%%%%%%%%%%%%%%%%%%%%%%%%%%%%%%%%%%%%
\clearpage

\question
A function \(f:\mathbb{R}^N \rightarrow \mathbb{R}\) is said to be convex if for every 
\(x_1,x_2 \in\mathbb{R}^N\) and every \(t \in [0,1]\),
\[f(tx_1 + (1-t)x2) \leq t f(x_1)+ (1-t)f(x_2).\]
Show that each norm on \(\mathbb{R}^N\) is convex. 
That is, recalling the definition of a norm on \(\mathbb{R}^N\) given on Homework 3, show that 
the function \(f:\mathbb{R}^N \rightarrow \mathbb{R}, f(x) := \|x\|\) has the above property.
\begin{solution}
    To show that an arbitrary norm in \(\mathbb{R}^N\) is convex,
    we may use the properties of a norm.
    Using properties 1 (\(\|cx\| = |c|\|x\| \)) and 3 (\(\|x_1 + x_2\| \leq \|x_1\| + \|x_2\|\))
    and noting that \(t,(1-t)\geq0\),
    we can see,
    \begin{align}
        \|x_1 + x_2\|
        &= \|tx_1 + (1-t)x_1 + tx_2 + (1-t)x_2\| \\
        &= \|tx_1 + (1-t)x_2 + tx_2 + (1-t)x_1\| \\ 
        &\leq \|tx_1 + (1-t)x_2\| + \|tx_2 + (1-t)x_1\| \\
        &\leq \|tx_1\| + \|(1-t)x_2\| + \|tx_2\| + \|(1-t)x_1\| \\
        &= t\|x_1\| + (1-t)\|x_2\| + t\|x_2\| + (1-t)\|x_1\| \\
        &= t\|x_1\| + (1-t)\|x_1\| + (1-t)\|x_2\| + t\|x_2\| \\
        &= \|x_1\| + \|x_2\|.
    \end{align}
    From this series of expressions, we take 3 and 6,
    \[\|tx_1 + (1-t)x_2\| + \|tx_2 + (1-t)x_1\| 
        \leq t\|x_1\| + (1-t)\|x_1\| + (1-t)\|x_2\| + t\|x_2\|.\]
    To see this more clearly we may subtract
    \[ \|tx_2 + (1-t)x_1\| \leq \|tx_2\| + \|(1-t)x_1\| = t\|x_2\| + (1-t)\|x_1\|.\]
    Showing that not only does the convexity property hold,
    it is implied by the other properties of a norm.
\end{solution}
%%%%%%%%%%%%%%%%%%%%%%%%%%%%%%%%%%%%%%%%%%%%%%%%%%%%%%%%%%%%%
\clearpage

\question
The \emph{characteristic function} of a given subset \(\mathcal{S}\) of a set \(\mathcal{X}\) is
\[
    \chi\mathcal{S}: \mathcal{X}\rightarrow\mathbb{R},\quad
    \chi\mathcal{S}(x)=
    \begin{cases}
        1, & x\in \mathcal{S},\\
        0, & x\notin \mathcal{S}.
    \end{cases}
\]
For any subsets \(\mathcal{A}\), \(\mathcal{B}\) and \(\mathcal{C}\) of \(\mathcal{X}\), show that
\[
    \chi\mathcal{A}\cup\mathcal{B}\cup\mathcal{C}
    = \chi\mathcal{A} + \chi\mathcal{B} + \chi\mathcal{C}
    - \chi\mathcal{A}\cap\mathcal{B} - \chi\mathcal{A}\cap\mathcal{C} 
    - \chi\mathcal{B}\cap\mathcal{C} + \chi\mathcal{A}\cap\mathcal{B}\cap\mathcal{C}.
\]
\emph{Hint: Without further justification, you may use two basic facts about such functions,
namely that the characteristic function of the complement of a subset \(\mathcal{S}\)
of \(\mathcal{X}\) is obtained by subtracting that of \(\mathcal{S}\) from the
constant function 1, and that the characteristic function of the intersection of two subsets
\(\mathcal{A}\) and \(\mathcal{B}\) of \(\mathcal{X}\) is
the termwise product of their characteristic functions, namely that}
\[
    \chi\mathcal{S}^c(x) = 1-\chi\mathcal{S}(x), \quad
    \chi\mathcal{A}\cap\mathcal{B}(x) = \chi\mathcal{A}(x) \chi\mathcal{B}(x), \quad
    \forall x\in\mathcal{X}.
\]
\begin{solution}
    To show this equality, we can leverage DeMorgan's law 
    and the provided identities as follows:
    \begin{align*}
        \chi\mathcal{A}\cup\mathcal{B}\cup\mathcal{C}
        &= \left(\left(\chi\mathcal{A}\cup\mathcal{B}\cup\mathcal{C}\right)^c\right)^c\\
        &= 1-\left(\chi\mathcal{A}^c\cap\mathcal{B}^c\cap\mathcal{C}^c\right)\\
        &= 1-\chi\mathcal{A}^c\chi\mathcal{B}^c\chi\mathcal{C}^c\\
        &= 1-(1-\chi\mathcal{A})(1-\chi\mathcal{B})(1-\chi\mathcal{C})\\
        &= 1-\left(1 - \chi\mathcal{A} - \chi\mathcal{B} - \chi\mathcal{A}
            + \chi\mathcal{B}\chi\mathcal{C} + \chi\mathcal{A}\chi\mathcal{B}
            + \chi\mathcal{A}\chi\mathcal{C} 
            -\chi\mathcal{A}\chi\mathcal{B}\chi\mathcal{C}\right)\\
        &= \chi\mathcal{A} + \chi\mathcal{B} + \chi\mathcal{A}
            - \chi\mathcal{B}\chi\mathcal{C} - \chi\mathcal{A}\chi\mathcal{B}
            - \chi\mathcal{A}\chi\mathcal{C}
            + \chi\mathcal{A}\chi\mathcal{B}\chi\mathcal{C} \\
        &= \chi\mathcal{A} + \chi\mathcal{B} + \chi\mathcal{A}
            - \chi\mathcal{B}\cap\mathcal{C} - \chi\mathcal{A}\cap\mathcal{B}
            - \chi\mathcal{A}\cap\mathcal{C}
            + \chi\mathcal{A}\cap\mathcal{B}\cap\mathcal{C}.\\
    \end{align*}
    
\end{solution}
%%%%%%%%%%%%%%%%%%%%%%%%%%%%%%%%%%%%%%%%%%%%%%%%%%%%%%%%%%%%%
\clearpage

\question
From Homework 2, recall Bernoulli's inequality, 
namely that \((1+x)^n \geq 1+nx\) for any \(x \geq -1\) and \(n \in\mathbb{N}\).
Using this inequality, show that if \(0 < y < 1\) then 
\(\lim\limits_{n\rightarrow\infty} y^n=0\).\\
\emph{Hint: Let } \(x = \frac{1}{y}-1\).
\begin{solution}
    \emph{I struggled to find a method to use Bernoulli's inequality for this. I instead tried
    a different substitution.} \\[1em]
    For \(y\in(0,1)\) there exists an \(x\geq1\)
    such that \(x=\frac{1}{y}\).
    Then
    \(\lim\limits_{n\rightarrow\infty} y^n 
    = \lim\limits_{n\rightarrow\infty} \left(\frac{1}{x}\right)^n
    = \lim\limits_{n\rightarrow\infty} \frac{1}{x^n}\).
    Thus it suffices to show \(\lim\limits_{n\rightarrow\infty} \frac{1}{x^n}=0\)

    Note that for any \(n\geq N, x\geq 1\)
    we have \(x^n \geq x^N\)
    and thus, \(\frac{1}{x^n} \leq\frac{1}{x^N}\).

    Then, take \(\varepsilon>0\).
    There exists \(N\) such that
    \(\text{d}\left(\frac{1}{x^N},0\right) = \frac{1}{x^N} < \varepsilon\).
    Then for any \(n\geq N\) we have,
    \(\text{d}\left(\frac{1}{x^n},0\right) = \frac{1}{x^n}\leq\frac{1}{x^N} <\epsilon\).

\end{solution}
%%%%%%%%%%%%%%%%%%%%%%%%%%%%%%%%%%%%%%%%%%%%%%%%%%%%%%%%%%%%%
\clearpage

\question
Let \((\mathcal{X},\text{d})\) be a complete metric space. For each \(n\in\mathbb{N}\),
let \(\mathcal{S}_n\) be a closed and nonempty subset of \(\mathcal{X}\).
Further suppose that \(\mathcal{S}_{n+1} \subseteq \mathcal{S}_n\) for all \(n\in\mathbb{N}\),
and that there exists a sequence \((z_n)_{n=1}^{\infty}\) of nonnegative real
numbers such that d\((x,y)\leq z_n\) for all for all \(n\in\mathbb{N}\) and
\(x,y\in\mathcal{S}_n\), and such that \(\lim\limits_{n\rightarrow\infty} z_n =0\). \\
Show that there exists a unique element \(x_\infty\) of \(\mathcal{X}\) such that
\(\overset{\infty}{\underset{n=1}{\bigcap}}\mathcal{S}_n = \{x_\infty\}\).

\begin{solution}
    Since \(\mathcal{S}_{n+1} \subseteq \mathcal{S}_n\)
    we also have,
    \(\mathcal{S}_N = \cap_{n=1}^N \mathcal{S}_n\).
    Additionally, because
    d\((x,y)\leq z_n\) for all \(x,y\in\mathcal{S}_n\),
    we consequently have,
    \[\mathcal{S}_n \subset \text{b}(x,z_n)\quad \forall x\in \mathcal{S}_n.\]


    Next, because \((z_n)_{n=1}^{\infty}\) is a convergent sequence,
    for all \(\delta>0\) 
    there exists a \(N\in\mathbb{N}\)
    such that for all \(n>N\), we have
    \(z_n \leq z_N \leq \delta\)
    thus,
    \[\text{b}(x,z_n) \subseteq \text{b}(x,z_N) \subseteq \text{b}(x,\delta)
        \forall x\in \mathcal{S}_n.\]

    Let \(\mathcal{S}_\infty = \cap_{n=1}^\infty \mathcal{S}_n\).
    Then,
    \(\mathcal{S}_\infty \subset \text{b}(x,z_\infty)\)
    \vspace{2em}

    For this problem we wish to show that given a \(\mathcal{S}_n \subset \text{b}(x,z_n)\)
    that the convergent sequence \((z_n)_{n=1}^{\infty}\) contains a unique element.
    Given that the sequence \((z_n)_{n=1}^{\infty}\) converges to 0,
    we know that \(z_n \leq z_N\) for all \(n\geq N\).
    Thus, \(\text{b}(x,z_n) \subseteq \text{b}(x,z_N)\).

    Next, let \(\mathcal{S}_\infty = \cap_{n=1}^\infty \mathcal{S}_n\).
    \(\mathcal{S}_\infty\) is then the closed set that is contained within
    \(\text{b}(x,z_\infty)\).
    For the sake of contradiction,
    let \(x_1,x_2\) be distinct elements in \(\mathcal{S}_\infty\).
    Then d\((x_1,x_2)\leq z_\infty = 0\).
    Therefore, d\((x_1,x_2)=0\) which means that \(x_1=x_2\),
    and will be that element \(x_\infty\).
\end{solution}
%%%%%%%%%%%%%%%%%%%%%%%%%%%%%%%%%%%%%%%%%%%%%%%%%%%%%%%%%%%%%


\end{questions}
\end{document}
