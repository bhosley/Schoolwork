\documentclass[12 pt,letterpaper]{article}
\usepackage{amsmath}
\usepackage{amssymb}
\usepackage{amsthm}
\usepackage[left=0.5in, right=0.5in, bottom=0.75in, top=0.75in]{geometry}

\usepackage{mathptmx}
\SetSymbolFont{letters}{bold}{OML}{cmm}{b}{it}
\SetSymbolFont{operators}{bold}{OT1}{cmr}{bx}{n}
\DeclareMathAlphabet{\mathcal}{OMS}{cmsy}{m}{n}

\allowdisplaybreaks

\usepackage{lastpage}
\usepackage{fancyhdr}
\pagestyle{fancy}
\fancyhead[L]{Homework 1}
\fancyhead[C]{Due: October 11, 2024}
\fancyhead[R]{Page \thepage\ of \pageref{LastPage}}
\fancyfoot[L]{Air Force Institute of Technology}
\fancyfoot[R]{Fall 2024 (Hosley)}
\fancyfoot[C]{MATH 600: Mathematical Analysis}
\renewcommand{\headrulewidth}{0.25pt}
\renewcommand{\footrulewidth}{0.25pt}

\newcommand{\la}{\lambda}

\newcommand{\bbC}{\mathbb{C}}
\newcommand{\bbF}{\mathbb{F}}
\newcommand{\bbN}{\mathbb{N}}
\newcommand{\bbQ}{\mathbb{Q}}
\newcommand{\bbR}{\mathbb{R}}
\newcommand{\bbT}{\mathbb{T}}
\newcommand{\bbZ}{\mathbb{Z}}

\newcommand{\brI}{\mathbf{I}}
\newcommand{\brJ}{\mathbf{J}}
\newcommand{\brM}{\mathbf{M}}
\newcommand{\brR}{\mathbf{R}}
\newcommand{\brT}{\mathbf{T}}

\newcommand{\bfa}{\boldsymbol{a}}
\newcommand{\bfA}{\boldsymbol{A}}
\newcommand{\bfb}{\boldsymbol{b}}
\newcommand{\bfB}{\boldsymbol{B}}
\newcommand{\bfc}{\boldsymbol{c}}
\newcommand{\bfC}{\boldsymbol{C}}
\newcommand{\bfD}{\boldsymbol{D}}
\newcommand{\bfe}{\boldsymbol{e}}
\newcommand{\bfE}{\boldsymbol{E}}
\newcommand{\bff}{\boldsymbol{f}}
\newcommand{\bfF}{\boldsymbol{F}}
\newcommand{\bfG}{\boldsymbol{G}}
\newcommand{\bfH}{\boldsymbol{H}}
\newcommand{\bfI}{\boldsymbol{I}}
\newcommand{\bfJ}{\boldsymbol{J}}
\newcommand{\bfK}{\boldsymbol{K}}
\newcommand{\bfL}{\boldsymbol{L}}
\newcommand{\bfM}{\boldsymbol{M}}
\newcommand{\bfp}{\boldsymbol{p}}
\newcommand{\bfP}{\boldsymbol{P}}
\newcommand{\bfR}{\boldsymbol{R}}
\newcommand{\bfT}{\boldsymbol{T}}
\newcommand{\bfu}{\boldsymbol{u}}
\newcommand{\bfU}{\boldsymbol{U}}
\newcommand{\bfv}{\boldsymbol{v}}
\newcommand{\bfV}{\boldsymbol{V}}
\newcommand{\bfw}{\boldsymbol{w}}
\newcommand{\bfW}{\boldsymbol{W}}
\newcommand{\bfx}{\boldsymbol{x}}
\newcommand{\bfX}{\boldsymbol{X}}
\newcommand{\bfy}{\boldsymbol{y}}
\newcommand{\bfY}{\boldsymbol{Y}}
\newcommand{\bfz}{\boldsymbol{z}}
\newcommand{\bfZ}{\boldsymbol{Z}}

\newcommand{\bfone}{\boldsymbol{1}}
\newcommand{\bfzero}{\boldsymbol{0}}
\newcommand{\bfxi}{\boldsymbol{\xi}}
\newcommand{\bfXi}{\boldsymbol{\Xi}}
\newcommand{\bfphi}{\boldsymbol{\varphi}}
\newcommand{\bfPhi}{\boldsymbol{\varPhi}}
\newcommand{\bfPi}{\boldsymbol{\varPi}}
\newcommand{\bfpsi}{\boldsymbol{\psi}}
\newcommand{\bfPsi}{\boldsymbol{\varPsi}}
\newcommand{\bfchi}{\boldsymbol{\chi}}
\newcommand{\bfzeta}{\boldsymbol{\zeta}}
\newcommand{\bfdelta}{\boldsymbol{\delta}}
\newcommand{\bfDelta}{\boldsymbol{\varDelta}}
\newcommand{\bfGamma}{\boldsymbol{\varGamma}}
\newcommand{\bfOmega}{\boldsymbol{\varOmega}}
\newcommand{\bfSigma}{\boldsymbol{\varSigma}}
\newcommand{\bfLambda}{\boldsymbol{\varLambda}}

\newcommand{\calA}{\mathcal{A}}
\newcommand{\calB}{\mathcal{B}}
\newcommand{\calC}{\mathcal{C}}
\newcommand{\calD}{\mathcal{D}}
\newcommand{\calE}{\mathcal{E}}
\newcommand{\calF}{\mathcal{F}}
\newcommand{\calG}{\mathcal{G}}
\newcommand{\calH}{\mathcal{H}}
\newcommand{\calI}{\mathcal{I}}
\newcommand{\calK}{\mathcal{K}}
\newcommand{\calM}{\mathcal{M}}
\newcommand{\calN}{\mathcal{N}}
\newcommand{\calS}{\mathcal{S}}
\newcommand{\calT}{\mathcal{T}}
\newcommand{\calU}{\mathcal{U}}
\newcommand{\calV}{\mathcal{V}}
\newcommand{\calX}{\mathcal{X}}
\newcommand{\calY}{\mathcal{Y}}

\newcommand{\rmA}{\mathrm{A}}
\newcommand{\rmB}{\mathrm{B}}
\newcommand{\rmc}{\mathrm{c}}
\newcommand{\rmC}{\mathrm{C}}
\newcommand{\rmd}{\mathrm{d}}
\newcommand{\rme}{\mathrm{e}}
\newcommand{\rmi}{\mathrm{i}}
\newcommand{\rmj}{\mathrm{j}}
\newcommand{\rmk}{\mathrm{k}}
\newcommand{\rms}{\mathrm{s}}
\newcommand{\rmS}{\mathrm{S}}
\newcommand{\rmT}{\mathrm{T}}

\newcommand{\Tr}{\mathrm{Tr}}
\newcommand{\Part}{\mathrm{part}}
\newcommand{\Null}{\mathrm{null}}
\newcommand{\Span}{\operatorname{span}}
\newcommand{\rank}{\operatorname{rank}}
\newcommand{\real}{\operatorname{Re}}
\newcommand{\imag}{\operatorname{Im}}
\newcommand{\cond}{\operatorname{cond}}

\newcommand{\im}{\operatorname{im}}

\newcommand{\conv}[2]{{#1}\ast{#2}}
\newcommand{\argmin}[1]{\underset{#1}{\mathrm{argmin}}}
\newcommand{\argmax}[1]{\underset{#1}{\mathrm{argmax}}}

\newcommand{\mat}[2]{\left[\begin{array}{#1}#2\end{array}\right]}
\newcommand{\dmat}[2]{\left|\begin{array}{#1}#2\end{array}\right|}
\newcommand{\arbset}[1]{\left\{{#1}\right\}}
\newcommand{\arbparen}[1]{\left({#1}\right)}
\newcommand{\sign}{\operatorname{sign}}

\newcommand{\abs}[1]{|{#1}|}
\newcommand{\bigabs}[1]{\bigl|{#1}\bigr|}
\newcommand{\Bigabs}[1]{\Bigl|{#1}\Bigr|}
\newcommand{\biggabs}[1]{\biggl|{#1}\biggr|}
\newcommand{\Biggabs}[1]{\Biggl|{#1}\Biggr|}

\newcommand{\paren}[1]{({#1})}
\newcommand{\bigparen}[1]{\bigl({#1}\bigr)}
\newcommand{\Bigparen}[1]{\Bigl({#1}\Bigr)}
\newcommand{\biggparen}[1]{\biggl({#1}\biggr)}
\newcommand{\Biggparen}[1]{\Biggl({#1}\Biggr)}

\newcommand{\bracket}[1]{[{#1}]}
\newcommand{\bigbracket}[1]{\bigl[{#1}\bigr]}
\newcommand{\Bigbracket}[1]{\Bigl[{#1}\Bigr]}
\newcommand{\biggbracket}[1]{\biggl[{#1}\biggr]}
\newcommand{\Biggbracket}[1]{\Biggl[{#1}\Biggr]}

\newcommand{\set}[1]{\{{#1}\}}
\newcommand{\bigset}[1]{\bigl\{{#1}\bigr\}}
\newcommand{\Bigset}[1]{\Bigl\{{#1}\Bigr\}}
\newcommand{\biggset}[1]{\biggl\{{#1}\biggr\}}
\newcommand{\Biggset}[1]{\Biggl\{{#1}\Biggr\}}

\newcommand{\norm}[1]{\|{#1}\|}
\newcommand{\bignorm}[1]{\bigl\|{#1}\bigr\|}
\newcommand{\Bignorm}[1]{\Bigl\|{#1}\Bigr\|}
\newcommand{\biggnorm}[1]{\biggl\|{#1}\biggr\|}
\newcommand{\Biggnorm}[1]{\Biggl\|{#1}\Biggr\|}

\newcommand{\ip}[2]{\langle{#1},{#2}\rangle}
\newcommand{\bigip}[2]{\bigl\langle{#1},{#2}\bigr\rangle}
\newcommand{\Bigip}[2]{\Bigl\langle{#1},{#2}\Bigr\rangle}
\newcommand{\biggip}[2]{\biggl\langle{#1},{#2}\biggr\rangle}
\newcommand{\Biggip}[2]{\Biggl\langle{#1},{#2}\Biggr\rangle}

\newcommand{\alphi}{\renewcommand{\labelenumi}{(\alph{enumi})}}
\newcommand{\alphii}{\renewcommand{\labelenumii}{(\alph{enumii})}}
\newcommand{\alphiii}{\renewcommand{\labelenumiii}{(\alph{enumiii})}}
\newcommand{\romani}{\renewcommand{\labelenumi}{(\roman{enumi})}}
\newcommand{\romanii}{\renewcommand{\labelenumii}{(\roman{enumii})}}
\newcommand{\romaniii}{\renewcommand{\labelenumiii}{(\roman{enumiii})}}
\newcommand{\arabici}{\renewcommand{\labelenumi}{(\arabic{enumi})}}
\newcommand{\arabicii}{\renewcommand{\labelenumii}{(\arabic{enumii})}}
\newcommand{\arabiciii}{\renewcommand{\labelenumiii}{(\arabic{enumiii})}}

\begin{document}

\noindent
Let $\calX$ be a set,
and let $\set{\calS_n}_{n\in\calN}$ be a collection of subsets of $\calX$, indexed by the members of an (arbitrary) set $\calN$.
The \textit{union} of $\set{\calS_n}_{n\in\calN}$ is defined to be the set of all points in $\calX$ that are contained in at least one subset $\calS_n$:
\begin{equation*}
\bigcup_{n\in\calN}\calS_n
:=\set{x\in\calX: \exists\, n_0\in\calN\text{ such that }x\in\calS_{n_0}}.
\end{equation*}
Meanwhile, the \textit{intersection} of $\set{\calS_n}_{n\in\calN}$ is defined to be the set of all points in $\calX$ that are contained in every $\calS_n$:
\begin{equation*}
\bigcap_{n\in\calN}\calS_n
:=\set{x\in\calX: x\in\calS_n\ \forall\,n\in\calN}.
\end{equation*}
For example, when $\calX=\bbR$, $\calN=\bbN=\set{1,2,3,\dotsc}$ and $\calS_n=(-\frac1n,\frac1n)=\set{x\in\bbR: -\frac1n<x<\frac1n}$ for each $n\in\bbN$,
\begin{equation*}
\bigcup_{n\in\calN}\calS_n
=\bigcup_{n=1}^\infty(-\tfrac1n,\tfrac1n)=(-1,1),
\quad
\bigcap_{n\in\calN}\calS_n
=\bigcap_{n=1}^\infty(-\tfrac1n,\tfrac1n)=\set{0}.
\end{equation*}

\begin{enumerate}
%%%%%%%%%%%%%%%%%%%%%%%%%%%%%%%%%%%%%%%%%%%%%%%%%%%%%%%%%%%%%%%%
\item
The \textit{complement} of a subset $\calS$ of $\calX$ is defined to be the set of all points in $\calX$ that are not members of $\calS$:
\begin{equation*}
\calS^\rmc:=\set{x\in\calX: x\notin\calS}.
\end{equation*}
Clearly, we have $(\calS^\rmc)^\rmc=\calS$ for any $\calS\subseteq\calX$.
Prove \textit{De Morgan's Laws}, namely that:
\begin{equation*}
\Bigparen{\,\bigcup_{n\in\calN}\calS_n}^\rmc
=\bigcap_{n\in\calN}\calS_n^\rmc,
\quad
\Bigparen{\,\bigcap_{n\in\calN}\calS_n}^\rmc
=\bigcup_{n\in\calN}\calS_n^\rmc.
\end{equation*}
\end{enumerate}

\noindent
A \textit{function} from $\calX$ to $\calY$ is a fixed rule that assigns an element $f(x)$ of $\calY$ to every element $x$ of $\calX$.
(More formally, a \textit{function} is a subset $\calF$ of
$\calX\times\calY:=\set{(x,y): x\in\calX, y\in\calY}$
that ``passes the vertical line test," that is,
has the property that if $(x_1,y_1),(x_2,y_2)\in\calF$ and $x_1=x_2$ then $y_1=y_2$.)
We usually denote such a function as ``$f:\calX\rightarrow\calY$,"
and refer to $\calX$ as the \textit{domain} of $f$ and to $\calY$ as the \textit{codomain} of $f$.
Meanwhile the \textit{image (range)} of $f$ is
\begin{equation*}
f(\calX)
:=\set{f(x): x\in\calX}
=\set{y\in\calY: \exists\, x\in\calX\text{ such that }f(x)=y},
\end{equation*}
which is a subset of $\calY$.

\begin{enumerate}
\setcounter{enumi}{1}
%%%%%%%%%%%%%%%%%%%%%%%%%%%%%%%%%%%%%%%%%%%%%%%%%%%%%%%%%%%%%%%%
\item
Let $f:\calX\rightarrow\calY$.
The \textit{image (under $f$)} of a subset $\calS$ of $\calX$ is the set
$f(\calS):=\set{f(x): x\in\calS}$.

For any subsets $\set{\calS_n}_{n\in\calN}$ of $\calX$, show that
\begin{equation*}
f\Bigparen{\,\bigcup_{n\in\calN}\calS_n}
=\,\bigcup_{n\in\calN}f(\calS_n),
\qquad
f\Bigparen{\,\bigcap_{n\in\calN}\calS_n}
\subseteq\bigcap_{n\in\calN}f(\calS_n).
\end{equation*}

Moreover, by means of an example, prove that the second statement can not be strengthened to an equality in general:
give an example of $f:\bbR\rightarrow\bbR$ and subsets $\calS_1,\calS_2$ of $\bbR$ such that $f(\calS_1\cap\calS_2)\neq f(\calS_1)\cap f(\calS_2)$.

%%%%%%%%%%%%%%%%%%%%%%%%%%%%%%%%%%%%%%%%%%%%%%%%%%%%%%%%%%%%%%%%
\item
Let $f:\calX\rightarrow\calY$.
The \textit{preimage (under $f$)} of a subset $\calS$ of $\calY$ is the set
$f^{-1}(\calS):=\set{x\in\calX: f(x)\in\calS}$.

\textit{Note: This notation is standard but can be misleading.
To be clear, we are NOT assuming here that $f$ is invertible and taking the image of $\calS$ under $f^{-1}$.}

Though this is not obvious, preimages behave more nicely with complements, unions and intersections than images do;
for any subsets $\calS$ and $\set{\calS_n}_{n\in\calN}$ of $\calY$, show that:
\begin{equation*}
f^{-1}(\calS^\rmc)
=[f^{-1}(\calS)]^\rmc,
\qquad
f^{-1}\Bigparen{\,\bigcup_{n\in\calN}\calS_n}
=\,\bigcup_{n\in\calN}f^{-1}(\calS_n),
\qquad
f^{-1}\Bigparen{\,\bigcap_{n\in\calN}\calS_n}
=\bigcap_{n\in\calN}f^{-1}(\calS_n).
\end{equation*}
Later on, we will use these facts to help understand \textit{continuous} functions.

\end{enumerate}

\clearpage
\section*{Question 1}

At risk of being unnecessarily verbose,
we will leverage three corollaries drawn from the definitions outlined in question 1.

\emph{Corollary 1.1:}
Because \(\mathcal{S}\) are defined as subsets of \(\mathcal{X}\),
then by the definition of complement,
\[x\notin\mathcal{S} \Leftrightarrow x\in\mathcal{S}^c\] 

\emph{Corollary 1.2:}
For an arbitrary element \(x_0\in\mathcal{X}\) and arbitrary \(\mathcal{S}\subseteq\mathcal{X}\),
the following statements are equivalent;
\begin{itemize}
    \item Element \(x_0\) is not within set \(\mathcal{S}\),
    \item Every element within \(\mathcal{S}\) is not equal to element \(x_0\),
    \item There does not exist any element in \(\mathcal{S}\) that is equal to \(x_0\).
\end{itemize}
\[
    x_0 \notin \mathcal{S}
    \quad \Leftrightarrow \quad
    x_0 \neq x \,\forall\ x \in \mathcal{S}
    \quad \Leftrightarrow \quad
    \nexists\ x \in \mathcal{S} \text{ s.t. } x = x_0
\]

\emph{Corollary 1.3:}
Combining the two previous corollaries we may further assert that the non-existence of element 
\(x_0\) within a set implies existence within the complimentary set,
\[
    \nexists\ x \in \mathcal{S} \text{ s.t. } x = x_0
    \quad \Leftrightarrow \quad
    x_0 \notin \mathcal{S}
    \quad \Leftrightarrow \quad
    x\in\mathcal{S}^c
    \quad \Leftrightarrow \quad
    \exists\ x \in \mathcal{S}^c \text{ s.t. } x = x_0
\]

Then, we demonstrate the first part of De Morgan's Law as follows,

\begin{align*}
    \left(\bigcup_{n\in\mathcal{N}} \mathcal{S}_n\right)^c
    &= \left\{x \in \mathcal{X} : \exists\, n_0 \in \mathcal{N}
        \text{ s.t. } x \in \mathcal{S}_{n_0}\right\}^c && \text{by definition of union}\\
    &= \left\{x \in \mathcal{X} : \nexists\, n_0 \in \mathcal{N}
        \text{ s.t. } x \in \mathcal{S}_{n_0}\right\} && \text{by Corollary 1.3}\\
    &= \left\{x \in \mathcal{X} : x \notin \mathcal{S}_{n}
        \,\forall\, n \in \mathcal{N} \right\} && \text{by Corollary 1.2}\\
    &= \left\{x \in \mathcal{X} : x \in \mathcal{S}_{n}^{c}
        \,\forall\, n \in \mathcal{N} \right\} && \text{by Corollary 1.1}\\
    &= \bigcap_{n\in\calN}\calS_n^\rmc, && \text{by definition of intersection}.
\end{align*}

The second part of DeMorgan's Law is shown by,

\begin{align*}
    \left(\bigcap_{n\in\mathcal{N}} \mathcal{S}_n\right)^c
    &= \left\{x \in \mathcal{X} : x \in \mathcal{S}_{n} 
        \,\forall\, n \in \mathcal{N}\right\}^c && \text{by definition of intersection} \\
    &= \left\{x \in \mathcal{X} : x \notin \mathcal{S}_{n}^{c} 
        \,\forall\, n \in \mathcal{N}\right\}^c && \text{by Corollary 1.1} \\
    &= \left\{x \in \mathcal{X} : \nexists\, n_0 \in \mathcal{N}
        \text{ s.t. } x \in \mathcal{S}_{n_0}^{c}\right\}^c && \text{by Corollary 1.2} \\
    &= \left\{x \in \mathcal{X} : \exists\, n_0 \in \mathcal{N}
        \text{ s.t. } x \in \mathcal{S}_{n_0}^{c}\right\} && \text{by Corollary 1.3} \\
    &= \bigcup_{n\in\calN}\calS_n^\rmc, && \text{by definition of union}.
\end{align*}

\clearpage
\section*{Question 2}

Let \(\mathcal{A} := \bigcup_{n\in\mathcal{N}} \mathcal{S}_n\), 
then \(\mathcal{S}_n \subseteq \mathcal{A}\ \forall n\in\mathcal{N}\).
That is, for arbitrary \(x\in\mathcal{X}, n\in\mathcal{N}\) such that 
\(x\in\mathcal{S}_n\) it is also the case that \(x\in\mathcal{A}\).
Explicating the elements of \(f(\mathcal{A})\) and elements of the union of \(f(\mathcal{S}_n)\)
we see that they are equivalent,
\begin{align*}
    \bigcup_{n\in\mathcal{N}} f(S_n) 
    &= \bigcup_{n\in\mathcal{N}} \left\{f(x): x\in\mathcal{S}_n \right\} \\
    &= \left\{f(x): \exists n\in\mathcal{N}\text{ s.t. }x\in\mathcal{S}_n\right\} \\
    &= \left\{f(x): x\in\mathcal{X}\exists n\in\mathcal{N}\text{ s.t. }x\in\mathcal{S}_n\right\} \\
    &= \left\{f(x): x\in\mathcal{A} \right\} \\
    &= f(\mathcal{A})
\end{align*}
\\
Next, let \(\mathcal{B} := \bigcap_{n\in\mathcal{N}} \mathcal{S}_n\).
Consequently, \(\forall n\in\mathcal{N},\ \mathcal{B} \subseteq \mathcal{S}_n\).
Which is, \[x\in\mathcal{B} \Rightarrow x\in\mathcal{S}_n,\ \forall n\in\mathcal{N},\]
thus, \[f(x)\in f(\mathcal{B}) \Rightarrow f(x)\in f(\mathcal{S}_n),\ \forall n\in\mathcal{N}.\]
Then, note that,
\[
    \{f(x)\in f(\mathcal{S}_n) \forall n\in\mathcal{N}\}
    = \bigcap_{n\in\mathcal{N}} f(\mathcal{S}_n),
\]
and we see that 
\[f(\mathcal{B}) \subseteq f(\bigcap_{n\in\mathcal{N}} f(\mathcal{S}_n)).\]
Next, we would show equality by completing mutual containment; showing that
\(f(\bigcap_{n\in\mathcal{N}} f(\mathcal{S}_n)) \subseteq f(\mathcal{B})\).
\\
However, this is not always the case, which will be shown in answer to the 
final part of the homework question.
\\[2em]
\emph{Counter Example:}

Let \(S_1:=\mathbb{N}=\mathbb{Z}^+_0\), \(S_2:=-\mathbb{N}=\mathbb{Z}^-_0\), and \(f(x) = \abs{x}\) 
(explicitly: the absolute value of x).
Then,
\[f(S_1\cap S_2) = \{0\} \quad \text{ and } \quad f(S_1)\cap f(S_2) = \mathbb{N} .\]


\clearpage
\section*{Question 3}

\setcounter{section}{3}
\subsection{}
For the first identity given in the question, the equality can be shown more explicitly
as follows:
\begin{align*}
    f^{-1}(\mathcal{S}^c)
    &= \left\{x\in\mathcal{X} : f^{-1}(x)\in\mathcal{S}^c\right\} \\
    &= \left\{x\in\mathcal{X}: f^{-1}(x)\notin\mathcal{S}\right\} \\
    &= \left\{x\in\mathcal{X}: f^{-1}(x)\in\mathcal{S}\right\}^c \\
    &= \left[f^{-1}(\mathcal{S})\right]^c.
\end{align*}

\subsection{}
Although the preimage concept itself is unintuitive, its behavior when using 
$f^{-1}(\calS):=\set{x\in\calX: f(x)\in\calS}$
makes showing the next two identities fairly straight-forward.
First,

\begin{align*}
    \bigcup_{n\in\mathcal{N}} f^{-1}\left(\mathcal{S}_n\right)
    &= \bigcup_{n\in\mathcal{N}} \left\{x\in\mathcal{X} : f^{-1}(x)\in\mathcal{S}_n\right\} \\
    &= \left\{x\in\mathcal{X} : \exists n_0\in\mathcal{N} \text{ s.t. }
    f^{-1}(x)\in\mathcal{S}_{n_0}\right\} \\
    &= \left\{x\in\mathcal{X} : f^{-1}(x)\in\bigcup_{n\in\mathcal{N}} \mathcal{S}_n\right\} \\
    &= f^{-1}\left(\bigcup_{n\in\mathcal{N}} \mathcal{S}_n\right).
\end{align*}

\subsection{}
Finally,

\begin{align*}
    \bigcap_{n\in\mathcal{N}} f^{-1}\left(\mathcal{S}_n\right)
    &= \bigcap_{n\in\mathcal{N}} \left\{x\in\mathcal{X} : f^{-1}(x)\in\mathcal{S}_n\right\} \\
    &= \left\{x\in\mathcal{X} : f^{-1}(x)\in\mathcal{S}_n \forall n\in\mathcal{N}\right\} \\
    &= \left\{x\in\mathcal{X} : f^{-1}(x)\in \bigcap_{n\in\mathcal{N}} \mathcal{S}_n \right\} \\
    &= f^{-1}\left(\bigcap_{n\in\mathcal{N}} \mathcal{S}_n\right).
\end{align*}

\end{document} 
