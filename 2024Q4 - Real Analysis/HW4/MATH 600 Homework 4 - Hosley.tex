\documentclass[12 pt,letterpaper]{article}
\usepackage{amsmath}
\usepackage{amssymb}
\usepackage{amsthm}
\usepackage[left=0.5in, right=0.5in, bottom=0.75in, top=0.75in]{geometry}

\usepackage{mathptmx}
\SetSymbolFont{letters}{bold}{OML}{cmm}{b}{it}
\SetSymbolFont{operators}{bold}{OT1}{cmr}{bx}{n}
\DeclareMathAlphabet{\mathcal}{OMS}{cmsy}{m}{n}

\allowdisplaybreaks

\usepackage{lastpage}
\usepackage{fancyhdr}
\pagestyle{fancy}
\fancyhead[L]{Homework 4}
\fancyhead[C]{Due: November 4, 2024}
\fancyhead[R]{Page \thepage\ of \pageref{LastPage}}
\fancyfoot[L]{Air Force Institute of Technology}
\fancyfoot[R]{Fall 2024 (Hosley)}
\fancyfoot[C]{MATH 600: Mathematical Analysis}
\renewcommand{\headrulewidth}{0.25pt}
\renewcommand{\footrulewidth}{0.25pt}

\newcommand{\la}{\lambda}

\newcommand{\bbC}{\mathbb{C}}
\newcommand{\bbF}{\mathbb{F}}
\newcommand{\bbN}{\mathbb{N}}
\newcommand{\bbQ}{\mathbb{Q}}
\newcommand{\bbR}{\mathbb{R}}
\newcommand{\bbT}{\mathbb{T}}
\newcommand{\bbZ}{\mathbb{Z}}

\newcommand{\brI}{\mathbf{I}}
\newcommand{\brJ}{\mathbf{J}}
\newcommand{\brM}{\mathbf{M}}
\newcommand{\brR}{\mathbf{R}}
\newcommand{\brT}{\mathbf{T}}

\newcommand{\bfa}{\boldsymbol{a}}
\newcommand{\bfA}{\boldsymbol{A}}
\newcommand{\bfb}{\boldsymbol{b}}
\newcommand{\bfB}{\boldsymbol{B}}
\newcommand{\bfc}{\boldsymbol{c}}
\newcommand{\bfC}{\boldsymbol{C}}
\newcommand{\bfD}{\boldsymbol{D}}
\newcommand{\bfe}{\boldsymbol{e}}
\newcommand{\bfE}{\boldsymbol{E}}
\newcommand{\bff}{\boldsymbol{f}}
\newcommand{\bfF}{\boldsymbol{F}}
\newcommand{\bfG}{\boldsymbol{G}}
\newcommand{\bfH}{\boldsymbol{H}}
\newcommand{\bfI}{\boldsymbol{I}}
\newcommand{\bfJ}{\boldsymbol{J}}
\newcommand{\bfK}{\boldsymbol{K}}
\newcommand{\bfL}{\boldsymbol{L}}
\newcommand{\bfM}{\boldsymbol{M}}
\newcommand{\bfp}{\boldsymbol{p}}
\newcommand{\bfP}{\boldsymbol{P}}
\newcommand{\bfR}{\boldsymbol{R}}
\newcommand{\bfT}{\boldsymbol{T}}
\newcommand{\bfu}{\boldsymbol{u}}
\newcommand{\bfU}{\boldsymbol{U}}
\newcommand{\bfv}{\boldsymbol{v}}
\newcommand{\bfV}{\boldsymbol{V}}
\newcommand{\bfw}{\boldsymbol{w}}
\newcommand{\bfW}{\boldsymbol{W}}
\newcommand{\bfx}{\boldsymbol{x}}
\newcommand{\bfX}{\boldsymbol{X}}
\newcommand{\bfy}{\boldsymbol{y}}
\newcommand{\bfY}{\boldsymbol{Y}}
\newcommand{\bfz}{\boldsymbol{z}}
\newcommand{\bfZ}{\boldsymbol{Z}}

\newcommand{\bfone}{\boldsymbol{1}}
\newcommand{\bfzero}{\boldsymbol{0}}
\newcommand{\bfxi}{\boldsymbol{\xi}}
\newcommand{\bfXi}{\boldsymbol{\Xi}}
\newcommand{\bfphi}{\boldsymbol{\varphi}}
\newcommand{\bfPhi}{\boldsymbol{\varPhi}}
\newcommand{\bfPi}{\boldsymbol{\varPi}}
\newcommand{\bfpsi}{\boldsymbol{\psi}}
\newcommand{\bfPsi}{\boldsymbol{\varPsi}}
\newcommand{\bfchi}{\boldsymbol{\chi}}
\newcommand{\bfzeta}{\boldsymbol{\zeta}}
\newcommand{\bfdelta}{\boldsymbol{\delta}}
\newcommand{\bfDelta}{\boldsymbol{\varDelta}}
\newcommand{\bfGamma}{\boldsymbol{\varGamma}}
\newcommand{\bfOmega}{\boldsymbol{\varOmega}}
\newcommand{\bfSigma}{\boldsymbol{\varSigma}}
\newcommand{\bfLambda}{\boldsymbol{\varLambda}}

\newcommand{\calA}{\mathcal{A}}
\newcommand{\calB}{\mathcal{B}}
\newcommand{\calC}{\mathcal{C}}
\newcommand{\calD}{\mathcal{D}}
\newcommand{\calE}{\mathcal{E}}
\newcommand{\calF}{\mathcal{F}}
\newcommand{\calG}{\mathcal{G}}
\newcommand{\calH}{\mathcal{H}}
\newcommand{\calI}{\mathcal{I}}
\newcommand{\calK}{\mathcal{K}}
\newcommand{\calM}{\mathcal{M}}
\newcommand{\calN}{\mathcal{N}}
\newcommand{\calS}{\mathcal{S}}
\newcommand{\calT}{\mathcal{T}}
\newcommand{\calU}{\mathcal{U}}
\newcommand{\calV}{\mathcal{V}}
\newcommand{\calX}{\mathcal{X}}
\newcommand{\calY}{\mathcal{Y}}

\newcommand{\rmA}{\mathrm{A}}
\newcommand{\rmB}{\mathrm{B}}
\newcommand{\rmc}{\mathrm{c}}
\newcommand{\rmC}{\mathrm{C}}
\newcommand{\rmd}{\mathrm{d}}
\newcommand{\rme}{\mathrm{e}}
\newcommand{\rmi}{\mathrm{i}}
\newcommand{\rmj}{\mathrm{j}}
\newcommand{\rmk}{\mathrm{k}}
\newcommand{\rms}{\mathrm{s}}
\newcommand{\rmS}{\mathrm{S}}
\newcommand{\rmT}{\mathrm{T}}

\newcommand{\Tr}{\mathrm{Tr}}
\newcommand{\Part}{\mathrm{part}}
\newcommand{\Null}{\mathrm{null}}
\newcommand{\Span}{\operatorname{span}}
\newcommand{\rank}{\operatorname{rank}}
\newcommand{\real}{\operatorname{Re}}
\newcommand{\imag}{\operatorname{Im}}
\newcommand{\cond}{\operatorname{cond}}

\newcommand{\inte}{\operatorname{int}}
\newcommand{\cl}{\operatorname{cl}}

\newcommand{\im}{\operatorname{im}}

\newcommand{\conv}[2]{{#1}\ast{#2}}
\newcommand{\argmin}[1]{\underset{#1}{\mathrm{argmin}}}
\newcommand{\argmax}[1]{\underset{#1}{\mathrm{argmax}}}

\newcommand{\mat}[2]{\left[\begin{array}{#1}#2\end{array}\right]}
\newcommand{\dmat}[2]{\left|\begin{array}{#1}#2\end{array}\right|}
\newcommand{\arbset}[1]{\left\{{#1}\right\}}
\newcommand{\arbparen}[1]{\left({#1}\right)}
\newcommand{\sign}{\operatorname{sign}}

\newcommand{\abs}[1]{|{#1}|}
\newcommand{\bigabs}[1]{\bigl|{#1}\bigr|}
\newcommand{\Bigabs}[1]{\Bigl|{#1}\Bigr|}
\newcommand{\biggabs}[1]{\biggl|{#1}\biggr|}
\newcommand{\Biggabs}[1]{\Biggl|{#1}\Biggr|}

\newcommand{\paren}[1]{({#1})}
\newcommand{\bigparen}[1]{\bigl({#1}\bigr)}
\newcommand{\Bigparen}[1]{\Bigl({#1}\Bigr)}
\newcommand{\biggparen}[1]{\biggl({#1}\biggr)}
\newcommand{\Biggparen}[1]{\Biggl({#1}\Biggr)}

\newcommand{\bracket}[1]{[{#1}]}
\newcommand{\bigbracket}[1]{\bigl[{#1}\bigr]}
\newcommand{\Bigbracket}[1]{\Bigl[{#1}\Bigr]}
\newcommand{\biggbracket}[1]{\biggl[{#1}\biggr]}
\newcommand{\Biggbracket}[1]{\Biggl[{#1}\Biggr]}

\newcommand{\set}[1]{\{{#1}\}}
\newcommand{\bigset}[1]{\bigl\{{#1}\bigr\}}
\newcommand{\Bigset}[1]{\Bigl\{{#1}\Bigr\}}
\newcommand{\biggset}[1]{\biggl\{{#1}\biggr\}}
\newcommand{\Biggset}[1]{\Biggl\{{#1}\Biggr\}}

\newcommand{\norm}[1]{\|{#1}\|}
\newcommand{\bignorm}[1]{\bigl\|{#1}\bigr\|}
\newcommand{\Bignorm}[1]{\Bigl\|{#1}\Bigr\|}
\newcommand{\biggnorm}[1]{\biggl\|{#1}\biggr\|}
\newcommand{\Biggnorm}[1]{\Biggl\|{#1}\Biggr\|}

\newcommand{\ip}[2]{\langle{#1},{#2}\rangle}
\newcommand{\bigip}[2]{\bigl\langle{#1},{#2}\bigr\rangle}
\newcommand{\Bigip}[2]{\Bigl\langle{#1},{#2}\Bigr\rangle}
\newcommand{\biggip}[2]{\biggl\langle{#1},{#2}\biggr\rangle}
\newcommand{\Biggip}[2]{\Biggl\langle{#1},{#2}\Biggr\rangle}

\newcommand{\alphi}{\renewcommand{\labelenumi}{(\alph{enumi})}}
\newcommand{\alphii}{\renewcommand{\labelenumii}{(\alph{enumii})}}
\newcommand{\alphiii}{\renewcommand{\labelenumiii}{(\alph{enumiii})}}
\newcommand{\romani}{\renewcommand{\labelenumi}{(\roman{enumi})}}
\newcommand{\romanii}{\renewcommand{\labelenumii}{(\roman{enumii})}}
\newcommand{\romaniii}{\renewcommand{\labelenumiii}{(\roman{enumiii})}}
\newcommand{\arabici}{\renewcommand{\labelenumi}{(\arabic{enumi})}}
\newcommand{\arabicii}{\renewcommand{\labelenumii}{(\arabic{enumii})}}
\newcommand{\arabiciii}{\renewcommand{\labelenumiii}{(\arabic{enumiii})}}

\usepackage{parskip}

\begin{document}

\noindent
Let $(\calX,\rmd)$ be a metric space.
From class, recall that a subset $\calS$ of $\calX$ is \textit{open} if for every $x\in\calS$, there exists $\delta>0$ such that
$\rmB(x,\delta)=\set{\hat{x}\in\calX: \rmd(x,\hat{x})<\delta}\subseteq\calS$.
Also recall that $\calS\subseteq\calX$ is \textit{closed} if
$\calS^\rmc=\set{x\in\calX: x\notin\calS}$ is open.

\begin{enumerate}
%%%%%%%%%%%%%%%%%%%%%%%%%%%%%%%%%%%%%%%%%%%%%%%%%%%%%%%%%%%%%%%%
\item
Show that a union of an arbitrary number of open sets is an open set.

That is, for each $n\in\calN$, let $\calS_n$ be any open set in $\calX$,
and show that $\displaystyle\bigcup_{n\in\calN}\calS_n$ is open.

%%%%%%%%%%%%%%%%%%%%%%%%%%%%%%%%%%%%%%%%%%%%%%%%%%%%%%%%%%%%%%%%
\item
Show that the intersection of a finite number of open sets is an open set.

That is, for each $n\in\set{1,\dotsc,N}$, let $\calS_n$ be any open set in $\calX$,
and show that $\displaystyle\bigcap_{n=1}^N\calS_n$ is open.

Also give an example of an infinite number of open sets in $\bbR$ whose intersection is not open.

\end{enumerate}

The results from Problems 1 and 2 can be restated in terms of closed sets:
in light of De Morgan's laws, the intersection of any number of closed sets is a closed set,
and the union of a finite number of closed sets is closed.
Moreover, a union of an infinite number of closed sets is not necessarily closed.
For example, $\bigcup_{n=1}^{\infty}[\tfrac1n,1]=(0,1]$.

In light of these ideas, we now define the \textit{interior} of a subset $\calS$ of $\calX$ to be the union of all the open sets that it contains,
and define the \textit{closure} of $\calS$ to be the intersection of all closed sets that contain it:
\begin{equation*}
\inte(\calS):=\bigcup_{\substack{\calE\subseteq\calS\\\calE\text{ is open}}}\calE,
\qquad
\cl(\calS):=\bigcap_{\substack{\calF\supseteq\calS\\\calF\text{ is closed}}}\calF.
\end{equation*}
Being a union of open subsets of $\calS$,
$\inte(\calS)$ is itself an open subset of $\calS$.
Moreover, by definition, any open set $\calE$ that is a subset of $\calS$ is also a subset of $\inte(\calS)$ (the union of all such sets).
Together, these facts imply that
$\inte(\calS)$ \textit{is the largest open set that $\calS$ contains}.
A similar argument shows $\cl(\calS)$ is the \textit{smallest closed set that contains $\calS$}.

\begin{enumerate}
\setcounter{enumi}{2}
%%%%%%%%%%%%%%%%%%%%%%%%%%%%%%%%%%%%%%%%%%%%%%%%%%%%%%%%%%%%%%%%
\item
Using De Morgan's laws,
show that $\cl(\calS)=[\inte(\calS^\rmc)]^\rmc$ for any subset $\calS$ of $\calX$.

\textit{Note: Applying this result with ``$\calS$" being $\calS^\rmc$ and then taking complements gives $\inte(\calS)=[\cl(\calS^\rmc)]^\rmc$.}

%%%%%%%%%%%%%%%%%%%%%%%%%%%%%%%%%%%%%%%%%%%%%%%%%%%%%%%%%%%%%%%%
\item
A point $x\in\calX$ is called an \textit{interior point} of a subset $\calS$ of $\calX$ if there exists $\delta>0$ such that $\rmB(x,\delta)\subseteq\calS$.

For any subset $\calS$ of $\calX$,
show that $x\in\inte(\calS)$ if and only if $x$ is an interior point of $\calS$.

\textit{Hint: From class, we know that open balls are open sets.}

%%%%%%%%%%%%%%%%%%%%%%%%%%%%%%%%%%%%%%%%%%%%%%%%%%%%%%%%%%%%%%%%
\item
A point $x\in\calX$ is called a \textit{limit point} of a subset $\calS$ of $\calX$ if for every $\delta>0$, there exists $\hat{x}\in\calS$ such that $0<\rmd(x,\hat{x})<\delta$.

For any subset $\calS$ of $\calX$, show that $x\in\cl(\calS)$ if and only if $x$ is either a member of $\calS$ or is a limit point of $\calS$.

\end{enumerate}

\clearpage
\section{}
Let \(x\in\underset{n\in\mathcal{N}}{\bigcup} \mathcal{S}_n\) be arbitrary.
Then, clearly, \(\exists n\in\mathcal{N}\) such that \(x\in \mathcal{S}_n\)
further, since \(\mathcal{S}_n\) is open, \(\exists\delta>0\) such that
\[
    \text{B}(x,\delta) =\{\hat{x}\in\mathcal{X}:\text{d}(\hat{x},x)<\delta\}
    \subseteq \mathcal{S}_n
    \subseteq \bigcup_{n\in\mathcal{N}} \mathcal{S}_n.
\]
In words, every element in an open set has an open ball,
contained within that sam open set, which implies containment
in the unionized superset.

\section{}

Let \(x\in\underset{n\in N}{\bigcap} \mathcal{S}_n\) be arbitrary.
Since all \(\mathcal{S}_n\) are open sets,
\(\exists\delta_n>0\) such that \(\text{B}(x,\delta_n)\subseteq\mathcal{S}_n\).
Let \(\delta=\underset{n\in N}{\min}(\delta_n)\).
Then, 
\[
    \text{B}(x,\delta)
    \subseteq\underset{n\in N}{\bigcap}\text{B}(x,\delta_n)
    \subseteq\text{B}(x,\delta_n)\forall n\in N.
\]
Moreover, 
\[
    \text{B}(x,\delta)
    \subseteq\underset{n\in N}{\bigcap}\text{B}(x,\delta_n)
    \subseteq \underset{n\in N}{\bigcap}\mathcal{S}_n
    \subseteq\mathcal{S}_n \forall n\in N.
\]
To close the proof we reflect upon the perhaps unintuitive need for a finite \(N\).
Recall that \(\delta=\underset{n\in N}{\min}(\delta_n)\),
while it is possible that an infinite set has a minimum,
it is guaranteed for a non-empty finite set.
An example of an intersection that is not open when \(N=\infty\) is
\[  \bigcap_{n=1}^{\infty} \left(-\frac{1}{n},\frac{1}{n}\right) = \{0\} = 0.  \]
For finite \(N\) we see that the intersection is instead \(\left(-\frac{1}{N},\frac{1}{N}\right)\).

\section{}

We start with the definition of interior, applied to the complement of subset \(\mathcal{S}\),
\[
    \text{int}(\mathcal{S}^c)
    = \bigcup_{\substack{\mathcal{E}\subseteq\mathcal{S}^c\\\mathcal{E}\text{ is open}}}
    \mathcal{E}.
\]
Using DeMorgan's law we can evaluate the compliment of the interior of \(\mathcal{S}^c\),
\[
    \left[\text{int}(\mathcal{S}^c)\right]^c
    = \bigcap_{\substack{\mathcal{E}\subseteq\mathcal{S}^c\\\mathcal{E}\text{ is open}}}
    \mathcal{E}^c
\]
Next, let \(\mathcal{E}^c := \mathcal{F}\). We may then rewrite the previous expression as
\begin{align*}
    \left[\text{int}(\mathcal{S}^c)\right]^c
    &= \bigcap_{\substack{\mathcal{F}^c\subseteq\mathcal{S}^c\\\mathcal{F}^c\text{ is open}}}
    \mathcal{F} \\
    &= \bigcap_{\substack{\mathcal{F}\supseteq\mathcal{S}\\\mathcal{F}\text{ is closed}}}
    \mathcal{F} \\
    &= \text{cl}(\mathcal{S}).
\end{align*}

\clearpage

\section{}
(\(\Leftarrow\)) Assuming \(x\) is an interior point of \(\mathcal{S}\).

This is defined as \(\exists\delta>0\) such that \(\text{B}(x,\delta)\subseteq\mathcal{S}\),
\(x\in\text{B}(x,\delta)\).
Referencing theorem 2.8.1, we know that an open ball is an open subset,
thus containment in \(\mathcal{S}\) implies membership of the 
open subsets of \(\mathcal{S}\) which is to be in the interior of \(\mathcal{S}\).

(\(\Rightarrow\)) Assuming \(x\in\text{int}(\mathcal{S})\).

This means that \(x\) is a member of the union of all empty sets \(\mathcal{S}\),
which, as seen in problem 2, is not necessarily open itself,
so we say there exists a an empty subset \(\mathcal{S}_0\subseteq\mathcal{S}\)
such that \(x\in\mathcal{S}_0\).
Recalling the in class theorem 2.8.2 an open set is a union of open balls,
thus, \(\exists\delta>0\) such that
\[
    \text{B}(x,\delta)\subseteq\mathcal{S}_0 \subseteq
    \bigcap_{\substack{\mathcal{E}\subseteq\mathcal{S}\\
        \mathcal{E}\text{is open}}}\mathcal{E}.
\]
Because \(x\in\text{B}(x,\delta)\) we conclude that \(x\) is an interior point.

\section{}
(\(\Rightarrow\)) Assuming \(x\in\text{cl}(\mathcal{S})\).
We wish to show that \(x\in\mathcal{S}\) or is a limit point of \(\mathcal{S}\).

From problem 3 we saw that,
\[ x\in\text{cl}(\mathcal{S}) = \left[\text{int}(\mathcal{S}^c)\right]^c, \]
thus,
\[ x\notin\text{int}(\mathcal{S}^c). \]
This means that \(x\) is not a member of any open subset of \(\mathcal{S}^c\).
Which implies that either \(x\notin\mathcal{S}^c\), and thus \(x\in\mathcal{S}\);
or 
\(\nexists\delta>0\) such that \(\text{B}(x,\delta)\subseteq\mathcal{S}^c\).
Thus \(\forall\delta>0\), there exists \(\hat{x}\in\text{B}(x,\delta)\)
where \(\hat{x}\notin\mathcal{S}^c\).
Equivalently,
\(\forall\delta>0\)
\(\exists\hat{x}\in\mathcal{S}\)
such that
\(0<\text{d}(x,\hat{x})<\delta\),
the definition of \(x\) being a limit point of \(\mathcal{S}\).

(\(\Leftarrow\)) Assuming \(x\in\mathcal{S}\) or \(x\) is a limit point of \(\mathcal{S}\).
We will show that \(x\in\text{cl}(\mathcal{S})\)

In the first case we have,
\[x\in\mathcal{S}\subseteq\text{cl}(\mathcal{S}).\]

If \(x\notin\mathcal{S}\) but is a limit point, 
then \(\forall\delta>0\)
there exists an
\(\hat{x}\in\mathcal{S}\)
such that \(0<\text{d}(x,\hat{x})<\delta\).
Thus, there does not exist
\(\delta>0\)
such that
\(\text{B}(x,\delta)\subseteq\mathcal{S}^c\).
Consequently, and as above,
\[
    x\notin\text{int}(\mathcal{S}^c) 
    \Rightarrow x\in\left[\text{int}(\mathcal{S}^c)\right]^c
    =\text{cl}(\mathcal{S}).\]



\vspace{2em}













\end{document}