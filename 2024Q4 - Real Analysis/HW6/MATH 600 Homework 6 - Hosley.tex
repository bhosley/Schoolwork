\documentclass[12 pt,letterpaper]{article}
\usepackage{amsmath}
\usepackage{amssymb}
\usepackage{amsthm}
\usepackage[left=0.5in, right=0.5in, bottom=0.75in, top=0.75in]{geometry}

\usepackage{mathptmx}
\SetSymbolFont{letters}{bold}{OML}{cmm}{b}{it}
\SetSymbolFont{operators}{bold}{OT1}{cmr}{bx}{n}
\DeclareMathAlphabet{\mathcal}{OMS}{cmsy}{m}{n}

\allowdisplaybreaks

\usepackage{lastpage}
\usepackage{fancyhdr}
\pagestyle{fancy}
\fancyhead[L]{Homework 6}
\fancyhead[C]{Due: November 22, 2024}
\fancyhead[R]{Page \thepage\ of \pageref{LastPage}}
\fancyfoot[L]{Air Force Institute of Technology}
\fancyfoot[R]{Fall 2024 (Hosley)}
\fancyfoot[C]{MATH 600: Mathematical Analysis}
\renewcommand{\headrulewidth}{0.25pt}
\renewcommand{\footrulewidth}{0.25pt}

\newcommand{\la}{\lambda}

\newcommand{\bbC}{\mathbb{C}}
\newcommand{\bbF}{\mathbb{F}}
\newcommand{\bbN}{\mathbb{N}}
\newcommand{\bbQ}{\mathbb{Q}}
\newcommand{\bbR}{\mathbb{R}}
\newcommand{\bbT}{\mathbb{T}}
\newcommand{\bbZ}{\mathbb{Z}}

\newcommand{\brI}{\mathbf{I}}
\newcommand{\brJ}{\mathbf{J}}
\newcommand{\brM}{\mathbf{M}}
\newcommand{\brR}{\mathbf{R}}
\newcommand{\brT}{\mathbf{T}}

\newcommand{\bfa}{\boldsymbol{a}}
\newcommand{\bfA}{\boldsymbol{A}}
\newcommand{\bfb}{\boldsymbol{b}}
\newcommand{\bfB}{\boldsymbol{B}}
\newcommand{\bfc}{\boldsymbol{c}}
\newcommand{\bfC}{\boldsymbol{C}}
\newcommand{\bfD}{\boldsymbol{D}}
\newcommand{\bfe}{\boldsymbol{e}}
\newcommand{\bfE}{\boldsymbol{E}}
\newcommand{\bff}{\boldsymbol{f}}
\newcommand{\bfF}{\boldsymbol{F}}
\newcommand{\bfG}{\boldsymbol{G}}
\newcommand{\bfH}{\boldsymbol{H}}
\newcommand{\bfI}{\boldsymbol{I}}
\newcommand{\bfJ}{\boldsymbol{J}}
\newcommand{\bfK}{\boldsymbol{K}}
\newcommand{\bfL}{\boldsymbol{L}}
\newcommand{\bfM}{\boldsymbol{M}}
\newcommand{\bfp}{\boldsymbol{p}}
\newcommand{\bfP}{\boldsymbol{P}}
\newcommand{\bfR}{\boldsymbol{R}}
\newcommand{\bfT}{\boldsymbol{T}}
\newcommand{\bfu}{\boldsymbol{u}}
\newcommand{\bfU}{\boldsymbol{U}}
\newcommand{\bfv}{\boldsymbol{v}}
\newcommand{\bfV}{\boldsymbol{V}}
\newcommand{\bfw}{\boldsymbol{w}}
\newcommand{\bfW}{\boldsymbol{W}}
\newcommand{\bfx}{\boldsymbol{x}}
\newcommand{\bfX}{\boldsymbol{X}}
\newcommand{\bfy}{\boldsymbol{y}}
\newcommand{\bfY}{\boldsymbol{Y}}
\newcommand{\bfz}{\boldsymbol{z}}
\newcommand{\bfZ}{\boldsymbol{Z}}

\newcommand{\bfone}{\boldsymbol{1}}
\newcommand{\bfzero}{\boldsymbol{0}}
\newcommand{\bfxi}{\boldsymbol{\xi}}
\newcommand{\bfXi}{\boldsymbol{\Xi}}
\newcommand{\bfphi}{\boldsymbol{\varphi}}
\newcommand{\bfPhi}{\boldsymbol{\varPhi}}
\newcommand{\bfPi}{\boldsymbol{\varPi}}
\newcommand{\bfpsi}{\boldsymbol{\psi}}
\newcommand{\bfPsi}{\boldsymbol{\varPsi}}
\newcommand{\bfchi}{\boldsymbol{\chi}}
\newcommand{\bfzeta}{\boldsymbol{\zeta}}
\newcommand{\bfdelta}{\boldsymbol{\delta}}
\newcommand{\bfDelta}{\boldsymbol{\varDelta}}
\newcommand{\bfGamma}{\boldsymbol{\varGamma}}
\newcommand{\bfOmega}{\boldsymbol{\varOmega}}
\newcommand{\bfSigma}{\boldsymbol{\varSigma}}
\newcommand{\bfLambda}{\boldsymbol{\varLambda}}

\newcommand{\calA}{\mathcal{A}}
\newcommand{\calB}{\mathcal{B}}
\newcommand{\calC}{\mathcal{C}}
\newcommand{\calD}{\mathcal{D}}
\newcommand{\calE}{\mathcal{E}}
\newcommand{\calF}{\mathcal{F}}
\newcommand{\calG}{\mathcal{G}}
\newcommand{\calH}{\mathcal{H}}
\newcommand{\calI}{\mathcal{I}}
\newcommand{\calK}{\mathcal{K}}
\newcommand{\calM}{\mathcal{M}}
\newcommand{\calN}{\mathcal{N}}
\newcommand{\calS}{\mathcal{S}}
\newcommand{\calT}{\mathcal{T}}
\newcommand{\calU}{\mathcal{U}}
\newcommand{\calV}{\mathcal{V}}
\newcommand{\calX}{\mathcal{X}}
\newcommand{\calY}{\mathcal{Y}}

\newcommand{\rmA}{\mathrm{A}}
\newcommand{\rmB}{\mathrm{B}}
\newcommand{\rmc}{\mathrm{c}}
\newcommand{\rmC}{\mathrm{C}}
\newcommand{\rmd}{\mathrm{d}}
\newcommand{\rme}{\mathrm{e}}
\newcommand{\rmi}{\mathrm{i}}
\newcommand{\rmj}{\mathrm{j}}
\newcommand{\rmk}{\mathrm{k}}
\newcommand{\rms}{\mathrm{s}}
\newcommand{\rmS}{\mathrm{S}}
\newcommand{\rmT}{\mathrm{T}}

\newcommand{\Tr}{\mathrm{Tr}}
\newcommand{\Part}{\mathrm{part}}
\newcommand{\Null}{\mathrm{null}}
\newcommand{\Span}{\operatorname{span}}
\newcommand{\rank}{\operatorname{rank}}
\newcommand{\real}{\operatorname{Re}}
\newcommand{\imag}{\operatorname{Im}}
\newcommand{\cond}{\operatorname{cond}}

\newcommand{\inte}{\operatorname{int}}
\newcommand{\cl}{\operatorname{cl}}

\newcommand{\im}{\operatorname{im}}

\newcommand{\conv}[2]{{#1}\ast{#2}}
\newcommand{\argmin}[1]{\underset{#1}{\mathrm{argmin}}}
\newcommand{\argmax}[1]{\underset{#1}{\mathrm{argmax}}}

\newcommand{\mat}[2]{\left[\begin{array}{#1}#2\end{array}\right]}
\newcommand{\dmat}[2]{\left|\begin{array}{#1}#2\end{array}\right|}
\newcommand{\arbset}[1]{\left\{{#1}\right\}}
\newcommand{\arbparen}[1]{\left({#1}\right)}
\newcommand{\sign}{\operatorname{sign}}

\newcommand{\abs}[1]{|{#1}|}
\newcommand{\bigabs}[1]{\bigl|{#1}\bigr|}
\newcommand{\Bigabs}[1]{\Bigl|{#1}\Bigr|}
\newcommand{\biggabs}[1]{\biggl|{#1}\biggr|}
\newcommand{\Biggabs}[1]{\Biggl|{#1}\Biggr|}

\newcommand{\paren}[1]{({#1})}
\newcommand{\bigparen}[1]{\bigl({#1}\bigr)}
\newcommand{\Bigparen}[1]{\Bigl({#1}\Bigr)}
\newcommand{\biggparen}[1]{\biggl({#1}\biggr)}
\newcommand{\Biggparen}[1]{\Biggl({#1}\Biggr)}

\newcommand{\bracket}[1]{[{#1}]}
\newcommand{\bigbracket}[1]{\bigl[{#1}\bigr]}
\newcommand{\Bigbracket}[1]{\Bigl[{#1}\Bigr]}
\newcommand{\biggbracket}[1]{\biggl[{#1}\biggr]}
\newcommand{\Biggbracket}[1]{\Biggl[{#1}\Biggr]}

\newcommand{\set}[1]{\{{#1}\}}
\newcommand{\bigset}[1]{\bigl\{{#1}\bigr\}}
\newcommand{\Bigset}[1]{\Bigl\{{#1}\Bigr\}}
\newcommand{\biggset}[1]{\biggl\{{#1}\biggr\}}
\newcommand{\Biggset}[1]{\Biggl\{{#1}\Biggr\}}

\newcommand{\norm}[1]{\|{#1}\|}
\newcommand{\bignorm}[1]{\bigl\|{#1}\bigr\|}
\newcommand{\Bignorm}[1]{\Bigl\|{#1}\Bigr\|}
\newcommand{\biggnorm}[1]{\biggl\|{#1}\biggr\|}
\newcommand{\Biggnorm}[1]{\Biggl\|{#1}\Biggr\|}

\newcommand{\ip}[2]{\langle{#1},{#2}\rangle}
\newcommand{\bigip}[2]{\bigl\langle{#1},{#2}\bigr\rangle}
\newcommand{\Bigip}[2]{\Bigl\langle{#1},{#2}\Bigr\rangle}
\newcommand{\biggip}[2]{\biggl\langle{#1},{#2}\biggr\rangle}
\newcommand{\Biggip}[2]{\Biggl\langle{#1},{#2}\Biggr\rangle}

\newcommand{\alphi}{\renewcommand{\labelenumi}{(\alph{enumi})}}
\newcommand{\alphii}{\renewcommand{\labelenumii}{(\alph{enumii})}}
\newcommand{\alphiii}{\renewcommand{\labelenumiii}{(\alph{enumiii})}}
\newcommand{\romani}{\renewcommand{\labelenumi}{(\roman{enumi})}}
\newcommand{\romanii}{\renewcommand{\labelenumii}{(\roman{enumii})}}
\newcommand{\romaniii}{\renewcommand{\labelenumiii}{(\roman{enumiii})}}
\newcommand{\arabici}{\renewcommand{\labelenumi}{(\arabic{enumi})}}
\newcommand{\arabicii}{\renewcommand{\labelenumii}{(\arabic{enumii})}}
\newcommand{\arabiciii}{\renewcommand{\labelenumiii}{(\arabic{enumiii})}}

\begin{document}

\noindent
On this assignment, we consider \textit{series} of real numbers.
For any sequence $(x_n)_{n=1}^{\infty}$ of real numbers,
and each natural number $m$,
we define the \textit{$m$th partial sum} of $(x_n)_{n=1}^{\infty}$ to be
$s_m:=\sum_{n=1}^{m}x_n$.
If the sequence $(s_m)_{m=1}^{\infty}$ of these partial sums converges to some real number $s_\infty$,
we refer to $s_\infty$ as the \textit{series} of $(x_n)_{n=1}^{\infty}$,
and denote it as
\begin{equation*}
\sum_{n=1}^{\infty}x_n
:=\lim_{m\rightarrow\infty}\sum_{n=1}^{m}x_n
=\lim_{m\rightarrow\infty}s_m
=s_\infty.
\end{equation*}
When instead $(s_m)_{m=1}^{\infty}$ does not converge,
we say that the series $\displaystyle\sum_{n=1}^{\infty}x_n$ \textit{diverges}.

\begin{enumerate}
%%%%%%%%%%%%%%%%%%%%%%%%%%%%%%%%%%%%%%%%%%%%%%%%%%%%%%%%%%%%%%%%
\item
For any sequences $(x_n)_{n=1}^{\infty}$ and $(y_n)_{n=1}^{\infty}$ of real numbers whose series converge, show that the series of $(x_n+y_n)_{n=1}^{\infty}$ converges with
\begin{equation*}
\sum_{n=1}^{\infty}(x_n+y_n)
=\sum_{n=1}^{\infty}x_n+\sum_{n=1}^{\infty}y_n.
\end{equation*}

\textit{Hint: Use (don't reprove) facts from class about limits of sequences and sums.  No $\varepsilon$'s needed!}

%%%%%%%%%%%%%%%%%%%%%%%%%%%%%%%%%%%%%%%%%%%%%%%%%%%%%%%%%%%%%%%%
\item
For any $c\in\bbR$ and sequence $(x_n)_{n=1}^{\infty}$ of real numbers whose series converges,
show that the series of $(cx_n)_{n=1}^{\infty}$ converges with
\begin{equation*}
\sum_{n=1}^{\infty}cx_n
=c\sum_{n=1}^{\infty}x_n.
\end{equation*}

%%%%%%%%%%%%%%%%%%%%%%%%%%%%%%%%%%%%%%%%%%%%%%%%%%%%%%%%%%%%%%%%
\item
Prove the \textit{geometric series formula}: if $\abs{x}<1$ then
$\displaystyle\sum_{n=0}^{\infty}x^n=\frac{1}{1-x}$.

\textit{Hint:
Without further justification,
you may use the geometric sum formula,
namely that $\sum_{n=0}^{m-1}x^n=\frac{1-x^m}{1-x}$ for any $m\in\bbN$ and $x\neq 1$,
and moreover the fact from class that $\lim_{m\rightarrow\infty} y^m=0$ for any $y\in[0,1)$.}

\end{enumerate}

\newpage

\noindent
Series of real numbers are particularly well-behaved in the special case where the terms are nonnegative.
To be precise, if $x_n\geq0$ for all $n\in\bbN$,
the sequence $(s_m)_{m=1}^{\infty}$ of partial sums of $(x_n)_{n=1}^{\infty}$ is increasing,
satisfying
\begin{equation*}
s_{m+1}
=\sum_{n=1}^{m+1}x_n
=\sum_{n=1}^{m}x_n+x_{m+1}
=s_m+x_{m+1}
\geq s_m,
\end{equation*}
for all $m\in\bbN$.
From class,
we moreover know that any increasing sequence of real numbers is either bounded above or not, in which cases it converges to its supremum or diverges, respectively.
In particular,
if $x_n\geq0$ for all $n\in\bbN$,
then either there exists $b\in\bbR$ such that $\sum_{n=1}^{m}x_n\leq b$ for all $m$,
in which case $\sum_{n=1}^{\infty}x_n$ converges to $\sup\set{s_m: m\in\bbN}$,
or no such $b$ exists, in which case $\sum_{n=1}^{\infty}x_n$ diverges.
As shorthand notation,
we denote these two scenarios as ``$\sum_{n=1}^{\infty}x_n<\infty$" and ``$\sum_{n=1}^{\infty}x_n=\infty$," respectively.
That is, for any sequence $(x_n)_{n=1}^\infty$ of real numbers:
\begin{itemize}

\item
We use ``$\sum_{n=1}^{\infty}x_n<\infty$" as an abbreviation for the scenario in which both $x_n\geq0$ for all $n\in\bbN$ and there exists $b\in\bbR$ such that $\sum_{n=1}^{m}x_n\leq b$ for all $m\in\bbN$, in which case the series converges to
\begin{equation*}
\sum_{n=1}^{\infty}x_n
=\lim_{m\rightarrow\infty}\sum_{n=1}^m x_n
=\sup\biggset{\sum_{n=1}^{m}x_n : m\in\bbN}.
\end{equation*}

\item
We use ``$\sum_{n=1}^{\infty}x_n=\infty$" as an abbreviation for the scenario in which both $x_n\geq0$ for all $n\in\bbN$ and, for any $b\in\bbR$,
there exists $m\in\bbN$ such that $\sum_{n=1}^{m}x_n> b$, in which case the series diverges.

\end{itemize}

\begin{enumerate}
\setcounter{enumi}{3}

%%%%%%%%%%%%%%%%%%%%%%%%%%%%%%%%%%%%%%%%%%%%%%%%%%%%%%%%%%%%%%%%
\item
We say that the series of a sequence $(x_n)_{n=1}^{\infty}$ of real numbers \textit{converges absolutely} if $\displaystyle\sum_{n=1}^{\infty}\abs{x_n}<\infty$.

Show that if this occurs then $\displaystyle\sum_{n=1}^\infty x_n$ converges,
namely that absolute convergence implies convergence.

\textit{Hint: For each $m\in\bbN$, let
$\displaystyle s_m:=\sum_{n=1}^{m}x_n$
and $\displaystyle t_m:=\sum_{n=1}^{m}\abs{x_n}$.
Since $(t_m)_{m=1}^{\infty}$ converges, it is Cauchy.}

%%%%%%%%%%%%%%%%%%%%%%%%%%%%%%%%%%%%%%%%%%%%%%%%%%%%%%%%%%%%%%%%
\item
Prove the \textit{comparison test}:
if $(x_n)_{n=1}^{\infty}$ and $(y_n)_{n=1}^{\infty}$ are sequences of real numbers with $\abs{x_n}\leq y_n$ for all $n\in\bbN$ and $\sum_{n=1}^{\infty}y_n<\infty$ then
$\sum_{n=1}^{\infty}x_n$ converges absolutely.

%%%%%%%%%%%%%%%%%%%%%%%%%%%%%%%%%%%%%%%%%%%%%%%%%%%%%%%%%%%%%%%%
\item
Using the comparison test, show that the \textit{harmonic series}
$\sum_{n=1}^{\infty}\frac1n$ diverges.

\textit{Hint:
$1+\frac12+\frac13+\frac14+\frac15+\frac16+\frac17+\frac18+\dotsb
\geq1+\frac12+\frac14+\frac14+\frac18+\frac18+\frac18+\frac18+\dotsb$.}
\end{enumerate}

\clearpage
\section{}
To show that for convergent sequences \((x_n)_{n=1}^\infty\), \((y_n)_{n=1}^\infty\)
the sequence \((x_n+y_n)_{n=1}^\infty\) also converges we can utilize the identities
provided in this question supplemented with the in-class theorem 4.5.a,
\begin{align*}
    \sum_{n=1}^{\infty}(x_n+y_n)
    = \lim\limits_{m\rightarrow\infty} \sum_{n=1}^{m}(x_n+y_n)
    = \lim\limits_{m\rightarrow\infty} \left( \sum_{n=1}^{m}x_n + \sum_{n=1}^{m}y_n \right)
    = \lim\limits_{m\rightarrow\infty} \sum_{n=1}^{m}x_n 
    + \lim\limits_{m\rightarrow\infty} \sum_{n=1}^{m}y_n
    = \sum_{n=1}^{\infty}x_n + \sum_{n=1}^{\infty}y_n.
\end{align*}

\section{}
To show that \(\sum_{n=1}^{\infty}cx_n = c\sum_{n=1}^{\infty}x_n\)
with a convergent sequence \((x_n)_{n=1}^\infty\), we will use the in-class theorem 4.5.b,
\begin{align*}
    \sum_{n=1}^{\infty}cx_n
    = \lim\limits_{m\rightarrow\infty} \sum_{n=1}^{m}cx_n
    = \lim\limits_{m\rightarrow\infty} c \sum_{n=1}^{m}x_n
    = \lim\limits_{m\rightarrow\infty} c \lim\limits_{m\rightarrow\infty} \sum_{n=1}^{m}x_n
    = c \lim\limits_{m\rightarrow\infty} \sum_{n=1}^{m}x_n
    = c \sum_{n=1}^{\infty}x_n.
\end{align*}

\section{}
We claim that if \(|x|<1\) then \(\sum_{n=1}^{\infty}x^n = \frac{1}{1-x}\).
Using the identities shown in the previous two problems, the geometric sum formula,
and that \(\lim\limits_{n\rightarrow\infty}y^n=0\) for \(y\in[0,1)\),
our claim can be shown by as follows:
\begin{align*}
    \sum_{n=1}^{\infty}x^n
    &= \lim\limits_{m\rightarrow\infty} \sum_{n=1}^{m}x^n \\
    &= \lim\limits_{m\rightarrow\infty} \left( x^m + \sum_{n=1}^{m-1}x^n \right) \\
    &= \lim\limits_{m\rightarrow\infty} x^m 
    + \lim\limits_{m\rightarrow\infty} \sum_{n=1}^{m-1}x^n \\
    % n -> \infty, x<1, x^n = 0 \\
    &= 0 + \lim\limits_{m\rightarrow\infty} \frac{1-x^m}{1-x} \\
    &= \lim\limits_{m\rightarrow\infty} \left( \frac{1}{1-x} - \frac{x^m}{1-x} \right) \\
    &= \frac{1}{1-x} - \lim\limits_{m\rightarrow\infty} \frac{x^m}{1-x} \\
    &= \frac{1}{1-x} - \frac{1}{1-x}\lim\limits_{m\rightarrow\infty} x^m \\
    &= \frac{1}{1-x} - \frac{1}{1-x}(0) \\
    &= \frac{1}{1-x}.
\end{align*}

\clearpage
\section{}
Without loss of generality let $m_1,m_2\in\mathbb{N}$ be such that $m_1<m_2$.
Then we can see that the distance between two partial sums \(s_m\) is
\begin{align*}
    |s_{m_1} - s_{m_2}|
    = \left|\sum_{n=1}^{m_1} x_n - \sum_{n=1}^{m_2} x_n \right|
    = \left|\sum_{n=m_1}^{m_2} x_n \right|
    \leq \sum_{n=m_1}^{m_2} |x_n|
    = \sum_{n=1}^{m_2} |x_n| - \sum_{n=1}^{m_1} |x_n|
    = |t_{m_2}| - |t_{m_1}|
    \leq |t_{m_2} - t_{m_1}|.
\end{align*}
Since \(t_m\) converges, it is cauchy; and since the distance between two \(s_m\)
is less than or equal to the corresponding pairs of \(t_m\) it is also cauchy.
Additionally, \(|s_\infty|\leq t_\infty <\infty\) provides bounds for \(s_\infty\).
Since \(s_\infty\) is Cauchy and bounded it converges,
which is true of all Cauchy sequences in \(\mathbb{R}\) (theorem 3.9.e).

\section{}
Because \(|x_n|<y_n\) for all \(n\in \mathbb{N}\); for all \(m\in\mathbb{N}\),
\[\sum_{n=1}^{m} |x_n| \leq \sum_{n=1}^{m} y_n.\]
Using this in conjunction with the properties used in problems 1 and 2 we have,
\begin{align*}
    \sum_{n=1}^{\infty} |x_n| =
    \lim\limits_{m\rightarrow\infty} \sum_{n=1}^{m} |x_n|
    &\leq \lim\limits_{m\rightarrow\infty} \sum_{n=1}^{m} y_n
    = \sum_{n=1}^{\infty} y_n
    < \infty.
\end{align*}

\section{}
Consider the series \(1+\frac{1}{2}+\frac{1}{4}+\frac{1}{4}
+\frac{1}{8}+\frac{1}{8}+\frac{1}{8}+\frac{1}{8}+\ldots\)
That is, a series constructed from concatenating \(1\) 
with \(2^{k-1}\) elements of \(\frac{1}{2^k}\).
We can compare these two series as provided in the \emph{hint}.
\[
    1+\frac{1}{2}+\frac{1}{3}+\frac{1}{4}+\frac{1}{5}+\frac{1}{6}+\frac{1}{7}+\frac{1}{8}+\ldots
    \geq
    1+\frac{1}{2}+\frac{1}{4}+\frac{1}{4}+\frac{1}{8}+\frac{1}{8}+\frac{1}{8}+\frac{1}{8}+\ldots
\]
From this hint we can derive an intuition for how the harmonic series is element-wise 
greater-than or equal to this \emph{power of two} series.
There is perhaps a more elegant method to compare the two using methods outside the 
scope of this course; we instead compare a sub-series of each using powers of two elements.
\[s_k:= \sum_{n=1}^{2^k}\frac{1}{n} \geq 1+\frac{k}{2} =: t_k.\]
As the number of elements approaches infinity, \(k\) in the subseries also
approaches infinity, as such,
\[\lim\limits_{k\rightarrow\infty} s_k \geq \lim\limits_{k\rightarrow\infty} t_k = \infty.\]
Which allows us to conclude that the harmonic series is also divergent.

\end{document} 