\documentclass[12 pt,letterpaper]{article}
\usepackage{amsmath}
\usepackage{amssymb}
\usepackage{amsthm}
\usepackage[left=0.5in, right=0.5in, bottom=0.75in, top=0.75in]{geometry}

\usepackage{mathptmx}
\SetSymbolFont{letters}{bold}{OML}{cmm}{b}{it}
\SetSymbolFont{operators}{bold}{OT1}{cmr}{bx}{n}
\DeclareMathAlphabet{\mathcal}{OMS}{cmsy}{m}{n}

\allowdisplaybreaks

\usepackage{lastpage}
\usepackage{fancyhdr}
\pagestyle{fancy}
\fancyhead[L]{Homework 5}
\fancyhead[C]{Due: November 15, 2024}
\fancyhead[R]{Page \thepage\ of \pageref{LastPage}}
\fancyfoot[L]{Air Force Institute of Technology}
\fancyfoot[R]{Fall 2024 (Hosley)}
\fancyfoot[C]{MATH 600: Mathematical Analysis}
\renewcommand{\headrulewidth}{0.25pt}
\renewcommand{\footrulewidth}{0.25pt}

\newcommand{\la}{\lambda}

\newcommand{\bbC}{\mathbb{C}}
\newcommand{\bbF}{\mathbb{F}}
\newcommand{\bbN}{\mathbb{N}}
\newcommand{\bbQ}{\mathbb{Q}}
\newcommand{\bbR}{\mathbb{R}}
\newcommand{\bbT}{\mathbb{T}}
\newcommand{\bbZ}{\mathbb{Z}}

\newcommand{\brI}{\mathbf{I}}
\newcommand{\brJ}{\mathbf{J}}
\newcommand{\brM}{\mathbf{M}}
\newcommand{\brR}{\mathbf{R}}
\newcommand{\brT}{\mathbf{T}}

\newcommand{\bfa}{\boldsymbol{a}}
\newcommand{\bfA}{\boldsymbol{A}}
\newcommand{\bfb}{\boldsymbol{b}}
\newcommand{\bfB}{\boldsymbol{B}}
\newcommand{\bfc}{\boldsymbol{c}}
\newcommand{\bfC}{\boldsymbol{C}}
\newcommand{\bfD}{\boldsymbol{D}}
\newcommand{\bfe}{\boldsymbol{e}}
\newcommand{\bfE}{\boldsymbol{E}}
\newcommand{\bff}{\boldsymbol{f}}
\newcommand{\bfF}{\boldsymbol{F}}
\newcommand{\bfG}{\boldsymbol{G}}
\newcommand{\bfH}{\boldsymbol{H}}
\newcommand{\bfI}{\boldsymbol{I}}
\newcommand{\bfJ}{\boldsymbol{J}}
\newcommand{\bfK}{\boldsymbol{K}}
\newcommand{\bfL}{\boldsymbol{L}}
\newcommand{\bfM}{\boldsymbol{M}}
\newcommand{\bfp}{\boldsymbol{p}}
\newcommand{\bfP}{\boldsymbol{P}}
\newcommand{\bfR}{\boldsymbol{R}}
\newcommand{\bfT}{\boldsymbol{T}}
\newcommand{\bfu}{\boldsymbol{u}}
\newcommand{\bfU}{\boldsymbol{U}}
\newcommand{\bfv}{\boldsymbol{v}}
\newcommand{\bfV}{\boldsymbol{V}}
\newcommand{\bfw}{\boldsymbol{w}}
\newcommand{\bfW}{\boldsymbol{W}}
\newcommand{\bfx}{\boldsymbol{x}}
\newcommand{\bfX}{\boldsymbol{X}}
\newcommand{\bfy}{\boldsymbol{y}}
\newcommand{\bfY}{\boldsymbol{Y}}
\newcommand{\bfz}{\boldsymbol{z}}
\newcommand{\bfZ}{\boldsymbol{Z}}

\newcommand{\bfone}{\boldsymbol{1}}
\newcommand{\bfzero}{\boldsymbol{0}}
\newcommand{\bfxi}{\boldsymbol{\xi}}
\newcommand{\bfXi}{\boldsymbol{\Xi}}
\newcommand{\bfphi}{\boldsymbol{\varphi}}
\newcommand{\bfPhi}{\boldsymbol{\varPhi}}
\newcommand{\bfPi}{\boldsymbol{\varPi}}
\newcommand{\bfpsi}{\boldsymbol{\psi}}
\newcommand{\bfPsi}{\boldsymbol{\varPsi}}
\newcommand{\bfchi}{\boldsymbol{\chi}}
\newcommand{\bfzeta}{\boldsymbol{\zeta}}
\newcommand{\bfdelta}{\boldsymbol{\delta}}
\newcommand{\bfDelta}{\boldsymbol{\varDelta}}
\newcommand{\bfGamma}{\boldsymbol{\varGamma}}
\newcommand{\bfOmega}{\boldsymbol{\varOmega}}
\newcommand{\bfSigma}{\boldsymbol{\varSigma}}
\newcommand{\bfLambda}{\boldsymbol{\varLambda}}

\newcommand{\calA}{\mathcal{A}}
\newcommand{\calB}{\mathcal{B}}
\newcommand{\calC}{\mathcal{C}}
\newcommand{\calD}{\mathcal{D}}
\newcommand{\calE}{\mathcal{E}}
\newcommand{\calF}{\mathcal{F}}
\newcommand{\calG}{\mathcal{G}}
\newcommand{\calH}{\mathcal{H}}
\newcommand{\calI}{\mathcal{I}}
\newcommand{\calK}{\mathcal{K}}
\newcommand{\calM}{\mathcal{M}}
\newcommand{\calN}{\mathcal{N}}
\newcommand{\calS}{\mathcal{S}}
\newcommand{\calT}{\mathcal{T}}
\newcommand{\calU}{\mathcal{U}}
\newcommand{\calV}{\mathcal{V}}
\newcommand{\calX}{\mathcal{X}}
\newcommand{\calY}{\mathcal{Y}}

\newcommand{\rmA}{\mathrm{A}}
\newcommand{\rmB}{\mathrm{B}}
\newcommand{\rmc}{\mathrm{c}}
\newcommand{\rmC}{\mathrm{C}}
\newcommand{\rmd}{\mathrm{d}}
\newcommand{\rme}{\mathrm{e}}
\newcommand{\rmi}{\mathrm{i}}
\newcommand{\rmj}{\mathrm{j}}
\newcommand{\rmk}{\mathrm{k}}
\newcommand{\rms}{\mathrm{s}}
\newcommand{\rmS}{\mathrm{S}}
\newcommand{\rmT}{\mathrm{T}}

\newcommand{\Tr}{\mathrm{Tr}}
\newcommand{\Part}{\mathrm{part}}
\newcommand{\Null}{\mathrm{null}}
\newcommand{\Span}{\operatorname{span}}
\newcommand{\rank}{\operatorname{rank}}
\newcommand{\real}{\operatorname{Re}}
\newcommand{\imag}{\operatorname{Im}}
\newcommand{\cond}{\operatorname{cond}}

\newcommand{\inte}{\operatorname{int}}
\newcommand{\cl}{\operatorname{cl}}

\newcommand{\im}{\operatorname{im}}

\newcommand{\conv}[2]{{#1}\ast{#2}}
\newcommand{\argmin}[1]{\underset{#1}{\mathrm{argmin}}}
\newcommand{\argmax}[1]{\underset{#1}{\mathrm{argmax}}}

\newcommand{\mat}[2]{\left[\begin{array}{#1}#2\end{array}\right]}
\newcommand{\dmat}[2]{\left|\begin{array}{#1}#2\end{array}\right|}
\newcommand{\arbset}[1]{\left\{{#1}\right\}}
\newcommand{\arbparen}[1]{\left({#1}\right)}
\newcommand{\sign}{\operatorname{sign}}

\newcommand{\abs}[1]{|{#1}|}
\newcommand{\bigabs}[1]{\bigl|{#1}\bigr|}
\newcommand{\Bigabs}[1]{\Bigl|{#1}\Bigr|}
\newcommand{\biggabs}[1]{\biggl|{#1}\biggr|}
\newcommand{\Biggabs}[1]{\Biggl|{#1}\Biggr|}

\newcommand{\paren}[1]{({#1})}
\newcommand{\bigparen}[1]{\bigl({#1}\bigr)}
\newcommand{\Bigparen}[1]{\Bigl({#1}\Bigr)}
\newcommand{\biggparen}[1]{\biggl({#1}\biggr)}
\newcommand{\Biggparen}[1]{\Biggl({#1}\Biggr)}

\newcommand{\bracket}[1]{[{#1}]}
\newcommand{\bigbracket}[1]{\bigl[{#1}\bigr]}
\newcommand{\Bigbracket}[1]{\Bigl[{#1}\Bigr]}
\newcommand{\biggbracket}[1]{\biggl[{#1}\biggr]}
\newcommand{\Biggbracket}[1]{\Biggl[{#1}\Biggr]}

\newcommand{\set}[1]{\{{#1}\}}
\newcommand{\bigset}[1]{\bigl\{{#1}\bigr\}}
\newcommand{\Bigset}[1]{\Bigl\{{#1}\Bigr\}}
\newcommand{\biggset}[1]{\biggl\{{#1}\biggr\}}
\newcommand{\Biggset}[1]{\Biggl\{{#1}\Biggr\}}

\newcommand{\norm}[1]{\|{#1}\|}
\newcommand{\bignorm}[1]{\bigl\|{#1}\bigr\|}
\newcommand{\Bignorm}[1]{\Bigl\|{#1}\Bigr\|}
\newcommand{\biggnorm}[1]{\biggl\|{#1}\biggr\|}
\newcommand{\Biggnorm}[1]{\Biggl\|{#1}\Biggr\|}

\newcommand{\ip}[2]{\langle{#1},{#2}\rangle}
\newcommand{\bigip}[2]{\bigl\langle{#1},{#2}\bigr\rangle}
\newcommand{\Bigip}[2]{\Bigl\langle{#1},{#2}\Bigr\rangle}
\newcommand{\biggip}[2]{\biggl\langle{#1},{#2}\biggr\rangle}
\newcommand{\Biggip}[2]{\Biggl\langle{#1},{#2}\Biggr\rangle}

\newcommand{\alphi}{\renewcommand{\labelenumi}{(\alph{enumi})}}
\newcommand{\alphii}{\renewcommand{\labelenumii}{(\alph{enumii})}}
\newcommand{\alphiii}{\renewcommand{\labelenumiii}{(\alph{enumiii})}}
\newcommand{\romani}{\renewcommand{\labelenumi}{(\roman{enumi})}}
\newcommand{\romanii}{\renewcommand{\labelenumii}{(\roman{enumii})}}
\newcommand{\romaniii}{\renewcommand{\labelenumiii}{(\roman{enumiii})}}
\newcommand{\arabici}{\renewcommand{\labelenumi}{(\arabic{enumi})}}
\newcommand{\arabicii}{\renewcommand{\labelenumii}{(\arabic{enumii})}}
\newcommand{\arabiciii}{\renewcommand{\labelenumiii}{(\arabic{enumiii})}}

%\usepackage{parskip}

\begin{document}

\noindent
Much of mathematics has been developed to solve \textit{inverse problems}:
given a function $f:\calX\rightarrow\calY$ and some $y\in\calY$,
find any and all $x\in\calX$ such that $f(x)=y$.
Such problems can be subtle:
for example, when $f:\bbF\rightarrow\bbF$, $f(x)=x^2$ and $y=2$,
this problem has two solutions when $\bbF=\bbR$,
and no solution when $\bbF=\bbQ$.
On this assignment,
we explore some concepts that help us understand such problems in general,
beginning with inverse functions.

\begin{enumerate}

%%%%%%%%%%%%%%%%%%%%%%%%%%%%%%%%%%%%%%%%%%%%%%%%%%%%%%%%%%%%%%%%
\item
For any function $f:\calX\rightarrow\calY$,
show that the following are equivalent:

\begin{enumerate}
\romanii

\item
There exists a function $g:\calY\rightarrow\calX$ such that $f(g(y))=y$ for all $y\in\calY$.

\item
The image of $f$ equals its codomain, that is, $\calY=f(\calX)=\set{f(x): x\in\calX}$.

\item
For any $y\in\calY$, there is at least one $x\in\calX$ such that $f(x)=y$.

\end{enumerate}

When any one of these three equivalent properties holds,
$f$ is said to be \textit{onto (surjective)}.

\textit{Hint: It suffices to show that (i) implies (ii), that (ii) implies (iii), and that (iii) implies (i).}

%%%%%%%%%%%%%%%%%%%%%%%%%%%%%%%%%%%%%%%%%%%%%%%%%%%%%%%%%%%%%%%%
\item
For any function $f:\calX\rightarrow\calY$,
show that the following are equivalent:

\begin{enumerate}
\romanii

\item
There exists a function $g:\calY\rightarrow\calX$ such that $g(f(x))=x$ for all $x\in\calX$.

\item
For any $x_1,x_2\in\calX$, if $f(x_1)=f(x_2)$ then $x_1=x_2$.

\item
For any $y\in\calY$, there is at most one $x\in\calX$ such that $f(x)=y$.

\end{enumerate}

When any one of these three equivalent properties holds,
$f$ is said to be \textit{one-to-one (injective)}.

%%%%%%%%%%%%%%%%%%%%%%%%%%%%%%%%%%%%%%%%%%%%%%%%%%%%%%%%%%%%%%%%
\item
For any function $f:\calX\rightarrow\calY$,
show that the following are equivalent:

\begin{enumerate}
\romanii

\item
There exists a function $g:\calY\rightarrow\calX$ such that $g(f(x))=x$ for all $x\in\calX$ and $f(g(y))=y$ for all $y\in\calY$.

\item
$f$ is one-to-one and onto.

\item
For any $y\in\calY$, there is exactly one $x\in\calX$ such that $f(x)=y$.

\end{enumerate}

Further show that when this occurs, the function $g$ is unique,
and moreover that for any $x\in\calX$ and $y\in\calY$,
we have $f(x)=y$ if and only if $x=g(y)$.

When any one of these three equivalent properties holds,
$f$ is said to be \textit{invertible (bijective)},
$g$ is called the \textit{inverse} of $f$,
and $g$ is often denoted as $f^{-1}$.

\textit{Hint:
Though much of this is an immediate consequence of the previous two problems,
how do we guarantee that the two ``$g$" functions from them are identical?
We caution that in general, the ``half-inverse" functions considered in the previous two problems are not unique.}

\newpage

%%%%%%%%%%%%%%%%%%%%%%%%%%%%%%%%%%%%%%%%%%%%%%%%%%%%%%%%%%%%%%%%
\item
Use induction to prove that for any $n\in\bbN$,
the function $f_n:[0,\infty)\rightarrow[0,\infty)$, $f_n(x)=x^n$ in \textit{strictly increasing}, i.e., satisfies $f_n(x_1)<f_n(x_2)$ for all $x_1,x_2\in[0,\infty)$ such that $x_1<x_2$.
Conclude that $f$ is one-to-one.

%%%%%%%%%%%%%%%%%%%%%%%%%%%%%%%%%%%%%%%%%%%%%%%%%%%%%%%%%%%%%%%%
\item
In class, we prove the \textit{intermediate value theorem}:
if $a<b$ and $f:[a,b]\rightarrow\bbR$ is continuous,
then for every $y$ in between $f(a)$ and $f(b)$,
there exists $x\in[a,b]$ such that $f(x)=y$.

Using this result,
and the fact from class that every polynomial function on $\bbR$ is continuous,
show for any $n\in\bbN$ that the function $f_n:[0,\infty)\rightarrow[0,\infty)$, $f_n(x)=x^n$ is onto.

\textit{Hint: Use Bernoulli's inequality.}

%%%%%%%%%%%%%%%%%%%%%%%%%%%%%%%%%%%%%%%%%%%%%%%%%%%%%%%%%%%%%%%%
\item
For any $n\in\bbN$, the previous results imply that the function $f_n:[0,\infty)\rightarrow[0,\infty)$, $f_n(x)=x^n$ is one-to-one and onto and thus invertible.
Letting $f_n^{-1}$ be its inverse,
commonly denoted by $y^{\frac1n}:=f_n^{-1}(y)$ for any $y\in[0,\infty)$,
we thus have
\begin{equation*}
(x^n)^{\frac1n}=f_n^{-1}(f_n(x))=x,\
\quad
(y^{\frac1n})^n=f_n(f_n^{-1}(y))=y,
\quad\forall\,x,y\geq 0.
\end{equation*}
Show that $(y_1y_2)^{\frac1n}=y_1^{\frac1n}y_2^{\frac1n}$ for all $y_1,y_2\geq0$.

\textit{Note: Without further justification, you may use the fact that
$(x_1x_2)^n=x_1^n x_2^n$ for all $x_1,x_2\in[0,\infty)$.}

\end{enumerate}

\parskip=5pt

\clearpage
\section{}
To prove the equivalence of these properties we will show that each property implies the 
next using the order suggested in the assignement:

\noindent
(i\(\Rightarrow\)ii):

Assume that property i. holds.
Note that \(f(\mathcal{X})\subseteq\mathcal{Y}\).
By property i. we also have that \(f(g(\mathcal{Y}))=\mathcal{Y}\)
and that \(g(\mathcal{Y})\subseteq\mathcal{X}\).
The latter implies \(f(g(\mathcal{Y}))\subseteq f(\mathcal{X})\).
% https://en.wikipedia.org/wiki/Image_(mathematics)#Properties
Combining these observations we may see that
\[\mathcal{Y} = f(g(\mathcal{Y})) \subseteq f(\mathcal{X}) \subseteq \mathcal{Y}.\]
And thus conclude that \(f(\mathcal{X})=\mathcal{Y}\).

\vspace{5pt}\noindent
(ii\(\Rightarrow\)iii):

Assume that property ii. holds.
Then, for arbitrary \(y\in\mathcal{Y}\) we have, \(y\in\{f(x):x\in\mathcal{X}\}\).
Thus for all \(y\) there exists an \(x\in\mathcal{X}\) such that \(f(x) = y\).
Note that \(f(x)\) does not necessarily correspond to a unique \(x\in\mathcal{X}\).

\vspace{5pt}\noindent
(iii\(\Rightarrow\)i):

Assume that property iii. holds.
Then for arbitrary \(y\in\mathcal{Y}\)
there exists at least one \(x_y\in\mathcal{X}\) such that \(f(x_y)=y\).
We define \(g:\mathcal{Y}\rightarrow\mathcal{X}\) 
in terms of any one of those \(x_y\) as \(g(y)=x_y\).
Because property iii uses arbitrary \(y\in\mathcal{Y}\)
the domain is the entirety of \(\mathcal{Y}\) 
and the image is \(g(\mathcal{Y})\subseteq\mathcal{X}\).

\section{}
Once again we will prove property equivalence cyclically:

\noindent
(i\(\Rightarrow\)ii):

Assume that property i. holds.
Pick two \(x_1,x_2\in\mathcal{X}\) such that \(f(x_1)=f(x_2)\).
Then, by property i. we have
\[x_1 = g(f(x_1)) = g(f(x_2)) = x_2,\]
confirming property ii.

\vspace{5pt}\noindent
(ii\(\Rightarrow\)iii):

Assume that property ii. holds.
Let \(y\in\mathcal{Y}\) be arbitrary.
Then, take every \(x_n\in\mathcal{X}\) where \(f(x_n) = y\).
Then, by property ii.,
\[
    y = f(x_1) = f(x_2) = \ldots = f(x_n)
    \quad\Rightarrow\quad
    x_1 = x_2 = \ldots = x_n.
\]
Thus, we have two plausible outcomes, either there are no \(x\)
or there is one \(x\) such that \(f(x)=y\) (property iii).

\vspace{5pt}\noindent
(iii\(\Rightarrow\)i):

Assume that property iii. holds.
We can define a function \(g:\mathcal{Y}\rightarrow\mathcal{X}\) based upon property iii.
We do so in two parts; the first provides property i., 
the second part addresses the remainder of the domain that isn't covered by the first part.

For \(y_1\in f(\mathcal{X}) \subseteq \mathcal{Y}\)
there is an \(x_y\in\mathcal{X}\) where \(f(x_y)=y_1\).
For this case we define \(g(y_1)=x_y\).

For \(y_2\in\mathcal{Y}/f(\mathcal{X})\) there is no \(x\in\mathcal{X}\)
where \(f(x) = y_2\).
In this case \(g(y):=x_0\) where \(x_0\in\mathcal{X}\) can be chosen by any arbitrary method.

\clearpage
\section{}
Once again we will prove property equivalence cyclically:

\noindent
(i\(\Rightarrow\)ii):

\noindent
\(f(g(y))=y\forall\ y\in\mathcal{Y}\) is the property i. in problem one,
thus \(f\) is \emph{onto}. \\
\(g(f(x))=x\forall\ x\in\mathcal{X}\) is the property i. in problem two,
thus \(f\) is \emph{one-to-one}.

\vspace{5pt}\noindent
(ii\(\Rightarrow\)iii):

\noindent
\(f\) is onto, thus has problem one property iii. 
thus there is at least one \(x\in\mathcal{X}\) such that \(f(x)=y\).\\
\(f\) is one-to-one, thus has problem two property iii. 
thus there is at most one \(x\in\mathcal{X}\) such that \(f(x)=y\).\\
From both we conclude that there is exactly one \(x\in\mathcal{X}\) such that \(f(x)=y\).

\vspace{5pt}\noindent
(iii\(\Rightarrow\)i):

By property iii., for any \(y\in\mathcal{Y}\)
there is exactly one \(x_y\in\mathcal{X}\) such that \(f(x_y)=y\).
Consequently we might define a function \(g:\mathcal{Y}\rightarrow\mathcal{X}\)
such that \(g(y)=x_y\).
The result is property i.,
\[
    f(g(y)) = f(x_y) = y
    \quad\text{and}\quad
    g(f(x_y)) = g(y) = x_y.
\]

Moreover, we can evaluate the uniqueness of \(g\) through contradiction.
For arbitrary \(y\in\mathcal{Y}\), let \(g_1(y) = x_1\) and \(g_2(y) = x_2\).
Then using property i. we can see that
\[
    f(g_1(y)) = f(x_1) = y = f(x_2) = f(g_2(y)).
\]
Further, by property iii., \(f(x_1) = f(x_2)\) implies that \(x_1 = x_2\).
This expression demonstrates the uniqueness of \(g\) and a relation to \(f\),
and hopefully an intuitive example of why under the previous properties \(g\)
would also be labeled as \(f^{-1}\).

\section{}

Let \(n=1\). Then,
\[f_n(x_1) = x_1^n = x_1^1 = x_1 < x_2 = x_2^1 = x_2^n = f_n(x_2).\]
Next, for sake of induction, assume \(x_1 < x_2\) implies \(f_n(x_1) < f_n(x_2)\). Then,
\[ f_{n+1}(x_1) = x_1^{n+1} = x_1^{n}(x_1) < x_2^{n}(x_2) = x_2^{n+1} = f_{n+1}(x_2). \]
Thus we conclude that \(f_n(x) := x^n\) is strictly increasing.
Next, we wish to demonstrate that this function is also \emph{one-to-one},
and will do so using property ii. as shown in problem 2.
%
Pick any \(x_1,x_2\in[0,\infty)\) such that \(f_n(x_1) = f_n(x_2)\)
Since \(x_1,x_2 \geq 0\),
\[x_1^n = f_n(x_1) = f_n(x_2) = x_2^n \quad\Rightarrow\quad x_1 = x_2.\]

\clearpage
\section{}

We will show that over the interval \([0,\infty)\), \(f_n(x)=x^n\)
is \emph{onto} by showing that property iii. of problem one holds.
In this case, that for any \(y\in[0,\infty)\)
there exists an \(x\in[0,\infty)\) such that \(f(x) = y\).
We can use Bernoulli's inequality to find an interval
(really an upper bound) on the \(x\) corresponding to any \(y\).

Let \(b:=x+1\). Substitute \(b\) into Bernoulli's inequality,
\begin{align*}
    f_n(b)
    = (b)^n
    = ((b-1)+1)^n
    \geq n(b-1)+1.
\end{align*}
Then for an arbitrary \(y\),
\begin{align*}
    y &\leq n(b-1)+1 \leq b^n\\
    y-1 &\leq n(b-1) \\
    \frac{y-1}{n} &\leq b-1 \\
    \frac{y-1}{n}+1 &\leq b.
\end{align*}
Thus, because \(f_n(x)\) is continuous,
for any \(y\in[0,\infty)\)
there is an \(x\in[0,\infty)\)
such that \(f_n(x) = y\).
Specifically, \[ 0\leq x \leq \frac{y-1}{n}+1. \]

\section{}
For arbitrary \(y_1,y_2\in[0,\infty)\), we see
\[
    \left(y_1 y_2 \right)^\frac{1}{n} 
    = \left((y_1^{\frac{1}{n}})^n (y_2^{\frac{1}{n}})^n \right)^\frac{1}{n}
    = \left(\left(y_1^{\frac{1}{n}} y_2^{\frac{1}{n}}\right) ^n \right)^\frac{1}{n}
    = \left(y_1^{\frac{1}{n}} y_2^{\frac{1}{n}} \right)^{n\frac{1}{n}}
    = y_1^{\frac{1}{n}} y_2^{\frac{1}{n}}.
\]

\end{document} 