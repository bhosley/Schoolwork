\chapter{MatPlotLib}
\label{ch:MatPlotLib}

\section{Py Plot}
\begin{minted}{python}
import matplotlib.pyplot as plt
# ... #
plt.show
\end{minted}
Plots can be called multiple times on same instance of 
pyplot for overlaying multiple items.

\subsection{Line Plots}
\includegraphics[width=0.5\linewidth]{SinCosPlot}
\begin{minted}{python}
plt.plot(X,Y)
\end{minted}
Plot will use a pair of Lists. 

\subsection{Scatter Plots}
\begin{minted}{python}
plt.scatter(X,Y)
\end{minted}
Scatter will use a pair of Lists. 

\subsection{Bar Charts}
\begin{minted}{python}
plt.bar(X,Y)
plt.hbar(X,Y)
\end{minted}
Bar will use a pair of Lists. 

\subsubsection{Multiple Bar Charts}
\paragraph{Side-By-Side}
\begin{minted}{python}
plt.bar(X + 0.0, Y0, color = 'b', width = 0.5)
plt.bar(X + 0.5, Y1, color = 'r', width = 0.5)
\end{minted}
For adjacent bars the X-axis will be used as the bottom.
Therefore it will be necessary to adjust the placement laterally on the X.
Likewise, it will be necessary to adjust the width of the bars to prevent overlap.

\paragraph{Stacked Bar Charts}
\begin{minted}{python}
plt.bar(X,Y0, color = 'b')
plt.bar(X,Y1, color = 'r', Bottom = Y0)
\end{minted}
For stacked bars the bottom of each bar will be the top of the previous bar.
This can be done consecutively.  

\paragraph{Back-to-back Bar Charts}
\begin{minted}{python}
plt.barh(X, Y0)
plt.barh(X, -Y1)
\end{minted}
Back-to-back is as simple as making the y value of the second plot negative. 

\subsection{Pie Charts}
\begin{minted}{python}
plt.pie(X)
\end{minted}
Pie takes a single List. 

\subsection{Histograms}
\begin{minted}{python}
plt.hist(X, bins = 10)
\end{minted}
Histogram takes a single List; divides data into 10 bins by default. 

\subsection{Boxplots}
\begin{minted}{python}
plt.boxplot(X)
\end{minted}
Boxplot takes a single List.

\section{Colors}
\label{sec:MPLColors}
MatLibPlot will use HTML color names, or a Triplet/Quadruplet syntax.
\begin{description}
	\item[Triplet] (1.0,0.0,0.5) Scaled to $[0.0,1.0]$ (R,G,B)
	\item[Quadruplet] Like a triplet but with transparency. (R,G,B,A)
	\item[Color Names] b Blue; g Green; r Red; c Cyan; m Magenta; y Yellow; k Black; w White
	\item[HTML Color Strings] \#RRGGBB ; 8bit values $[0-f]$
	\item[Shades of Gray] color='.75'; Like Black with transparency
\end{description}

\subsection{Color Maps}
\begin{minted}{python}
import matplotlib.cm as cm
\end{minted}

\section{Labels}
\begin{minted}{python}
# create a figure and axis
fig, ax = plt.subplots()

ax.set_title('...')
ax.set_xlabel('...')
ax.set_ylabel('...')

\end{minted}

\section{Markers}
\label{sec:mplMarkers}
