\documentclass[]{article}
\usepackage[english]{babel}
\usepackage[utf8x]{inputenc}
\usepackage{amsmath}
\usepackage[cache=false]{minted}
\usepackage{graphicx}
\usepackage{caption}
\graphicspath{ {./images/} }

%opening
\title{}
\author{}

\begin{document}
\section{R-Language Basic Syntax}
\subsection{Scripting and Preamble}
	\begin{minted}{r}
?hist
help.search("search term")
	\end{minted}
Input/Output
	\begin{minted}{r}
install.packages("Package Name")
library("Package Name")

data <- read.csv("\filePath\fileName.csv")
	\end{minted}

\subsection{Data}
	\begin{minted}{r}
D <- read.table(“path”,sep=“,”,header=TRUE)
D\$Column
d <- c(1,2,3,4)
# Defining a column vector
m <- cbind(c1,c2,c3)
# Binding a number of columns into a matrix
# Must be the same data type
df <- data.frame(x1,x2,x3)
# Data frames can hold multiple data types
v <- rnorm(100,5,2)
# 100 numbers, mean of 5, SD of 2
	\end{minted}

\subsection{Functions}
	\begin{minted}{r}
expr <- function(x,y){
	statements
	return(result)
}

for(i in 1:n){
	statement
}

while (test_expression) {
	statement
}

repeat {
	statement
}

ifelse(condition, truthy return, falsey return)

if (test_expression) {
	statement
} else {
	statement
}
	\end{minted}

\subsection{Import/Export}
	\begin{minted}{r}
write.table(dataset, path, sep=“,”, row.names=FALSE, col.names=FALSE)
# Exports csv
	\end{minted}
\section{Regular Plots}
\subsection{Basic ggplot2}
Boxplot
\begin{minted}{r}
plot(xData, yData)
boxplot(DataSource, aes(x = xData, y = yData)) + geom_boxplot()
# In either case xData is a vector of FACTORS. (~Qualitative or grouped items)
# ex: Pokemon EV ratings: Okay, Good, Fairly Good, Very Good, Excellent, Perfect
boxplot(yData ~ xData,data =DataSource)
boxplot(DataSource, aes(x = xData, y = yData, fill = xData)) + 
	geom_boxplot(outlier.color = "black", outlier.shape = 16, outlier.size = 2)
# This will fill each of multiple xData-based box and whiskers with a different color
\end{minted}
Function Curve
\begin{minted}{r}
curve(x=y)
curve(func1(x), from=xMin, to=xMax)
# To add more lines
curve(func2(), add = TRUE, col = "color")
\end{minted}

\section{Bar Graphs}

\subsection{Parameters}
\begin{minted}{r}
ggplot(DataSet, 
	aes(
		x= xData,
		y= yData,
		fill= zData  ))
\end{minted}
\begin{tabular}{l l}
	Factors: & Bar charts are better for Categorical data \\
	Bins: & x= factor(data) \\
	Fill: & Will divide data by the passed column.  \\
\end{tabular}

\begin{minted}{r}
+ geom_bar( ... )
	stat= "",
	fill= "",
	color= "",
	position= "",
	width= 0.9
\end{minted}
\begin{tabular}{l l}
	stat: & the name of the column that the bars will be labeled from \\
	fill: & Infill color \\
	color: & Outline color \\
	position:" & "dodge" will change bars from stacked to adjacent \\
	width: & Default, 0.9; width of bars \\
\end{tabular}

\begin{minted}{r}
+ geom_text( ... ) 
	aes(label= xData),
	color= "",
	y= i,
	vjust= i
	position= position_dodge()
\end{minted}
\begin{tabular}{l l}
	label: & the column name from which the label data will be derived \\
	color: & color of the text \\
	y: & sets the height of the label on the bar \\
	vjust; & vertical justify; -0.2 puts it just above bar, 1.5 just below \\
\end{tabular}

\begin{minted}{r}
+ guides(fill = guide_legend(reverse=TRUE))
+ scale_fill_brewer(palette= "Pastel1")
+ scale_fill_manual(values:c("#669933", "#FFCC66"))
\end{minted}
\begin{tabular}{l l}
	guides: & Control things relating to legend; shown is how to reverse order \\
	brewer: & Uses palettes for coloring \\
	manual:	& Allows custom color palettes \\
\end{tabular}

\subsection{Examples}

\includegraphics[width=0.5\linewidth]{"Bar Chart 4"}
\includegraphics[width=0.5\linewidth]{"Bar Chart 3"}
\begin{minted}{r}
# 1 Stacked; default
# 2 geom_bar(position= "dodge")
\end{minted}
\includegraphics[width=0.5\linewidth]{"Bar Chart 2"}
\includegraphics[width=0.5\linewidth]{"Bar Chart 5"}
\begin{minted}{r}
# 3 aes(x = reorder(xData, yData), y = yData)
# 4 geom_text(aes(label=Weight),vjust=-0.2)
\end{minted}

\subsection{Proportional Stacked Graph:}
\includegraphics[width=0.6\linewidth]{"Proportional Bar Chart"}
\begin{minted}{r}
library("plyr")
ddply(cabbage_exp,"Date",transform,percent_weight=Weight/sum(Weight)*100)
# apply as D<-ddply(...)
ddply(DataSource,"Split_Data_By_This",transform,new_col_name=expr())
\end{minted}

\subsection{Cleveland Dot Plot:}
\includegraphics[width=0.5\linewidth]{"Cleveland Dot Plot 1"}
\includegraphics[width=0.5\linewidth]{"Cleveland Dot Plot 2"}
\begin{minted}{r}
library(gcookbook)
tophit <-tophitters2001[1:25, ] 
ggplot(tophit,aes(x=avg,y=reorder(name,avg))) 
	+ geom_point(size=3) 
	+ theme_bw() 
	+ theme(panel.grid.major.x =element_blank(), 
		panel.grid.minor.x =element_blank(), 
		panel.grid.major.y =element_line(
				colour="grey60",
				linetype="dashed"))
\end{minted}
Graph 2:
\begin{minted}{r}
nameorder <-tophit$name[order(tophit$lg,tophit$avg)]
tophit$name <-factor(tophit$name,levels=nameorder)
ggplot(tophit,aes(x=avg,y=name)) 
	+ geom_segment(aes(yend= name), xend= 0, colour= "grey50") 
	+ geom_point(size= 3, aes(colour= lg)) 
	+ scale_colour_brewer(palette="Set1",limits=c("NL","AL"),guide=FALSE) 
	+ theme_bw() 
	+ theme(panel.grid.major.y= element_blank()) 
	+ facet_grid(lg ~ ., scales= "free_y",space= "free_y")
\end{minted}
\begin{tabular}{l l}
	geom segment: & Makes the bar line; xend sets it to end at the point \\
	brewer: 	& Setting palette, setting league labels, eliminating the legend \\
	theme: 		& Set white background, setting grid \\
	facet grid: & Sorts and labels based on a facet; in this case lg (Player's League) \\
\end{tabular}

\section{Line Graphs}

\subsection{Parameters}
\begin{minted}{r}
ggplot(DataSet, 
	aes(
		x= xData, 
		y= yData, 
		fill= "", 
		order= desc("")))
\end{minted}
\begin{tabular}{l l}
	Line Graphs & Typically represent data across a continuum \\
	fill('color'): & Colors in the area below the plot \\
	order: & Allows you to swap order of Stacked Area graph \\
\end{tabular}

\begin{minted}{r}
+ geom_line( ... )
	position_dodge= i,
	linetype= "dashed",
	size= i,
	color= "blue"
\end{minted}
\begin{tabular}{l l}
	position\_dodge(0.9): & 
		offsets overlapped data, prevents collision of multiple lines \\
	linetype: & Style of line (solid, dashed, dotted, etc)\\
	size(i): & Thickness of line \\
\end{tabular}

\begin{minted}{r}
+ geom_point( ... )
	position_dodge= i,
	size= i,
	color= "",
	shape= i,
	fill= ""
\end{minted}
\begin{tabular}{l l}
	position\_dodge(0.9): & offsets overlap, prevents collision of multiple points \\
	shape: & Integer index to different shapes \\
	color: & Outline color \\
	fill: & Infill color \\
\end{tabular}

\begin{minted}{r}
+ geom_area( ... )
	color= "",
	fill= "",
	alpha= 0.i
\end{minted}
\begin{tabular}{l l}
	color: & Outline color \\
	fill: & Infill color \\
	alpha: & transparency (lower i is more transparent) \\
\end{tabular} \\
* If you want to avoid lines on the sides of the area, draw a geom\_line over top instead of using the color option here.

\begin{minted}{r}
+ expand_limits(y= i, x= j)
+ scale_color_brewer(palette= "")
+ scale_fill_manual(c(" "," "))
+ scale_y_log10(),
+ geom_ribbon()
\end{minted}
\begin{tabular}{l l}
	expand limits: & Increases the shown axis limits \\
	scale color brewer: & Using built-in palette options \\
	scale fill manual(): & c('','') Custom Palette options \\
	scale y log10: & Will change the y scale to logarithmic \\
	geom ribbon: & (ymin=,ymax=) a confidence area bound by those functions \\
\end{tabular} \\
*Specify Points after line so that they are drawn on top

\subsection{Multiple Lines}
\begin{minted}{r}
library(plyr)
tg <-ddply(ToothGrowth,c("supp", "dose"), summarise, length=mean(len))
\end{minted}
Creates a dataframe in which groups are broken up by supp(ly) and measured by dose; \\
Summarise takes the average of overlapping points.

\subsection{Examples}
	\includegraphics[width=0.5\linewidth]{"Line Graph 1"}
	\includegraphics[width=0.5\linewidth]{"Line Graph 2"}
	\includegraphics[width=0.5\linewidth]{"Line Graph 3"}
	\includegraphics[width=0.5\linewidth]{"Line Graph 4"}
\begin{minted}{r}
# 1 geom_line() + geom_point()
# 2 As above + scale_y_log10()
# 3 Graphed with a ddply transform
# 4 Graphed with a geom_area(fill())
\end{minted}

	\includegraphics[width=0.5\linewidth]{"Line Graph 5"}
	\includegraphics[width=0.5\linewidth]{"Line Graph 6"}
\begin{minted}{r}
ggplot(uspopage,aes(x=Year,y=Thousands,fill=AgeGroup)) 
	+ geom_area(colour="black",size=.2,alpha=.4) 
	+ scale_fill_brewer(palette="Blues",breaks=rev(levels(uspopage$AgeGroup)))
\end{minted}

	\includegraphics[width=0.5\linewidth]{"Line Graph 7"}
	\includegraphics[width=0.5\linewidth]{"Line Graph 8"}
\begin{minted}{r}
ggplot(clim,aes(x=Year,y=Anomaly10y)) 
	+ geom_ribbon(aes(ymin=Anomaly10y-Unc10y,ymax=Anomaly10y+Unc10y),alpha=0.2) 
	+ geom_line()
ggplot(clim,aes(x=Year,y=Anomaly10y)) 
	+ geom_line(aes(y=Anomaly10y-Unc10y),colour="grey50",linetype="dotted") 
	+ geom_line(aes(y=Anomaly10y+Unc10y),colour="grey50",linetype="dotted") 
	+ geom_line()
\end{minted}

\section{Scatter Plots}
\subsection{Parameters}
\begin{minted}{r}
ggplot(DataSet,
	aes(
		x= xData,
		y= yData,
		color= "col",
		shape= "col",
		size= "col"		
\end{minted}
\begin{tabular}{l l}
	Color: & Sort by color based on 'col' data \\
	Shape: & Sort by shape based on 'col' data \\
	Size: & Will make a Balloon plot \\
\end{tabular}

\begin{minted}{r}
+ geom_point(...)
	shape= i,
	size= 2,
	color= "",
	alpha= i,
	position= "jitter",
	position= position_jitter(width= i, height= i)
\end{minted}
\begin{tabular}{l l}
	alpha: & Transparency of the point (Useful to combat over-plotting) \\
	Jittering: & Gives data a lateral offset, combats over-plotting and looks good \\
\end{tabular}

\begin{minted}{r}
+ stat_smooth(...)
	method= lm,
	level= i,
	se= TRUE,
\end{minted}
\begin{tabular}{l l}
	Stat Smooth: & Fits a linear regression line and 95\% confidence interval \\
	Method: & auto: Chosen based on largest group available \\
		& loess: local weighted polynomial (for <1000) \\
		& lm: Linear Method \\
		& glm: gam:  \\
	Level: & Sets the confidence interval \\
	SE: & True is default; False turns off confidence interval \\
\end{tabular}

\begin{minted}{r}
+ scale_color_brewer(Palette)
+ scale_shape_manual(i,j,...)
+ stat_bin2d(bins= OptionalDef)
+ stat_binhex()
+ scale_fill_gradient(low= "color", high= "color", limits= c(i,I)
+ geom_line(data= predicted, ...)
+ annotate("text", label= "...", parse= TRUE, x=, y= )
+ geom_text(aes(), size, vjust, hjust, ...)
\end{minted}
\begin{tabular}{l l}
	Stat Bin 2D: & condense overplotted points into bins \\
	Stat Bin Hex: & Like 2d but hexagons instead of squares \textbf{Lib(hexbin)} \\
	Scale Gradient & Build a gradient between the two colors matching the two limits \\
	Geom Line: & A standard line can be overlaid; if data setting is predicted \\
	Annotate: & Place a note at the define coordinates \\
		& Parse: will try to properly show math formula \\
	Geom Text & Will label all of the data as provided \\
\end{tabular}

\subsection{Examples}
\includegraphics[width=0.5\linewidth]{"Dot Plot 1"}
\includegraphics[width=0.5\linewidth]{"Dot Plot 2"}
\begin{minted}{r}
# 1 Standard
# 2 geom_point( alpha = 0.1 )
\end{minted}

\includegraphics[width=0.5\linewidth]{"Dot Plot 3"}
\includegraphics[width=0.5\linewidth]{"Dot Plot 4"}
\begin{minted}{r}
# 3 geom_point( alpha = 0.01 )
# 4 + stat_bin2d()
\end{minted}

\includegraphics[width=0.5\linewidth]{"Dot Plot 5"}
\includegraphics[width=0.5\linewidth]{"Dot Plot 6"}
\begin{minted}{r}
# 5 + stat_binhex()
# 6 + geom_point(position= position_jitter(width=.5,height=0)
\end{minted}

\includegraphics[width=0.5\linewidth]{"Dot Plot 7"}
\includegraphics[width=0.5\linewidth]{"Dot Plot 8"}
\begin{minted}{r}
# 7 + geom_smooth()
# 8 + geom_line(data= predicted)
\end{minted}

\includegraphics[width=0.5\linewidth]{"Dot Plot 9"}
\includegraphics[width=0.5\linewidth]{"Dot Plot 10"}
\begin{minted}{r}
# 9 + annotate("text",x=4350,y=5.4,label="Canada") 
	+ annotate("text",x=7400,y=6.8,label="USA")
# 10 + geom_text(aes(x=healthexp+100,label=Name),size=4,hjust=0)
\end{minted}

\includegraphics[width=0.5\linewidth]{"Dot Plot 11"}
\includegraphics[width=0.5\linewidth]{"Dot Plot 12"}
\begin{minted}{r}
# 11 ggplot(..., size= GDP) 
# 12 pairs(countries2009[2:5]) # Dot Plot Matrix
\end{minted}

\section{Distribution Summaries}
\subsection{Histograms}
\begin{minted}{r}
ggplot(DataSet,
	aes(
		x= ""
+ geom_histogram(...)
	binwidth= i,
	fill= "",
	color= "",
	alpha= 0.i,
	position= "",
\end{minted}
\begin{tabular}{l l}
	Histograms: & DataSet can be null if x is passed a complete vector \\
	Binwidth: & The range of data that each bar will cover \\
	Fill: & Either a color or a Factor \\
	Position: & "identity": Will overlay two sets of data \\
		& "dodge": Positions data sets adjacent \\
\end{tabular}
\begin{minted}{r}
+ facet_grid(...)
	factor ~ .,
	scales= "free"
\end{minted}
\begin{tabular}{l l}
	Factor: & The factor by which the facets will divide the data \\
	Scales: & Sets scaling. Free lets each facet scale to fit \\
\end{tabular} \\

\includegraphics[width=0.5\linewidth]{"Histogram 1"} 
\includegraphics[width=0.5\linewidth]{"Histogram 2"} \\
\begin{minted}{r}
# 1 
ggplot(birthwt,aes(x=bwt)) 
	+ geom_histogram(fill="white",colour="black") + facet_grid(smoke ~ .)
# 2 
library(MASS);  birthwt1$smoke <-factor(birthwt1$smoke)
ggplot(birthwt1,aes(x=bwt, fill=smoke)) 
	+ geom_histogram(position="identity", alpha=0.4)
\end{minted}

\subsection{Density Curves}
\begin{minted}{r}
ggplot(DataSet,
	aes(
		x= "",
		fill= "",
		color= ""
+ geom_density(...)
	
+ geom_line(...)
	stat= "density",
	adjust= i,
	
\end{minted}
\begin{tabular}{l l}
	Geom Density: & Draws the density curve as a complete polygon (incl. sides) \\
	Geom Line: & Will do a density line without lines on sides and bottom \\
	Adjust: & Determine bandwidth and smoothing \\
\end{tabular} \\
* Using Geom first then line allows you to fill the area 
and then map the curve as an overlay. \\
** Overlay density about a histogram to show how prediction curve
meets observations. Use aes(y=..density..) in the histogram options.\\
\includegraphics[width=0.5\linewidth]{"Density Curve 1"}
\includegraphics[width=0.5\linewidth]{"Density Curve 2"} \\
\begin{minted}{r}
# 1 
ggplot(faithful,aes(x=waiting)) 
	+ geom_line(stat="density", adjust=.25,colour="red") 
	+ geom_line(stat="density") 
	+ geom_line(stat="density", adjust=2,colour="blue")
# 2 
ggplot(faithful,aes(x=waiting, y=..density..)) 
	+ geom_histogram(fill="cornsilk", colour="grey60",size=.2) 
	+ geom_density() 
	+ xlim(35, 105)
\end{minted}

\subsection{Box Plot}
\begin{minted}{r}
ggplot(DataSet,
	aes(
		x= "",
		y= ""
+ geom_boxplot(...)
	outlier.size= i,
	outlier.shape= i,
	notch= "TRUE" or "FALSE"
\end{minted}
\begin{tabular}{l l}
	Outlier: & Controls Outlier point options \\
	Notch: & Shrinks the width of the median point for easier comparison \\	
\end{tabular} \\
\includegraphics[width=0.5\linewidth]{"Boxplot 1"}
\includegraphics[width=0.5\linewidth]{"Boxplot 2"} \\
\begin{minted}{r}
# 1 ggplot(birthwt,aes(x=factor(race),y=bwt)) + geom_boxplot(notch=TRUE)
# 2 ggplot(birthwt, aes(x=factor(race),y=bwt)) + geom_boxplot() 
	+ stat_summary(fun.y="mean", geom="point", shape=23,size=3,fill="white")
\end{minted}

\subsection{Violin Plot}
\begin{minted}{r}
ggplot(DataSet,
	aes(
		x= "",
		y= ""
+ geom_violin(...)
	trim= "TRUE",
	scale= "count",
	adjust= i
\end{minted}
\begin{tabular}{l l}
	Trim: & Trims the tails off either end of the plot (default is true) \\
	Scale: & Changes the scale; Normal is each plot having the same total area \\
	Adjust: & Smoothing; Less smoothing < 1 < More Smoothing \\
\end{tabular}
\includegraphics[width=0.5\linewidth]{"Violin Plot 1"}
\includegraphics[width=0.5\linewidth]{"Violin Plot 2"} \\
\begin{minted}{r}
# 1 ggplot(heightweight,aes(x=sex,y=heightIn)) 
	+ geom_violin() 
	+ geom_boxplot(width=.1,fill="black", outlier.colour=NA) 
	+ stat_summary(fun.y=median, geom="point",fill="white", shape=21,size=2.5)
# 2 ggplot(heightweight,aes(x=sex,y=heightIn,fill=sex)) 
	+ geom_violin(alpha=0.4, trim="FALSE")
\end{minted}

\subsection{Frequency Dot Plot}
\begin{minted}{r}
ggplot(DataSet,
	aes(
		x= "",
+ geom_dotplot(...)
	binwidth= i,
	stackdir= "",
\end{minted}
\begin{tabular}{l l}
	Binwidth: & Acts like Histo-Binwidths \\
	Stack Direction: & Manner in which dots stack \\
		& center, center\_whole, 
\end{tabular} \\
\includegraphics[width=0.5\linewidth]{"Frequency Dot Plot 1"}
\includegraphics[width=0.5\linewidth]{"Frequency Dot Plot 2"} \\
\begin{minted}{r}
# 1 ggplot(countries2009,aes(x=infmortality))
	+ geom_dotplot(binwidth=.25,stackdir="centerwhole") 
	+ scale_y_continuous(breaks=NULL) 
	+ theme(axis.title.y=element_blank())
# 2 ggplot(heightweight, aes(x=sex,y=heightIn)) 
	+ geom_dotplot(binaxis="y",binwidth=.5,stackdir="center")
\end{minted}

\subsection{2-Dimensional Density Plot}
\begin{minted}{r}
ggplot(DataSet,
	aes(
		x= "",
+ stat_density2d(...)
\end{minted}
\begin{tabular}{l l}
\end{tabular} \\
\includegraphics[width=0.5\linewidth]{"2D Density 1"}
\includegraphics[width=0.5\linewidth]{"2D Density 2"} \\
\begin{minted}{r}
# 1 ggplot(faithful, aes(x=eruptions,y=waiting))
+ geom_point() + stat_density2d()
# 2 ggplot(faithful, aes(x=eruptions,y=waiting))
+ stat_density2d(aes(fill=..density..), geom="raster", contour=FALSE, h=c(.5,5))
\end{minted}

\section{Annotations}

\subsection{Text}
\begin{minted}{r}
+ annotate("text",
	x= i,
	y= i, 
	label= "",
	color= "",
	alpha= 0.i,
	size= i,
	family= "",
	fontface= "",
	parse= FALSE, TRUE, F, or T,
\end{minted}
\begin{tabular}{l l}
	x,y: & Sets the center of the text, uses graph's scale \\
		& Inf, -Inf, mean(range(data\$stat)) \\
	Label: & Contents of the annotation \\
	Family: & "serif", \\
	Fontface: & "italic", \\
	Parse: & Will format a math formula for show from the label \\
		& 'Single Quotes' to escape parser \\
\end{tabular} \\

\subsection{Lines}
\begin{minted}{r}
+ geom_hline(yintercept= i)
+ geom_vline(xintercept= i)
+ geom_abline(intercept= i, slope= i)
+ annotate("segment", x=i,xend=i,y=i,yend=i)
	color= "",
	size= i,
	linetype= i,
\end{minted}
\begin{tabular}{l l}
\end{tabular} \\

\subsection{Showing Error}
\begin{minted}{r}
+ geom_errorbar(aes(ymin=Stat-se,ymax=Stat+se) ... )
\end{minted}
\begin{tabular}{l l}
\end{tabular} \\


\subsection{Examples}
\includegraphics[width=0.5\linewidth]{"Annotate 1"}
\includegraphics[width=0.5\linewidth]{"Annotate 2"}
\begin{minted}{r}
# 1
p + annotate("text", x=3,y=48, label="Group 1", 
	family="serif", fontface="italic", colour="darkred", size=3) 
  + annotate("text", x=4.5,y=66, label="Group 2", 
  	family="serif", fontface="italic", colour="darkred", size=3)
# 2
p + annotate("text", x=2,y=0.3,parse=TRUE, 
	label="frac(1, sqrt(2 * pi)) * e ^ {-x^2 / 2}")
\end{minted}

\includegraphics[width=0.5\linewidth]{"Annotate 3"}
\includegraphics[width=0.5\linewidth]{"Annotate 4"}
\begin{minted}{r}
# 3
p + geom_hline(yintercept=60) 
  + geom_vline(xintercept=14) 
  + geom_abline(intercept=37.4,slope=1.75)
# 4
p + geom_hline(aes(yintercept=heightIn,colour=sex), 
	data=hw_means, linetype="dashed", size=1)
\end{minted}

\includegraphics[width=0.5\linewidth]{"Annotate 5"}
\includegraphics[width=0.5\linewidth]{"Annotate 6"}
\begin{minted}{r}
# 5
p +annotate("segment",x=1850,xend=1820,y=-.8,yend=-.95,colour="blue",size=2,
	arrow=arrow()) 
  +annotate("segment", x=1950,xend=1980,y=-.25,yend=-.25,
  	arrow=arrow(ends="both",angle=90,length=unit(.2,"cm")))
# 6
p <-ggplot(subset(climate, Source=="Berkeley"), aes(x=Year,y=Anomaly10y)) 
  + geom_line()
p + annotate("rect", xmin=1950,xmax=1980,ymin=-1,ymax=1,alpha=.1, fill="blue")
\end{minted}

\includegraphics[width=0.5\linewidth]{"Annotate 7"}
\includegraphics[width=0.5\linewidth]{"Annotate 8"}
\begin{minted}{r}
# 7
ggplot(cabbage_exp,aes(x=Date,y=Weight,fill=Cultivar)) 
+ geom_bar(stat= "identity",position="dodge") 
+ geom_errorbar(aes(ymin=Weight-se,ymax=Weight+se), 
	position=position_dodge(0.9),width=.2) 
# 8
p <-ggplot(mpg,aes(x=displ,y=hwy)) +geom_point() + facet_grid(.~ drv)
f_labels <-data.frame(drv =c("4", "f", "r"), label =c("4wd", "Front", "Rear"))
p + geom_text(x=6,y=40,aes(label=label),  data=f_labels)
# If you use annotate(), the label will appear in all facets
p + annotate("text", x=6,y=42,label="label text")
\end{minted}

\section{Axes}

\subsection{Basic Operations}
\begin{minted}{r}
+ ylim(min,max)
+ xlim(max,min) # Reverse values for a reversed axis
+ coord_flip()
+ scale_x_discrete(limits=rev(levels( DAT$X )))
	limits= c(cat1, cat2, ...)
+ scale_y_continuous(...)
	limits= c(min,max),
	breaks= TRUE or FALSE,
	breaks= seq(beginning, end, interval),
	breaks= c(a,b,c,d),
	name= "Axis Label"
+ scale_y_reverse(limits= c(i,i))
\end{minted}
\begin{tabular}{l l}
	Coord Flip: & Swaps x and y axes \\
	Discrete Axis: & \\
		& Reverse Limits: Reverses order of a set of Factors \\	
		& c() To manually define order of Categorical Data \\
	Continuous: & \\
	 	& Breaks: Sets whether or not axis will have lines \\
\end{tabular} \\

\subsection{Scaling and Labels}
\begin{minted}{r}
+ coord_fixed()
	ratio= i	# y=ix
+ theme( ... )
	axis.text.y =element_blank()
		=element_text(
			angle= i,
			vjust= i,
			hjust= i,
			family= "",
			face= "",
			color= "",
			size= i
	axis.ticks.x =element_blank()
	axis.title.y
	axis.line
+ xlab("Label Literal")
+ labs(x= "", y=""
\end{minted}
\begin{tabular}{l l}
	Fixed Coordinates: & Equal scaling on both Axes \\
	Axis Text: & Leaves tick marks, but removes labels \\
	Axis. & Items may be addressed as both, .x, or .y \\
\end{tabular} \\



\subsection{Examples}
\includegraphics[width=0.5\linewidth]{"Axes 1"}
\includegraphics[width=0.5\linewidth]{"Axes 2"} \\
\begin{minted}{r}
# 1 
+ scale_y_continuous(breaks=c(50, 56, 60, 66, 72), 
	labels=c("Tiny", "Really\nshort", "Short", "Medium", "Tallish"))
# 2 library(scales)
ggplot(Animals,aes(x=body,y=brain,label=rownames(Animals))) + 
	geom_text(size=3) + 
	annotation_logticks() + 
	scale_x_log10(breaks = trans_breaks("log10", function(x) 10^x),  
		labels = trans_format("log10",math_format(10^.x)), 
		minor_breaks =log10(5) + -2:5) + 
	scale_y_log10(breaks =trans_breaks("log10", function(x) 10^x), 
		labels = trans_format("log10",math_format(10^.x)), 
		minor_breaks =log10(5) + -1:3) + 
	coord_fixed() + 
	theme_bw()
\end{minted}

\includegraphics[width=0.5\linewidth]{"Axes 3"}
\includegraphics[width=0.5\linewidth]{"Axes 4"} \\
\begin{minted}{r}
# 3
ggplot(wind,aes(x=DirCat,fill=SpeedCat)) + 
	geom_histogram(binwidth=15,origin=-7.5) + 
	coord_polar() + 
	scale_x_continuous(limits=c(0,360))
# 4
ggplot(www,aes(x=minute,y=users)) +
	geom_line() + 
	scale_x_continuous(name="time",breaks=seq(0, 100,by=10),
		labels=timeHM_formatter)
\end{minted}

\section{Overall Appearances}

\subsection{Title}
\begin{minted}{r}
+ ggtitle( "..." )
+ labs(title= "" )
+ theme(plot.title= "")
\end{minted}
\begin{tabular}{l l}
\end{tabular}

\subsection{Themes}
\begin{minted}{r}
+ theme_grey(...)
+ theme_bw(...)
	<Options from this entire packet>
my_theme <- theme_bw() + theme( ... )
\end{minted}
\begin{tabular}{l l}
\end{tabular}

\subsection{Examples}
\includegraphics[width=0.5\linewidth]{"Theme 1"}
\includegraphics[width=0.5\linewidth]{"Theme 2"} \\
\begin{minted}{r}
# 1 p + 
theme(panel.grid.major = element_line(colour="red"), 
	panel.grid.minor = element_line(colour="red", linetype="dashed", size=0.2),
	panel.background = element_rect(fill="lightblue"), 
	panel.border = element_rect(colour="blue", fill=NA, size=2))
# 2 p + 
	facet_grid(sex ~ .) +
	theme(strip.background =element_rect(fill="pink"), 
		strip.text.y =element_text(size=14,angle=-90,face="bold"))
\end{minted}

\section{Legends}
\begin{minted}{r}
+ guides( ... ) 
	fill= FALSE #To remove legend
+ theme(
	legend.position= "none" # To Remove
	legend.position= "top", "bottom"
	legend.position= c(i,i) # Coordinates
	legend.justification= c(i,i)
	legend.background= element_rect(fill='', color='')
	legend.background= element_blank()
	legend.key= element_blank()	# Remove background behind key-symbols
	legend.title= element_text(...)
		face= "",
		family="",
		color="",
		size=""
	legend.text= element_text(...) # As Above
\end{minted}
\begin{tabular}{l l}
	Position: & Coordinate Space starts (0,0) at the bottom left, 
				and goes to (1,1) at the top right. \\
	Justification:& Which part of the legend is treated as the positioning point.\\
	Key: & A box around the key symbols in the legend. \\
\end{tabular}

\subsection{Examples}
\includegraphics[width=0.5\linewidth]{"Legend 1"}
\includegraphics[width=0.5\linewidth]{"Legend 2"} \\
\begin{minted}{r}
# 1 p + 
scale_shape_discrete(labels=c("Female", "Male")) + 
scale_color_discrete(labels=c("Female", "Male"))
# 2 library(grid)
p + 
scale_fill_discrete(labels=
	c("Control", "Type 1\ntreatment", "Type 2\ntreatment")) + 
theme(legend.text=element_text(lineheight=.8), 
	legend.key.height=unit(1, "cm")) 
\end{minted}

\section{General Use}

Many Examples use: \\
"library(gcookbook)" \\
"library(MASS)" \\

\subsection{Palettes}
ColorBrewer Palettes \\
\includegraphics[width=\linewidth]{"ColorBrewer"} \\
Viridis Palettes \\
\includegraphics[width=\linewidth]{"Viridis"} \\

\section{Other Examples}
\includegraphics[width=\linewidth]{"Viridis Demo 1"} \\
\begin{minted}{r}
ggplot(data.frame(x=rnorm(10000),y=rnorm(10000)), aes(x=x,y=y)) 
+ geom_hex() 
+ coord_fixed() 
+ scale_fill_viridis() 
+ theme_bw()
\end{minted}

\includegraphics[width=\linewidth]{"Viridis Demo 2"} \\
\inputminted[]{r}{"Viridis Map Demo.rhistory"}


\end{document}

% Darling Cartogram
% Demers Cartogram
% Gridded Hex Cartogram
% http://r-statistics.co/Top50-Ggplot2-Visualizations-MasterList-R-Code.html