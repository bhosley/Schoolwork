\chapter{Methodology}
\label{ch:methodology}
\glsresetall

\section{Data Sanitization}

The datasets employed in this study encompass statistics from the 2019 and 2020 seasons, 
juxtaposed with the points per game from the 2020 and 2021 seasons. 
Initial data cleaning ensured the absence of missing values and consistent data types across all columns. 
The dataframe summaries were scrutinized for anomalous instances. 
The notable instance in the 2019 data revealed a player who participated in 17 games within the season. This player was identified as Emmanuel Sanders, who transitioned from the Denver Broncos to the San Francisco 49ers mid-season. While this anomaly was not replicated in 2020, the 2021 season saw the NFL shift to a 17-game standard season, leading to 82 players with similar statistics.

\section{Variable Selection}

In order to gain a deeper understanding of the data, we analyzed the relationship between each feature and the points per game. 
The outcomes of this preliminary analysis are depicted in Figures \ref{app:featuresPPG}. 
This exploration reveals a distinct correlation between specific features and points per game, 
which is contingent on a player's role within the team. 
This observation implies that segregating player positions could be advantageous, 
enabling dimensionality reduction by eliminating unused features. 
A case in point is the passing features; apart from quarterbacks, only one player has a non-zero value, 
illustrating the specificity of certain features to particular positions.

The relationship with the 'teams' feature is not depicted in the presented collection of charts, 
as the initial plots did not exhibit any discernible pattern.
To further investigate this, Figure \ref{fig:teamppgsyearon} provides a comparison of the 
distribution of PPG across teams over the seasons under study.
The teams are arranged in order of their average Per-Player PPG, to ensure a non-arbitrary sequence.
This visualization underscores the lack of valuable insights based on team performance.

Figure \ref{app:positionPlots} presents the plots for each position following the elimination of their constant variables. Further understanding was obtained by examining the correlation between features. Figures \ref{app:positionCorrPlots}, which display the correlation plots, are referenced when discussing correlation-related issues.

Initially, a comprehensive model was constructed for the Quarterbacks to serve as a benchmark. The resulting coefficients are displayed in Table \ref{tab:full_lm_QB}. 
The dredge tool was considered for identifying an optimal model. However, the constraints of the available computational environment rendered full enumeration impractical.
Instead, we employed a step-wise variable selection function that operated bidirectionally and 
utilized the Akaike Information Criterion (AIC) \cite{akaike1998} as the metric. 
Concurrently, Elasticnet regularization was performed. 
The former approach yielded a marginally superior model.
Finally, a Bayesian linear model was implemented, which demonstrated comparable performance to the other two methods.
This procedure was replicated for each player position.

In the final stage, the partial regressions are combined into a single multiple regression, 
with the player's position serving as an interaction term. 
The built-in library offered a practical approach for this integration of multiple, separate regressions. 
The coefficients derived from this unified model are very close to those from the individual regressions, 
when adjusted for the new inclusion of the shared regression terms.

\section{Influential Variables}

Throughout the analysis phase we screened for outliers. 
In the initial multiple regression, 26 outliers were detected within the data, as determined by the Mahalanobis distance. 
Within the model summary itself, two outliers were identified based on Cook's distance. 
These model-reported outliers were iteratively removed until no further outliers were detected. 
Concurrently, the outliers within the dataset were eliminated from a separate model variant. 
The final revised models did not reveal any additional outliers.

Among all the scenarios considered, the model derived from eliminating the outliers, 
as identified from the model summary, based on Cook's distance, 
exhibited superior performance in terms of $R^2$ and Standard Error. 
Consequently, this model was selected for further use in this project.
The outliers that were removed are shown in Table \ref{tab:outliers}.

\section{Adequacy Checking}

The assumption of error independence is demonstrated in Table \ref{fig:residualplot}. This table also presents the plot of residuals against observed values for PPG.Next, which appears to exhibit homoscedasticity, indicating a constant variance of errors across all levels of the independent variables. The assumption of error normality is further supported by the Q-Q Plot depicted in Figure \ref{fig:qqplot}. These visualizations collectively affirm the adequacy of our multiple regression model.


\section{Model Validation}

The validation of our regression model involved a comprehensive examination of various aspects. The Q-Q plot (Figure \ref{fig:qqplot}) confirmed the normality of the error terms, a critical assumption of regression analysis.

We further refined our model using the Variance Inflation Factor (VIF). This iterative process led to the removal of several predictors from the model, namely 'Receiving Yards', 'Games', 'Passing Interceptions', and 'Fumbles Lost'.

The impact of this refinement on the model's performance was assessed using the R-squared and adjusted R-squared statistics. The R-squared value decreased slightly from 0.66 to 0.64 following the VIF analysis. However, the adjusted R-squared increased from 0.54 to 0.61, indicating an improved fit of the model to the data.

In conclusion, our rigorous process of validation and refinement has resulted in a robust and reliable regression model. The model's performance statistics are as follows: R-squared = 0.64, Adjusted R-squared = 0.61, F-statistic = 17.34 on 39, p-value = 2.2e-16. The coefficients of the final model are presented in Table \ref{tab:final_model}.

\begin{figure}[h]
	\centering
	\includegraphics[width=0.8\linewidth]{Figures/results_plot}
	\caption{Predicted v. Observed}
	\label{fig:resultsplot}
\end{figure}

%%%%%  Pre Edit  %%%%%%





