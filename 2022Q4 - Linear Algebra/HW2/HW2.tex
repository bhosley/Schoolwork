\documentclass[12pt]{amsart}


%\usepackage[notref,notcite]{showkeys}
\usepackage{verbatim}
\usepackage{fullpage}
\usepackage{amsfonts}
\usepackage{bbm}
\usepackage{wrapfig}
\usepackage{enumerate}
\usepackage{float}
\usepackage{hyperref}
\usepackage[left=0.5in, right=0.5in, bottom=0.75in, top=0.75in]{geometry}
\setlength{\parindent}{0pt}


\usepackage{color}
\usepackage{colortbl}
\usepackage{graphicx}
\usepackage{epstopdf}
\usepackage{amsmath, amsthm, amssymb}
\usepackage{fancybox}

\newcommand{\1}{\mathbbm{1}}

\newcommand{\eps}{\varepsilon}
\newcommand{\la}{\lambda}
\newcommand{\Contradiction}{\Rightarrow\Leftarrow}
\DeclareMathOperator{\lspan}{span}
\DeclareMathOperator{\tr}{tr}
\DeclareMathOperator{\ran}{ran}
\DeclareMathOperator{\diag}{diag}
\providecommand{\abs}[1]{\lvert#1\rvert}
\providecommand{\norm}[1]{\lVert#1\rVert}
\renewcommand{\theenumi}{\alph{enumi}}
\renewcommand{\labelenumi}{(\theenumi)}

\newcounter{Theorem}
\newcounter{Definition}
\numberwithin{equation}{section}
\numberwithin{Theorem}{section}

\theoremstyle{plain} %% This is the default, anyway
\newtheorem{thm}[Theorem]{Theorem}
\newtheorem{cor}[Theorem]{Corollary}
\newtheorem{lem}[Theorem]{Lemma}
\newtheorem{prop}[Theorem]{Proposition}
%\usepackage{upgreek}

\theoremstyle{definition}
\newtheorem{defn}[Theorem]{Definition}

\theoremstyle{remark}
\newtheorem{remark}{Remark}[section]
\newtheorem{ex}[Theorem]{Example}
\newtheorem{nota}[Theorem]{Notation}

\newif\ifanswers

\answersfalse

\begin{document}

\thispagestyle{empty}

\noindent{\Large Homework 2\ifanswers\ Solutions\fi - Hosley. (Due October 14 at 8am)}\bigskip


\begin{enumerate}[1.]
\item Assume \(\mathcal{V}\) is a vector space over a field \(\mathbb{F}\), \(\mathcal{N}\) is a finite nonempty set, and \(\{\mathbf{v}_{n}\}_{n\in\mathcal{N}}\) is a sequence of vectors in \(\mathcal{V}\). Let \(\mathbf{V}:\mathbb{F}^{\mathcal{N}}\to\mathcal{V}\) be the synthesis operator of \(\{\mathbf{v}_{n}\}_{n\in\mathcal{N}}\), that is, for each \(\mathbf{x}\in\mathbb{F}^{\mathcal{N}}\)
\[\mathbf{V}(\mathbf{x}) = \sum_{n\in\mathcal{N}}\mathbf{x}(n)\mathbf{v}_{n}.\]
Show that for any subspace \(\mathcal{X}\) of \(\mathbb{F}^{\mathcal{N}}\), the set 
\[\mathbf{V}(\mathcal{X}) = \{\mathbf{V}(\mathbf{x}) : \mathbf{x}\in\mathcal{X}\}\]
is a subspace of \(\mathcal{V}\).
\bigskip

\begin{align*}
	\mathbf{0} = \sum_{n\in\mathcal{N}}  0(n)v_n = \mathbf{V}(0)
\end{align*}

\newpage

\item Assume \(\mathcal{V}\) is a vector space over a field \(\mathbb{F}\), \(\mathcal{N}\) is a finite nonempty set, and \(\{\mathbf{v}_{n}\}_{n\in\mathcal{N}}\) is a sequence of vectors in \(\mathcal{V}\). Fix \(n_{0}\in\mathcal{N}\) and define the set
\[\mathcal{N}_{0} := \mathcal{N}\setminus\{n_{0}\} = \{n\in\mathcal{N} : n\neq n_{0}\}.\]
 Suppose \(\{\mathbf{v}_{n}\}_{n\in\mathcal{N}_{0}}\) is linearly independent and \(\mathbf{v}_{n_{0}}\in\mathcal{V}\setminus\operatorname{span}\{\mathbf{v}_{n}\}_{n\in\mathcal{N}_{0}}\) (this means \(\mathbf{v}_{n_{0}}\in\mathcal{V}\) and \(\mathbf{v}_{n_{0}}\notin\operatorname{span}\{\mathbf{v}_{n}\}_{n\in\mathcal{N}_{0}}\)). Prove that \(\{\mathbf{v}_{n}\}_{n\in\mathcal{N}}\) is linearly independent.


Hint: Proofs that a sequence such as \(\{\mathbf{v}_{n}\}_{n\in\mathcal{N}}\) is linearly independent often take the following form: Suppose \(\mathbf{x}\in\mathbb{F}^{\mathcal{N}}\) is a vector such that \(\sum_{n\in\mathcal{N}}\mathbf{x}(n)\mathbf{v}_{n} = \mathbf{0}\), then use the properties you know about the sequence \(\{\mathbf{v}_{n}\}_{n\in\mathcal{N}}\) to argue that \(\mathbf{x}(n) = 0\) for each \(n\in\mathcal{N}\). In this case, you know two things about \(\{\mathbf{v}_{n}\}_{n\in\mathcal{N}}\).

\bigskip
\newpage

\item Assume \(\mathcal{V}\) is a vector space over a field \(\mathbb{F}\), \(\mathcal{N}\) is a finite nonempty set, and \(\{\mathbf{v}_{n}\}_{n\in\mathcal{N}}\) is a sequence of vectors in \(\mathcal{V}\). Suppose \(\{\mathbf{v}_{n}\}_{n\in\mathcal{N}}\) is a dependent set.  Prove that there is some \(n_{0}\in\mathcal{N}\) such that
\[\operatorname{span}\{\mathbf{v}_{n}\}_{n\in\mathcal{N}\setminus\{n_{0}\}} = \operatorname{span}\{\mathbf{v}_{n}\}_{n\in\mathcal{N}}.\]
Hint: In this problem you are trying to show that two sets are equal. Proofs that two sets \(A\) and \(B\) are equal often take the following form: Let \(x\in A\) be arbitrary, and argue that \(x\in B\). This shows that \(A\subset B\). Next, let \(y\in B\) be arbitrary, and argue that \(y\in A\). This shows that \(B\subset A\). Thus \(A=B\).

\bigskip



\item Go to \href{https://eigenquiz.app/}{Eigenquiz.app} and complete activity a901772. Upload a completion certificate for this activity to your homework folder.
\end{enumerate}



\end{document}