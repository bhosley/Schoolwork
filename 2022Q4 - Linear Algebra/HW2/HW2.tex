\documentclass[12pt]{amsart}


%\usepackage[notref,notcite]{showkeys}
\usepackage{verbatim}
\usepackage{fullpage}
\usepackage{amsfonts}
\usepackage{bbm}
\usepackage{wrapfig}
\usepackage{enumerate}
\usepackage{float}
\usepackage{hyperref}
\usepackage[left=0.5in, right=0.5in, bottom=0.75in, top=0.75in]{geometry}
\setlength{\parindent}{0pt}


\usepackage{color}
\usepackage{colortbl}
\usepackage{graphicx}
\usepackage{epstopdf}
\usepackage{amsmath, amsthm, amssymb}
\usepackage{fancybox}

\newcommand{\1}{\mathbbm{1}}

\newcommand{\eps}{\varepsilon}
\newcommand{\la}{\lambda}
\newcommand{\Contradiction}{\Rightarrow\Leftarrow}
\DeclareMathOperator{\lspan}{span}
\DeclareMathOperator{\tr}{tr}
\DeclareMathOperator{\ran}{ran}
\DeclareMathOperator{\diag}{diag}
\providecommand{\abs}[1]{\lvert#1\rvert}
\providecommand{\norm}[1]{\lVert#1\rVert}
\renewcommand{\theenumi}{\alph{enumi}}
\renewcommand{\labelenumi}{(\theenumi)}

\newcounter{Theorem}
\newcounter{Definition}
\numberwithin{equation}{section}
\numberwithin{Theorem}{section}

\theoremstyle{plain} %% This is the default, anyway
\newtheorem{thm}[Theorem]{Theorem}
\newtheorem{cor}[Theorem]{Corollary}
\newtheorem{lem}[Theorem]{Lemma}
\newtheorem{prop}[Theorem]{Proposition}
%\usepackage{upgreek}

\theoremstyle{definition}
\newtheorem{defn}[Theorem]{Definition}

\theoremstyle{remark}
\newtheorem{remark}{Remark}[section]
\newtheorem{ex}[Theorem]{Example}
\newtheorem{nota}[Theorem]{Notation}

\newif\ifanswers

\answersfalse

\begin{document}

\thispagestyle{empty}

\noindent{\Large Homework 2\ifanswers\ Solutions\fi - Hosley. (Due October 14 at 8am)}\bigskip


\begin{enumerate}[1.]
\item Assume \(\mathcal{V}\) is a vector space over a field \(\mathbb{F}\), \(\mathcal{N}\) is a finite nonempty set, and \(\{\mathbf{v}_{n}\}_{n\in\mathcal{N}}\) is a sequence of vectors in \(\mathcal{V}\). Let \(\mathbf{V}:\mathbb{F}^{\mathcal{N}}\to\mathcal{V}\) be the synthesis operator of \(\{\mathbf{v}_{n}\}_{n\in\mathcal{N}}\), that is, for each \(\mathbf{x}\in\mathbb{F}^{\mathcal{N}}\)
\[\mathbf{V}(\mathbf{x}) = \sum_{n\in\mathcal{N}}\mathbf{x}(n)\mathbf{v}_{n}.\]
Show that for any subspace \(\mathcal{X}\) of \(\mathbb{F}^{\mathcal{N}}\), the set 
\[\mathbf{V}(\mathcal{X}) = \{\mathbf{V}(\mathbf{x}) : \mathbf{x}\in\mathcal{X}\}\]
is a subspace of \(\mathcal{V}\). \\
\hrule
\bigskip
We will show that any set $\mathbf{V}(\mathcal{X})$ is a subspace of \(\mathcal{V}\) using Theorem 1.8,by proving that $\mathbf{V}(\mathcal{X})$ has the properties outlined in the theorem; contains a $\mathbf{0}$, is closed under addition, and is closed under multiplication.

\textbf{Proof.} Since $\mathcal{X}$ is a subspace, it follows that $0\in\mathcal{X}$. The zero vector can be represented as
\begin{align*}
	\mathbf{0} = \sum_{n\in\mathcal{N}}  \mathbf{0}(n)\mathbf{v}_n = \mathbf{V}(\mathbf{0}).
\end{align*}

Next, let $u_1, u_2 \in \mathbf{V}(\mathcal{X})$ such that  $u_1 = \mathbf{V}(\mathbf{x}_1)$ and $ u_2 = \mathbf{V}(\mathbf{x}_2)$.
\begin{align*}
	\mathbf{V}(\mathbf{x}_1)+\mathbf{V}(\mathbf{x}_2) 
	&= \sum_{n\in\mathcal{N}}\mathbf{x}_1(n)\mathbf{v}_n + \sum_{n\in\mathcal{N}}\mathbf{x}_2(n)\mathbf{v}_n \\
	&= \sum_{n\in\mathcal{N}}\mathbf{x}_1(n)\mathbf{v}_n + \mathbf{x}_2(n)\mathbf{v}_n \\
	&= \sum_{n\in\mathcal{N}}(\mathbf{x}_1(n) + \mathbf{x}_2(n))\mathbf{v}_n \\
	&= \sum_{n\in\mathcal{N}}(\mathbf{x}_1+\mathbf{x}_2(n))\mathbf{v}_n \\
	&= \mathbf{V}(\mathbf{x}_1+ \mathbf{x}_2).
\end{align*}
Since $\mathbf{V}(\mathbf{x}_1)+\mathbf{V}(\mathbf{x}_2) = \mathbf{V}(\mathbf{x}_1+ \mathbf{x}_2)$, $\mathbf{V}(\mathcal{X}$ is closed under addition.

Finally, let $c\in\mathbb{F}$ and $u\in\mathbf{V}(\mathcal{X})$ such that $u=\mathbf{V}(\mathbf{x})$.
Then, for arbitrary $\mathbf{x}\in\mathcal{X}$ it can be shown that  
\begin{align*}
	c\mathbf{V}(\mathbf{x})
	&= \sum_{n\in\mathcal{N}}\mathbf{x}(n)\mathbf{v}_n \\
	&= \sum_{n\in\mathcal{N}}c(\mathbf{x}(n)\mathbf{v}_n) \\
	&= \sum_{n\in\mathcal{N}}c\mathbf{x}(n)\mathbf{v}_n \\
	&= \mathbf{V}(c\mathbf{x}).
\end{align*}
Thus, $\mathbf{V}(\mathcal{X})$ is closed under multiplication. \textbf{QED.}

\newpage

\item Assume \(\mathcal{V}\) is a vector space over a field \(\mathbb{F}\), \(\mathcal{N}\) is a finite nonempty set, and \(\{\mathbf{v}_{n}\}_{n\in\mathcal{N}}\) is a sequence of vectors in \(\mathcal{V}\). Fix \(n_{0}\in\mathcal{N}\) and define the set
\[\mathcal{N}_{0} := \mathcal{N}\setminus\{n_{0}\} = \{n\in\mathcal{N} : n\neq n_{0}\}.\]
 Suppose \(\{\mathbf{v}_{n}\}_{n\in\mathcal{N}_{0}}\) is linearly independent and \(\mathbf{v}_{n_{0}}\in\mathcal{V}\setminus\operatorname{span}\{\mathbf{v}_{n}\}_{n\in\mathcal{N}_{0}}\) (this means \(\mathbf{v}_{n_{0}}\in\mathcal{V}\) and \(\mathbf{v}_{n_{0}}\notin\operatorname{span}\{\mathbf{v}_{n}\}_{n\in\mathcal{N}_{0}}\)). Prove that \(\{\mathbf{v}_{n}\}_{n\in\mathcal{N}}\) is linearly independent.


Hint: Proofs that a sequence such as \(\{\mathbf{v}_{n}\}_{n\in\mathcal{N}}\) is linearly independent often take the following form: Suppose \(\mathbf{x}\in\mathbb{F}^{\mathcal{N}}\) is a vector such that \(\sum_{n\in\mathcal{N}}\mathbf{x}(n)\mathbf{v}_{n} = \mathbf{0}\), then use the properties you know about the sequence \(\{\mathbf{v}_{n}\}_{n\in\mathcal{N}}\) to argue that \(\mathbf{x}(n) = 0\) for each \(n\in\mathcal{N}\). In this case, you know two things about \(\{\mathbf{v}_{n}\}_{n\in\mathcal{N}}\).
\\
\hrule
\bigskip

Suppose $\mathbf{x}\in\mathbb{F}^{\mathcal{N}}$ be a vector such that $\sum_{n\in\mathcal{N}}\mathbf{x}(n)\mathbf{v}_{n} = \mathbf{0}$. Then,\\

\begin{align*}
	\mathbf{0} &=\sum_{n\in\mathcal{N}}\mathbf{x}(n)\mathbf{v}_{n} \\
	\mathbf{0} &= \sum_{n\in\mathcal{N}_{0}}\mathbf{x}(n)\mathbf{v}_{n} +  \mathbf{x}(n_0)\mathbf{v}_{n_{0}}\\
	-\mathbf{x}(n_0)\mathbf{v}_{n_{0}} &= \sum_{n\in\mathcal{N}_{0}}\mathbf{x}(n)\mathbf{v}_{n} \\
	\mathbf{v}_{n_{0}}  &= \frac{1}{-\mathbf{x}(n_0)} \sum_{n\in\mathcal{N}_{0}}\mathbf{x}(n)\mathbf{v}_{n}. \\
\end{align*}

Thus we see that if $\mathbf{x}(n_0)\neq0$ then $\mathbf{v}_{n_{0}}$ could be written as a linear combination in $\sum\limits_{n\in\mathcal{N}_{0}}\mathbf{x}(n)\mathbf{v}_{n}$. 
However, since $\mathbf{v}_{n_{0}}$ is not in $\operatorname{span}\{\mathbf{v}_{n}\}_{n\in\mathcal{N}_{0}}$ we conclude that $\mathbf{x}(n_0)=0$ and therefore $\sum\limits_{n\in\mathcal{N}}\mathbf{x}(n)\mathbf{v}_{n}$ is also linearly independent.



\newpage

\item Assume \(\mathcal{V}\) is a vector space over a field \(\mathbb{F}\), \(\mathcal{N}\) is a finite nonempty set, and \(\{\mathbf{v}_{n}\}_{n\in\mathcal{N}}\) is a sequence of vectors in \(\mathcal{V}\). Suppose \(\{\mathbf{v}_{n}\}_{n\in\mathcal{N}}\) is a dependent set.  Prove that there is some \(n_{0}\in\mathcal{N}\) such that
\[\operatorname{span}\{\mathbf{v}_{n}\}_{n\in\mathcal{N}\setminus\{n_{0}\}} = \operatorname{span}\{\mathbf{v}_{n}\}_{n\in\mathcal{N}}.\]
Hint: In this problem you are trying to show that two sets are equal. Proofs that two sets \(A\) and \(B\) are equal often take the following form: Let \(x\in A\) be arbitrary, and argue that \(x\in B\). This shows that \(A\subset B\). Next, let \(y\in B\) be arbitrary, and argue that \(y\in A\). This shows that \(B\subset A\). Thus \(A=B\). \\
\hrule
\bigskip

Note that 
$ \operatorname{span}\{\mathbf{v}_{n}\}_{n\in\mathcal{N}\setminus\{n_{0}\}} =
\sum\limits_{n\in\mathcal{N}\setminus\{n_0\}}\mathbf{x}(n)\mathbf{v}_{n} $
and 
$ \operatorname{span}\{\mathbf{v}_{n}\}_{n\in\mathcal{N}} = 
 \sum\limits_{n\in\mathcal{N}}\mathbf{x}(n)\mathbf{v}_{n} $.

Let $y\in \operatorname{span}\{\mathbf{v}_{n}\}_{n\in\mathcal{N}\setminus\{n_{0}\}} $ be arbitrary.
Because this span is a vector space, there exists an $\textbf{x}\in\mathbb{F}$ such that

$ \sum\limits_{n\in\mathcal{N}\setminus\{n_0\}}\mathbf{x}(n)\mathbf{v}_{n} = y $

Suppose that $\mathbf{x}(n_0) = 0$ Then,

$ \sum\limits_{n\in\mathcal{N}\setminus\{n_0\}}\mathbf{x}(n)\mathbf{v}_{n} + \mathbf{x}(n_0)\mathbf{v}_{n_0} $

$y+0 = y \in   \operatorname{span}\{\mathbf{v}_{n}\}_{n\in\mathcal{N}} $



\vspace*{\fill}
\item Go to \href{https://eigenquiz.app/}{Eigenquiz.app} and complete activity a901772. Upload a completion certificate for this activity to your homework folder.
\end{enumerate}



\end{document}