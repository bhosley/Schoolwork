\documentclass[12pt]{amsart}


%\usepackage[notref,notcite]{showkeys}
\usepackage{verbatim}
\usepackage{fullpage}
\usepackage{amsfonts}
\usepackage{bbm}
\usepackage{wrapfig}
\usepackage{enumerate}
\usepackage{float}
\usepackage{hyperref}
\usepackage[left=0.5in, right=0.5in, bottom=0.75in, top=0.75in]{geometry}
\setlength{\parindent}{0pt}


\usepackage{color}
\usepackage{colortbl}
\usepackage{graphicx}
\usepackage{epstopdf}
\usepackage{amsmath, amsthm, amssymb}
\usepackage{fancybox}

\newcommand{\1}{\mathbbm{1}}

\newcommand{\eps}{\varepsilon}
\newcommand{\la}{\lambda}
\newcommand{\Contradiction}{\Rightarrow\Leftarrow}
\DeclareMathOperator{\lspan}{span}
\DeclareMathOperator{\tr}{tr}
\DeclareMathOperator{\ran}{ran}
\DeclareMathOperator{\diag}{diag}
\providecommand{\abs}[1]{\lvert#1\rvert}
\providecommand{\norm}[1]{\lVert#1\rVert}
\renewcommand{\theenumi}{\alph{enumi}}
\renewcommand{\labelenumi}{(\theenumi)}

\newcounter{Theorem}
\newcounter{Definition}
\numberwithin{equation}{section}
\numberwithin{Theorem}{section}

\theoremstyle{plain} %% This is the default, anyway
\newtheorem{thm}[Theorem]{Theorem}
\newtheorem{cor}[Theorem]{Corollary}
\newtheorem{lem}[Theorem]{Lemma}
\newtheorem{prop}[Theorem]{Proposition}
%\usepackage{upgreek}

\theoremstyle{definition}
\newtheorem{defn}[Theorem]{Definition}

\theoremstyle{remark}
\newtheorem{remark}{Remark}[section]
\newtheorem{ex}[Theorem]{Example}
\newtheorem{nota}[Theorem]{Notation}



\begin{document}

\thispagestyle{empty}

\noindent{\Large Homework 5 (Due November 4 at 8am)}
\hspace{\fill} {\Large B. Hosley}
\bigskip


\begin{enumerate}[1.]



\item Let \(\mathcal{V}\) be a vector space over a field \(\mathbb{F}\), and let \(\mathcal{W}\) be a subspace of \(\mathcal{V}\). For each \(\mathbf{v}\in\mathcal{V}\) define the set
\[[\mathbf{v}]:=\{\mathbf{v}+\mathbf{w} : \mathbf{w}\in\mathcal{W}\}.\]
Set
\[\mathcal{V}/\mathcal{W}: = \{[\mathbf{v}] : \mathbf{v}\in\mathcal{V}\}.\]
The set \(\mathcal{V}/\mathcal{W}\) is called the \textit{quotient space} of \(\mathcal{V}\) by \(\mathcal{W}\). We read the symbols \(\mathcal{V}/\mathcal{W}\) as ``\(\mathcal{V}\) mod \(\mathcal{W}\).''



\bigskip

\begin{enumerate}[(a)]



\item Show that \([\mathbf{v}_{1}]=[\mathbf{v}_{2}]\) if and only if \(\mathbf{v}_{1}-\mathbf{v}_{2}\in\mathcal{W}\).

\bigskip

%%%%%%%%%%%%%%%%%%%%%%%%%%%%%%%
%	\begin{ Question 1 (a)}	  %
%%%%%%%%%%%%%%%%%%%%%%%%%%%%%%%
\begin{proof}
($\Rightarrow$) Assuming \([\mathbf{v}_1] = [\mathbf{v}_2]\): \\
Let \(\mathbf{v}_1,\mathbf{v}_2\in\mathcal{V}\) and suppose \([\mathbf{v}_1] = [\mathbf{v}_2]\).
Let \(\mathbf{w}_1,\mathbf{w}_2\in\mathcal{W}\) be arbitrary, then
\[
[\mathbf{v}_1] = [\mathbf{v}_2]
\mathbf{v}_1 + \mathbf{w}_1 = \mathbf{v}_2 + \mathbf{w}_2
\mathbf{v}_1 - \mathbf{v}_1 = \mathbf{w}_2 - \mathbf{w}_1
.\]
Since \(\mathcal{W}\) is closed under addition \(w_2 - w_1\in\mathcal{W}\) and therefore \(\mathbf{v}_1 - \mathbf{v}_2\in\mathcal{W}\)

($\Leftarrow$) Assuming \(\mathbf{v}_1-\mathbf{v}_2\in\mathcal{W}\): \\
Let \(\mathbf{v}_1,\mathbf{v}_2\in\mathcal{V}\) and \(\mathbf{w}\in\mathcal{W}\) be arbitrary and
and suppose \(\mathbf{v}_1-\mathbf{v}_2\in\mathcal{W}\) then
\begin{align*}
[\mathbf{v}_1] 
= \mathbf{v}_1 + \mathbf{w}
= \mathbf{v}_1 + (\mathbf{v}_2-\mathbf{v}_1) + (\mathbf{v}_1-\mathbf{v}_2) + \mathbf{w}
\intertext{and since $(\mathbf{v}_1-\mathbf{v}_2)\in\mathcal{W}$ we get}
\mathbf{v}_1 + (\mathbf{v}_2-\mathbf{v}_1) + \mathbf{w}
= \mathbf{v}_2 + \mathbf{w}
= [\mathbf{v}_2] .
\end{align*}

\end{proof}
%%%%%%%%%%%%%%%%%%%%%%%%%%%%%%%
%	\end{ Question 1 (a)}	  %
%%%%%%%%%%%%%%%%%%%%%%%%%%%%%%%

\item Let \(\mathbf{x}\in\mathcal{V}/\mathcal{W}\) and \(\mathbf{v}\in\mathcal{V}\). Show that \(\mathbf{x}=[\mathbf{v}]\) if and only if \(\mathbf{v}\in\mathbf{x}\).

\bigskip

%%%%%%%%%%%%%%%%%%%%%%%%%%%%%%%
%	\begin{ Question 1 (b)}	  %
%%%%%%%%%%%%%%%%%%%%%%%%%%%%%%%
	
	
	
%%%%%%%%%%%%%%%%%%%%%%%%%%%%%%%
%	\end{ Question 1 (b)}	  %
%%%%%%%%%%%%%%%%%%%%%%%%%%%%%%%

\end{enumerate}

\noindent Given \(\mathbf{v}_{1},\mathbf{v}_{2}\in\mathcal{V}\) and \(c\in\mathbb{F}\) define
\[[\mathbf{v}_{1}]+[\mathbf{v}_{2}]:=[\mathbf{v}_{1}+\mathbf{v}_{2}]\quad\text{and}\quad c[\mathbf{v}_{1}] := [c\mathbf{v}_{1}].\]
Note that it is not obvious that these operations are well defined. For example, if \(\mathbf{w}_{1}\neq \mathbf{w}_{2}\), but \([\mathbf{w}_{1}]=[\mathbf{w}_{2}]\), then we must show that \([c\mathbf{w}_{1}]=[c\mathbf{w}_{2}]\) in order for scalar multiplication to be well defined in \(\mathcal{V}/\mathcal{W}\).\bigskip

\begin{enumerate}[(a)]

\addtocounter{enumii}{2}

\item Show that addition and scalar multiplication are well defined. That is, if \(\mathbf{v}_{1},\mathbf{v}_{2},\mathbf{u}_{1},\mathbf{u}_{2}\in\mathcal{V}\) and \(c\in\mathbb{F}\) such that \([\mathbf{v}_{1}]=[\mathbf{v}_{2}]\) and \([\mathbf{u}_{1}]=[\mathbf{u}_{2}]\), then \([\mathbf{v}_{1}+\mathbf{v}_{2}]=[\mathbf{u}_{1}+\mathbf{u}_{2}]\) and \([c\mathbf{v}_{1}]=[c\mathbf{v}_{2}]\).

\bigskip

%%%%%%%%%%%%%%%%%%%%%%%%%%%%%%%
%	\begin{ Question 1 (c)}	  %
%%%%%%%%%%%%%%%%%%%%%%%%%%%%%%%

(i) Since \([\mathbf{v}_{1}]=[\mathbf{v}_{2}]\) and \([\mathbf{u}_{1}]=[\mathbf{u}_{2}]\), 
\begin{align*}
	[\mathbf{v}_{1}] + [\mathbf{u}_{1}] = [\mathbf{v}_{2}] + [\mathbf{u}_{2}] 
\intertext{and using the definition of addition from above,}
	[\mathbf{v}_{1} + \mathbf{u}_{1}] = [\mathbf{v}_{2} + \mathbf{u}_{2}] 
\end{align*}
	
(ii) Since \([\mathbf{v}_{1}]=[\mathbf{v}_{2}]\)
\[
[c\mathbf{v}_1] = c[\mathbf{v}_1] = c[\mathbf{v}_2] = [c\mathbf{v}_2]
\]
	
%%%%%%%%%%%%%%%%%%%%%%%%%%%%%%%
%	\end{ Question 1 (c)}	  %
%%%%%%%%%%%%%%%%%%%%%%%%%%%%%%%

\item Prove that \(\mathcal{V}/\mathcal{W}\) is a vector space with the addition and scalar multiplication defined above. What set is the additive identity in \(\mathcal{V}/\mathcal{W}\)?

\bigskip

%%%%%%%%%%%%%%%%%%%%%%%%%%%%%%%
%	\begin{ Question 1 (d)}	  %
%%%%%%%%%%%%%%%%%%%%%%%%%%%%%%%

Associativity
Commutativity
Scalar field multiplication
Multiplicative identity

Distributive under vector addition
Distributive under field addition	
Inverse elements
Additive identity
	
%%%%%%%%%%%%%%%%%%%%%%%%%%%%%%%
%	\end{ Question 1 (d)}	  %
%%%%%%%%%%%%%%%%%%%%%%%%%%%%%%%

\item Suppose that \(\mathcal{V}\) is a finite-dimensional space. Prove that \(\operatorname{dim}(\mathcal{V}/\mathcal{W}) = \operatorname{dim}(\mathcal{V}) - \operatorname{dim}(\mathcal{W})\). (Hint: Take a basis \(\{\mathbf{w}_{n}\}_{n=1}^{M}\) for \(\mathcal{W}\), where \(M=\operatorname{dim}\mathcal{W}\), extend it to a basis \(\{\mathbf{w}_{n}\}_{n=1}^{N}\) for \(\mathcal{V}\), where \(N=\operatorname{dim}\mathcal{V}\). Show that \(\{[\mathbf{w}_{n}]\}_{n=M+1}^{N}\) is a basis for \(\mathcal{V}/\mathcal{W}\).)

\bigskip

\end{enumerate}

%%%%%%%%%%%%%%%%%%%%%%%%%%%%%%%
%	\begin{ Question 1 (e)}	  %
%%%%%%%%%%%%%%%%%%%%%%%%%%%%%%%
	
	
	
%%%%%%%%%%%%%%%%%%%%%%%%%%%%%%%
%	\end{ Question 1 (e)}	  %
%%%%%%%%%%%%%%%%%%%%%%%%%%%%%%%

\clearpage

\item Let \(\mathcal{V}\) and \(\mathcal{U}\) be vector spaces over a field \(\mathbb{F}\), and let \(\mathbf{L}:\mathcal{V}\to\mathcal{U}\) be a linear map. In this exercise we will show that
\[\mathbf{L}(\mathcal{V})\cong \mathcal{V}/\operatorname{ker}(\mathbf{L}).\]
This statement is known as the \textit{first isomorphism theorem}.

\bigskip

\begin{enumerate}[(a)]

\item For each \(\mathbf{x}\in\mathcal{V}/\operatorname{ker}(\mathbf{L})\) show that the set \(\mathcal{L}_{\mathbf{x}}:=\{\mathbf{L}(\mathbf{v}) : \mathbf{v}\in\mathbf{x}\}\) contains exactly one element.

\bigskip

%%%%%%%%%%%%%%%%%%%%%%%%%%%%%%%
%	\begin{ Question 2 (a)}	  %
%%%%%%%%%%%%%%%%%%%%%%%%%%%%%%%
	
	
	
%%%%%%%%%%%%%%%%%%%%%%%%%%%%%%%
%	\end{ Question 2 (a)}	  %
%%%%%%%%%%%%%%%%%%%%%%%%%%%%%%%

\item Let \(\mathbf{\Phi}(\mathbf{x})\) be the unique element in \(\mathcal{L}_{\mathbf{x}}\) for each \(\mathbf{x}\in\mathcal{V}/\operatorname{ker}(\mathbf{L})\). Show that if \(\mathbf{x}\in\mathcal{V}/\operatorname{ker}(\mathbf{L})\) and \(\mathbf{v}\in\mathbf{x}\), then \(\mathbf{\Phi}(\mathbf{x}) = \mathbf{L}(\mathbf{v})\). In particular, \(\mathbf{\Phi}([\mathbf{v}]) = \mathbf{L}(\mathbf{v})\) for any \(\mathbf{v}\in\mathcal{V}\). 

\bigskip

%%%%%%%%%%%%%%%%%%%%%%%%%%%%%%%
%	\begin{ Question 2 (b)}	  %
%%%%%%%%%%%%%%%%%%%%%%%%%%%%%%%
	
	
	
%%%%%%%%%%%%%%%%%%%%%%%%%%%%%%%
%	\end{ Question 2 (b)}	  %
%%%%%%%%%%%%%%%%%%%%%%%%%%%%%%%

\item Show that \(\mathbf{\Phi}: \mathcal{V}/\operatorname{ker}(\mathbf{L})\to\mathbf{L}(\mathcal{V})\) is an isomorphism.

\bigskip

%%%%%%%%%%%%%%%%%%%%%%%%%%%%%%%
%	\begin{ Question 2 (c)}	  %
%%%%%%%%%%%%%%%%%%%%%%%%%%%%%%%
	
	
	
%%%%%%%%%%%%%%%%%%%%%%%%%%%%%%%
%	\end{ Question 2 (c)}	  %
%%%%%%%%%%%%%%%%%%%%%%%%%%%%%%%

\end{enumerate}

\item Use problems 1(e) and 2(c) to prove the Rank-Nullity Theorem.

\bigskip

%%%%%%%%%%%%%%%%%%%%%%%%%%%%%%%
%	\begin{ Question 3}	  %
%%%%%%%%%%%%%%%%%%%%%%%%%%%%%%%
	
	
	
%%%%%%%%%%%%%%%%%%%%%%%%%%%%%%%
%	\end{ Question 3}	  %
%%%%%%%%%%%%%%%%%%%%%%%%%%%%%%%

\vspace*{\fill}

\item Go to \href{https://eigenquiz.app/}{Eigenquiz.app} and complete activity a584703. Upload a completion certificate for this activity to your homework folder.

\end{enumerate}

\end{document}