\documentclass[12pt]{amsart}


%\usepackage[notref,notcite]{showkeys}
\usepackage{verbatim}
\usepackage{fullpage}
\usepackage{amsfonts}
\usepackage{bbm}
\usepackage{wrapfig}
\usepackage{enumerate}
\usepackage{float}
\usepackage{hyperref}
\usepackage[left=0.5in, right=0.5in, bottom=0.75in, top=0.75in]{geometry}
\setlength{\parindent}{0pt}


\usepackage{color}
\usepackage{colortbl}
\usepackage{graphicx}
\usepackage{epstopdf}
\usepackage{amsmath, amsthm, amssymb}
\usepackage{fancybox}
\usepackage{mathtools}

\newcommand{\1}{\mathbbm{1}}

\newcommand{\eps}{\varepsilon}
\newcommand{\la}{\lambda}
\newcommand{\Contradiction}{\Rightarrow\Leftarrow}
\DeclareMathOperator{\lspan}{span}
\DeclareMathOperator{\tr}{tr}
\DeclareMathOperator{\ran}{ran}
\DeclareMathOperator{\diag}{diag}
\providecommand{\abs}[1]{\lvert#1\rvert}
\providecommand{\norm}[1]{\lVert#1\rVert}
\renewcommand{\theenumi}{\alph{enumi}}
\renewcommand{\labelenumi}{(\theenumi)}

\newcounter{Theorem}
\newcounter{Definition}
\numberwithin{equation}{section}
\numberwithin{Theorem}{section}

\theoremstyle{plain} %% This is the default, anyway
\newtheorem{thm}[Theorem]{Theorem}
\newtheorem{cor}[Theorem]{Corollary}
\newtheorem{lem}[Theorem]{Lemma}
\newtheorem{prop}[Theorem]{Proposition}
%\usepackage{upgreek}

\theoremstyle{definition}
\newtheorem{defn}[Theorem]{Definition}

\theoremstyle{remark}
\newtheorem{remark}{Remark}[section]
\newtheorem{ex}[Theorem]{Example}
\newtheorem{nota}[Theorem]{Notation}



\begin{document}

\thispagestyle{empty}

\noindent{\Large Homework 3 (Due October 21 at 8am)}
\hspace{\fill} {\Large B. Hosley}
\bigskip


\begin{enumerate}[1.]

\item This problem is a warmup for the next:

Consider the sequences \(\{\mathbf{v}_{n}\}_{n=1}^{3}\) and \(\{\mathbf{u}_{n}\}_{n=1}^{3}\) in \(\mathbb{R}^{3}\) and \(\mathbb{R}^{2}\), respectively, where
\[\mathbf{v}_{1} = \begin{bmatrix} 1\\ 0\\ 0\end{bmatrix},\ \mathbf{v}_{2} = \begin{bmatrix} 2\\ 1\\ 0\end{bmatrix},\ \mathbf{v}_{3} = \begin{bmatrix} 0\\ -1\\ 1\end{bmatrix},\ \mathbf{u}_{1} = \begin{bmatrix} 1\\ 2\end{bmatrix},\ \mathbf{u}_{2} = \begin{bmatrix} 1\\ 2\end{bmatrix},\ \mathbf{u}_{3} = \begin{bmatrix} 0\\ -1\end{bmatrix}.\]

Since we have proved that all synthesis operators are linear, part (a) shows that \(\mathbf{V}^{-1}\) is linear, and part (b) shows that there exists a \textit{linear} map \(\mathbf{L}:\mathbb{R}^{3}\to\mathbb{R}^{2}\) with the property that \(\mathbf{L}(\mathbf{v}_{n}) = \mathbf{u}_{n}\) for each \(n\in[3]\).

\begin{enumerate}

\item Let \(\mathbf{V}:\mathbb{R}^{3}\to\mathbb{R}^{3}\) be the synthesis operator of \(\{\mathbf{v}_{n}\}_{n=1}^{3}\). What theorem from class tells us that \(\mathbf{V}\) is an invertible function? Find a sequence \(\{\mathbf{w}_{n}\}_{n=1}^{3}\) in \(\mathbb{R}^{3}\) such that \(\mathbf{V}^{-1}\) is the synthesis operator of \(\{\mathbf{w}_{n}\}_{n=1}^{3}\). Show that your answer is correct.

\bigskip

\item Find a sequence \(\{\boldsymbol{\ell}_{n}\}_{n=1}^{3}\) such that if \(\mathbf{L}:\mathbb{R}^{3}\to\mathbb{R}^{2}\) is the synthesis operator of \(\{\boldsymbol{\ell}_{n}\}_{n=1}^{3}\), then \(\mathbf{L}(\mathbf{v}_{n}) = \mathbf{u}_{n}\) for all \(n\in[3]\). (You don't need to include the proof that the sequence you give has the desired property, but you should prove it to yourself!)

\bigskip

\end{enumerate}
\hrule
\bigskip

a) Theorem 5.5(d) from the Dr. Fickus notes state that "(ii) $\mathbf{V}$ [synthesis operator] is invertible", is equivalent to "(i) ${\mathbf{v}_n}_{n\in \mathcal{N}}$ is a basis for $\mathbf{V}$". To evaluate if the given synthesis function constitutes a basis consider 
\begin{align*}
	\mathbf{V}(\mathbf{x}) = \sum_{n\in\mathcal{N}}\mathbf{x}(n)\mathbf{v}_n = 
	\mathbf{x}(1)\begin{bmatrix} 1 \\ 0 \\ 0 \end{bmatrix} +
	\mathbf{x}(2)\begin{bmatrix} 2 \\ 1 \\ 0 \end{bmatrix} +
	\mathbf{x}(3)\begin{bmatrix} 0 \\ -1 \\ 1 \end{bmatrix} =
	\begin{bmatrix}
		\mathbf{x}(1) + 2\mathbf{x}(2) \\
		\mathbf{x}(2) - \mathbf{x}(3) \\
		\mathbf{x}(3)
	\end{bmatrix}
	\begin{matrix} \\ \\ . \end{matrix}
\end{align*}


To evaluate linear independence it is sufficient to solve
\[
\begin{bmatrix}
	\mathbf{x}(1) + 2\mathbf{x}(2) \\
	\mathbf{x}(2) - \mathbf{x}(3) \\
	\mathbf{x}(3)
\end{bmatrix}
= 
\begin{bmatrix} 0 \\ 0 \\ 0 \end{bmatrix}
\begin{matrix} \\ \\ . \end{matrix}
\]

This shows that to produce $\mathbf{0}$, each $\mathbf{x}(n)$ must equal zero, and therefore that this sequence of vectors is linearly independent. 
Additionally, because this set of \(\{\mathbf{v}_{n}\}_{n=1}^{3}\) contains three linearly independent vectors in $\mathbb{R}^3$ we know that this sequence spans $\mathbb{R}^3$. 
Since this sequence is linearly independent and spans $\mathbb{R}^3$ it constitutes a basis. 
Recalling theorem 5.5(d), that this sequence is a basis implies that \(\mathbf{V}\) is invertible.

To calculate the members of \(\{\mathbf{u}_{n}\}_{n=1}^{3}\) let
$\boldsymbol{\delta}$ be the standard basis and
consider that for $n\in[3]$ 
\[
\mathbf{V}^{-1}(\boldsymbol{\delta}_n) = \mathbf{w}_n  
\]
and since $\mathbf{V}^{-1}$ is invertable
\[
\mathbf{V}(\mathbf{w}_n) = \boldsymbol{\delta}_n .
\].
Evaluated at each $n\in[3]$ we see
\begin{align*}
	\mathbf{V}(\mathbf{w}_1) =
	\begin{bmatrix}
		\mathbf{w}_1(1) + 2\mathbf{w}_1(2) \\
		\mathbf{w}_1(2) -  \mathbf{w}_1(3) \\
		\mathbf{w}_1(3)
	\end{bmatrix}
	&= 
	\begin{bmatrix} 1 \\ 0 \\ 0 \end{bmatrix}
	\rightarrow
	\mathbf{w}_1 =
	\begin{bmatrix} 1 \\ 0 \\ 0 \end{bmatrix}
	\\
	\mathbf{V}(\mathbf{w}_2) =
	\begin{bmatrix}
		\mathbf{w}_2(1) + 2\mathbf{w}_2(2) \\
		\mathbf{w}_2(2) -  \mathbf{w}_2(3) \\
		\mathbf{w}_2(3)
	\end{bmatrix}
	&= 
	\begin{bmatrix} 0 \\ 1 \\ 0 \end{bmatrix}
	\rightarrow
	\mathbf{w}_3 =
	\begin{bmatrix} -2 \\ 1 \\ 0 \end{bmatrix}
	\\
	\mathbf{V}(\mathbf{w}_3) =
	\begin{bmatrix}
		\mathbf{w}_3(1) + 2\mathbf{w}_3(2) \\
		\mathbf{w}_3(2) -  \mathbf{w}_3(3) \\
		\mathbf{w}_3(3)
	\end{bmatrix}
	&= 
	\begin{bmatrix} 0 \\ 0 \\ 1 \end{bmatrix}
	\rightarrow
	\mathbf{w}_3 =
	\begin{bmatrix} -2 \\ 1 \\ 1 \end{bmatrix}
	\begin{matrix} \\ \\ . \end{matrix}
\end{align*}
This is the sequence \(\{\mathbf{w}_{n}\}_{n=1}^{3}\) such that \(\mathbf{V}^{-1}\) is the synthesis operator of \(\{\mathbf{w}_{n}\}_{n=1}^{3}\).


\clearpage
b)
To find a sequence 
\(\{\boldsymbol{\ell}_{n}\}_{n=1}^{3}\) such that 
\(\mathbf{L}(\mathbf{v}_{n}) = \mathbf{u}_{n}\) is the synthesis operator,
we may evaluate 
\(\{\boldsymbol{\ell}_{n}\}_{n=1}^{3}\)
for each 
\(\{\mathbf{v}_{n}\}_{n=1}^{3}\) and
\(\{\mathbf{u}_{n}\}_{n=1}^{3}\) for $n\in[3]$.

The value of $\boldsymbol{\ell}_{1}$ can be calculated by evaluating
\begin{align*}
	\mathbf{L}(\mathbf{v}_1) 
	= 
	\mathbf{v}_1(1)\boldsymbol{\ell}_{1} +
	\mathbf{v}_1(2)\boldsymbol{\ell}_{2} +
	\mathbf{v}_1(3)\boldsymbol{\ell}_{3} =
	\mathbf{u}_1 &= \begin{bmatrix} 1 \\ 2 \end{bmatrix} \\
%		
	1\boldsymbol{\ell}_{1} +
	0\boldsymbol{\ell}_{2} +
	0\boldsymbol{\ell}_{3} &=
	\begin{bmatrix}1 \\ 2 \end{bmatrix} \\
%		
	\boldsymbol{\ell}_{1} &=
	\begin{bmatrix}1 \\ 2 \end{bmatrix} 
	\begin{matrix} \\ \\ . \end{matrix}
\end{align*}
The value of $\boldsymbol{\ell}_{2}$ can be calculated by evaluating  
\begin{align*}
	\mathbf{L}(\mathbf{v}_2) 
	= 
	\mathbf{v}_2(1)\boldsymbol{\ell}_{1} +
	\mathbf{v}_2(2)\boldsymbol{\ell}_{2} +
	\mathbf{v}_2(3)\boldsymbol{\ell}_{3} =
	\mathbf{u}_2 &= \begin{bmatrix} 1 \\ 2 \end{bmatrix} \\
	%		
	2\boldsymbol{\ell}_{1} +
	1\boldsymbol{\ell}_{2} +
	0\boldsymbol{\ell}_{3} &=
	\begin{bmatrix}1 \\ 2 \end{bmatrix} \\
	%		
	\begin{bmatrix} 2 \\ 4 \end{bmatrix} +
	\boldsymbol{\ell}_{2} &=
	\begin{bmatrix}1 \\ 2 \end{bmatrix} \\
	\boldsymbol{\ell}_{2} &=
	\begin{bmatrix} -1 \\ -2 \end{bmatrix}
	\begin{matrix} \\ \\ . \end{matrix}
\end{align*}
The value of $\boldsymbol{\ell}_{3}$ can be calculated by evaluating
\begin{align*}
	\mathbf{L}(\mathbf{v}_3) 
	= 
	\mathbf{v}_3(1)\boldsymbol{\ell}_{1} +
	\mathbf{v}_3(2)\boldsymbol{\ell}_{2} +
	\mathbf{v}_3(3)\boldsymbol{\ell}_{3} =
	\mathbf{u}_3 &= \begin{bmatrix} 0 \\ -1 \end{bmatrix} \\
	%		
	0\boldsymbol{\ell}_{1} +
	(-1)\boldsymbol{\ell}_{2} +
	1\boldsymbol{\ell}_{3} &=
	\begin{bmatrix} 0 \\ -1 \end{bmatrix} \\
	%		
	\begin{bmatrix} 1 \\ 2 \end{bmatrix} +
	\boldsymbol{\ell}_{3} &=
	\begin{bmatrix} 0 \\ -1 \end{bmatrix} \\
	%
	\boldsymbol{\ell}_{3} &=
	\begin{bmatrix} -1 \\ -3 \end{bmatrix}
	\begin{matrix} \\ \\ . \end{matrix}
\end{align*}

Thus we find 
\[
\boldsymbol{\ell}_{1} =
\begin{bmatrix}1 \\ 2 \end{bmatrix} 
,
\boldsymbol{\ell}_{2} =
\begin{bmatrix} -1 \\ -2 \end{bmatrix}
,
\boldsymbol{\ell}_{3} =
\begin{bmatrix} -1 \\ -3 \end{bmatrix}
\]
a viable sequence \(\{\boldsymbol{\ell}_{n}\}_{n=1}^{3}\) for the synthesis operator
\(\mathbf{L}:\mathbb{R}^{3}\to\mathbb{R}^{2}\) as defined above.



\clearpage

\item In this problem we will show that if we have a basis \(\{\mathbf{v}_{n}\}_{n\in\mathcal{N}}\) for a vector space \(\mathcal{V}\), and for each basis element \(\mathbf{v}_{n}\) we select some vector \(\mathbf{u}_{n}\) in the vector space \(\mathcal{U}\), then there is a unique linear map \(\mathbf{L}:\mathcal{V}\to\mathcal{U}\) such that \(\mathbf{L}(\mathbf{v}_{n}) = \mathbf{u}_{n}\). This has two important consequences:

\begin{itemize}
\item If we simply prescribe where each element of a basis for \(\mathcal{V}\) gets mapped to, then there exists a \textbf{linear} map on the \textbf{whole space} \(\mathcal{V}\) that maps each basis element in the domain to the prescribed vectors in the codomain. That is, we can define a linear map by simply describing how it acts on a basis for the domain.

\item If two linear functions \(\mathbf{L},\mathbf{M}:\mathcal{V}\to\mathcal{U}\) have the same output for all inputs in some basis for \(\mathcal{V}\), then the two functions have the same output for \textbf{every} input in \(\mathcal{V}\), that is, they are the same function. That is, a linear function is entirely determined by how it acts on a basis.
\end{itemize}

Let \(\mathcal{V}\) and \(\mathcal{U}\) be vector spaces over the field \(\mathbb{F}\). Suppose \(\{\mathbf{v}_{n}\}_{n\in\mathcal{N}}\) is a basis for \(\mathcal{V}\) and \(\{\mathbf{u}_{n}\}_{n\in\mathcal{N}}\) is a sequence in \(\mathcal{U}\). 




\bigskip

\begin{enumerate}

\item Let \(\mathbf{V}:\mathbb{F}^{\mathcal{N}}\to\mathcal{V}\) be the synthesis operator of the sequence \(\{\mathbf{v}_{n}\}_{n\in\mathcal{N}}\). Since \(\{\mathbf{v}_{n}\}_{n\in\mathcal{N}}\) is a basis, by a theorem from class the synthesis operator is a bijective function. Therefore, \(\mathbf{V}\) has an inverse \(\mathbf{V}^{-1}:\mathcal{V}\to\mathbb{F}^{\mathcal{N}}\). Prove that
\[\mathbf{w} = \sum_{n\in\mathcal{N}}\big(\mathbf{V}^{-1}(\mathbf{w})\big)(n)\mathbf{v}_{n}\quad\text{for all }\mathbf{w}\in\mathcal{V}.\]

\bigskip


\item Show that \(\mathbf{V}^{-1}\) is a linear map. (Hint: Use Part (a) to write \(c_{1}\mathbf{w}_{1} + c_{2}\mathbf{w}_{2}\) as a linear combination of \(\{\mathbf{v}_{n}\}_{n\in\mathcal{N}}\) in two different ways.)

\bigskip

\item Let \(\mathbf{U}:\mathbb{F}^{\mathcal{N}}\to\mathcal{U}\) be the synthesis operator of \(\{\mathbf{u}_{n}\}_{n\in\mathcal{N}}\). Show that the composition \(\mathbf{L}:=\mathbf{U}\circ\mathbf{V}^{-1}:\mathcal{V}\to\mathcal{U}\) is linear.

\bigskip

\item Prove \(\mathbf{L}:\mathcal{V}\to\mathcal{U}\) is the unique linear map such that
\[\mathbf{L}(\mathbf{v}_{n}) = \mathbf{u}_{n}\quad\text{for all }n\in\mathcal{N}.\]

\bigskip
\end{enumerate}
\hrule
\bigskip

(a) 
\begin{proof}
Let $\mathbf{x}\in\mathbb{F}^\mathcal{N}$ such that 
$\mathbf{V}(\mathbf{x}) = \mathbf{w}$. Then for all $\mathbf{w}\in\mathcal{V}$
\begin{align*}
	\mathbf{w} 
	&= \mathbf{V}(\mathbf{x}) \\
	&= \sum_{n\in\mathcal{N}}\mathbf{x}(n)\mathbf{v}_n \\
	&= \sum_{n\in\mathcal{N}}(\mathbf{V}^{-1}(\mathbf{w}))(n)\mathbf{v}_n .
\end{align*}
\end{proof}

\clearpage
(b)
To show that $\mathbf{V}^{-1}$ is a linear map using
definition 2.1 from the Dr. Fickus notes which states that for vector spaces $\mathcal{V}$ and $\mathcal{U}$ over common field $\mathbb{F}$ a function $\mathbf{L}:\mathcal{V}\rightarrow\mathcal{U}$ is linear if for all $c_1,c_2\in\mathbb{F},\mathbf{v}_1,\mathbf{v}_2\in\mathcal{V}$,
\[
\mathbf{L}(c_1\mathbf{v}_1+c_2\mathbf{v}_2) = c_1\mathbf{L}(\mathbf{v}_1)+c_2\mathbf{L}(\mathbf{v}_2).
\]

\begin{proof}
For any $\mathbf{w}_1,\mathbf{w}_2\in\mathcal{V}$ and any $c_1,c_2\in\mathbb{F}$,
\begin{align*}
	c_1\mathbf{w}_1 + c_2\mathbf{w}_2
	&= c_1 \sum_{n\in\mathcal{N}}(\mathbf{V}^{-1}(\mathbf{w}_1))(n)\mathbf{v}_n +
		c_2 \sum_{n\in\mathcal{N}}(\mathbf{V}^{-1}(\mathbf{w}_2))(n)\mathbf{v}_n \\
	&= \sum_{n\in\mathcal{N}}c_1(\mathbf{V}^{-1}(\mathbf{w}_1))(n)\mathbf{v}_n +
		\sum_{n\in\mathcal{N}}c_2(\mathbf{V}^{-1}(\mathbf{w}_2))(n)\mathbf{v}_n \\
	&= \sum_{n\in\mathcal{N}}c_1(\mathbf{V}^{-1}(\mathbf{w}_1))(n)\mathbf{v}_n +
		c_2(\mathbf{V}^{-1}(\mathbf{w}_2))(n)\mathbf{v}_n \\
	&= \sum_{n\in\mathcal{N}}\big(c_1\mathbf{V}^{-1}(\mathbf{w}_1) +
	c_2\mathbf{V}^{-1}(\mathbf{w}_2)\big)(n)\mathbf{v}_n \\
\shortintertext{and}
	(c_1\mathbf{w}_1 + c_2\mathbf{w}_2)
	&= \sum_{n\in\mathcal{N}} \mathbf{V}^{-1}(c_1\mathbf{w}_1 + c_2\mathbf{w}_2)(n)\mathbf{v}_n .
\shortintertext{Thus,}
	\sum_{n\in\mathcal{N}} \mathbf{V}^{-1}(c_1\mathbf{w}_1 + c_2\mathbf{w}_2)(n)\mathbf{v}_n
	&=
	\sum_{n\in\mathcal{N}}\big(c_1\mathbf{V}^{-1}(\mathbf{w}_1) +
	c_2\mathbf{V}^{-1}(\mathbf{w}_2)\big)(n)\mathbf{v}_n
\intertext{and therefore}
	\mathbf{V}^{-1}(c_1\mathbf{w}_1 + c_2\mathbf{w}_2)
	&= 
	c_1\mathbf{V}^{-1}(\mathbf{w}_1) +
	c_2\mathbf{V}^{-1}(\mathbf{w}_2).
\end{align*}
\end{proof}

(c) 
\begin{proof}
Let
\(\mathbf{U}:\mathbb{F}^{\mathcal{N}}\to\mathcal{U}\) be the synthesis operator of \(\{\mathbf{u}_{n}\}_{n\in\mathcal{N}}\),
\(\mathbf{L}:=\mathbf{U}\circ\mathbf{V}^{-1}:\mathcal{V}\to\mathcal{U}\),
$\mathbf{w}_1,\mathbf{w}_2\in\mathcal{V}$, and
$c_1,c_2\in\mathbb{F}$.
Then,
\begin{align*}
	\mathbf{L}(c_1\mathbf{w}_1 + c_2\mathbf{w}_2)
	&= \mathbf{U}(\mathbf{V}^{-1}(c_1\mathbf{w}_1 + c_2\mathbf{w}_2)) \\
	&= \mathbf{U}(\mathbf{V}^{-1}(c_1\mathbf{w}_1) + \mathbf{V}^{-1}(c_2\mathbf{w}_2)) \\
	&= \mathbf{U}(c_1\mathbf{V}^{-1}(\mathbf{w}_1) + c_2\mathbf{V}^{-1}(\mathbf{w}_2)) \\
	&= \sum_{n\in\mathcal{N}}\mathbf{u}_n(c_1\mathbf{V}^{-1}(\mathbf{w}_1) + c_2\mathbf{V}^{-1}(\mathbf{w}_2))(n) \\
	&= \sum_{n\in\mathcal{N}}\mathbf{u}_n(c_1\mathbf{V}^{-1}(\mathbf{w}_1))(n) + \sum_{n\in\mathcal{N}}\mathbf{u}_n(c_2\mathbf{V}^{-1}(\mathbf{w}_2))(n) \\
	&= \sum_{n\in\mathcal{N}}c_1\mathbf{u}_n(\mathbf{V}^{-1}(\mathbf{w}_1))(n) + \sum_{n\in\mathcal{N}}c_2\mathbf{u}_n(\mathbf{V}^{-1}(\mathbf{w}_2))(n) \\
	&= c_1\sum_{n\in\mathcal{N}}\mathbf{u}_n(\mathbf{V}^{-1}(\mathbf{w}_1))(n) + c_2\sum_{n\in\mathcal{N}}\mathbf{u}_n(\mathbf{V}^{-1}(\mathbf{w}_2))(n) \\
	&= c_1\mathbf{U}(\mathbf{V}^{-1}(\mathbf{w}_1)) + c_2\mathbf{U}(\mathbf{V}^{-1}(\mathbf{w}_2)) \\
	&= c_1\mathbf{L}(\mathbf{w}_1 + c_2\mathbf{L}(\mathbf{w}_2).
\end{align*}
Since 
$\mathbf{L}(c_1\mathbf{w}_1 + c_2\mathbf{w}_2) = 
c_1\mathbf{L}(\mathbf{w}_1 + c_2\mathbf{L}(\mathbf{w}_2)$
for all
$\mathbf{w}_1,\mathbf{w}_2\in\mathcal{V}$ and
$c_1,c_2\in\mathbb{F}$,
$\mathbf{L}$ is linear.
\end{proof}

\clearpage
(d)
\begin{proof}
Let $\mathbf{w}\in\mathbb{F}^\mathcal{N}$ and $\mathbf{x} \in \text{span }\{\mathbf{v}_{n}\}_{n\in\mathcal{N}}$
be such that $\mathbf{x} = \sum_{n\in\mathcal{N}} \mathbf{w}(n)\mathbf{v}_n$.
Assume for the same of contradiction that $\mathbf{L}$ and $\mathbf{M}$
are distinct linear mappings such that
\(\mathbf{L},\mathbf{M}:\mathcal{V}\to\mathcal{U}\),
\(\mathbf{L}(\mathbf{v}_{n}) = \mathbf{u}_{n}\) and
\(\mathbf{M}(\mathbf{v}_{n}) = \mathbf{u}_{n}\)
for all \(n\in\mathcal{N}\). As distinct mappings it is expected that
$\mathbf{L}(\mathbf{x})\ne\mathbf{M}(\mathbf{x})$, however, 

\begin{align*}
	\mathbf{L}(\mathbf{x})
	&= \mathbf{L}(\sum_{n\in\mathcal{N}} \mathbf{w}(n)\mathbf{v}_n) \\
	&= \sum_{n\in\mathcal{N}} \mathbf{L}(\mathbf{w}(n)\mathbf{v}_n) \\
	&= \sum_{n\in\mathcal{N}} \mathbf{w}(n)\mathbf{L}(\mathbf{v}_n) \\
	&= \sum_{n\in\mathcal{N}} \mathbf{w}(n)\mathbf{u}_n \\
	&= \sum_{n\in\mathcal{N}} \mathbf{w}(n)\mathbf{M}(\mathbf{v}_n) \\
	&= \sum_{n\in\mathcal{N}} \mathbf{M}(\mathbf{w}(n)\mathbf{v}_n) \\
	&= \mathbf{M}(\sum_{n\in\mathcal{N}} \mathbf{w}(n)\mathbf{v}_n) \\
	&= \mathbf{M}(\mathbf{x}).
\end{align*}
Thus $\mathbf{L}(\mathbf{x})$ and $\mathbf{M}(\mathbf{x})$ cannot have all of the same outputs for all inputs and be distinct.
\end{proof}

\vspace*{\fill}
\item Go to \href{https://eigenquiz.app/}{Eigenquiz.app} and complete activity a697132. Upload a completion certificate for this activity to your homework folder.
\end{enumerate}

\end{document}