\documentclass[12pt]{amsart}


%\usepackage[notref,notcite]{showkeys}
\usepackage{verbatim}
\usepackage{fullpage}
\usepackage{amsfonts}
\usepackage{bbm}
\usepackage{wrapfig}
\usepackage{enumerate}
\usepackage{float}
\usepackage{hyperref}
\usepackage[left=0.5in, right=0.5in, bottom=0.75in, top=0.75in]{geometry}
\setlength{\parindent}{0pt}


\usepackage{color}
\usepackage{colortbl}
\usepackage{graphicx}
\usepackage{epstopdf}
\usepackage{amsmath, amsthm, amssymb}
\usepackage{fancybox}

\newcommand{\1}{\mathbbm{1}}

\newcommand{\eps}{\varepsilon}
\newcommand{\la}{\lambda}
\newcommand{\Contradiction}{\Rightarrow\Leftarrow}
\DeclareMathOperator{\lspan}{span}
\DeclareMathOperator{\tr}{tr}
\DeclareMathOperator{\ran}{ran}
\DeclareMathOperator{\diag}{diag}
\providecommand{\abs}[1]{\lvert#1\rvert}
\providecommand{\norm}[1]{\lVert#1\rVert}
\renewcommand{\theenumi}{\alph{enumi}}
\renewcommand{\labelenumi}{(\theenumi)}

\newcounter{Theorem}
\newcounter{Definition}
\numberwithin{equation}{section}
\numberwithin{Theorem}{section}

\theoremstyle{plain} %% This is the default, anyway
\newtheorem{thm}[Theorem]{Theorem}
\newtheorem{cor}[Theorem]{Corollary}
\newtheorem{lem}[Theorem]{Lemma}
\newtheorem{prop}[Theorem]{Proposition}
%\usepackage{upgreek}

\theoremstyle{definition}
\newtheorem{defn}[Theorem]{Definition}

\theoremstyle{remark}
\newtheorem{remark}{Remark}[section]
\newtheorem{ex}[Theorem]{Example}
\newtheorem{nota}[Theorem]{Notation}



\begin{document}

\thispagestyle{empty}

\noindent{\Large Homework 3 (Due October 21 at 8am)}\bigskip





\begin{enumerate}[1.]

\item This problem is a warmup for the next:

Consider the sequences \(\{\mathbf{v}_{n}\}_{n=1}^{3}\) and \(\{\mathbf{u}_{n}\}_{n=1}^{3}\) in \(\mathbb{R}^{3}\) and \(\mathbb{R}^{2}\), respectively, where
\[\mathbf{v}_{1} = \begin{bmatrix} 1\\ 0\\ 0\end{bmatrix},\ \mathbf{v}_{2} = \begin{bmatrix} 2\\ 1\\ 0\end{bmatrix},\ \mathbf{v}_{3} = \begin{bmatrix} 0\\ -1\\ 1\end{bmatrix},\ \mathbf{u}_{1} = \begin{bmatrix} 1\\ 2\end{bmatrix},\ \mathbf{u}_{2} = \begin{bmatrix} 1\\ 2\end{bmatrix},\ \mathbf{u}_{3} = \begin{bmatrix} 0\\ -1\end{bmatrix}.\]

Since we have proved that all synthesis operators are linear, part (a) shows that \(\mathbf{V}^{-1}\) is linear, and part (b) shows that there exists a \textit{linear} map \(\mathbf{L}:\mathbb{R}^{3}\to\mathbb{R}^{2}\) with the property that \(\mathbf{L}(\mathbf{v}_{n}) = \mathbf{u}_{n}\) for each \(n\in[3]\).

\begin{enumerate}

\item Let \(\mathbf{V}:\mathbb{R}^{3}\to\mathbb{R}^{3}\) be the synthesis operator of \(\{\mathbf{v}_{n}\}_{n=1}^{3}\). What theorem from class tells us that \(\mathbf{V}\) is an invertible function? Find a sequence \(\{\mathbf{w}_{n}\}_{n=1}^{3}\) in \(\mathbb{R}^{3}\) such that \(\mathbf{V}^{-1}\) is the synthesis operator of \(\{\mathbf{w}_{n}\}_{n=1}^{3}\). Show that your answer is correct.

\bigskip

\item Find a sequence \(\{\boldsymbol{\ell}_{n}\}_{n=1}^{3}\) such that if \(\mathbf{L}:\mathbb{R}^{3}\to\mathbb{R}^{2}\) is the synthesis operator of \(\{\boldsymbol{\ell}_{n}\}_{n=1}^{3}\), then \(\mathbf{L}(\mathbf{v}_{n}) = \mathbf{u}_{n}\) for all \(n\in[3]\). (You don't need to include the proof that the sequence you give has the desired property, but you should prove it to yourself!)

\bigskip

\end{enumerate}

a) Theorem 5.5(d) from the Dr. Fikus notes state that a synthesis operator is by definition invertable. \\


To find a sequence \(\{\mathbf{w}_{n}\}_{n=1}^{3}\) with a synthesis operator \(\mathbf{V}^{-1}\)

Define $\mathbf{x}$




\[
\begin{bmatrix}
	\mathbf{w}(1) + 2\mathbf{w}(2) \\
	\mathbf{w}(2) - \mathbf{w}(3) \\
	\mathbf{w}(3)
\end{bmatrix}
\]








\clearpage

\item In this problem we will show that if we have a basis \(\{\mathbf{v}_{n}\}_{n\in\mathcal{N}}\) for a vector space \(\mathcal{V}\), and for each basis element \(\mathbf{v}_{n}\) we select some vector \(\mathbf{u}_{n}\) in the vector space \(\mathcal{U}\), then there is a unique linear map \(\mathbf{L}:\mathcal{V}\to\mathcal{U}\) such that \(\mathbf{L}(\mathbf{v}_{n}) = \mathbf{u}_{n}\). This has two important consequences:

\begin{itemize}
\item If we simply prescribe where each element of a basis for \(\mathcal{V}\) gets mapped to, then there exists a \textbf{linear} map on the \textbf{whole space} \(\mathcal{V}\) that maps each basis element in the domain to the prescribed vectors in the codomain. That is, we can define a linear map by simply describing how it acts on a basis for the domain.

\item If two linear functions \(\mathbf{L},\mathbf{M}:\mathcal{V}\to\mathcal{U}\) have the same output for all inputs in some basis for \(\mathcal{V}\), then the two functions have the same output for \textbf{every} input in \(\mathcal{V}\), that is, they are the same function. That is, a linear function is entirely determined by how it acts on a basis.
\end{itemize}

Let \(\mathcal{V}\) and \(\mathcal{U}\) be vector spaces over the field \(\mathbb{F}\). Suppose \(\{\mathbf{v}_{n}\}_{n\in\mathcal{N}}\) is a basis for \(\mathcal{V}\) and \(\{\mathbf{u}_{n}\}_{n\in\mathcal{N}}\) is a sequence in \(\mathcal{U}\). 




\bigskip

\begin{enumerate}

\item Let \(\mathbf{V}:\mathbb{F}^{\mathcal{N}}\to\mathcal{V}\) be the synthesis operator of the sequence \(\{\mathbf{v}_{n}\}_{n\in\mathcal{N}}\). Since \(\{\mathbf{v}_{n}\}_{n\in\mathcal{N}}\) is a basis, by a theorem from class the synthesis operator is a bijective function. Therefore, \(\mathbf{V}\) has an inverse \(\mathbf{V}^{-1}:\mathcal{V}\to\mathbb{F}^{\mathcal{N}}\). Prove that
\[\mathbf{w} = \sum_{n\in\mathcal{N}}\big(\mathbf{V}^{-1}(\mathbf{w})\big)(n)\mathbf{v}_{n}\quad\text{for all }\mathbf{w}\in\mathcal{V}.\]

\bigskip


\item Show that \(\mathbf{V}^{-1}\) is a linear map. (Hint: Use Part (a) to write \(c_{1}\mathbf{w}_{1} + c_{2}\mathbf{w}_{2}\) as a linear combination of \(\{\mathbf{v}_{n}\}_{n\in\mathcal{N}}\) in two different ways.)

\bigskip

\item Let \(\mathbf{U}:\mathbb{F}^{\mathcal{N}}\to\mathcal{U}\) be the synthesis operator of \(\{\mathbf{u}_{n}\}_{n\in\mathcal{N}}\). Show that the composition \(\mathbf{L}:=\mathbf{U}\circ\mathbf{V}^{-1}:\mathcal{V}\to\mathcal{U}\) is linear.

\bigskip

\item Prove \(\mathbf{L}:\mathcal{V}\to\mathcal{U}\) is the unique linear map such that
\[\mathbf{L}(\mathbf{v}_{n}) = \mathbf{u}_{n}\quad\text{for all }n\in\mathcal{N}.\]


\bigskip


\end{enumerate}



\vspace*{\fill}
\item Go to \href{https://eigenquiz.app/}{Eigenquiz.app} and complete activity a697132. Upload a completion certificate for this activity to your homework folder.
\end{enumerate}

\end{document}