\documentclass[12pt]{amsart}


%\usepackage[notref,notcite]{showkeys}
\usepackage{verbatim}
\usepackage{fullpage}
\usepackage{amsfonts}
\usepackage{bbm}
\usepackage{wrapfig}
\usepackage{enumerate}
\usepackage{float}
\usepackage{hyperref}
\usepackage[left=0.5in, right=0.5in, bottom=0.75in, top=0.75in]{geometry}
\setlength{\parindent}{0pt}


\usepackage{color}
\usepackage{colortbl}
\usepackage{graphicx}
\usepackage{epstopdf}
\usepackage{amsmath, amsthm, amssymb}
\usepackage{fancybox}

\newcommand{\1}{\mathbbm{1}}

\newcommand{\eps}{\varepsilon}
\newcommand{\la}{\lambda}
\newcommand{\Contradiction}{\Rightarrow\Leftarrow}
\DeclareMathOperator{\lspan}{span}
\DeclareMathOperator{\tr}{tr}
\DeclareMathOperator{\ran}{ran}
\DeclareMathOperator{\diag}{diag}
\providecommand{\abs}[1]{\lvert#1\rvert}
\providecommand{\norm}[1]{\lVert#1\rVert}
\renewcommand{\theenumi}{\alph{enumi}}
\renewcommand{\labelenumi}{(\theenumi)}

\newcounter{Theorem}
\newcounter{Definition}
\numberwithin{equation}{section}
\numberwithin{Theorem}{section}

\theoremstyle{plain} %% This is the default, anyway
\newtheorem{thm}[Theorem]{Theorem}
\newtheorem{cor}[Theorem]{Corollary}
\newtheorem{lem}[Theorem]{Lemma}
\newtheorem{prop}[Theorem]{Proposition}
%\usepackage{upgreek}

\theoremstyle{definition}
\newtheorem{defn}[Theorem]{Definition}

\theoremstyle{remark}
\newtheorem{remark}{Remark}[section]
\newtheorem{ex}[Theorem]{Example}
\newtheorem{nota}[Theorem]{Notation}

\newif\ifanswers

%\answerstrue

\begin{document}

\thispagestyle{empty}

\noindent{\Large Homework 1\ifanswers\ Solutions\fi. (Due October 7 at 8am)}\bigskip



On this homework, we consider a way to combine two vector spaces to produce another vector space,
as well as two ways to combine two subspaces of a common vector space to produce other subspaces of it.\bigskip

\begin{enumerate}[1.]
\item
Let $\mathcal{V}$ and $\mathcal{W}$ be two vector spaces over the same field $\mathbb{F}$.
The \textit{direct sum} $\mathcal{V}\oplus\mathcal{W}$ of $\mathcal{V}$ and $\mathcal{W}$ is their Cartesian product $\mathcal{V}\times\mathcal{W}:=\{(\mathbf{v},\mathbf{w}): \mathbf{v}\in\mathcal{V}, \mathbf{w}\in\mathcal{W}\}$ equipped with entrywise addition and scalar multiplication:
\begin{equation*}
(\mathbf{v}_1,\mathbf{w}_1)+(\mathbf{v}_2,\mathbf{w}_2):=(\mathbf{v}_1+\mathbf{v}_2,\mathbf{w}_1+\mathbf{w}_2),
\quad
c(\mathbf{v},\mathbf{w}):=(c\mathbf{v},c\mathbf{w}),
\end{equation*}
for all \(c\in\mathbb{F}\) and \(\mathbf{v},\mathbf{v}_1,\mathbf{v}_2\in\mathcal{V},\,\mathbf{w},\mathbf{w}_1,\mathbf{w}_2\in\mathcal{W}\). Show that $\mathcal{V}\oplus\mathcal{W}$ is a vector space by directly proving that it satisfies the eight properties of a vector space.

\textit{Hint: These properties are ``inherited" from $\mathcal{V}$ and $\mathcal{W}$.
For example, since $\mathcal{V}$ and $\mathcal{W}$ are vector spaces, their respective addition operations are commutative, satisfying $\mathbf{v}_1+\mathbf{v}_2=\mathbf{v}_2+\mathbf{v}_1$ for all $\mathbf{v}_1,\mathbf{v}_2\in\mathcal{V}$ and $\mathbf{w}_1+\mathbf{w}_2=\mathbf{w}_2+\mathbf{w}_1$ for all $\mathbf{w}_1,\mathbf{w}_2\in\mathcal{W}$,
and so for any $(\mathbf{v}_1,\mathbf{w}_1),(\mathbf{v}_2,\mathbf{w}_2)\in\mathcal{V}\oplus\mathcal{W}$,
\begin{equation*}
(\mathbf{v}_1,\mathbf{w}_1)+(\mathbf{v}_2,\mathbf{w}_2)
=(\mathbf{v}_1+\mathbf{v}_2,\mathbf{w}_1+\mathbf{w}_2)
=(\mathbf{v}_2+\mathbf{v}_1,\mathbf{w}_2+\mathbf{w}_1)
=(\mathbf{v}_2,\mathbf{w}_2)+(\mathbf{v}_1,\mathbf{w}_1),
\end{equation*}
as desired.
Use a similar argument to prove that $\mathcal{V}\oplus\mathcal{W}$ inherits the remaining seven properties of a vector space from the assumption that $\mathcal{V}$ and $\mathcal{W}$ satisfy them individually.}
%\end{enumerate}
\bigskip\hrule\bigskip

Above we have seen that $\mathcal{V}\oplus\mathcal{W}$ satisfies the commutative property. Below we will demonstrate the remaining properties of a vector space. \\
\textbf{Proof.} \\

\begin{itemize}
	\item[1.2] Addition is Associative. \\
	Let $v_1,v_2,v_3 \in \mathcal{V}$ and $w_1,w_2,w_3 \in \mathcal{W}$ be arbitrary.
	\begin{align*}
		(v_1,w_1)+[(v_2,w_2)+(v_3,w_3)] 
		&=   (v_1,w_1)+(v_2+v_3, w_2+w_3) \\
		&=   (v_1+ (v_2+v_3), w_1+ (w_2+w_3)) \\
		&=   (v_1+ (v_2+v_3),  w_1+ (w_2+w_3)) \\
		&=   (v_1+ v_2+v_3,  w_1+ w_2+w_3) \\
		&=   ((v_1+ v_2)+v_3,  (w_1+ w_2)+w_3) \\
		&=   (v_1+ v_2 , w_1+ w_2) + (v_3, w_3) \\
		&=   [(v_1,w_1)+(v_2,w_2)]+(v_3,w_3)  
	\end{align*}
	Therefore, $(v_1,w_1)+[(v_2,w_2)+(v_3,w_3)]  =  [(v_1,w_1)+(v_2,w_2)]+(v_3,w_3)$ \\
	
	\item[1.3]  Additive Identity Element. \\
	\textbf{Proof:}
	Let $(v_0,w_0) =(0,0)$ \\
	Let $v_1 \in \mathcal{V}$ and $w_1 \in \mathcal{W}$ be arbitrary.
	\begin{align*}
				(v_1,w_1)+(v_0,w_0)
		&=   (v_1+v_0, w_1+w_0) \\
		&=   (v_1+0, w_1+0) \\
		&=   (v_1, w_1)
	\end{align*}
	Since $(v_1,w_1)+(v_0,w_0) = (v_1, w_1)$ we show that $(v_0,w_0) =(0,0)$ is an additive identity. \\
	
	\item[1.4]  Additive Inverse Element. \\
	\textbf{Proof:}
	Let $v_n \in \mathcal{V}$ and $w_n \in \mathcal{W}$ be arbitrary. \\
	Let $x_n = -v_n$ and $x\in \mathcal{V}$ for arbitrary $n$. \\
	Let $y_n = -w_n$ and $x\in \mathcal{W}$ for arbitrary $n$. 
	\begin{align*}
			(v_n,w_n)+(x_n,y_n)
		&= (v_n+x_n, w_n+y_n) \\
		&= (v_n-v_n, w_n-w_n) \\
		&= (0,0)
	\end{align*}
	This shows $v_n,w_n)+(x_n,y_n) = (0,0)$ when $x_n = -v_n$ and $y_n = -w_n$ \\
	
	\item[1.5]  Compatibility of scalar multiplication with field multiplication. \\
	\textbf{Proof:}
	Let $c_1, c_2 \in\mathbb{F}$
	and $v \in \mathcal{V}$ and $w \in \mathcal{W}$ be arbitrary.
	\begin{align*}
		[c_1(c_2(v,w))]
		&= (c_1(c_2v,c_2w)) \\
		&= (c_1(c_2v),c_1(c_2w)) \\
		&= (c_1c_2v,c_1c_2w) \\
		&= ((c_1c_2)v,(c_1c_2)w) \\
		&= (c_1c_2)(v,w) \\
	\end{align*}
	This shows $[c_1(c_2(v,w))] = (c_1c_2)(v,w)$ \\
	
	\item[1.6]  Multiplicative Identity Element. \\
	\textbf{Proof:}
	Let $v \in \mathcal{V}$ and $w \in \mathcal{W}$ and be arbitrary.
	\begin{align*}
		1(v,w) = (1v,1w) = (v,w) 
	\end{align*}
	Therefore $1(v,w) = (v,w)$ \\
	
	\item[1.7]  Distributivity of scalar multiplication with respect to vector addition. \\
	\textbf{Proof:}
	Let $c \in\mathbb{F}$
	and $v_1,v_2 \in \mathcal{V}$ and $w_1,w_2 \in \mathcal{W}$ and be arbitrary. 
	\begin{align*}
		c[(v_1,w_1)+(v_2,w_2)]
		&= c(v_1+v_2,w_1+w_2) \\
		&= (c(v_1+v_2),c(w_1+w_2)) \\
		&= ((cv_1+cv_2),(cw_1+cw_2)) \\
		&= (cv_1+cv_2,cw_1+cw_2) \\
		&= (cv_1,cw_1) + (cv_2,cw_2) \\
		&= c(v_1,w_1) + c(v_2,w_2)
	\end{align*}
	Thus $c[(v_1,w_1)+(v_2,w_2)] = c(v_1,w_1) + c(v_2,w_2)$ \\
	
	\item[1.8]  Distributivity of scalar multiplication with respect to field addition. \\
	\textbf{Proof:}
	Let $c_1, c_2 \in\mathbb{F}$
	and $v \in \mathcal{V}$ and $w \in \mathcal{W}$ be arbitrary. 
	\begin{align*}
		(c_1+c_2)(v,w)
		&= ((c_1+c_2)v,(c_1+c_2)w) \\
		&= (c_1v+c_2v,c_1w+c_2w) \\
		&= (c_1v,c_1w)+(c_2v,c_2w) \\
		&= c_1(v,w)+C-2(v,w) \\
	\end{align*}
	Thus $(c_1+c_2)(v,w) = c_1(v,w)+c_2(v,w)$. \\
	
	
\end{itemize}


\end{enumerate}
\bigskip\hrule\bigskip

The remaining problems involve \textit{subspaces} of vector spaces.
As discussed in class, if $\mathcal{V}$ is a vector space over a field $\mathbb{F}$,
then a subset $\mathcal{U}$ of $\mathcal{V}$ is a \textit{subspace} of $\mathcal{V}$ if $\mathbf{0}\in\mathcal{U}$ and $c_1\mathbf{u}_1+c_2\mathbf{u}_2\in\mathcal{U}$ for all $c_1,c_2\in\mathbb{F}$ and $\mathbf{u}_1,\mathbf{u}_2\in\mathcal{U}$.\bigskip

\begin{enumerate}[1.]
\setcounter{enumi}{1}
%%%%%%%%%%%%%%%%%%%%%%%%%%%%%%%%%%%%%%%%%%%%%%%%%%%%%%%%%%%%%%%%%%%%%%%%%%%%%%%%%%%%%%%%%%%%%%%%%%%%%%%%%%%%%%%%%%%%%%%%%%%%%%%%
\item
The \textit{intersection} of two subsets $\mathcal{U}_1$ and $\mathcal{U}_2$ of a set $\mathcal{V}$ is the set of points $\mathcal{V}$ that belong to both of them:
\begin{equation*}
\mathcal{U}_1\cap\mathcal{U}_2:=\{\mathbf{v}\in\mathcal{V}: \mathbf{v}\in\mathcal{U}_1\ \text{and}\ \mathbf{v}\in\mathcal{U}_2\}.
\end{equation*}
Show that if $\mathcal{V}$ is a vector space and $\mathcal{U}_1$ and $\mathcal{U}_2$ are both subspaces of $\mathcal{V}$ then $\mathcal{U}_1\cap\mathcal{U}_2$ is a subspace of $\mathcal{V}$.\bigskip

%%%%%%%%%%%%%%%%%%%%%%%%%%%%%%%%%%%%%%%%%%%%%%%%%%%%%%%%%%%%%%%%%%%%%%%%%%%%%%%%%%%%%%%%%%%%%%%%%%%%%%%%%%%%%%%%%%%%%%%%%%%%%%%%
\item
The \textit{union} of two subsets $\mathcal{U}_1$ and $\mathcal{U}_2$ of a set $\mathcal{V}$ is the set of points in $\mathcal{V}$ that belong to either of them:
\begin{equation*}
\mathcal{U}_1\cup\mathcal{U}_2:=\{\mathbf{v}\in\mathcal{V}: \mathbf{v}\in\mathcal{U}_1\ \text{or}\ \mathbf{v}\in\mathcal{U}_2\}.
\end{equation*}
Give an example of two subspaces $\mathcal{U}_1$ and $\mathcal{U}_2$ of $\mathbb{R}^2$ with the property that $\mathcal{U}_1\cup\mathcal{U}_2$ is not a subspace of $\mathbb{R}^2$.\bigskip

%%%%%%%%%%%%%%%%%%%%%%%%%%%%%%%%%%%%%%%%%%%%%%%%%%%%%%%%%%%%%%%%%%%%%%%%%%%%%%%%%%%%%%%%%%%%%%%%%%%%%%%%%%%%%%%%%%%%%%%%%%%%%%%%
\item
The \textit{sum} of two subspaces $\mathcal{U}_1$ and $\mathcal{U}_2$ of a set $\mathcal{V}$ is the of all points in $\mathcal{V}$ that can be written as a sum of a vector in $\mathcal{U}_1$ with a vector in $\mathcal{U}_2$:
\begin{equation*}
\mathcal{U}_1+\mathcal{U}_2:=\{\mathbf{u}_1+\mathbf{u}_2: \mathbf{u}_1\in\mathcal{U}_1,\mathbf{u}_2\in\mathcal{U}_2\}.
\end{equation*}
Show that $\mathcal{U}_1+\mathcal{U}_2$ is the smallest subspace of $\mathcal{V}$ that contains $\mathcal{U}_1\cup\mathcal{U}_2$ by proving the following three things:
\begin{enumerate}[(a)]
\item
$\mathcal{U}_1+\mathcal{U}_2$ is a subspace of $\mathcal{V}$;
\item
$\mathcal{U}_1\cup\mathcal{U}_2\subseteq\mathcal{U}_1+\mathcal{U}_2$;
\item
If $\mathcal{U}$ is any subspace of $\mathcal{V}$ such that $\mathcal{U}_1\cup\mathcal{U}_2\subseteq\mathcal{U}$ then $\mathcal{U}_1+\mathcal{U}_2\subseteq\mathcal{U}$.
\end{enumerate}\bigskip

\textit{Note:
In general, ``$\mathcal{A}\subseteq\mathcal{B}$" denotes that ``$\mathcal{A}$ is a subset of $\mathcal{B}$," namely that ``$\mathcal{A}$ is contained in $\mathcal{B}$" or equivalently that ``$\mathcal{B}$ contains $\mathcal{A}$"; these all just mean that every element of $\mathcal{A}$ is also an element of $\mathcal{B}$.}\bigskip

\item Go to \href{https://eigenquiz.app/}{Eigenquiz.app} and complete activity a966457. Upload a completion certificate for this activity to your homework folder.
\end{enumerate}



\end{document}