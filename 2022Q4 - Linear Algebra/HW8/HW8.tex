\documentclass[12pt]{amsart}


%\usepackage[notref,notcite]{showkeys}
\usepackage{verbatim}
\usepackage{fullpage}
\usepackage{amsfonts}
\usepackage{bbm}
\usepackage{wrapfig}
\usepackage{enumerate}
\usepackage{float}
\usepackage{hyperref}
\usepackage[left=0.5in, right=0.5in, bottom=0.75in, top=0.75in]{geometry}
\setlength{\parindent}{0pt}


\usepackage{color}
\usepackage{colortbl}
\usepackage{graphicx}
\usepackage{epstopdf}
\usepackage{amsmath, amsthm, amssymb}
\usepackage{fancybox}

\newcommand{\1}{\mathbbm{1}}

\newcommand{\eps}{\varepsilon}
\newcommand{\la}{\lambda}
\newcommand{\Contradiction}{\Rightarrow\Leftarrow}
\DeclareMathOperator{\lspan}{span}
\DeclareMathOperator{\tr}{tr}
\DeclareMathOperator{\ran}{ran}
\DeclareMathOperator{\diag}{diag}
\providecommand{\abs}[1]{\lvert#1\rvert}
\providecommand{\norm}[1]{\lVert#1\rVert}
\renewcommand{\theenumi}{\alph{enumi}}
\renewcommand{\labelenumi}{(\theenumi)}

\newcounter{Theorem}
\newcounter{Definition}
\numberwithin{equation}{section}
\numberwithin{Theorem}{section}

\theoremstyle{plain} %% This is the default, anyway
\newtheorem{thm}[Theorem]{Theorem}
\newtheorem{cor}[Theorem]{Corollary}
\newtheorem{lem}[Theorem]{Lemma}
\newtheorem{prop}[Theorem]{Proposition}
%\usepackage{upgreek}

\theoremstyle{definition}
\newtheorem{defn}[Theorem]{Definition}

\theoremstyle{remark}
\newtheorem{remark}{Remark}[section]
\newtheorem{ex}[Theorem]{Example}
\newtheorem{nota}[Theorem]{Notation}



\begin{document}

\thispagestyle{empty}

\noindent{\Large Homework 8 (Due December 2 at 8am)}\bigskip


\begin{enumerate}[1.]

\item Let \(\mathcal{V}\) be a finite-dimensional inner product space. Suppose \(\{\mathbf{v}_{n}\}_{n=1}^{N}\) is a linearly independent sequence in \(\mathcal{V}\).
For \(k\in[N]\) set
\[\mathbf{e}_{k} = \begin{cases} \mathbf{v}_{1} & k=1\\ \displaystyle{\mathbf{v}_{k} - \sum_{n=1}^{k-1}\frac{\langle \mathbf{e}_{n},\mathbf{v}_{k}\rangle}{\|\mathbf{e}_{n}\|^{2}} \mathbf{e}_{n}} & k=2,3,\ldots,N.\end{cases}\]
We shall see that \(\{\mathbf{e}_{n}\}_{n=1}^{N}\) is an independent sequence with the same span as \(\{\mathbf{v}_{n}\}_{n=1}^{N}\), but the vectors \(\{\mathbf{e}_{n}\}_{n=1}^{N}\) are all orthogonal to each other. Computing the vectors \(\{\mathbf{e}_{n}\}_{n=1}^{N}\) is called the \textit{Gram-Schmidt orthogonalization procedure}.\bigskip

\begin{enumerate}[(a)]

\item In the case that \(\mathcal{V} = \mathbb{R}^{3}\) with the standard inner product, apply the Gram-Schmidt orthogonalization procedure to the vectors 
\[\begin{bmatrix} 1\\1\\1\end{bmatrix},\ \begin{bmatrix} 1\\1\\0\end{bmatrix},\ \begin{bmatrix} 1\\0\\0\end{bmatrix}.\]
\bigskip

\item Show that \(\langle \mathbf{e}_{1},\mathbf{e}_{2}\rangle = 0\) and \(\operatorname{span}\{\mathbf{e}_{1},\mathbf{e}_{2}\} = \operatorname{span}\{\mathbf{v}_{1},\mathbf{v}_{2}\}\).\bigskip

\item Use induction on \(N\) to show that \(\langle \mathbf{e}_{n},\mathbf{e}_{m}\rangle = 0\) for \(n\neq m\) and \(\operatorname{span} \{\mathbf{e}_{n}\}_{n=1}^{N} = \operatorname{span}\{\mathbf{v}_{n}\}_{n=1}^{N}\).\bigskip

\end{enumerate}\bigskip



\item\label{gsp} Use the Gram-Schmidt orthogonalization procedure on the basis \(\{p_{0},p_{1},p_{2}\}\) for \(\mathbb{P}_{2}\) to find an orthogonal basis for \(\mathbb{P}_{2}\).\bigskip

\item Find a quadratic polynomial \(q\in\mathbb{P}_{2}\) such that \(\langle p_{3}-q,p\rangle = 0\) for all \(p\in\mathbb{P}_{2}\). Is \(q\) unique? (Hint: You might find it useful to use the orthogonal basis for \(\mathbb{P}_{2}\) that you found in Problem \ref{gsp}.) \bigskip

\item Let \(\mathbb{Z}_{N}\) be the set \(\{0,1,2,\ldots,N-1\}\) with addition modulo \(N\). Let \(\mathbf{T}:\mathbb{C}^{\mathbb{Z}_{N}}\to\mathbb{C}^{\mathbb{Z}_{N}}\) be given by \[(\mathbf{Tx})(n) = \mathbf{x}(n-1).\]
Find \(\mathbf{T}^{\ast}\) and show that \(\mathbf{TT}^{\ast}=\mathbf{T}^{\ast}\mathbf{T}\), that is, \(\mathbf{T}\) commutes with its adjoint. Operators that commute with their adjoint are called \textit{normal}.\bigskip




\item Go to \href{https://eigenquiz.app/}{Eigenquiz.app} and complete activity a279595. Upload a completion certificate for this activity to your homework folder.

\end{enumerate}

\end{document}


\end{proof}

