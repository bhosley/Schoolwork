\documentclass[12pt]{amsart}


%\usepackage[notref,notcite]{showkeys}
\usepackage{verbatim}
\usepackage{fullpage}
\usepackage{amsfonts}
\usepackage{bbm}
\usepackage{wrapfig}
\usepackage{enumerate}
\usepackage{float}
\usepackage{hyperref}
\usepackage[left=0.5in, right=0.5in, bottom=0.75in, top=0.75in]{geometry}
\setlength{\parindent}{0pt}


\usepackage{color}
\usepackage{colortbl}
\usepackage{graphicx}
\usepackage{epstopdf}
\usepackage{amsmath, amsthm, amssymb}
\usepackage{fancybox}

\newcommand{\1}{\mathbbm{1}}

\newcommand{\eps}{\varepsilon}
\newcommand{\la}{\lambda}
\newcommand{\Contradiction}{\Rightarrow\Leftarrow}
\DeclareMathOperator{\lspan}{span}
\DeclareMathOperator{\tr}{tr}
\DeclareMathOperator{\ran}{ran}
\DeclareMathOperator{\diag}{diag}
\providecommand{\abs}[1]{\lvert#1\rvert}
\providecommand{\norm}[1]{\lVert#1\rVert}
\renewcommand{\theenumi}{\alph{enumi}}
\renewcommand{\labelenumi}{(\theenumi)}

\newcounter{Theorem}
\newcounter{Definition}
\numberwithin{equation}{section}
\numberwithin{Theorem}{section}

\theoremstyle{plain} %% This is the default, anyway
\newtheorem{thm}[Theorem]{Theorem}
\newtheorem{cor}[Theorem]{Corollary}
\newtheorem{lem}[Theorem]{Lemma}
\newtheorem{prop}[Theorem]{Proposition}
%\usepackage{upgreek}

\theoremstyle{definition}
\newtheorem{defn}[Theorem]{Definition}

\theoremstyle{remark}
\newtheorem{remark}{Remark}[section]
\newtheorem{ex}[Theorem]{Example}
\newtheorem{nota}[Theorem]{Notation}
\usepackage{nicefrac}


\begin{document}

\thispagestyle{empty}

\noindent{\Large Homework 8 (Due December 2 at 8am)}\bigskip


\begin{enumerate}[1.]

\item Let \(\mathcal{V}\) be a finite-dimensional inner product space. Suppose \(\{\mathbf{v}_{n}\}_{n=1}^{N}\) is a linearly independent sequence in \(\mathcal{V}\).
For \(k\in[N]\) set
\[\mathbf{e}_{k} = \begin{cases} \mathbf{v}_{1} & k=1\\ \displaystyle{\mathbf{v}_{k} - \sum_{n=1}^{k-1}\frac{\langle \mathbf{e}_{n},\mathbf{v}_{k}\rangle}{\|\mathbf{e}_{n}\|^{2}} \mathbf{e}_{n}} & k=2,3,\ldots,N.\end{cases}\]
We shall see that \(\{\mathbf{e}_{n}\}_{n=1}^{N}\) is an independent sequence with the same span as \(\{\mathbf{v}_{n}\}_{n=1}^{N}\), but the vectors \(\{\mathbf{e}_{n}\}_{n=1}^{N}\) are all orthogonal to each other. Computing the vectors \(\{\mathbf{e}_{n}\}_{n=1}^{N}\) is called the \textit{Gram-Schmidt orthogonalization procedure}.\bigskip

\begin{enumerate}[(a)]

\item In the case that \(\mathcal{V} = \mathbb{R}^{3}\) with the standard inner product, apply the Gram-Schmidt orthogonalization procedure to the vectors 
\[\begin{bmatrix} 1\\1\\1\end{bmatrix},\ \begin{bmatrix} 1\\1\\0\end{bmatrix},\ \begin{bmatrix} 1\\0\\0\end{bmatrix}.\]
\bigskip

\item Show that \(\langle \mathbf{e}_{1},\mathbf{e}_{2}\rangle = 0\) and \(\operatorname{span}\{\mathbf{e}_{1},\mathbf{e}_{2}\} = \operatorname{span}\{\mathbf{v}_{1},\mathbf{v}_{2}\}\).\bigskip

\item Use induction on \(N\) to show that \(\langle \mathbf{e}_{n},\mathbf{e}_{m}\rangle = 0\) for \(n\neq m\) and \(\operatorname{span} \{\mathbf{e}_{n}\}_{n=1}^{N} = \operatorname{span}\{\mathbf{v}_{n}\}_{n=1}^{N}\).\bigskip

\end{enumerate}\bigskip

%%%%%%%%%%%%%%%%%%%%%%%%%%%%%%%
%	\begin{ Question 1 }	  %
%%%%%%%%%%%%%%%%%%%%%%%%%%%%%%%
\hrule
\bigskip

\begin{enumerate}[(a)]
	\item 
	Let the vectors above be known as \(\mathbf{v}_1, \mathbf{v}_2\) and \(\mathbf{v}_3\) respectively. 
	Then let the result of Graham-Schmidt orthogonalization applied to these particular vectors be \(\mathbf{e}_1,\mathbf{e}_2,\mathbf{e}_3\in\mathcal V\). Then
	\[\mathbf{e}_1=\mathbf{v}_1=\begin{bmatrix} 1\\1\\1	\end{bmatrix}\]
	and
	\[\mathbf{e}_2=
	\mathbf{v}_2
	- \frac{\langle\mathbf{e}_1,\mathbf{v}_2\rangle}{\|\mathbf{e}_1\|^2}\mathbf{e}_1
	= \begin{bmatrix} 1\\1\\0 \end{bmatrix} 
	- \frac{2}{3}\begin{bmatrix} 1\\1\\1 \end{bmatrix}
	= \begin{bmatrix} \nicefrac{1}{3}\\\nicefrac{1}{3}\\-\nicefrac{2}{3} \end{bmatrix} \]
	and
	\[\mathbf{e}_3=
	\mathbf{v}_3
	- \frac{\langle\mathbf{e}_1,\mathbf{v}_3\rangle}{\|\mathbf{e}_1\|^2}\mathbf{e}_1
	- \frac{\langle\mathbf{e}_2,\mathbf{v}_3\rangle}{\|\mathbf{e}_2\|^2}\mathbf{e}_2
	= \begin{bmatrix} 1\\0\\0 \end{bmatrix} 
	- \frac{1}{3}\begin{bmatrix} 1\\1\\1 \end{bmatrix}
	- \frac{\nicefrac{1}{3}}{\nicefrac{2}{3}} 
		\begin{bmatrix} \nicefrac{1}{3}\\\nicefrac{1}{3}\\-\nicefrac{2}{3} \end{bmatrix}
	= \begin{bmatrix} \nicefrac{1}{2}\\-\nicefrac{1}{2}\\0 \end{bmatrix}. \]
	
	\item 
	\begin{proof}
		Let \(\mathbf{e}_1,\mathbf{e}_2\) be the first two elements in \(\{\mathbf{e}_{n}\}_{n=1}^{N}\). Then	
	\begin{align*}
		\langle \mathbf{e}_1,\mathbf{e}_2 \rangle
		&= \langle \mathbf{v}_1, \mathbf{v}_2 - \frac{\langle\mathbf{e}_1,\mathbf{v}_2\rangle}{\|\mathbf{e}_1\|^2}\mathbf{e}_1 \rangle  \\
		&= \langle \mathbf{v}_1, \mathbf{v}_2 \rangle - \langle \mathbf{v}_1, \frac{\langle\mathbf{e}_1,\mathbf{v}_2\rangle}{\|\mathbf{e}_1\|^2}\mathbf{e}_1 \rangle  \\
		&= \langle \mathbf{v}_1, \mathbf{v}_2 \rangle - \frac{\langle\mathbf{e}_1,\mathbf{v}_2\rangle}{\|\mathbf{e}_1\|^2} \langle \mathbf{v}_1, \mathbf{e}_1 \rangle  \\
		&= \langle \mathbf{v}_1, \mathbf{v}_2 \rangle - \frac{\langle\mathbf{v}_1,\mathbf{v}_2\rangle}{\|\mathbf{e}_1\|^2} \langle \mathbf{e}_1, \mathbf{e}_1 \rangle  \\
		&= \langle \mathbf{v}_1, \mathbf{v}_2 \rangle - 
		\langle\mathbf{v}_1,\mathbf{v}_2\rangle   \\
		&= 0.  \\
	\end{align*}
	\end{proof}
	\begin{proof}
		Let \(c_1, c_2 \in\mathbb{F}\) be arbitrary. Then we may see for \(\mathbf{v}_1,\mathbf{v}_2 \in \{\mathbf{v}_{n}\}_{n=1}^{N}\) and the corresponding orthogonalized vectors \(\mathbf{e}_1,\mathbf{e}_2 \in\{\mathbf{e}_{n}\}_{n=1}^{N}\) that
		\begin{align*}
			c_1\mathbf{v}_1+c_2\mathbf{v}_2
			&= c_1\mathbf{e}_1+c_2\left( \mathbf{e}_2 + \frac{\langle\mathbf{e}_1,\mathbf{v}_2\rangle}{\|\mathbf{e}_1\|^2}\mathbf{e}_1 \right) \\
			&= c_1\mathbf{e}_1+ c_2\mathbf{e}_2 + c_2\left( \frac{\langle\mathbf{e}_1,\mathbf{v}_2\rangle}{\|\mathbf{e}_1\|^2}\mathbf{e}_1 \right) \\
			&= \left(c_1 + c_2 \frac{\langle\mathbf{e}_1,\mathbf{v}_2\rangle}{\|\mathbf{e}_1\|^2} \right)\mathbf{e}_1 + c_2\mathbf{e}_2.
		\end{align*}
		Thus \(\operatorname{span}\{\mathbf{v}_{1},\mathbf{v}_{2}\} \subseteq \operatorname{span}\{\mathbf{e}_{1},\mathbf{e}_{2}\}\). Similarly, 
		\begin{align*}
			c_1\mathbf{e}_1+c_2\mathbf{e}_2
			&= c_1\mathbf{v}_1+c_2\left( \mathbf{v}_2 - \frac{\langle\mathbf{e}_1,\mathbf{v}_2\rangle}{\|\mathbf{e}_1\|^2}\mathbf{e}_1 \right) \\
			&= c_1\mathbf{v}_1- c_2\left( \frac{\langle\mathbf{e}_1,\mathbf{v}_2\rangle}{\|\mathbf{e}_1\|^2}\mathbf{e}_1 \right) + c_2\mathbf{v}_2 \\
			&= c_1\mathbf{v}_1- c_2\left( \frac{\langle\mathbf{e}_1,\mathbf{v}_2\rangle}{\|\mathbf{e}_1\|^2}\mathbf{e}_1 \right) + c_2\mathbf{v}_2 \\
			&= c_1\mathbf{v}_1- c_2\left( \frac{\langle\mathbf{e}_1,\mathbf{v}_2\rangle}{\|\mathbf{e}_1\|^2}\mathbf{v}_1 \right) + c_2\mathbf{v}_2 \\
			&= \left(c_1- c_2 \frac{\langle\mathbf{e}_1,\mathbf{v}_2\rangle}{\|\mathbf{e}_1\|^2} \right)\mathbf{v}_1 + c_2\mathbf{v}_2.
		\end{align*}
		Thus \(\operatorname{span}\{\mathbf{e}_{1},\mathbf{e}_{2}\} \subseteq \operatorname{span}\{\mathbf{v}_{1},\mathbf{v}_{2}\}\). Because each span is a subset of the other we can conclude \(\operatorname{span}\{\mathbf{e}_{1},\mathbf{e}_{2}\} = \operatorname{span}\{\mathbf{v}_{1},\mathbf{v}_{2}\}\).
	\end{proof}

	\item 
	\begin{proof}
		Clearly, the case \(N=1\) is exclusive to the condition \(n\neq m\). 
		The base case of \(N=2\) is shown to be true in part b.
		Assume that \(\langle \mathbf{e}_{n},\mathbf{e}_{m}\rangle = 0\) 
		for \(n\neq m\) and \(n,m\leq N\).
		Then observe that for \(N+1\)
		\begin{align*}
			\langle \mathbf{e}_{m},\mathbf{e}_{N+1}\rangle
			&= \langle \mathbf{e}_{m}, \mathbf{v}_{N+1}-\sum_{n=1}^{N}\frac{\langle\mathbf{e}_n,\mathbf{v}_{N+1}\rangle}{\|\mathbf{e}_n\|^2}\mathbf{e}_n\rangle \\
			&= \langle \mathbf{e}_{m}, \mathbf{v}_{N+1}\rangle- \langle \mathbf{e}_{m},\sum_{n=1}^{N}\frac{\langle\mathbf{e}_n,\mathbf{v}_{N+1}\rangle}{\|\mathbf{e}_n\|^2}\mathbf{e}_n\rangle \\
			&= \langle \mathbf{e}_{m}, \mathbf{v}_{N+1}\rangle- \sum_{n=1}^{N}\frac{\langle\mathbf{e}_n,\mathbf{v}_{N+1}\rangle}{\|\mathbf{e}_n\|^2}\langle \mathbf{e}_{m},\mathbf{e}_n\rangle
		\intertext{then, because \(\langle \mathbf{e}_{n},\mathbf{e}_{m}\rangle = 0\) 
				for all \(n\neq m\) only \(\mathbf e_m\) remains}
			&= \langle \mathbf{e}_{m}, \mathbf{v}_{N+1}\rangle- \frac{\langle\mathbf{e}_m,\mathbf{v}_{N+1}\rangle}{\|\mathbf{e}_m\|^2}\langle \mathbf{e}_{m},\mathbf{e}_m\rangle \\
			&= \langle \mathbf{e}_{m}, \mathbf{v}_{N+1}\rangle- \langle\mathbf{e}_m,\mathbf{v}_{N+1}\rangle \\
			&= 0.
		\end{align*}
		Thus \(\langle \mathbf{e}_{n},\mathbf{e}_{m}\rangle = 0\) 
		where \(n\neq m\) holds for all \(N\geq2\).
	
		Similarly, that \(\operatorname{span} \{\mathbf{e}_{n}\}_{n=1}^{N} = \operatorname{span}\{\mathbf{v}_{n}\}_{n=1}^{N}\) holds for \(N=2\)
		was shown in part b. Assume that it holds for \(n,m\leq N\), 
		and let \(\{c_{n}\}_{n=1}^{N+1}, \{b_{n}\}_{n=1}^{N+1} \in \mathbb F\), 
		then we will see that for any 
		\(\mathbf w\in\operatorname{span} \{\mathbf{v}_{n}\}_{n=1}^{N+1}\)
		\begin{align*} \mathbf w
			&= \sum_{n=1}^{N+1}c_n\mathbf v_n \\
			&= \sum_{n=1}^{N}c_n\mathbf v_n + c_{N+1}\mathbf v_{N+1} \\
			&= \sum_{n=1}^{N}b_n\mathbf e_n + c_{N+1}\mathbf v_{N+1} \\
			%
			&= \sum_{n=1}^{N}b_n\mathbf e_n + c_{N+1} \left( \mathbf{e}_{N+1}+\sum_{m=1}^{N}\frac{\langle\mathbf{e}_m,\mathbf{v}_{N+1}\rangle}{\|\mathbf{e}_m\|^2}\mathbf{e}_m \right) \\
			&= \sum_{n=1}^{N}b_n\mathbf e_n + c_{N+1} \mathbf{e}_{N+1}+c_{N+1}\left( \sum_{m=1}^{N}\frac{\langle\mathbf{e}_m,\mathbf{v}_{N+1}\rangle}{\|\mathbf{e}_m\|^2}\mathbf{e}_m \right) \\
		\end{align*}
		Since \(\mathbf w\) was arbitrary, any element in \(\operatorname{span} \{\mathbf{v}_{n}\}_{n=1}^{N+1}\) can be expressed as a 
		linear combination of elements in 
		\(\{\mathbf{e}_{n}\}_{n=1}^{N+1}\).
		Then for arbitrary \(\mathbf u\in\operatorname{span} \{\mathbf{e}_{n}\}_{n=1}^{N+1}\)
		\begin{align*} \mathbf u
			&= \sum_{n=1}^{N+1}b_n\mathbf e_n \\
			&= \sum_{n=1}^{N}b_n\mathbf e_n + b_{N+1}\mathbf e_{N+1} \\
			&= \sum_{n=1}^{N}c_n\mathbf v_n + b_{N+1}\mathbf e_{N+1} \\
			&= \sum_{n=1}^{N}c_n\mathbf v_n + b_{N+1} \left( \mathbf{v}_{N+1}-\sum_{m=1}^{N}\frac{\langle\mathbf{e}_m,\mathbf{v}_{N+1}\rangle}{\|\mathbf{e}_m\|^2}\mathbf{e}_m \right) \\
			&= \sum_{n=1}^{N}c_n\mathbf v_n + b_{N+1} \mathbf{v}_{N+1}+b_{N+1}\left( \sum_{m=1}^{N}\frac{\langle\mathbf{e}_m,\mathbf{v}_{N+1}\rangle}{\|\mathbf{e}_m\|^2}\mathbf{e}_m \right) \\
		\end{align*}
		Since \(\mathbf u\) was arbitrary, any element in \(\operatorname{span} \{\mathbf{e}_{n}\}_{n=1}^{N+1}\) can be expressed as a 
		linear combination of elements in 
		\(\{\mathbf{v}_{n}\}_{n=1}^{N+1}\);
		Thus, \( \operatorname{span} \{\mathbf{v}_{n}\}_{n=1}^{N+1} =
		\operatorname{span} \{\mathbf{e}_{n}\}_{n=1}^{N+1}\)
	\end{proof}

	


\end{enumerate}


% \end{ Question 1 }

\item\label{gsp} Use the Gram-Schmidt orthogonalization procedure on the basis \(\{p_{0},p_{1},p_{2}\}\) for \(\mathbb{P}_{2}\) to find an orthogonal basis for \(\mathbb{P}_{2}\).\bigskip

%%%%%%%%%%%%%%%%%%%%%%%%%%%%%%%
%	\begin{ Question 2 }	  %
%%%%%%%%%%%%%%%%%%%%%%%%%%%%%%%
\hrule
\bigskip

Let \(\varrho_n\) represent the orthogonalized basis element corresponding to \(p_n\).
The first element remains the same, \(\varrho_0=p_0\). 
Applied to the second element we see
\begin{align*}
	\varrho_1
	&=p_1-\frac{\langle\varrho_0,p_1\rangle}{\|\varrho_0\|^2}\varrho_0 \\
	&=p_1-\frac{\int_{-1}^{1}1\cdot x\,dx}{\int_{-1}^{1}1\cdot1\,dx}\cdot1 \\
	&=p_1-\frac{\left.\nicefrac{x^2}{2}\right|_{-1}^1}{2} \\
	&=p_1-\frac{0}{2} \\
	&=p_1 \\
\intertext{and to the third element}
	\varrho_2(x)
	&=p_2-\frac{\langle\varrho_0,p_2\rangle}{\|\varrho_0\|^2}\varrho_0 -\frac{\langle\varrho_1(x),p_2\rangle}{\|\varrho_1(x)\|^2}\varrho_1 \\
	&=p_2-\frac{\int_{-1}^{1}x^2\,dx}{\int_{-1}^{1}1\,dx}\varrho_0 -\frac{\int_{-1}^{1}x^3}{\int_{-1}^{1}x^2}\varrho_1 \\
	&=p_2-\frac{\nicefrac{2}{3}}{2}\varrho_0 -\frac{0}{2}\varrho_1 \\
	&=p_2-\frac{1}{3}p_0. \\
\end{align*}

Thus the orthoganalized basis corresponding to \(\{p_0,p_1,p_2\}\) is \(\{p_0,p_1,p_2-\frac{p_0}{3}\}\).

% \end{ Question 2 }
\clearpage

\item Find a quadratic polynomial \(q\in\mathbb{P}_{2}\) such that \(\langle p_{3}-q,p\rangle = 0\) for all \(p\in\mathbb{P}_{2}\). Is \(q\) unique? (Hint: You might find it useful to use the orthogonal basis for \(\mathbb{P}_{2}\) that you found in Problem \ref{gsp}.) \bigskip

%%%%%%%%%%%%%%%%%%%%%%%%%%%%%%%
%	\begin{ Question 3 }	  %
%%%%%%%%%%%%%%%%%%%%%%%%%%%%%%%
\hrule
\bigskip

First extending the orthogonalized basis from problem 2, we see
\begin{align*}
	\varrho_3
	&=p_3-\frac{\langle\varrho_0,p_3\rangle}{\|\varrho_0\|^2}\varrho_0 -\frac{\langle\varrho_1,p_3\rangle}{\|\varrho_1\|^2}\varrho_1
	-\frac{\langle\varrho_2,p_3\rangle}{\|\varrho_2\|^2}\varrho_2 \\
	&=p_3-\frac{\int_{-1}^{1}x^3\,dx}{\|\varrho_0\|^2}\varrho_0 -\frac{\int_{-1}^{1}x^4\,dx}{\|\varrho_1\|^2}\varrho_1
	-\frac{\int_{-1}^{1}x^5-\frac{1}{3}x^3\,dx}{\|\varrho_2\|^2}\varrho_2 \\
	&=p_3-\frac{0}{\|\varrho_0\|^2}\varrho_0 -\frac{\frac{2}{5}}{\|\varrho_1\|^2}\varrho_1
	-\frac{0}{\|\varrho_2(x)\|^2}\varrho_2 \\
	&=p_3-\frac{2}{5\int_{-1}^{1}x^2\,dx}p_1 \\
	&=p_3-\frac{2\cdot3}{5\cdot2}p_1 \\
	&=p_3-\frac{3}{5}p_1. \\
\end{align*}
Since \(p_0=\varrho_0\) and \(p_1=\varrho_1\) then 
\(q=\frac{3}{5}p_1\) for \(p_0\) and \(p_1\). 
Since \(p_2\neq\varrho_2\) the same value for \(q\) may not hold.
On inspection we see that
\[
\langle p_3-\frac{3}{5}p_1, p_2 \rangle
= \int_{-1}^{1} (x^3-\frac{3}{5}x)x^2\,dx
= \int_{-1}^{1} x^5-\frac{3}{5}x^3\,dx
= \left. \frac{1}{5}x^6-\frac{1}{5}x^4\right|_{-1}^{1}
= 0.
\]
Thus \(q=\frac{3}{5}p_1\) holds for all \(p\in\mathbb P\). 
To evaluate the uniqueness of \(q=\frac{3}{5}p_1\) let \(c\in\mathbb{F}\) then we may evaluate \(\langle p_{3}-cq,p\rangle = 0\), for \(p_1\)
\begin{align*}
	0= \langle p_3-cp_1, p_1 \rangle
	= \int_{-1}^{1} x^4-cx^2 \,dx
	= \left. \frac{1}{5}x^5-\frac{c}{3}x^3 \right|_{-1}^{1}
	= (\frac{1}{5}-\frac{c}{3})-(-\frac{1}{5}+\frac{c}{3})
	= \frac{2}{5}-\frac{2c}{3}
	= \frac{3}{5}-c.
\end{align*}
Thus \(q\) is uniquely \(\frac{3}{5}p_1\). \\
% \end{ Question 3 }

\clearpage

\item Let \(\mathbb{Z}_{N}\) be the set \(\{0,1,2,\ldots,N-1\}\) with addition modulo \(N\). Let \(\mathbf{T}:\mathbb{C}^{\mathbb{Z}_{N}}\to\mathbb{C}^{\mathbb{Z}_{N}}\) be given by \[(\mathbf{Tx})(n) = \mathbf{x}(n-1).\]
Find \(\mathbf{T}^{\ast}\) and show that \(\mathbf{TT}^{\ast}=\mathbf{T}^{\ast}\mathbf{T}\), that is, \(\mathbf{T}\) commutes with its adjoint. Operators that commute with their adjoint are called \textit{normal}.\bigskip

%%%%%%%%%%%%%%%%%%%%%%%%%%%%%%%
%	\begin{ Question 4 }	  %
%%%%%%%%%%%%%%%%%%%%%%%%%%%%%%%
\hrule
\bigskip
% Example 10.4 in notes


Let \(\mathbf{x},\mathbf{y}\in\mathbb{C}^{\mathbb{Z}_{N}}\)
be arbitrary and and let \(n'=n-1\).
Then,
\begin{align*}
	\langle \mathbf{y},\mathbf{Tx} \rangle
	&= \sum_{n\in\mathbb{Z}_N} \overline{\mathbf{y}(n)}(\mathbf{Tx})(n) \\
	&= \sum_{n\in\mathbb{Z}_N} \overline{\mathbf{y}(n)}\mathbf{x}(n-1)  \\
	&= \sum_{n\in\mathbb{Z}_N} \overline{\mathbf{y}(n'+1)}\mathbf{x}(n') \\
	&= \sum_{n\in\mathbb{Z}_N} \overline{(\mathbf{T}^{-1}\mathbf{y})(n')}\mathbf{x}(n') \\
	&= \langle \mathbf{T}^{-1}\mathbf{y},\mathbf{x} \rangle
\end{align*}

Thus we see that \(\mathbf{T}^{-1}=\mathbf{T^*}\).
Additionally, from this relationship we can see
\begin{align*}
	\mathbf{TT^*}
	= \mathbf{TT}^{-1}
	= \mathbf{I}
	= \mathbf{T}^{-1}\mathbf{T}
	= \mathbf{T^*T}.
\end{align*}

Showing that \(\mathbf T\) is a \textit{normal} operator.

% \end{ Question 4 }

\vspace{\fill}

\item Go to \href{https://eigenquiz.app/}{Eigenquiz.app} and complete activity a279595. Upload a completion certificate for this activity to your homework folder.

\end{enumerate}

\end{document}


\end{proof}

