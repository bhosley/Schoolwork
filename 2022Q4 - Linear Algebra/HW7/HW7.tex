\documentclass[12pt]{amsart}


%\usepackage[notref,notcite]{showkeys}
\usepackage{verbatim}
\usepackage{fullpage}
\usepackage{amsfonts}
\usepackage{bbm}
\usepackage{wrapfig}
\usepackage{enumerate}
\usepackage{float}
\usepackage{hyperref}
\usepackage[left=0.5in, right=0.5in, bottom=0.75in, top=0.75in]{geometry}
\setlength{\parindent}{0pt}


\usepackage{color}
\usepackage{colortbl}
\usepackage{graphicx}
\usepackage{epstopdf}
\usepackage{amsmath, amsthm, amssymb}
\usepackage{fancybox}

\newcommand{\1}{\mathbbm{1}}

\newcommand{\eps}{\varepsilon}
\newcommand{\la}{\lambda}
\newcommand{\Contradiction}{\Rightarrow\Leftarrow}
\DeclareMathOperator{\lspan}{span}
\DeclareMathOperator{\tr}{tr}
\DeclareMathOperator{\ran}{ran}
\DeclareMathOperator{\diag}{diag}
\providecommand{\abs}[1]{\lvert#1\rvert}
\providecommand{\norm}[1]{\lVert#1\rVert}
\renewcommand{\theenumi}{\alph{enumi}}
\renewcommand{\labelenumi}{(\theenumi)}

\newcounter{Theorem}
\newcounter{Definition}
\numberwithin{equation}{section}
\numberwithin{Theorem}{section}

\theoremstyle{plain} %% This is the default, anyway
\newtheorem{thm}[Theorem]{Theorem}
\newtheorem{cor}[Theorem]{Corollary}
\newtheorem{lem}[Theorem]{Lemma}
\newtheorem{prop}[Theorem]{Proposition}
%\usepackage{upgreek}

\theoremstyle{definition}
\newtheorem{defn}[Theorem]{Definition}

\theoremstyle{remark}
\newtheorem{remark}{Remark}[section]
\newtheorem{ex}[Theorem]{Example}
\newtheorem{nota}[Theorem]{Notation}



\begin{document}

\thispagestyle{empty}

\noindent{\Large Homework 7 (Due November 23 at 8am)}\bigskip


\begin{enumerate}[1.]

\item For \(N\in\mathbb{N}\cup\{0\}\) let \(\mathbb{P}_{N}\) be the real vector space defined in Homework 6, Problem 1. For \(f,g\in\mathbb{P}_{N}\) set
\[\langle f,g\rangle = \int_{-1}^{1}f(x)g(x)\,dx.\]

\begin{enumerate}[(a)]
\item Show that \(\langle\cdot,\cdot\rangle:\mathbb{P}_{N}\times\mathbb{P}_{N}\to\mathbb{R}\) is an inner product.\medskip

\item In the case \(N=3\), find a basis for 
\[\operatorname{span}\{p_{0}\}^{\bot}:=\{f\in\mathbb{P}_{N} : \langle p_{0},f\rangle = 0\}.\]

\item Consider the linear map \(\mathbf{E}:\mathbb{P}_{3}\to\mathbb{R}\) given by \(\mathbf{E}f = f(0)\). Find a vector \(g\in\mathbb{P}_{2}\) such that \(\mathbf{E}f = \langle g,f\rangle\) for every \(f\in\mathbb{P}_{3}\).

\end{enumerate}\bigskip


\item Let \(\mathcal{V}\) be a finite-dimensional inner product space, and let \(\mathcal{W}\subset \mathcal{V}\) be a subspace. Let \(\{\mathbf{w}_{n}\}_{n=1}^{M}\) be a basis for \(\mathcal{W}\), and define the map \(\mathbf{B}:\mathcal{V}\to\mathcal{V}\) by
\[\mathbf{B}\mathbf{v} = \sum_{n=1}^{M}\langle \mathbf{w}_{n},\mathbf{v}\rangle \mathbf{w}_{n}.\]\medskip

\begin{enumerate}[(a)]

\item Show that \(\mathbf{B}\) is linear.\medskip

\item Show that \(\operatorname{ker}(\mathbf{B}) = \mathcal{W}^{\bot}\).\medskip

\item Show that \(\mathcal{W}\cap\mathcal{W}^{\bot}=\{\mathbf{0}\}\).\medskip

\item Show that \(\{\mathbf{B}(\mathbf{w}_{n})\}_{n=1}^{M}\) is linearly independent. Use this to determine \(\operatorname{rank}\mathbf{B}\).\medskip

\item Use the Rank-Nullity Theorem to show that
\[\dim\mathcal{W} + \dim\mathcal{W}^{\bot} = \dim\mathcal{V}.\]

\end{enumerate}\bigskip

\item No eigenquiz this week.

\end{enumerate}

\end{document}

