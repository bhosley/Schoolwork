\documentclass[12pt]{amsart}


%\usepackage[notref,notcite]{showkeys}
\usepackage{verbatim}
\usepackage{fullpage}
\usepackage{amsfonts}
\usepackage{bbm}
\usepackage{wrapfig}
\usepackage{enumerate}
\usepackage{float}
\usepackage{hyperref}
\usepackage[left=0.5in, right=0.5in, bottom=0.75in, top=0.75in]{geometry}
\setlength{\parindent}{0pt}


\usepackage{color}
\usepackage{colortbl}
\usepackage{graphicx}
\usepackage{epstopdf}
\usepackage{amsmath, amsthm, amssymb}
\usepackage{fancybox}
\usepackage{comment}

\newcommand{\1}{\mathbbm{1}}

\newcommand{\eps}{\varepsilon}
\newcommand{\la}{\lambda}
\newcommand{\Contradiction}{\Rightarrow\Leftarrow}
\DeclareMathOperator{\lspan}{span}
\DeclareMathOperator{\tr}{tr}
\DeclareMathOperator{\ran}{ran}
\DeclareMathOperator{\diag}{diag}
\providecommand{\abs}[1]{\lvert#1\rvert}
\providecommand{\norm}[1]{\lVert#1\rVert}
\renewcommand{\theenumi}{\alph{enumi}}
\renewcommand{\labelenumi}{(\theenumi)}

\newcounter{Theorem}
\newcounter{Definition}
\numberwithin{equation}{section}
\numberwithin{Theorem}{section}

\theoremstyle{plain} %% This is the default, anyway
\newtheorem{thm}[Theorem]{Theorem}
\newtheorem{cor}[Theorem]{Corollary}
\newtheorem{lem}[Theorem]{Lemma}
\newtheorem{prop}[Theorem]{Proposition}
%\usepackage{upgreek}

\theoremstyle{definition}
\newtheorem{defn}[Theorem]{Definition}

\theoremstyle{remark}
\newtheorem{remark}{Remark}[section]
\newtheorem{ex}[Theorem]{Example}
\newtheorem{nota}[Theorem]{Notation}



\begin{document}

\thispagestyle{empty}

\noindent{\Large Homework 7 (Due November 23 at 8am)}
\hspace{\fill} {\Large B. Hosley}
\bigskip


\begin{enumerate}[1.]

\item For \(N\in\mathbb{N}\cup\{0\}\) let \(\mathbb{P}_{N}\) be the real vector space defined in Homework 6, Problem 1. For \(f,g\in\mathbb{P}_{N}\) set
\[\langle f,g\rangle = \int_{-1}^{1}f(x)g(x)\,dx.\]

\begin{enumerate}[(a)]
\item Show that \(\langle\cdot,\cdot\rangle:\mathbb{P}_{N}\times\mathbb{P}_{N}\to\mathbb{R}\) is an inner product.\medskip

\item In the case \(N=3\), find a basis for 
\[\operatorname{span}\{p_{0}\}^{\bot}:=\{f\in\mathbb{P}_{N} : \langle p_{0},f\rangle = 0\}.\]

\item Consider the linear map \(\mathbf{E}:\mathbb{P}_{3}\to\mathbb{R}\) given by \(\mathbf{E}f = f(0)\). Find a vector \(g\in\mathbb{P}_{2}\) such that \(\mathbf{E}f = \langle g,f\rangle\) for every \(f\in\mathbb{P}_{3}\).

\end{enumerate}\bigskip

%%%%%%%%%%%%%%%%%%%%%%%%%%%%%%%
%	\begin{ Question 1 }	  %
%%%%%%%%%%%%%%%%%%%%%%%%%%%%%%%
\hrule
\bigskip
\begin{enumerate}[(a)]
\item
\begin{proof}
Let \(f,f_1,f_2,f_3\in\mathbb{P}_N\) and \(x,c\in\mathbb{R}\) be arbitrary.
Then, per the definition of inner product space:
\begin{enumerate}[i.]
	\item Additive in the second part,
	\begin{align*}
		\langle f_1(x), f_2(x)+f_3(x) \rangle
		&= \int_{-1}^{1}f_1(x) (f_2(x)+f_3(x))\,dx \\
		&= \int_{-1}^{1}f_1(x)f_2(x)+ f_1(x)f_3(x)\,dx \\
		&= \int_{-1}^{1}f_1(x)f_2(x)+ \int_{-1}^{1}f_1(x)f_3(x)\,dx \\
		&= \langle f_1(x),f_2(x)\rangle + \langle f_1(x),f_3(x)\rangle .
	\end{align*}
	
	\item Multiplicative in the second part,
	\begin{align*}
		\langle f_1(x),cf_2(x)\rangle
		= \int_{-1}^{1}f_1(x) cf_2(x)\,dx 
		= c\int_{-1}^{1}f_1(x)f_2(x)\,dx 
		= c\langle f_1(x),f_2(x)\rangle .
	\end{align*}
	
	\item Conjugate symmetry,
	\begin{align*}
		\langle f_1(x),f_2(x)\rangle
		= \int_{-1}^{1}f_1(x)f_2(x)\,dx 
		&= \int_{-1}^{1}f_2(x)f_1(x)\,dx 
		= \langle f_2(x),f_1(x)\rangle
		\intertext{and since the domain is $\mathbb{R}$, }
		\langle f_2(x),f_1(x)\rangle
		&= \overline{\langle f_2(x),f_1(x)\rangle} .
	\end{align*}
	
	\item Positive definiteness,
	\begin{align*}
		\langle f(x),f(x)\rangle
		= \int_{-1}^{1}(f(x))^2\,dx 
		= \int_{-1}^{1}\bigg(\sum_{n=0}^{N}a_np_n(x)\bigg)^2\,dx 
	\end{align*}
	Because \((f(x))^2\) is positive on the 
	interval \([-1,1]\) when \(f(x)\neq0\) 
	the integral is also positive on that interval. 
	\vspace{-1.4em}
\end{enumerate} 	
\end{proof}

\item
A basis for \(\operatorname{span}\{p_{0}\}^\perp\)
is \(\{q_{n}\}_{n=1}^{3}\) in which
\begin{align*}
	q_1(x) = x, \quad
	q_2(x) = x^3, \quad
	\text{and} \quad
	q_3(x) = 3x^2-1 \quad
	\text{for all } x\in\mathbb{R}.
\end{align*}

\item
If there was a typo, then a vector \(g\in\mathbb{P}_{3}\)such that \(\mathbf{E}f = \langle g,f\rangle = f(0)\) is
\[g_0(x) = \frac{9}{8}-\frac{15}{8}x^2.\]
If there was no typo, then a vector \(g\in\mathbb{P}_{2}\) such that \(\mathbf{E}f = \langle g,f\rangle = f(0)\) is
\[g_0(x) = \frac{9}{8}-\frac{15}{8}x^2.\]

Even though it wasn't asked, this may be found by
\begin{align*}
	\left\langle g, p_0\right\rangle &= 1 
	=\int_{-1}^1\left(b_0 p_0+b_1 p_1+b_2 p_2+b_3 p_3\right)(t) dt 
	&= b_0 t+b_1 \frac{1}{2} t^2+\left.b_2 \frac{1}{3} t^3\right|_{-1}^1
	&=2 b_0 + \frac{2}{3}b_2\\
	%
	\left\langle g, p_1\right\rangle&= 0 
	=\int_{-1}^1 p_1(t)\left(b_0 p_0+b_1 p_1+b_2 p_2+b_3 p_3\right)(t) dt 
	&= b_0 t^2+b_1 \frac{1}{3} t^3+b_2 \frac{1}{4} t^4+\left.b_3 \frac{1}{5} t^5\right|_{-1} ^1
	&=\frac{2}{3} b_1+\frac{2}{5} b_3 \\
	%
	\left\langle g, p_2\right\rangle&=0
	=\int_{-1}^1 p_2(t)\left(b_0 p_0+b_1 p_1+b_2 p_2+b_3 p_3\right)(t) dt 
	&=b_0 \frac{1}{3} t^3+b_1 \frac{1}{4} t^4+b_2 \frac{1}{5} t^5+\left.b_3 \frac{1}{6} t^6\right|_{-1} ^1
	&=\frac{2}{3} b_0+\frac{2}{5} b_2 \\
	%
	\left\langle g, p_3\right\rangle&=0
	=\int_{-1}^1 p_3(t)\left(b_0 p_0+b_1 p_1+b_2 p_2+b_3 p_3\right)(t) dt 
	&=b_0 \frac{1}{4} t^4+b_1 \frac{1}{5} t^5+b_2 \frac{1}{6} t^6+\left.b_3 \frac{1}{7} t^7\right|_{-1} ^1
	&=\frac{2}{5} b_1+\frac{2}{7} b_3
\end{align*}
clearly,
\[b_1 = 0 \quad \text{and} \quad b_3 = 0\]
and slightly less clearly, by solving the first and third equations,
\[
b_0 + \frac{2}{3}b_2 = 1 \quad \text{and} \quad 
\frac{2}{3} b_0+\frac{2}{5} b_2 = 0
\]
elucidates
\[b_0 = \frac{9}{8} \quad \text{and} \quad b_2 = -\frac{15}{8}.\]

\end{enumerate}
%%%%%%%%%%%%%%%%%%%%%%%%%%%%%%%
%	 \end{ Question 1 }	 	  %
%%%%%%%%%%%%%%%%%%%%%%%%%%%%%%%

\clearpage

\item Let \(\mathcal{V}\) be a finite-dimensional inner product space, and let \(\mathcal{W}\subset \mathcal{V}\) be a subspace. Let \(\{\mathbf{w}_{n}\}_{n=1}^{M}\) be a basis for \(\mathcal{W}\), and define the map \(\mathbf{B}:\mathcal{V}\to\mathcal{V}\) by
\[\mathbf{B}\mathbf{v} = \sum_{n=1}^{M}\langle \mathbf{w}_{n},\mathbf{v}\rangle \mathbf{w}_{n}.\]\medskip

\begin{enumerate}[(a)]

\item Show that \(\mathbf{B}\) is linear.\medskip

\item Show that \(\operatorname{ker}(\mathbf{B}) = \mathcal{W}^{\bot}\).\medskip

\item Show that \(\mathcal{W}\cap\mathcal{W}^{\bot}=\{\mathbf{0}\}\).\medskip

\item Show that \(\{\mathbf{B}(\mathbf{w}_{n})\}_{n=1}^{M}\) is linearly independent. Use this to determine \(\operatorname{rank}\mathbf{B}\).\medskip

\item Use the Rank-Nullity Theorem to show that
\[\dim\mathcal{W} + \dim\mathcal{W}^{\bot} = \dim\mathcal{V}.\]

\end{enumerate}\bigskip

%%%%%%%%%%%%%%%%%%%%%%%%%%%%%%%
%	\begin{ Question 2 }	  %
%%%%%%%%%%%%%%%%%%%%%%%%%%%%%%%
\hrule
\bigskip
\begin{enumerate}[(a)]
	\itemsep1em 
\item 
\begin{proof}
Let \(c_1,c_2\in\mathbb{R}\) and \(\mathbf{v}_1,\mathbf{v}_2\in\mathcal{V}\) be arbitrary. Then, linearity can be shown by
\begin{align*}
	\mathbf{B}(c_1\mathbf{v}_1+c_2\mathbf{v}_2)
	&= \sum_{n=1}^{M} \langle \mathbf{w}_n, c_1\mathbf{v}_1+c_2\mathbf{v}_2 \rangle\mathbf{w}_n \\
	&= \sum_{n=1}^{M} \langle \mathbf{w}_n, c_1\mathbf{v}_1 \rangle\mathbf{w}_n + \langle \mathbf{w}_n,c_2\mathbf{v}_2 \rangle\mathbf{w}_n \\
	&= \sum_{n=1}^{M} c_1\langle \mathbf{w}_n,\mathbf{v}_1 \rangle\mathbf{w}_n + c_2\langle \mathbf{w}_n,\mathbf{v}_2 \rangle\mathbf{w}_n \\
	&=  c_1\sum_{n=1}^{M}\langle \mathbf{w}_n,\mathbf{v}_1 \rangle\mathbf{w}_n + c_2\sum_{n=1}^{M}\langle \mathbf{w}_n,\mathbf{v}_2 \rangle\mathbf{w}_n \\
	&= c_1\mathbf{B}\mathbf{v}_1 + c_2\mathbf{B}\mathbf{v}_2 .
\end{align*}
\end{proof}

\item 
\begin{proof}
\begin{comment}
Let \(\{\mathbf{u}_n\}_{n=1}^N\) be a basis for for \(\mathcal{W}^\perp\). Then, for arbitrary \(\mathbf{u}\in\mathcal{W}^\perp\)
	\mathbf{B}\mathbf{u}
	= \sum_{n=1}^{M} \langle \mathbf{w}_n,\mathbf{u} \rangle\mathbf{w}_n
	= \sum_{n=1}^{M} 0 \, \mathbf{w}_n
	= 0 .
\end{comment}
Let \(\mathbf{u}\in\ker(\mathbf{B})\), then 
\begin{align*}
	0
	= \mathbf{B}\mathbf{u}
	= \sum_{n=1}^{M} \langle \mathbf{w}_n,\mathbf{u} \rangle\mathbf{w}_n
	= \sum_{n=1}^{M} \langle \mathbf{w}_n,\mathbf{u} \rangle
\end{align*}
Thus every non-trivial element of \(\ker(\mathbf{B})\) is \(\mathbf{u}\)
such that \(\langle \mathbf{w}_n,\mathbf{u} \rangle = 0\)
which is by definition \(\mathcal{W}^\perp\).
\end{proof}


\item
\begin{proof}
Let \(\mathbf{v}\in\mathcal{W}\cap\mathcal{W}^\perp\)
then note that 
\begin{align*}
	\langle \mathbf{v},\mathbf{v} \rangle = 0.
\end{align*}
By the contraposition of Definition 9.3(iv),
\(\langle \mathbf{v},\mathbf{v} \rangle \ngtr 0\)
implies that \(\mathbf{v} = \mathbf{0}\).
\end{proof}

\clearpage

\item 
\begin{proof}
Let \(\{c_n\}_{n=1}^{M}\in\mathbb{F}\) be a set of scalars such that
\begin{align*}
0
= \sum_{n=1}^{M} c_n\mathbf{B}(\mathbf{w}_{n})
= \sum_{n=1}^{M}\sum_{m=1}^{M} c_n\langle \mathbf{w}_{m},\mathbf{w}_{n}\rangle \mathbf{w}_{m}
= \sum_{m=1}^{M} \langle \mathbf{w}_{m},\sum_{n=1}^{M}c_n\mathbf{w}_{n}\rangle \mathbf{w}_{m}.
\end{align*}
Because \(\mathbf{w}\) is independent, 
\(c_n=0\) for each \(c\in\{c_n\}_{n=1}^{M}\),
and therefore \(\{\mathbf{B}(\mathbf{w}_{n})\}_{n=1}^{M}\)
is also independent.
Additionally, since any \(\langle \mathbf{w}_{m},\mathbf{w}_{n}\rangle\)
is a scalar, \(\mathbf{B}\mathbf{v}\)
is necessarily a scalar multiple of \(\mathbf{w}\)
thus rank\(\,\mathbf{B}=M\).

\end{proof}


\item 
\begin{proof}
Using the rank nullity theorem applied to \(\mathbf{B}\) we see that
\[ \text{Rank } \mathbf{B} + \text{Nullity } \mathbf{B} = \dim (\text{Domain } \mathbf{B}). \]
Part (c) showed that \(\operatorname{ker}(\mathbf{B}) = \mathcal{W}^{\bot}\)
and (d) showed \(\operatorname{rank}\mathbf{B}=M\), 
which is \(\dim\mathcal{W}\), thus

\[\dim\mathcal{W} + \dim\mathcal{W}^{\bot} = \dim\mathcal{V}.\]
\end{proof}

	
\end{enumerate}

%%%%%%%%%%%%%%%%%%%%%%%%%%%%%%%
%	 \end{ Question 2 }	 	  %
%%%%%%%%%%%%%%%%%%%%%%%%%%%%%%%

\vspace{\fill}
\item No eigenquiz this week.

\end{enumerate}

\end{document}

