\documentclass[12pt]{amsart}


%\usepackage[notref,notcite]{showkeys}
\usepackage{verbatim}
\usepackage{fullpage}
\usepackage{amsfonts}
\usepackage{bbm}
\usepackage{wrapfig}
\usepackage{enumerate}
\usepackage{float}
\usepackage{hyperref}
\usepackage[left=0.5in, right=0.5in, bottom=0.75in, top=0.75in]{geometry}
\setlength{\parindent}{0pt}


\usepackage{color}
\usepackage{colortbl}
\usepackage{graphicx}
\usepackage{epstopdf}
\usepackage{amsmath, amsthm, amssymb}
\usepackage{fancybox}

\newcommand{\1}{\mathbbm{1}}

\newcommand{\eps}{\varepsilon}
\newcommand{\la}{\lambda}
\newcommand{\Contradiction}{\Rightarrow\Leftarrow}
\DeclareMathOperator{\lspan}{span}
\DeclareMathOperator{\tr}{tr}
\DeclareMathOperator{\ran}{ran}
\DeclareMathOperator{\diag}{diag}
\providecommand{\abs}[1]{\lvert#1\rvert}
\providecommand{\norm}[1]{\lVert#1\rVert}
\renewcommand{\theenumi}{\alph{enumi}}
\renewcommand{\labelenumi}{(\theenumi)}

\newcounter{Theorem}
\newcounter{Definition}
\numberwithin{equation}{section}
\numberwithin{Theorem}{section}

\theoremstyle{plain} %% This is the default, anyway
\newtheorem{thm}[Theorem]{Theorem}
\newtheorem{cor}[Theorem]{Corollary}
\newtheorem{lem}[Theorem]{Lemma}
\newtheorem{prop}[Theorem]{Proposition}
%\usepackage{upgreek}

\theoremstyle{definition}
\newtheorem{defn}[Theorem]{Definition}

\theoremstyle{remark}
\newtheorem{remark}{Remark}[section]
\newtheorem{ex}[Theorem]{Example}
\newtheorem{nota}[Theorem]{Notation}



\begin{document}

\thispagestyle{empty}

\noindent{\Large Homework 4 (Due October 28 at 8am)}\bigskip


\begin{enumerate}[1.]



%%%%%%%%%%%%%%%%%%%%%%%%%%%%%%%%%%%%%%%%%%%%%%%%%%%%%%%%%%%%%%%%
\item
For any integer $N>2$, the \textbf{difference operator (discrete derivative)} is the operator
\begin{equation*}
\mathbf{D}:\mathbb{F}^{N}\rightarrow\mathbb{F}^{N-1},\quad(\mathbf{D}\mathbf{x})(n):=\mathbf{x}(n+1)-\mathbf{x}(n).
\end{equation*}
Show that $\mathbf{D}$ is linear, and in the specific example of $N=4$, find the $3\times 4$ matrix representation of $\mathbf{D}$.

\bigskip

%%%%%%%%%%%%%%%%%%%%%%%%%%%%%%%%%%%%%%%%%%%%%%%%%%%%%%%%%%%%%%%%
\item
Let $\mathbf{1}\in\mathbb{F}^N$ be the \textbf{all-ones vector} defined by $\mathbf{1}(n)=1$ for all $n\in[N]$.

Show that $\mathbf{D}\mathbf{x}=\mathbf{0}$ if and only if $\mathbf{x}$ is a scalar multiple of $\mathbf{1}$, that is, that
\begin{equation*}
\ker(\mathbf{D})
=\{\mathbf{x}\in\mathbb{F}^N: \mathbf{D}\mathbf{x}=\mathbf{0}\}
=\{c\mathbf{1}: c\in\mathbb{F}\}.
\end{equation*}

\bigskip

%%%%%%%%%%%%%%%%%%%%%%%%%%%%%%%%%%%%%%%%%%%%%%%%%%%%%%%%%%%%%%%%
\item
For any integer $N>2$, the \textbf{cumulative sum (discrete integral)} is the operator
\begin{equation*}
\mathbf{S}:\mathbb{F}^{N-1}\rightarrow\mathbb{F}^{N},\quad(\mathbf{S}\mathbf{y})(n):=\left\{\begin{array}{cl}0,&n=1,\\\displaystyle\sum_{k=1}^{n-1}\mathbf{y}(k),&n>1.\end{array}\right.
\end{equation*}
Show that $\mathbf{S}$ is linear, and in the specific example of $N=4$, find the $4\times 3$ matrix representation of $\mathbf{S}$.

\bigskip

%%%%%%%%%%%%%%%%%%%%%%%%%%%%%%%%%%%%%%%%%%%%%%%%%%%%%%%%%%%%%%%%
\item
Prove the discrete version of one-half of the Fundamental Theorem of Calculus, namely that:
\begin{equation*}
\mathbf{D}(\mathbf{S}\mathbf{y})=\mathbf{y},\quad\forall\,\mathbf{y}\in\mathbb{F}^{N-1}.
\end{equation*}

\textit{Hint: For any $n\in[N-1]$,
note $[\mathbf{D}(\mathbf{S}\mathbf{y})](n)=(\mathbf{S}\mathbf{y})(n+1)-(\mathbf{S}\mathbf{y})(n)$ and then use the definition of $\mathbf{S}$.}

\textit{Note: In class, we will discuss how operator composition corresponds to matrix multiplication.
In particular, the statement that $\mathbf{D}(\mathbf{S}\mathbf{y})=\mathbf{y}$ for all $\mathbf{y}$ is equivalent to having $\mathbf{D}\mathbf{S}=\mathbf{I}$.
For example, when $N=4$, multiplying the $3\times 4$ and $4\times 3$ matrices you found in the previous problems should yield a $3\times 3$ identity matrix.
Here, we caution that $\mathbf{S}\mathbf{D}\neq\mathbf{I}$ and moreover that $\mathbf{D}$ is not invertible.}

\bigskip

%%%%%%%%%%%%%%%%%%%%%%%%%%%%%%%%%%%%%%%%%%%%%%%%%%%%%%%%%%%%%%%%
\item
Putting the previous problems together with results from class,
show that for any $\mathbf{y}\in\mathbb{F}^{N-1}$ the equation $\mathbf{D}\mathbf{x}=\mathbf{y}$ has a solution, and moreover, that the set of all solutions $\mathbf{x}$ to this equation is
\begin{equation*}
\{\mathbf{x}\in\mathbb{F}^N: \mathbf{D}\mathbf{x}=\mathbf{y}\}
=\{\mathbf{S}\mathbf{y}+c\mathbf{1}: c\in\mathbb{R}\}.
\end{equation*}

\textit{Note:
The fact that $\mathbf{D}\mathbf{x}=\mathbf{y}$ has a solution for any $\mathbf{y}\in\mathbb{F}^{N-1}$ implies that the image of $\mathbf{D}$ is its codomain.
In general, when this occurs, we say the operator is \textbf{onto (surjective)}.
In this particular instance, we see that any discrete function $\mathbf{y}$ is the discrete derivative of other discrete functions $\mathbf{x}$,
one of which is the discrete integral $\mathbf{S}\mathbf{y}$ of $\mathbf{y}$;
all other solutions are obtained by adding an arbitrary constant discrete function $c\mathbf{1}$ to it.}


\bigskip

\item Let \(\mathcal{V}\) be a finite-dimensional vector space over a field \(\mathbb{F}\). Prove that a subset \(\mathcal{W}\subset\mathcal{V}\) is a subspace if and only if there exists a linear function \(\mathbf{L}:\mathcal{V}\to\mathcal{V}\) such that \(\mathcal{W}=\operatorname{ker}(\mathbf{L})\). 

Hint: Let \(M = \operatorname{dim}\mathcal{W}\) and let \(\{w_{n}\}_{n=1}^{M}\) be a basis for \(\mathcal{W}\). Extend this to a basis (by a theorem from class) for \(\mathcal{V}\), define \(\mathbf{L}\) on this basis, and extend linearly (by a theorem from homework).


\bigskip


\item No EigenQuiz assignment this week!

\end{enumerate}



\end{document}