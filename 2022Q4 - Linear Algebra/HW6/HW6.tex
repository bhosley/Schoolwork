\documentclass[12pt]{amsart}


%\usepackage[notref,notcite]{showkeys}
\usepackage{verbatim}
\usepackage{fullpage}
\usepackage{amsfonts}
\usepackage{bbm}
\usepackage{wrapfig}
\usepackage{enumerate}
\usepackage{float}
\usepackage{hyperref}
\usepackage[left=0.5in, right=0.5in, bottom=0.75in, top=0.75in]{geometry}
\setlength{\parindent}{0pt}


\usepackage{color}
\usepackage{colortbl}
\usepackage{graphicx}
\usepackage{epstopdf}
\usepackage{amsmath, amsthm, amssymb}
\usepackage{fancybox}

\newcommand{\1}{\mathbbm{1}}

\newcommand{\eps}{\varepsilon}
\newcommand{\la}{\lambda}
\newcommand{\Contradiction}{\Rightarrow\Leftarrow}
\DeclareMathOperator{\lspan}{span}
\DeclareMathOperator{\tr}{tr}
\DeclareMathOperator{\ran}{ran}
\DeclareMathOperator{\diag}{diag}
\providecommand{\abs}[1]{\lvert#1\rvert}
\providecommand{\norm}[1]{\lVert#1\rVert}
\renewcommand{\theenumi}{\alph{enumi}}
\renewcommand{\labelenumi}{(\theenumi)}

\newcounter{Theorem}
\newcounter{Definition}
\numberwithin{equation}{section}
\numberwithin{Theorem}{section}

\theoremstyle{plain} %% This is the default, anyway
\newtheorem{thm}[Theorem]{Theorem}
\newtheorem{cor}[Theorem]{Corollary}
\newtheorem{lem}[Theorem]{Lemma}
\newtheorem{prop}[Theorem]{Proposition}
%\usepackage{upgreek}

\theoremstyle{definition}
\newtheorem{defn}[Theorem]{Definition}

\theoremstyle{remark}
\newtheorem{remark}{Remark}[section]
\newtheorem{ex}[Theorem]{Example}
\newtheorem{nota}[Theorem]{Notation}



\begin{document}

\thispagestyle{empty}

\noindent{\Large Homework 6 (Due November 18 at 8am)}
\hspace{\fill} {\Large B. Hosley}
\bigskip

\begin{enumerate}[1.]

\item For each \(n\in\mathbb{N}\cup\{0\}\) let \(p_{n}:\mathbb{R}\to\mathbb{R}\) where \(p_{n}(x) = x^{n}\) for all \(x\in\mathbb{R}\). The functions \(\{p_{n}\}_{n=0}^{\infty}\) are called the \textit{power functions}. Consider the real vector space 
\[\mathbb{P}_{N}:=\operatorname{span}\{p_{n}\}_{n=0}^{N} = \left\{\sum_{n=0}^{N}a_{n}p_{n} : a_{0},a_{1},\ldots,a_{N}\in\mathbb{R}\right\}.\]
Define the operator \(\mathbf{F}:\mathbb{P}_{N}\to\mathbb{P}_{N}\) by \[[\mathbf{F}f](x) = (x+1)f'(x)\quad\text{for all }x\in\mathbb{R}\]

\medskip

\begin{enumerate}[(a)]

\item Show that \(\mathbf{F}\) is linear.\medskip

\item When \(N=2\) find the matrix representation of \(\mathbf{F}\) with respect to the basis \(\{p_{n}\}_{n=0}^{2}\).\medskip

\item Find a basis for \(\mathbb{P}_{2}\) consisting of eigenvectors of \(\mathbf{F}:\mathbb{P}_{2}\to\mathbb{P}_{2}\).\medskip

\item Find the matrix representation of \(\mathbf{F}\) with respect to the eigenbasis you found in part (c).\medskip

\end{enumerate}

\bigskip
\hrule
\bigskip
%%%%%%%%%%%%%%%%%%%%%%%%%%%%%%%
%	\begin{ Question 1 }	  %
%%%%%%%%%%%%%%%%%%%%%%%%%%%%%%%
\begin{enumerate}[(a)]
	\item % (a)
	\begin{proof}
		Let \(f_1,f_2\in\mathbb{P}_N\) and \(c_1,c_2\in\mathbb{R}\).
		Then,
		\begin{align*}
			\mathbf{F}[c_1f_1+c_2f_2](x)
			&= (x+1)(c_1f'_1+c_2f'_2)(x) \\
			&= (x+1)(c_1f'_1(x)+c_2f'_2(x)) \\
			&= (x+1)c_1f'_1(x)+(x+1)c_2f'_2(x) \\
			&= c_1[(x+1)f'_1(x)]+c_2[(x+1)f'_2(x)] \\
			&= c_1\mathbf{F}(f_1)(x)+c_2\mathbf{F}(f_2)(x) \\
			&= c_1(\mathbf{F}f_1)(x)+c_2(\mathbf{F}f_2)(x).
		\end{align*}
	\end{proof} \bigskip

	\item 
	Let $A$ be the matrix representation of $\mathbf{F}$ when \(N=2\). Then,
	\[A = \begin{bmatrix}
		0 & 1 & 0 \\
		0 & 1 & 2 \\
		0 & 0 & 2
	\end{bmatrix}\begin{matrix}\\ \\.\end{matrix}\] \bigskip

	\item
	The eigenvalues associated $A$ are \(\la=0,1,2\). An eigenvector from each will form a basis for \(\mathbb{P}_2\). One such basis, drawn from each respectively, is
	\[
	v_1 = \begin{bmatrix} 1 \\ 0 \\ 0 \end{bmatrix},
	v_2 = \begin{bmatrix} 1 \\ 1 \\ 0 \end{bmatrix},
	v_3 = \begin{bmatrix} 1 \\ 2 \\ 1 \end{bmatrix}
	\begin{matrix}  \\  \\ . \end{matrix} 
	\] \bigskip
	
	\item 
	The matrix representation of $\mathbf{F}$ with respect to the eigenbasis found in part (c) is
	\begin{comment}
	\begin{align*}
		\mathbf{F}(a\mathbf{v}_0+b\mathbf{v}_1+c\mathbf{v}_2)
		&= \mathbf{F}(a+b+c +bx+2cx +cx^2) \\
		&= \mathbf{F}((a+b+c) +(b+2c)x +(c)x^2) \\
		&= (x+1)(b+2c +2cx) \\
		&= b+2c +2cx + (b+2c)x +2cx^2 \\
		&= a(0+0x+0x^2) + b(1+x+0x^2) + 2c(1+2x+x^2) \\
		&= a\mathbf{v}_0 + b\mathbf{v}_1 + 2c\mathbf{v}_2 \\
		&= \begin{bmatrix}	0 \\ b \\ 2c \end{bmatrix}
		%This seems incorrect
		\Rightarrow
	\end{align*}
	\end{comment}
	\[\begin{bmatrix}
		0 & 0 & 0 \\
		0 & 1 & 0 \\
		0 & 0 & 2
	\end{bmatrix}
	\begin{matrix}  \\  \\ . \end{matrix}\]

\end{enumerate}
%%%%%%%%%%%%%%%%%%%%%%%%%%%%%%%
%	 \end{ Question 1 }	 	  %
%%%%%%%%%%%%%%%%%%%%%%%%%%%%%%%
\clearpage

\item Suppose that \(\mathcal{V}\) is a finite-dimensional vector space over \(\mathbb{F}\). Assume \(\mathbf{L}:\mathcal{V}\to\mathcal{V}\) is a linear map such that every nonzero vector \(\mathbf{v}\in\mathcal{V}\) is an eigenvector of \(\mathbf{L}\). Show that \(\mathbf{L}\) is a scalar multiple of the identity map. Give an example of a linear map \(\mathbf{L}:\mathbb{R}^{2}\to\mathbb{R}^{2}\) and a basis for \(\mathbb{R}^{2}\) such that every element of the basis is an eigenvector of \(\mathbf{L}\), but \(\mathbf{L}\) is NOT a scalar multiple of the identity operator.

\bigskip
\hrule
\bigskip
%%%%%%%%%%%%%%%%%%%%%%%%%%%%%%%
%	\begin{ Question 2 }	  %
%%%%%%%%%%%%%%%%%%%%%%%%%%%%%%%
\begin{proof}
	Let \(\{\mathbf{v}_n\}_{n=1}^N\) be as basis for $\mathcal{V}$ where \(N=\dim(\mathcal{V})\). 
	Then, since \(\{\mathbf{v}_n\}_{n=1}^N\) is an eigenbasis for 
	\(\mathbf{L}\) by assumption let
	\(\la_n\) be the eigenvalue for each corresponding \(\mathbf{v}_n\).
	Then note that for any \(\mathbf{v}_1,\mathbf{v}_2,\mathbf{v}_3 \in\mathcal{V}\) such that
	\(\mathbf{v}_1+\mathbf{v}_2=\mathbf{v}_3\)
	\[
	\mathbf{L}(\mathbf{v}_3)
	= \mathbf{L}(\mathbf{v}_1+\mathbf{v}_2)
	= \mathbf{L}\mathbf{v}_1+\mathbf{L}\mathbf{v}_2
	= \la_1\mathbf{v}_1+\la_2\mathbf{v}_2
	= \la_3(\mathbf{v}_1+\mathbf{v}_2)
	.\]
	Since L is assumed to be linear \(\la_1=\la_2=\la_3\).
	Next let \(\boldsymbol{\Lambda}\) be a diagonal matrix where each non-zero entry is an eigenvalue of $\mathbf{L}$, in this case, all the same \(\la_n\).
	Then for some invertible $\mathbf{V}:\mathbb{F}^N\to\mathcal{V}$ we see that
	\[
	\mathbf{L}
	=\mathbf{V}\boldsymbol{\Lambda}\mathbf{V}^{-1}
	=\mathbf{V}\la\mathbf{I}\mathbf{V}^{-1}
	=\la\mathbf{V}\mathbf{I}\mathbf{V}^{-1}
	=\la\mathbf{V}\mathbf{V}^{-1}
	=\la\mathbf{I}.
	\]
	By theorem 8.4, L is diagonalizable, 
	and in this case equivalent to the identity matrix multiplied by the scalar \(\la\).		
\end{proof}

\vspace{2em}

Consider a linear map \(\mathbf{L}:\mathbb{R}^{2}\to\mathbb{R}^{2}\) 
represented by
\[ A = \begin{bmatrix} 1 & 1 \\	0 & 3 \end{bmatrix}
\begin{matrix}  \\ . \end{matrix}\]
Next, consider the vectors $\mathbf{b}_1$ and $\mathbf{b}_2$ 
that form a basis for \(\mathbb{R}^2\) where
\[ \mathbf{b}_1 = \begin{bmatrix} 1 \\ 0 \end{bmatrix}
\text{ and }
\mathbf{b}_2 = \begin{bmatrix} 2 \\ 4 \end{bmatrix}
\begin{matrix}  \\ . \end{matrix}\]
A is not a scalar multiple of the identity operator
and has \(\mathbf{b}_1\) and \(\mathbf{b}_2\) as eigenvectors.

%%%%%%%%%%%%%%%%%%%%%%%%%%%%%%%
%	 \end{ Question 2 }	 	  %
%%%%%%%%%%%%%%%%%%%%%%%%%%%%%%%
\clearpage

\item Let \(\mathcal{V}\) be a finite-dimensional vector space over a field \(\mathbb{F}\), and let \(\{\mathbf{v}_{n}\}_{n=1}^{N}\) be a basis for \(\mathcal{V}\). For each \(m\in[N]\) define the function \(\boldsymbol{\varphi}_{m}:\mathcal{V}\to\mathbb{F}\) by
\[\boldsymbol{\varphi}_{m}\left(\sum_{n=1}^{N}\mathbf{x}(n)\mathbf{v}_{n}\right) = \mathbf{x}(m)\quad\text{for each }\mathbf{x}\in\mathbb{F}^{N}.\]
Equivalently, let \(\boldsymbol{\varphi}_{m}:\mathcal{V}\to\mathbb{F}\) be the unique linear map such that
\[\boldsymbol{\varphi}_{m}(\mathbf{v}_{n}) = \begin{cases} 1 & n=m,\\ 0 & n\neq m.\end{cases}\]

\begin{enumerate}[(a)]

\item Show that \(\{\boldsymbol{\varphi}_{m}\}_{m=1}^{N}\) is a basis for \(\operatorname{Hom}(\mathcal{V},\mathbb{F})\).\medskip

\item Find the matrix representation of \(\boldsymbol{\varphi}_{1}\) with respect to the basis \(\{\mathbf{v}_{n}\}_{n=1}^{N}\) for \(\mathcal{V}\) and the standard basis for \(\mathbb{F}\).\medskip

\end{enumerate}

\noindent Let \(\mathbf{\Phi}:\mathcal{V}\to\operatorname{Hom}(\mathcal{V},\mathbb{F})\) be the unique linear map such that \(\mathbf{\Phi}(\mathbf{v}_{n}) = \boldsymbol{\varphi}_{n}\) for each \(n\in[N]\).\medskip

\begin{enumerate}[(a)]

\addtocounter{enumii}{2}

\item In the case that \(\mathcal{V}=\mathbb{R}^{N}\) and \(\mathbf{v}_{m}=\boldsymbol{\delta}_{m}\) for each \(m\in[N]\) find \(\big(\mathbf{\Phi}(\mathbf{v})\big)(\mathbf{v})\) for \(\mathbf{v}\in\mathbb{R}^{N}\). \medskip

\end{enumerate}

\bigskip
\hrule
\bigskip
%%%%%%%%%%%%%%%%%%%%%%%%%%%%%%%
%	\begin{ Question 3 }	  %
%%%%%%%%%%%%%%%%%%%%%%%%%%%%%%%
\begin{enumerate}[(a)]
	\item % (a)
	\begin{proof}
		Let \(\mathbf{H}\in\hom((\mathcal{V},\mathbb{F})\) be arbitrary.
		Note that the matrix representation of $\mathbf{H}$ will be a 
		\(1\times N\) matrix 
		and that \(\boldsymbol{\varphi}_{m}(\mathbf{x})\) evaluates to the 
		\(m^{\text{th}}\) value of any \(\mathbf{x}\).
		Then observe for that for any $\mathbf{H}$ or indeed any 
		\(1\times N\) matrix there exists scalars \(\{c_m\}^N_{m=1}\)
		such that
		\[
			\mathbf{H} = \sum_{m=1}^{N}c_m\boldsymbol{\varphi}_{m} .
		\]
		Thus \(\{\boldsymbol{\varphi}_{m}\}_{m=1}^{N}\) spans 
		\(\hom((\mathcal{V},\mathbb{F})\). Next, note that for 
		\[
			 \mathbf{0}^T = \sum_{m=1}^{N}c_m\boldsymbol{\varphi}_{m}
		\]
		all \(c\in\{c_m\}^N_{m=1}\) must be zero and therefore 
		\(\{\boldsymbol{\varphi}_{m}\}_{m=1}^{N}\) is a basis for \(\operatorname{Hom}(\mathcal{V},\mathbb{F})\). 
	\end{proof}
	\bigskip \medskip

	\item
	The matrix representation of \(\boldsymbol{\varphi}_{1}\)
	is a \(1\times N\) matrix with a 1 as the first element and
	zeroes for elements 2 through \(N\). \\
	
	\item
	In the case described, for each \(\mathbf{v}_n\in\mathbb{R}^N\) \(\big(\mathbf{\Phi}(\mathbf{v})\big)(\mathbf{v})\)
	may be evaluated as
	\[
	\big(\mathbf{\Phi}(\mathbf{v}_n)\big)(\mathbf{v}_n)
	= \big(\boldsymbol{\varphi}_{n}\big)(\boldsymbol{\delta}_{n})
	= 1.
	\]
	
\end{enumerate}
%%%%%%%%%%%%%%%%%%%%%%%%%%%%%%%
%	 \end{ Question 3 }	 	  %
%%%%%%%%%%%%%%%%%%%%%%%%%%%%%%%

\vspace{\fill}

\item Go to \href{https://eigenquiz.app/}{Eigenquiz.app} and complete activity a501177. Upload a completion certificate for this activity to your homework folder.

\end{enumerate}

\end{document}