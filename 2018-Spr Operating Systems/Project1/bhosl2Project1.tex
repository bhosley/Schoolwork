\documentclass[a4paper,man,natbib]{apa6}
\usepackage[english]{babel}
\usepackage[utf8x]{inputenc}
% Common Packages - Delete Unused Ones %
\usepackage{setspace}
%\usepackage{amsmath}
%\usepackage[cache=false]{minted}
\usepackage{graphicx}
\usepackage{caption}
\graphicspath{ {./images/} }
% End Packages %

\title{Project 1}
\shorttitle{P1}
\author{Brandon Hosley}
\date{2018 11 19}
\affiliation{Chung-Wei Lee}
%\abstract{}

\begin{document}
\maketitle
\subsection{Compilation}
The submitted program compiles and executes without error. \\
The most obvious improvement in this category will be to improve the program's handling of additional parameters.
\subsection{Arguments}
The program currently accepts integers between 1 and 10, if outside this range it will default to N = 10.
\subsection{Commands}
This program currently accepts any of the commands listed in the original assignment, calling the corresponding Linux function; cal, date, ls, ps, pwd, who, quit. A "help" function was additionally added to inform the user of which functions are available to them and what those functions do. 
\subsection{Forking}
The program will evoke the fork() function for any valid user command except for "quit" and "help" which will perform their function outright. \\
Each other function will be executed in a child process and passed to a member of the exec() family of functions, in this case, execvp() with no additional parameters.
\subsection{Difficulties}
The most profound difficulty arising in this project was the result of unfamiliarity with the C language, and with system calls. 
\clearpage
\includegraphics[width=\linewidth]{screenshot.png}

%\bibliographystyle{apacite}
%\bibliography{} %link to relevant .bib file
\end{document}