\documentclass[]{article}
\usepackage[english]{babel}
\usepackage{amsmath}
\usepackage[hypcap=false]{caption}
\usepackage{graphicx}
\usepackage{hyperref}
\hypersetup{
	hidelinks
	}

\title{Cloud Security: Assignment 4.4}
\author{Brandon Hosley}
\date{\today}

\begin{document}
	\maketitle
	
\section{Summary} 

In this paper\cite{Shucheng2010} Shucheng et al. describe methodologies for securing resources stored in a cloud environment.
The research team identifies that the shared nature of cloud environments present numerous security challenges to organizations wishing to utilize such services.
Additionally, the team describes challenges to securing resources that a data owner may wish to share with specific parties.
The researchers assume an honest-but-curious model for the cloud service provider, and communication channels with industry standard security.
The newly developed model will address concerns that the researchers had with previous cloud-access-control schemes;
high-end scalability of unique access lockbox-key models, 
low-end scalability within file-groups,
performance issues with large file-specific revocation lists,
potential key exposure with proxy re-encryption,
and using a mutual semi-trusted third party to protect data owner keys.

In simple terms the proposed method starts with the data-owner encrypting the data before storing it on the server. 
The data-owner provides part of the decryption key to the server.
The data-owner maintains an access control list.
When a requester is granted access to a file stored on the cloud server, the server sends a ciphertext that is the stored ciphertext lazily encrypted with the requester's public key.
The requester can then decrypt the ciphertext using their private key, and the two parts of the original key provided by the cloud provider and the data-owner.

\section{Analysis}

The authors do well in integrating good aspects of previously existing work.
The intuition of the proposed system seems to work well and address the concerns enumerated by the team.
While the system will be inferior to the others in certain ways, for the team's described purpose their system makes well warranted trade-offs.
Achieving these results with linear complexity should be acceptable to any practical real-world application.


\clearpage
\bibliographystyle{acm}
\bibliography{\jobname}
\end{document}