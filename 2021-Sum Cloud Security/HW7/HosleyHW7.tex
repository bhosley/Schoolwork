\documentclass[]{article}
\usepackage[english]{babel}
\usepackage{amsmath}
\usepackage[hypcap=false]{caption}
\usepackage{graphicx}
\usepackage{hyperref}
\hypersetup{
	hidelinks
	}

\title{Cloud Security: Assignment 7.4}
\author{Brandon Hosley}
\date{\today}

\begin{document}
	\maketitle
	
\section{Summary} 

This paper\cite{Teaho2012} was produced in an early period of the rapid rise in popular cloud computing.
The problem set that the team addresses is one of not only identity management but of attribute based access and protection of user identity from management by a single authority.
The attribute based encryption (CP\_ABE) used by this project generates keys based on attributes controlled by disparate authorities.
A user's attributes will be compared to a file's permissions trees for each type of action a user may wish to take.
They will request a private key from each authority. 
They will be able to use a composite private key to decrypt a file which was encrypted using a composite public key generated by any one of the authorities on behalf of the data owner.


\section{Analysis}

There are a number of potential weaknesses within CP-ABE when the subject paper is written. 
One such problem is user revocability.
Due to the method this model uses to produce anonymity for the user, every file associated with the (attribute owned by a) revocation authority will require re-encryption, and each user associated with the authority will require a new key.
For a small-scale implementation this may be acceptable, but it would rapidly grow to consume significant resources when considering the size of an organization, the turnover of employees, and the potential for a large amount of data to be produced or stored.
Additionally, while the team addressed the potential for a collusion attack, the rebuttal seemed to imply that every authority would perform this re-encryption and re-issuing key process for each revocation.
This method seems to be optimal for small internet-based communities in which there is a limited amount of data and a greater incentive to protect the attributive privacy of the users.



\clearpage
\bibliographystyle{acm}
\bibliography{\jobname}
\end{document}