\documentclass[]{article}
\usepackage[english]{babel}
\usepackage{amsmath}
\usepackage[hypcap=false]{caption}
\usepackage{graphicx}
\usepackage{hyperref}
\hypersetup{
	hidelinks
	}

\title{Cloud Security: Assignment 2.1}
\author{Brandon Hosley}
\date{\today}

\begin{document}
	\maketitle
	
\section{Summary} 

In this paper\cite{Wang2010} the authors seek to evaluate the cause of observations of unusual packet delay in virtualized data centers.
To isolate potential causes for the packet delays the authors check three parts of the cluster: processor sharing times, round trip delays, and TCP/UDP throughput.
They measure these effects across two different types of VM instance and within one availability zone.
One problem that arose was an issue with measuring tools being subject to irregular scheduling by the hypervisor causing missed events and incomplete evaluation periods.
The authors conclude that scheduling of VMs on the processor fits as a probable source for all the observed networking issues and they provide a few example implications that this would have in practice.

\section{Analysis}
The first two concerns presented are with scientific computing and network experiments using the cloud.
A cloud-based model cannot accurately represent a standard network in this format.
Interruptions to machine priority also mean that the VM is not always listening, and results may miss important observations.
The third example the authors provide is with network management systems hosted in a virtual environment; the problem experienced will be the same with interruptions preventing a consistent and uninterrupted observation period.
This last observation is one that will plague industries as they move to virtualizing more resources, and their network grows. 
A sufficiently small organization may not need much in the way of network monitoring and they may be able to operate within the services provided by the cloud provider.
Larger organizations, especially international organizations that will have networks spanning multiple availability zones or data centers in disparate places will have a much greater need for strong network monitoring tools to maintain the health and effectiveness of their network.

\clearpage
\bibliographystyle{acm}
\bibliography{\jobname}
\end{document}