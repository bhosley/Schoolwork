\documentclass[12pt,letterpaper]{exam}
\usepackage[letterpaper,left=0.5in,right=0.5in,top=0.5in,bottom=0.5in]{geometry}
%\usepackage{toc}

% --- Fonts ---
\usepackage[T1]{fontenc}
\usepackage[utf8]{inputenc}
\usepackage{libertine}

% --- Math ---
\usepackage{amssymb}
\usepackage{amsthm}
\usepackage{mathtools}
\usepackage{bbm}

% --- References ---
\usepackage{hyperref}

% --- Other Common Packages ---
%\usepackage{booktabs}
%\usepackage{graphicx}
%\usepackage{multicol}
%\usepackage[shortlabels]{enumitem}
%\usepackage[table]{xcolor}
%\usepackage{wrapfig}
%\usepackage{capt-of}
%\usepackage{tikz}
%\usepackage{pgfplots}
%\usetikzlibrary{shapes,arrows,positioning,patterns}
%\usepackage{pythonhighlight}
\usepackage{comment}

%\newcommand\chapter{ X }
\renewcommand{\thequestion}{\textbf{\thesection.\arabic{question}}}
\renewcommand{\questionlabel}{\thequestion}

% -------------------------------- Top Matter -------------------------------- %
\newcommand{\class}{PhD Specialty} % This is the name of the course 
\newcommand{\assignmentname}{Examination} % 
\newcommand{\authorname}{Hosley, Brandon} % 
\newcommand{\workdate}{\today} % 
%\printanswers % this includes the solutions sections

% --------------------------------- Document --------------------------------- %
\begin{document}
\pagestyle{plain}
\thispagestyle{empty}
\noindent
 
% ---------------------------------- Header ---------------------------------- %
\noindent
\begin{tabular*}{\textwidth}{l @{\extracolsep{\fill}} r @{\extracolsep{10pt}} l}
	\textbf{\class} & \textbf{\authorname} &\\%Your name here instead, obviously
	\textbf{\assignmentname } & \textbf{\workdate} & \\
\end{tabular*}\\ 
\rule{\textwidth}{2pt}

% ----------------------------------- Body ----------------------------------- %
\tableofcontents
\hrule


\section{Dr. Yielding's Questions}


While HARL is a subfield of MARL, the distinction is one that seems to be ill-%
defined int he literature. 

In the broadest sense the term could accurately refer to any case of MARL
in which the agents are not identical copies. 




	% What is the clear distinction between HARL and MARL? 
	% Is HARL just MARL where the agents may not share the same policy? 
	% Is it a distinct family of RL algorithms? 
	% There seems to be a whirlwind of terminology use of "HARL" and "heterogeneous" that would be wise to clarify. 
	% The only reference that specifically coins the term "HARL" seems to be an Irish conference paper in the intro, reference [9].
	% For this distinction, please clarify how it is, relates to, and/or is distinguished from the following, and where the concepts may overlap:

\begin{questions}
	\question
	% Specific heterogeneous versions of RL algorithms such as HAA2C (from A3C), 
	% HADDPG (from DDPG) etc (reference [53]). 
	% If it is specifically meant to be this, will you be implementing them yourself as policy classes in Rllib for your experiments? This is perhaps hinted at on page 25 under Contribution 1

	% -------------------------- End Question -------------------------- %


	\question 
	% How agents' policy (and thus ANNs), observations, and actions align during training or execution:
	% a. Training multiple agents under a single policy that controls all agents in the same action space (overlord ie one 'agent' controlling multiple actors)
	% VS
	% b. Training multiple agents that share the same policy but have individual observation/action (swarm/flock ie each actor is its own agent but the same policy)
	% VS
	% c. Training multiple agents that may or not share a policy in the same environment (ie a policy mapping function used like in Rllib, as shown in your Figure 9 from Rllib docs)

\begin{comment}
a - Decidedly not, I would argue that this may not even necessarily qual as MARL
b - Def MARL not HARL
c - * Here is were we live. Circumstantially HARL
\end{comment}

	% -------------------------- End Question -------------------------- %


	\question
	% 3. Training vs execution methods:
	% a. Training multiple agents together in an environment, then executing separately in different environments. 
	% Reference [37] calls this CTDE - Centralized Training Decentralized Execution Cooperative MARL, and might be the same as (a) above
	% b. Training multiple agents separately then executing in different environments. Reference [37] calls this DTDE - Decentralized Training Decentralized Execution Cooperative MARL and might be the same as (b) above
	
	% -------------------------- End Question -------------------------- %
	
	% This could be done perhaps as a mixture of an extra paragraph or two in the intro defining terminology
	% and editing elsewhere in the doc to use said terminology or adding 1-2 clarifying sentences throughout
	% the doc where relevant

\end{questions}

\section{Dr. Cox's Questions}
\section{Dr. Robbins' Questions}

\end{document}
