\documentclass[12pt]{amsart}
\usepackage[english]{babel}
\usepackage[left=0.75in, right=0.75in, bottom=0.75in, top=0.75in]{geometry}
\usepackage[utf8x]{inputenc}
\usepackage{amsmath,amssymb,amsthm}
\usepackage{enumerate}
\usepackage{graphicx}


\title{OPER 640 - Stochastic Modeling and Analysis}
\author{B. Hosley}
\date{\today}

\begin{document}
	\maketitle
	\raggedbottom

\section{Description}

% Problem description
% 	Paraphrase project description,
% 	formulate analysis questions of interest to leadership

\section{Modeling the Situation}

%DTMC model formulation (Lesson2 slide 18 (with single entity transition diagram))

% Define Xn as the state of the system at time n or the n-th observation (or event)
% Define the state space S as the set of all possible outcomes for Xn
% Verify the Markov Property is satisfied.
% Verify Time-Homogeneity
% Construct the Transition Probability Matrix. Or specify probability function p_ij.
% As needed:
% 	Formulate state transition function (X_n+1 as a function of X_n)
% 	Draw a Transition Diagram.
% 	Define a cost or reward function.



\section{Baseline Analysis}

% Baseline analysis - primary analysis questions
% 	How often will the squadron be able to achieve mission success, meeting 8-orbit mission requirements
% 	How long will it take for the squadron become FMC >= 8 aircraft
% 	What is the long-term sortie generation rate?
% 	What is the long-term mission capable aircraft availability rate?

\section{Exploratory Analysis}

% Sensitivity (excursion) analyses - the what-if questions, 
% 	explore possible solutions by modifying parameter values

\section{Conclusions and Recommendations}

% Conclusions and recommendations


\end{document}