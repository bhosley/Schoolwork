\documentclass[answers]{exam}
\usepackage[english]{babel}
\usepackage[utf8x]{inputenc}
\usepackage{amsmath,amssymb,amsthm}

\usepackage{xcolor}
\usepackage{graphicx}
\usepackage{enumitem}

\begin{document}

\noindent{\large OPER 640 - Stochastic Modeling and Analysis I%
	% Homework 2 % (Due Jan X at 10am)
	}
\hspace{\fill} {\large B. Hosley}
\bigskip

% \unframedsolutions

\begin{questions}

%%%%%%%%%%%%%%%%%%%%%%%%%%%
%	\begin{ Question 1}	  %
%%%%%%%%%%%%%%%%%%%%%%%%%%%
\question 
\textit{Computational Exercise 3.2 (page 79).}

Let \(T =\text{min}\{n\geq0:X_n =0\}\).Compute
\begin{enumerate}
	\item \(P(T \geq 3|X_0 = 1)\), 
	\item \(\text{E}(T|X_0 =1)\),
	\item \(\text{Var}(T |X_0 = 1)\).
\end{enumerate}


\textbf{3.2} Do the above problem with the following transition probability matrix:
\[\begin{bmatrix}
	0.0 & 1.0 & 0.0 & 0.0 \\
	0.2 & 0.0 & 0.8 & 0.0 \\
	0.0 & 0.8 & 0.0 & 0.2 \\
	0.0 & 0.0 & 1.0 & 0.0 
\end{bmatrix}.\]

\begin{solution}
\begin{enumerate}
	\item \(P(T \geq 3|X_0 = 1)\)
	
	
	\item \(\text{E}(T|X_0 =1)\)
	\begin{align*}
		\begin{bmatrix}
			0.0 & 0.8 & 0.0 \\
			0.8 & 0.0 & 0.2 \\
			0.0 & 1.0 & 0.0 
		\end{bmatrix}
	\end{align*}
	
	
	\item \(\text{Var}(T |X_0 = 1)\)
	
\end{enumerate}
\end{solution}
%\end{ Question 1}

%%%%%%%%%%%%%%%%%%%%%%%%%%%
%	\begin{ Question 2}	  %
%%%%%%%%%%%%%%%%%%%%%%%%%%%
\question 
\textit{Computational Exercise 3.4 (page 80). 
	Compute variance and coefficient of variation as well. 
	(Assume game time is measured in rounds, 
	and each player tosses one die per round.)}


\textbf{3.4} If two players are playing the game in Computational Exercise 3.3, 
compute the probability distribution of the time until the game ends 
(that is, when one of the two players lands on 16.) 
Assume that there is no interaction among the two players, 
except that they take turns. Compute the mean time until the game terminates.

\begin{center}
	\includegraphics[width=0.45\linewidth]{"Screenshot 2023-01-28 at 2.05.59 PM"}
	\\ Figure 3.4 \textit{The 4 by 4 chutes and ladders.}
\end{center}

\begin{solution}

[[0.0, 1/3, 1/3, 1/3,   0,   0,   0,   0,   0,   0,   0,   0],
[   0, 0.0, 1/3, 2/3,   0,   0,   0,   0,   0,   0,   0,   0],
[   0,   0, 0.0, 1/3, 1/3, 1/3,   0,   0,   0,   0,   0,   0],
[   0,   0,   0, 0.0, 1/3, 1/3, 1/3,   0,   0,   0,   0,   0],
[   0,   0,   0,   0, 0.0, 1/3, 1/3, 1/3,   0,   0,   0,   0],
[   0,   0,   0,   0,   0, 0.0, 1/3, 1/3,   0, 1/3,   0,   0],
[   0, 1/3,   0,   0,   0,   0, 0.0, 1/3,   0, 1/3,   0,   0],
[   0, 1/3,   0,   0,   0,   0,   0, 1/3, 1/3,   0,   0,   0],
[   0,   0,   0,   0,   0,   0, 1/3,   0, 0.0, 1/3, 1/3,   0],
[   0,   0,   0,   0,   0,   0, 1/3,   0,   0, 0.0, 1/3, 1/3],
[   0,   0,   0,   0,   0,   0, 1/3,   0,   0,   0, 1/3, 1/3],
[   0,   0,   0,   0,   0,   0,   0,   0,   0,   0,   0, 1.0]]

\end{solution}
%\end{ Question 2}

%%%%%%%%%%%%%%%%%%%%%%%%%%%
%	\begin{ Question 3}	  %
%%%%%%%%%%%%%%%%%%%%%%%%%%%
\question
\textit{Computational Exercise 3.18 (page 81). 
	Ensure the rows in the transition probability matrix sum to 1. 
	If not, normalize the necessary rows so that they do. 
	Compute variance and coefficient of variation as well.}

\textbf{3.18} Suppose a DNA sequence is appropriately modeled by a 
4-state DTMC with transition probability matrix P(1) given on page 22. 
Suppose the first base is A. 
Compute the expected length of the sequence until we see the triplet ACT .

\begin{solution}
	S
\end{solution}
%\end{ Question 3}

%%%%%%%%%%%%%%%%%%%%%%%%%%%
%	\begin{ Question 4}	  %
%%%%%%%%%%%%%%%%%%%%%%%%%%%
\question 
\textit{Computational Exercise 3.22 (page 82). 
	Ensure the rows in the transition probability matrix sum to 1. 
	If not, normalize the necessary rows so that they do.}

\textbf{3.22} In Computational Exercise 3.18 compute the probability 
that we see the triplet ACT before the triplet GCT.

\begin{solution}
	S
\end{solution}
%\end{ Question 4}

%%%%%%%%%%%%%%%%%%%%%%%%%%%
%	\begin{ Question 5}	  %
%%%%%%%%%%%%%%%%%%%%%%%%%%%
\question 
\textit{Computational Exercise 3.24 (page 82).}

\textbf{3.24} Consider the discrete time queue with 
Bernoulli arrivals and departures as described in Example 2.12 on page 17. 
Suppose the queue has i customers in it initially. 
Compute the probability that the system will eventually become empty.

\begin{solution}
	S
\end{solution}
%\end{ Question 5}

%%%%%%%%%%%%%%%%%%%%%%%%%%%
%	\begin{ Question 6}	  %
%%%%%%%%%%%%%%%%%%%%%%%%%%%
\question 
	Consider the maze shown in Figure {\color{red} 1}. 
	At time 0 a mouse is placed in Cell 1 and food in Cell 25. 
	The mouse stays in the current cell for one unit of time and 
	then chooses to move to an adjacent cell at random in a uniform manner 
	(no diagonal movement, cardinal directions only). 
	Its successive moves are independent and completely uninfluenced by the food.
	
	\begin{center}
		\includegraphics[width=0.7\linewidth]{"Screenshot 2023-01-27 at 10.42.23 AM"}
		\\ Figure 1: Mouse Maze
	\end{center}
	
	
	\begin{enumerate}[label=(\alph*)]
		\item What is the expected time required for the mouse to reach the food? 
		Compute variance and coefficient of variation as well.
		
		\item What is the probability that the mouse will reach the food in 50 time-steps or less?
		
		\item Suppose a cat is placed in cell 5. 
		The cat moves like the mouse but in an independent fashion. 
		When the cat and mouse occupy the same cell, the cat promptly eats the mouse. 
		What is the probability that the mouse will reach the cheese before the cat consumes the mouse? 
		(Assume that, if the cat and mouse arrive at the cheese at the same time, the mouse gets to eat the cheese.)
	\end{enumerate}
\begin{solution}
	S
\end{solution}
%\end{ Question 6}

\end{questions}
\end{document}