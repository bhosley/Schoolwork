\documentclass{afitthesis}

% --------------------------------- preamble -----------------------------------

% NOTE: The following macros are the meta data of your document, like the title,
% author, and department. Some of these apply to the whole document. Some, like
% in the section after this, apply to the SF 298 form only. Several are
% commented out because they are probably not needed for your situation.

% Main document information
\title{ WORKING TITLE:\\
    Efficiency, Efficacy, and Explainability in\\
    Heterogeneous Agent Reinforcement Learning }
\author{Brandon Hosley}
% \authorsecond{}
% \authorthird{}
% \authorfourth{}
\rank{Captain, USAF}
% \ranksecond{}
% \rankthird{}
% \rankfourth{}
\previousdegrees{B.Sc., M.Sc.}
\newdegree{Doctor of Philosophy in Data Science}
\graduationdate{May 2026}
\department{\ENS} % \ENY, \ENG, \ENP, \ENC, \ENS, or \ENV
\doctype{prospectus}
\docdesignator{AFIT-ENS-P-XX-X-XXX}
% \address{}
% \disclaimer{}
% \copyrightstatement{}
\committee{ % rank name, Ph.D. \\ role
    {Bruce Cox, Ph.D.\\Chair},
    {Matthew Robbins, Ph.D.\\Member},
    {Maj Nicholas Yielding, Ph.D.\\Member}}
%\abstract{Write your abstract here. This text should probably stay under 100
%    words or so. Make sure to not put any personally-identifiable information
%    about others in this text.}
%\keywords{earth; water; air; fire}
%\dedication{To the one who loves me most.}
%\acknowledgments{I would like to thank the entire committee for your great
%    support.}

% Distribution and Control
% \cui{
%     Controlled By: AETC \\
%     Controlled By: AFIT/ENG \\
%     CUI Category(ies): PRVCY \\
%     Distribution: \DistB{CATEGORY}{DATE}{OFFICE} \\
%     POC: John Smith, 555-123-4567}
% \classified{
%     Classified By: \\
%     Derived From: \\
%     Declassify On: }
% \banner{cui}

% SF 298 (Report Documentation Page) information
%\sfStartDate{Sep 2023}
%\sfEndDate{Mar 2024}
%\sfContractNumber{XXXXXX-XX-X-XXXX}
%\sfGrantNumber{}
%\sfProgramElementNumber{}
%\sfProjectNumber{XXXXXXXX}
%\sfTaskNumber{}
%\sfWorkUnitNumber{}
%\sfSponsorAgency{AFXX/XXXX\\
%    Building XXX, WPAFB OH 45433-7765\\
%    DSN XXX-XXXX, COMM 937-XXX-XXXX, Email: first.last@us.af.mil }
%\sfSponsorAcronyms{}
%\sfSponsorReportNumber{}
%\sfDistribution{\DistA}
%\sfSupplementaryNotes{}
%\sfReportClassification{}
%\sfAbstractClassification{}
%\sfPageClassification{}
%\sfAbstractLimitation{UU}
%\sfResponsiblePerson{Dr. Your Advisor, AFIT/ENG}
%\sfPhoneNumber{(937) XXX-XXXX}

    % SF 298 Override default variables
    % \sfReportDate{}
    % \sfReportType{}
    % \sfTitle{}
    % \sfAuthors{}
    % \sfDepartment{}
    % \sfAddress{}
    % \sfDocDesignator{}
    % \sfAbstract{}
    % \sfSubjectTerms{}
    % \sfPageCount{}
    % \sfClassification{}

\makeatletter
\def\input@path{{Chapters/}{other/}}
\makeatother
\usepackage{cleveref}

% NOTE: Google "Overleaf Getting started with BibLaTeX" for guidance.
\usepackage[style=ieee]{biblatex} % bibliographies
\addbibresource{other/Prospectus.bib} % name of the bibliography file

% Define the default location to look for figures.
% \graphicspath{{fig}} % Uncomment to use

% Define types and file locations of glossaries
\usepackage[nomain,abbreviations,symbols,section]{glossaries-extra}
\makeglossaries%
% AF and ...
\newabbreviation{afit}{AFIT}{Air Force Institute of Technology}
\newabbreviation{asic}{ASIC}{application specific integrated circuit}
\newabbreviation{api}{API}{application programming interface}
\newabbreviation{cots}{COTS}{commercial off-the-shelf}
\newabbreviation{dod}{DoD}{Department of Defense}
\newabbreviation{darpa}{DARPA}{Defense Advanced Research Projects Agency}
\newabbreviation{isr}{ISR}{intelligence, surveillance, and reconnaissance}
\newabbreviation{kl}{KL}{Kakade-Langford}
\newabbreviation{lidar}{LIDAR}{laser imaging, detection, and ranging}
\newabbreviation[shortplural={UAS}]{uas}{UAS}{unmanned aicraft system}
\newabbreviation{sota}{SOTA}{state of the art}
\newabbreviation{offset}{OFFSET}{OFFensive Swarm-Enabled Tactics}
% DARPA

% Stochastics and Games
\newabbreviation{etdr}{ETDR}{expected total discounted reward}
\newabbreviation{mas}{MAS}{multi-agent system}
\newabbreviation{mcts}{MCTS}{Monte-Carlo tree search}
\newabbreviation[longplural={Markov decision processes}]{mdp}{MDP}{Markov decision process}
\newabbreviation{nash}{NE}{Nash equilibrium}
\newabbreviation{posg}{POSG}{partially observable stochastic game}
\newabbreviation{td}{TD}{temporal-difference}
\newabbreviation{rts}{RTS}{real time strategy}
\newabbreviation{dec-pomdp}{Dec-POMDP}{decentralised partially observable \gls{mdp}}

% General AI
\newabbreviation{ai}{AI}{artificial intelligence}
\newabbreviation{ctde}{CTDE}{centralized training with decentralized execution}
\newabbreviation{dnn}{DNN}{deep neural network}
\newabbreviation{gnn}{GNN}{graph neural network}
\newabbreviation{harl}{HARL}{heterogeneous-agent reinforcement learning}
\newabbreviation{marl}{MARL}{multi-agent reinforcement learning}
\newabbreviation{mlp}{MLP}{mulit-layer perceptron}
\newabbreviation{rl}{RL}{reinforcement learning}

% RL Algorithms
\newabbreviation{a3c}{A3C}{asynchronous advantage actor-critic}
\newabbreviation{haa2c}{HAA2C}{heterogeneous-agent \gls{a3c}}
\newabbreviation{impala}{IMPALA}{importance weighted actor-learner architecture}

\newabbreviation{maddpg}{MADDPG}{multi-agent deep-deterministic policy gradient}
\newabbreviation{haddpg}{HADDPG}{heterogeneous-agent deep-deterministic policy gradient}
\newabbreviation{ippo}{IPPO}{independent \gls{ppo}}

\newabbreviation{td3}{TD3}{twin-delayed deep-deterministic policy gradient}
\newabbreviation{matd3}{MATD3}{multi-agent \gls{td3}}
\newabbreviation{hatd3}{HATD3}{heterogeneous-agent \gls{td3}}

\newabbreviation{ppo}{PPO}{proximal policy optimization}
\newabbreviation{happo}{HAPPO}{heterogeneous-agent \gls{ppo}}
\newabbreviation{mappo}{MAPPO}{multi-agent \gls{ppo}}

\newabbreviation{trpo}{TRPO}{trust region policy optimization}
\newabbreviation{hatrpo}{HATRPO}{heterogeneous-agent \gls{trpo}}
\newabbreviation{matrpo}{MATRPO}{multi-agent \gls{trpo}}

\newabbreviation{coma}{COMA}{counterfactual multi-agent policy gradients}
\newabbreviation{hpn}{HPN}{hyper policy network}
\newabbreviation{pic}{PIC}{permutation-invariant critic}

% Environments
\newabbreviation{grf}{GRF}{Google research football}
\newabbreviation{lbf}{LBF}{Level-based Foraging}
\newabbreviation{mpe}{MPE}{Multi Particle Environments}
\newabbreviation{sisl}{SISL}{Stanford Intelligent Systems Laboratory}

% Specialized command for this section:
\NewDocumentCommand{\glsNewSymbol}{m m m O{} O{#1}}{%
    \newglossaryentry{#1}{%
        text=\ensuremath{#2}, 
        name=\ensuremath{#3}, 
        description={#4},
        sort={#5}, 
        type=symbols, 
        % group={#6}
    }%
    % This function allows me to use the commands: 
    % \gls{x} -> \(x\) and \Gls{x} -> \(X\)
} 
% \glsNewSymbol{name}{printed in text}{printed in glossary}[description in glossary]

% -------------- General -------------- %
% Reals
\glsNewSymbol{reals}{\mathbb{R}}{\mathbb{R}}[Set of all real numbers.]


% -------------- Games -------------- %
% Agents
\glsNewSymbol{i}{i}{I}[Set of agents \(i\).]
% Time steps
\glsNewSymbol{t}{t}{T}[Set of time steps \(t\).]
% States
\glsNewSymbol{s}{s}{s,S,\bar{S}}[State, state space, set of terminal states.]
% Observation Space
\glsNewSymbol{O}{O}{O,O_i}[Observation space, marginal observation space of agent \(i\).]
\glsNewSymbol{o}{o}{o,o_i}[Joint observation, observation of agent \(i\).]
% Actions
\glsNewSymbol{A}{A}{A,A_i}[Action space, marginal action space of agent \(i\).]
\glsNewSymbol{a}{a}{a,a_i}[Joint action, action of agent \(i\).]
% Reward function
\glsNewSymbol{r}{r}{r,r_i}[Reward, reward for agent \(i\).]
\glsNewSymbol{rew}{\mathcal{R}}{\mathcal{R},\mathcal{R}_i}[
    Reward function, reward function for agent \(i\).]
% Transition Probability Matrix
\glsNewSymbol{P}{\textbf{P}}{\textbf{P}}[
    \(\text{dim}(\Gls{s})\times\text{dim}(\Gls{s})\) 
    State-transition probability matrix.]
% #TODO: Write a better description for probability transition function and transition probability
\glsNewSymbol{p}{p}{P}[Transition probability function.]
% Reward value
\glsNewSymbol{g}{g}{G_t}[Return at time \gls{t}.]


% -------------- Machine Learning -------------- %
% Policy
\glsNewSymbol{pi}{\pi}{\pi,\pi_i}[Policy (decision-making rule), Policy of agent \(i\).]
\glsNewSymbol{pi_opt}{\pi^*}{\pi^*}[Optimal joint policy.]
% Action Value
\glsNewSymbol{q}{q}{Q}[Set of state-values or state-value functions.]
\glsNewSymbol{q_pi}{q_\pi}{q_\pi(a|s)}[Value of action \gls{a} given state \gls{s} by policy \gls{pi}.]
\glsNewSymbol{q_*}{q_*}{q_*(a|s)}[Value of state action\gls{a} given state \gls{s} under optimal policy.]
% State Value
\glsNewSymbol{v}{v}{V}[Set of action-values or action-value functions.]
\glsNewSymbol{v_pi}{v_\pi}{v_\pi(s)}[Value of state \gls{s} given by policy \gls{pi}.]
\glsNewSymbol{v_*}{v_*}{v_*(s)}[Value of state \gls{s} under optimal policy.]


% -------------- Optimization -------------- %
% Expected Value
\glsNewSymbol{expRet}{\mathbb{E}}{\mathbb{E}_{\pi}[\cdot]}[Expected return from policy \gls{pi}.]
% Discount factor
\glsNewSymbol{discount}{\gamma}{\gamma}[Discount-rate.]
% Step size parameter
\glsNewSymbol{step-size}{\alpha}{\alpha}[Policy update step-size parameter.]


% -------------- Other -------------- %

% #OPT: Ensure function of symbol links

% Formatting tools
\usepackage{tcolorbox}  % For call out boxes
\usepackage{titlesec}
%\titleformat{\subsection}[runin]
%  {\normalfont\bfseries\itshape}{\thesubsection}{1em}{}

% ------------------------------ body of paper ---------------------------------

\begin{document}
\maketitle % This command will create all the prefatory pages.


% --------------------
\chapter{Introduction}
% Introduction %
\section{Motivation}%
\label{sec:motivation}

In August 2023, at the National Defense Industrial Association's 
Emerging Technologies conference, Deputy Secretary of Defense Kathleen Hicks 
announced the \emph{Replicator Initiative}~\cite{robertson2023}. 
This initiative aims to field all-domain attritable autonomous (ADA2) systems,
leveraging autonomous technology to address production disadvantages the 
United States may face in great-power competition~\cite{zotero-2656}. 
Unlike previous efforts that sought bespoke military drones~\cite{bajak2023}, 
\emph{Replicator} aims to leverage more readily available technologies, 
potentially inspired by the demonstrated effectiveness of \gls{cots} \glspl{uas} 
employed by both belligerents during the Russian invasion of Ukraine~\cite{bajak2023a}.

Despite the strategic push of the \emph{Replicator Initiative}, 
a Rand Corporation study from February 2024~\cite{gerstein2024} predicted 
that effective, intelligent swarms are still several years from realization. 
These advancements will require the confluence of developments in several 
different areas; communications and signals, manufacturing, \gls{ai},
and usability. In the study they highlight a difficulty in the transition 
from what they call \emph{Surrogate Swarms} (many \gls{uas} controlled
by at least one human) to fully autonomous ones.
The area they termed \gls*{ai} likely refers to the systems used for autonomous 
decision making, which presents a host of constituent problems to be solved.

One of the primary difficulties in achieving effective swarm behavior 
lies in the training of these agents. Traditional methods often 
involve predefined team sizes and uniform capabilities among agents, 
which limits their adaptability and scalability. To overcome these limitations, 
it is essential to develop training methodologies that enable agents to 
generalize across variable team sizes and diverse hardware configurations. 
This adaptability is crucial for deploying flexible and resilient autonomous 
systems in dynamic environments.

\Gls{rl} has emerged as the most promising paradigm for addressing 
these challenges. \Gls{rl} has become the cornerstone for developing 
autonomous decision-making systems~\cite{sutton2018}. However, 
the application of RL to multi-agent systems, particularly in swarm scenarios, 
introduces additional layers of complexity. 
Agents must not only learn to achieve individual goals but also to 
collaborate effectively to maximize the collective reward~\cite{cao2012}.
\Gls{marl} extends beyond single-agent \gls{rl} to provide solutions 
to many such problems, even significantly exceeding the capabilities 
of the latter within certain domains that do not necessitate multiple 
agents~\cite{gronauer2022}. 

In the pursuit of a flexible training framework
it is essential to consider the potential of heterogeneous agents. 
Leveraging open-source and commercial off-the-shelf (COTS) resources 
logically extends to a framework that can utilize disparate hardware. 
This flexibility allows for the integration of various sensors, processors, 
and capabilities, maximizing the potential of each agent within the swarm. 

\Gls{harl}, is an extension of \gls{marl} characterized by agents having 
distinct roles, capabilities, policies, or objectives. It is an 
underexplored area, but early results suggest that \gls{harl} algorithms
have the potential to significantly improve problem-solving efficiency 
and adaptability~\cite{calvo2018}.

The practical applications of heterogeneous actors are vast, particularly in 
scenarios where coordination and cooperation are crucial. For example, 
drone swarms in search and rescue missions can leverage diverse capabilities, 
such as various sensory equipment or distinct maneuverability traits, 
enabling more thorough area coverage and expedited victim detection
~\cite{hoang2023,kouzeghar2023}.
Similarly, agricultural robots outfitted with different sensors and tools can 
concurrently execute multiple tasks—ranging from harvesting to soil analysis and
pest control—thereby substantially enhancing efficiency and increasing crop 
yields~\cite{carbone2018,amarasinghe2019}.

While some scenarios necessitate a \gls{harl} approach, the central aim of 
this dissertation is to investigate strategies that improve the training 
efficiency of multi-agent systems, including, heterogeneous settings. 
Across three planned contributions, we explore methods that reduce the 
computational cost of learning while maintaining or improving final policy 
performance. These methods span policy upsampling, invariant observation processing, 
and progressive network expansion. The remainder of this document 
presents a unified background and research plan supporting these contributions.

%To fully realize the potential benefits of \gls{harl}, it is crucial to 
%understand the trade-offs involved. This \printdoctype will survey current 
%approaches for training multiple agents, 
%identifying the strengths and limitations of each. 
%In \cref{sec:problem_statement,sec:research_question} we will 
%highlight several under-explored areas that warrant further investigation. 
%By addressing these gaps, we aim to advance the field and enhance the 
%effectiveness of multi-agent systems in real-world applications.

\section{Background}%
\label{sec:background}

    \subsection*{Reinforcement Learning: From Deep Blue to AlphaStar}%

\Gls{rl} has marked a number of significant milestones in outperforming humans 
in competitive domains. One of the most pivotal events occurred in 1997 
when IBM's Deep Blue defeated world chess champion Garry Kasparov. Though 
powered mainly by brute force computation and hand-tuned algorithms rather than 
learning-based approaches~\cite{campbell2002}, Deep Blue's victory set the 
stage for the broader application of \gls{ai} in complex strategic games.

The field progressed significantly with DeepMind's introduction of AlphaGo 
in 2015. AlphaGo employed a combination of deep neural networks and 
\gls{mcts}~\cite{silver2016}, initially trained on human expert games and 
further improved through self-play.
This method enabled AlphaGo to defeat Lee Sedol, one of the world's top Go 
players, illustrating \gls{rl}'s potential to tackle challenges in games with 
vast state spaces and decisions typically driven by human intuition.

This breakthrough was quickly followed by the development of 
AlphaZero~\cite{silver2017}, which revolutionized the field by mastering chess,
Go, and Shogi through self-play alone, without any human-derived 
data~\cite{silver2017a}. The method of self-play demonstrated not only 
versatility across different games but also the capacity of RL systems to 
develop domain-independent strategies.

A subsequent major advancement was achieved with DeepMind's 
AlphaStar~\cite{vinyals2019}, 
which demonstrated that advanced RL models could handle complex strategies, 
real-time decision-making, and intricate player interactions. 
AlphaStar's success in defeating professional StarCraft II players
was particularly notable due to the game's demand for long-term strategic 
planning and quick tactical responses in an open-ended scenario.

To achieve the level of proficiency demonstrated in AlphaStar, 
Vinyals et al.~\cite{vinyals2019} employed a multifaceted approach 
that integrated deep learning, imitation learning, 
reinforcement learning, and multi-agent learning. 
The specifics of these contributions are explored in detail 
in~\Cref{ch:literature_review}.

    \subsection*{Multi-agent Reinforcement Learning}%:Learning to Work Together}

Well before the the rise of \gls*{rl}, research in game theory 
provided foundational work that would be indispensable to multi-agent systems.
As early as 1951, Brown~\cite{brown1951iterative} proposed a method for 
calculating \gls{nash} in two-player games through a process he termed 
fictitious play, which involves iteratively updating strategies. 
Unlike simultaneous strategy updates, Brown's method applies updates 
sequentially—a condition that Berger~\cite{berger2005, berger2007} 
later proved to be sufficient for guaranteed convergence to 
\gls{nash} in nondegenerate ordinal games. 

The development in this area remained comparatively stunted until 
significant strides were made in single-agent methods. 
Traditional Bellman-Equation-style solutions, while effective in single-agent 
settings and certain types of multi-agent games like zero-sum and 
common-payoff games, faced greater difficulty in stochastic or 
degenerate games~\cite{shoham2007}.
These challenges highlighted the limitations of extending single-agent 
frameworks directly to multi-agent environments without modifications.

The introduction of multiple independent agents in an environment introduces 
additional complexity; the game becomes non-stationary from 
the perspective of any single agent~\cite{busoniu2008}. 
This non-stationarity poses unique challenges as each agent must adapt 
to the actions of others whose strategies are also evolving, 
significantly complicating the learning process.

In this realm, the extension into \gls{marl} allows for the consideration 
of a wide spectrum of interactions as described in game theory, 
ranging from purely competitive to purely cooperative. 
\gls{marl} addresses the multitude of challenges associated with these 
diverse styles of interaction, offering frameworks and strategies that 
are adaptable to varying degrees of cooperation and competition among 
agents~\cite{lowe2020}.

In some cases, the interactions of interest in \gls{marl} are asymmetrical, 
adding another layer of complexity to strategy formulation and 
execution~\cite*{sun2023}.
Among the most notable successes in handling mixed modes of cooperation 
and competition is OpenAI's achievement with OpenAI Five. In this project, 
a team of agents reached superhuman performance in the multiplayer game Dota 2,
utilizing a blend of techniques including a unique method of skill transfer 
known as ``surgery'' and extensive use of self-play~\cite{berner2019}.
This milestone not only demonstrated the capability of \gls{marl} systems to 
manage and excel in intricate, dynamically shifting competitive environments 
but also showcased the potential for these systems to develop and refine 
collaborative strategies among heterogeneous agents.

    \subsection*{The Game Theoretical Concerns}%:

In both papers describing AlphaStar~\cite{vinyals2019} and OpenAI 
Five~\cite{berner2019}, the authors mention in sparse detail the 
``game theoretic'' concerns their respective frameworks seek to address. 
These concerns are primarily attributed to the potential pitfalls of self-play. 
Two major problems are highlighted.

The first problem, often called strategic 
collapse~\cite{berner2019,vinyals2019}, describes a phenomenon 
where an agent overfits to a self-defeating strategy, 
resulting in a feedback loop and a counter-intuitive observation where 
cumulative rewards per episode may suddenly drop and become unrecoverable 
during continued training.

The second problem is cyclic strategy chasing, where multiple agents 
converge on a set of strategies that balance wins against one strategy 
with losses to another. An example of this is the game rock-paper-scissors. 
Balduzzi et al. (2019) discuss this phenomenon in~\cite{balduzzi2019}.

To mitigate these risks, AlphaStar implemented a structured league-play 
schema that continuously pitted different agent policies against each other. 
OpenAI Five, on the other hand, used a simpler approach by maintaining a pool
of previous milestone agents for ongoing comparison and refinement.


    % --- Bringing it together.
    \subsection*{Towards Flexible Training Methodologies}

The advancements made by AlphaStar~\cite{vinyals2019} and 
OpenAI Five~\cite{berner2019} have underscored the potential of \gls{marl}
to achieve superhuman performance in complex, dynamic environments,
and inspired a large amount of follow-on research.
However, their approaches still involve training agents to 
operate as a team with a predefined number of members;
AlphaStar~\cite{vinyals2019} effectively a team of one,
and OpenAI Five~\cite{berner2019} always a team of five.

Smit et al.~\cite{smit2023} was inspired by AlphaStar~\cite{vinyals2019},
and attempted to make significant efficiency improvements.
One of the methods that they tried (ultimately unsuccessfully) was to train a 
subset of the final team, effectively the same as the scalability problem.
We revisit Smit et al.~\cite{smit2023} in \cref{ch:literature_review}.

While these pioneering efforts have demonstrated remarkable achievements, 
they also highlight significant challenges that remain unresolved. 
Notably, the ability to train agents that can generalize across variable 
team sizes and configurations is crucial for advancing the field of 
multi-agent reinforcement learning. Addressing these challenges requires 
innovative methodologies that enhance the flexibility and efficiency of 
training processes. 

This leads us to our core research questions, which aim to evaluate
the potential of \gls{marl} or \gls{harl} to overcome these limitations 
and achieve scalable, robust, and adaptable autonomous systems.

%\section{Problem statement}%
%\label{sec:problem_statement}%
%
%This \printdoctype aims to examine several key aspects that 
%contributed to the success of those projects, 
%with the overarching goal of identifying how these methods can be 
%applied not only to address the game-theoretic challenges impacting 
%\gls{marl} but also to improve scalability and flexibility in other contexts. 
%Understanding how different components of these training frameworks 
%contribute to the effectiveness of the resulting agents and their relative 
%costs is crucial for enabling further research and broader applications of 
%\gls{marl} technologies. 
%Additionally, we hypothesize that \gls{harl} techniques will offer observable 
%benefits during the training process, not only in addressing game-theoretic 
%problems but also in enhancing the resulting agents' flexibility when 
%deployed in novel configurations.

\section{Research Questions}%
\label{sec:research_question}%
\label{sec:relevance_and_importance}

% --- REVISED RESEARCH QUESTIONS AND CONTRIBUTIONS ---
\begin{description}
    \item[] \textbf{\hyperref[ch:contribution_1]{Contribution 1}:} Policy Upsampling
    \begin{itemize}
        \item[RQ 1.1:] Can pretraining smaller teams of agents and then scaling to the target 
        team size via policy duplication and retraining improve training efficiency 
        without sacrificing final policy performance in MARL?
        \item[RQ 1.2:] How does the effectiveness of this direct scaling strategy vary across 
        environments with different forms of agent heterogeneity 
        (e.g., behavioral vs. intrinsic)?
    \end{itemize}
    \item[]\textbf{\hyperref[ch:contribution_2]{Contribution 2}:} Invariant Observation Processing
    \begin{itemize}
        \item[RQ 2.1:] How does incorporating input-invariant structures—such as policies 
        robust to observation permutations and variable team sizes—impact learning efficiency 
        and team robustness in settings with heterogeneous observations?
        \item[RQ 2.2:] Do input-invariant architectures support more stable policy behavior 
        under changes to team size and partial observation loss during execution?
    \end{itemize}
    \item[] \textbf{\hyperref[ch:contribution_3]{Contribution 3}:} Progressive Network Expansion
    \begin{itemize}
        \item[RQ 3.1:] Can tensor-based projections be used to grow a policy networks 
        capacity during training while preserving its prior functional behavior?
        \item[RQ 3.2:] Can we identify appropriate transition points during training 
        when projecting to a larger policy network yields the greatest benefit?
        \item[RQ 3.3:] Does this progressive architectural growth strategy reduce total training 
        cost or improve final policy performance compared to fixed-size architectures?
    \end{itemize}
\end{description}


\section{Outline}%

\begin{comment} %%%% Dissertation Version %%%%
The remainder of this document is designed to systematically explore 
the complex field of multi-agent reinforcement learning, 
particularly focusing on heterogeneous-agent systems.
Following this introductory chapter, 
\ref{ch:literature_review}: Literature Review provides a comprehensive analysis 
of the seminal and recent literature pertinent to our research focus. 
\Cref{ch:methodology} details the experimental and analytical techniques 
employed. \Cref{ch:results} presents the data and findings from our research, 
followed by \cref{ch:discussion}, where these results are interpreted 
in the context of existing knowledge and their implications for future 
research are explored.
The \emph{dissertation} will conclude with~\cref{ch:conclusion}, 
which summarizes the research and suggests avenues for further investigation. 
Each chapter builds upon the previous to provide a comprehensive understanding 
of the topic, aiming to contribute valuable insights to the field of \gls{harl}.
\end{comment}

%%%% Prospectus Version %%%%
The remainder of this document is organized as a dissertation prospectus 
supporting a k-paper dissertation format. It begins with a literature review in 
\cref{ch:literature_review}, establishing foundational context across reinforcement 
learning, multi-agent systems, and heterogeneous coordination. 
Following this, \cref{ch:contribution_1,ch:contribution_2,ch:contribution_3} 
contain each of the three planned contributions:
\begin{itemize}
    \item \textbf{\hyperref[ch:contribution_1]{Contribution 1}:} Investigates whether direct 
        upsampling of policies from smaller teams can reduce training cost while preserving 
        final performance.
    \item \textbf{\hyperref[ch:contribution_2]{Contribution 2}:} Evaluates input-invariant 
        policy architectures as a method to support shared learning across heterogeneous 
        observation structures.
    \item \textbf{\hyperref[ch:contribution_3]{Contribution 3}:} Explores progressive network 
        growth using tensor projection to improve training efficiency in high-capacity networks.
\end{itemize}
Each contribution is framed as a standalone study but collectively supports 
the overarching goal of building scalable, flexible HARL systems. 
The document concludes with a timeline and project plan for completing the proposed research.

% -------------------------
\chapter{Literature Review}%
\label{ch:literature_review}
% #TODO: Literature Review
LITERATURE REVIEW PLACEHOLDER

Test: \gls{marl}

Testing an equation with links:

\begin{equation}
    A(s,a) = \Gls{q}(s,a) - \Gls{v}(s)
\end{equation}



% -------------------
%\chapter{Methodology}%
\label{ch:methodology}
\chapter{Contributions}
\section{Practical implications}
\section{Theoretical implications}

For each proposed contribution:

\begin{itemize}
    \item (Multiple pages probably) on the motivation and the specific problem. Call back to the appropriate references.
    \item Specify the academic contributions (Bulletized format makes for easier, clearer conveyance)
    \item Summary of specific methodology to be employed by each part
    \item Expected results. That is, the expected implications of whatever the results will be. We aren't predicting the answer to the questions, just what the answers will tell us.
\end{itemize}
In this chapter we propose contributions divided into three sections.
The sections are intended to represent coherent groups of publishable results.
The target venue of publication is \emph{Autonomous Agents and Multi-Agent Systems}
~\cite{zotero-2605} or publication with similar objectives.

\Cref{fig:timeline} details the proposed timeline for the contributions listed in the following
sections. The bars of the gantt chart represent the period time during with the subject of the
bar is expected to be a primary focus. The end of the bar coincides with the point at which
the associated paper is finished and has been submitted to some publication.

\begin{figure}[htbp]
    \begin{center}
    \begin{ganttchart}[y unit title=0.4cm, y unit chart=0.5cm,
    vgrid,hgrid, title label anchor/.style={below=-1.6ex},
    title left shift=.05, title right shift=-.05, title height=1,
    progress label text={}, bar height=0.8, bar top shift=0.1,
    group right shift=0, group top shift=.6,
    inline, 
    milestone inline label node/.append style={left=2mm},
    %bar label/.style={anchor=west},    bar top shift
    group height=.3]{1}{24}
    
        %labels
        \gantttitle{2024}{24} \\
        \gantttitle{}{10} 
        \gantttitle{Jun}{2} 
        \gantttitle{Jul}{2} 
        \gantttitle{Aug}{2} 
        \gantttitle{Sep}{2} 
        \gantttitle{Oct}{2} 
        \gantttitle{Nov}{2} 
        \gantttitle{Dec}{2} \\

        %tasks
        \ganttbar[bar inline label node/.style={left=10mm},]{Prospectus}{12}{12} \\
        \ganttbar[bar inline label node/.style={left=15mm},]{Specialty Exam}{14}{14} \\
        \ganttmilestone{Specialty Defense NLTD}{23} \\
        \ganttbar[bar/.style={fill=blue!15}]{Paper 1}{12}{18} \\
        \ganttbar[bar/.style={fill=blue!25}]{Paper 2}{19}{24} 

        %relations 
        \ganttlink{elem0}{elem1} 
    \end{ganttchart}
    \begin{ganttchart}[y unit title=0.4cm, y unit chart=0.5cm,
    vgrid,hgrid, title label anchor/.style={below=-1.6ex},
    title left shift=.05, title right shift=-.05, title height=1,
    progress label text={}, bar height=0.8, bar top shift=0.1,
    group right shift=0, group top shift=.6,
    inline,
    group height=.3]{1}{24}
    
        %labels
        \gantttitle{2025}{24} \\
        \gantttitle{Jan}{2} 
        \gantttitle{Feb}{2} 
        \gantttitle{Mar}{2} 
        \gantttitle{Apr}{2} 
        \gantttitle{May}{2} 
        \gantttitle{Jun}{2} 
        \gantttitle{Jul}{2} 
        \gantttitle{Aug}{2} 
        \gantttitle{Sep}{2} 
        \gantttitle{Oct}{2} 
        \gantttitle{Nov}{2} 
        \gantttitle{Dec}{2} \\

        %tasks
        \ganttbar[bar/.style={fill=blue!25}]{Paper 2}{1}{6} \\
        \ganttbar[bar/.style={fill=blue!35}]{Paper 3}{7}{16} \\
        \ganttbar{Dissertation/Def. Prep.}{17}{24} 
        %\ganttbar{Defense Prep}{}{} 
    \end{ganttchart}
    \end{center}
    \caption{Planned Timeline}
    \label{fig:timeline}
\end{figure}

\section{Contribution 1}
\label{sec:contribution1}

\subsection{Motivation}
Contribution 1 is motivated by the need to establish a broader baseline evaluation of relevant 
\gls{marl} and \gls{harl} algorithms. Specifically, we aim to determine if there are significant 
differences in the generalizability of the resulting policies given a fixed model class.
We will measure generalizability by assessing solution quality under deviations from the training 
conditions, i.e., when agents trained under a given policy are tasked with solving problems 
alongside teammates trained under separate instances.
This exploratory study aims to evaluate candidate training algorithms for use in the 
latter stages of our research.

\subsection{Methodology}
We will train a set of \gls{marl} and \gls{harl} algorithms across candidate scenarios 
multiple times, with the number of iterations determined by the available time and 
computational resources on the university cluster. This phase will provide data on training time, 
consistency of performance, and variance in training curves.

Next, we will evaluate the algorithms in purely cooperative tasks using teams constructed from 
agents drawn from separate training instances. We expect \gls{marl} training instances to produce 
agents with sufficiently similar policies, resulting in teams that demonstrate similar, 
albeit slightly reduced, levels of effectiveness. In contrast, \gls{harl} allows 
agents to converge on distinct policies, potentially leading to diverse team behaviors. 
The absence of existing research on this scenario suggests a wide range of possible outcomes.

For tasks where training instances converge on distinct agent roles, we hypothesize that evaluation
teams composed of similarly specialized agents will significantly under-perform compared to teams 
with appropriately diverse agents. Finally, we will assess the algorithms in competitive settings, 
introducing non-stationarity from adversarial agents. 
This allows for testing team mixing in a manner similar to the cooperative tasks and 
evaluating performance against adversarial agents trained in separate instances.

\subsection{Resources}
To perform this battery of tests, we intend to use a framework that facilitates the application 
of each algorithm to various environments, thereby improving interoperability. 
We believe that RLlib~\cite{liang2018}, part of the larger Ray project~\cite{zotero-2599}, 
is the ideal tool for this line of research. RLlib offers several important features that 
streamline the construction of the appropriate pipeline, integrating well with the \glspl{api} 
of other mature frameworks that provide beneficial functionalities.

\begin{description}
    \item[Data Collection:] 
    There are multiple options for data collection such as \emph{Tensorboard}~\cite{zotero-2601} 
    and \emph{Weights and Balances}~\cite{zotero-2603}. 
    The most common metrics observed in the papers reviewed for this \printdoctype 
    were episode returns and win rates, typically recorded at intervals~
    \cite{zhong2024,yu2022,papoudakis2021,lowe2020,zheng2020}.
    \item[Algorithms:] 
    RLlib already implements several popular algorithms, though not all algorithms of interest are 
    included. Some algorithms may need to be developed within the existing framework standards.
    \item[Environments:] 
    RLlib supports a wide variety of environments. Most of the environments used in 
    the referenced papers are already implemented within this framework.
    \item[Policy Management:] 
    RLlib includes functions for exporting policies in a pickled format. 
    More importantly, it offers functions for loading policies and managing agents 
    independently in \gls{marl} settings.
\end{description}
\begin{figure}[htbp]
    \includegraphics[width=\linewidth]{rllib_marl_policies.png}
    \caption{Multi-agent support in RLlib~\cite{zotero-2599}}
    \label{fig:rllib_marl_policies}
\end{figure}

\subsection{Anticipated Obstacles}
One anticipated obstacle is that some of the algorithms of interest are not currently 
implemented within the RLlib framework. While this does not prevent experimentation, 
it does limit the scope of our study. To address this, we plan to either translate 
these algorithms into the RLlib framework or build wrappers to ensure compatibility.

\subsection{Expected Contributions}
\begin{description}
    \item[Baseline Evaluation:] 
    Establishing a comprehensive baseline for \gls{marl} and \gls{harl} algorithms, 
    particularly those presented in \cite{zhong2024}, by rigorously examining their 
    performance across various scenarios.
    \item[Training Consistency:] 
    Providing insights into the relationship between training time, consistency of performance, 
    and variance in training curves for \gls{marl} and \gls{harl} algorithms.
    \item[Cooperative Task Performance:] 
    Evaluating the performance of agents in cooperative tasks when teams are constructed from 
    agents trained in separate instances, highlighting the differences between 
    \gls{marl} and \gls{harl} in terms of policy convergence and team effectiveness.
    \item[Competitive Task Evaluation:] 
    Assessing the performance of algorithms in competitive settings, examining the impact of 
    non-stationarity introduced by adversarial agents, and the effectiveness of team mixing.
    \item[Novel Insights:] 
    Generating new insights into the behavior of agents when deployed in novel configurations, 
    contributing to the broader understanding of multi-agent reinforcement 
    learning and its practical applications.
    \item[Algorithm Implementations:] 
    Contributing to a growing, open source project.
\end{description}



\section{Contribution 2}

\subsection{Motivation}
Contribution 2 is motivated by applying select principles from the league formulations outlined 
in \cite{vinyals2019} and \cite{berner2019} to address the unresolved problem highlighted by 
Smit et al.~\cite{smit2023}. Smit et al. utilized a football (soccer) simulation environment and 
encountered significant difficulties in developing a scalable training methodology. Their 
objective was to train agents in a 4v4 setting that could effectively operate in an 11v11 game.

While they observed some emergent organization of the players during the training phase, 
the additional agents clustered into groups that maintained the same formations and spacing 
developed during training when scaled up. This issue highlights the challenge of scaling 
training methodologies for more complex environments.

In this \printdoctype we propose that the organization observed by Smit et al. represents an 
emergent role-heterogeneity, which was distinct yet similar to the division of responsibilities 
among players in different positions when the game is played by humans. 
We intend to investigate the usability of this heterogeneity to inform curricula planning and
to address and potentially resolve the issues experienced by Smit et al.
We propose two extensions in the next subsection. These extensions aim to enhance the 
scalability and effectiveness of the training methodology for complex, large-scale environments.

\subsection{Methodology}
For the first proposed extension, we will introduce a period of training that targets varying 
the number of agents in assigned roles. In the second we propose a modified observation space.
We will employ two approaches to secondary training. In both cases, 
we will train a smaller team for 1,000,000 steps, consistent with the baseline training used in~
\cite{zhong2024} and similar to the duration used by Smit et al. in their paper.

The first approach for secondary training involves associating the agents' emergent roles with 
corresponding human positions and utilizing sub-games, such as those found in the 'academy' 
scenarios developed by Kurach et al. for Google Research Football~\cite{kurach2020}. 
This method aims to refine the agents' roles through specialized training scenarios.

The second approach for secondary training continues with the full game but varies the number of 
instances of distinct agents per episode. For example, let \(\{A,B,C,D\}\) represent four 
emergent roles. In each episode of secondary training, a random policy \(\pi\in\{A,B,C,D\}\) 
will be selected, and one to three additional copies of agents with that policy will be added. 
This variation aims to enhance the agents' adaptability to different team compositions.

For the second extension, we will apply transformations to the environmental part of the agents' 
observation space based on the number of agents on the team. 
In its simplest form, this may involve tiling. 
We also plan to experiment with a function that forms a convex hull around the agent, 
bounded by the true edge of the environment and using its nearest teammates as vertices. 
This approach aims to provide a more structured and relevant observation space, 
potentially improving the agents' situational awareness and decision-making capabilities.

\subsection{Resources}
At minimum we can utilize the same football environment, however, using the results from 
contribution 1 we hope to identify additional candidate environments that can provide 
meaningful insight into this extensibility challenge.

The framework used in \cref{sec:contribution1} is expected to 
provide the necessary components to test the curricula and observation space 
changes proposed in the preceding methodology section.

\subsection{Anticipated Obstacles}

A foreseeable weakness to the first part of this approach is that it still requires some human 
intervention when developing the training period targeted at the roles of the agents.
The second approach is a very simple approach to begin addressing this.
We anticipate that this aspect will provide a potential avenue for future work.

\subsection{Expected Contributions}
\begin{description}
    \item[Enhanced Role Adaptability:] 
    By introducing training that varies the number of agents in assigned roles, 
    we aim to enhance the adaptability of agents to different team compositions. 
    This approach is expected to improve the flexibility and robustness of the resulting 
    policies when scaled to larger teams.
    \item[Improved Training Methodologies:] 
    The use of sub-games and targeted training scenarios will refine agents' roles, 
    potentially leading to more specialized and effective behaviors. 
    This contribution will provide insights into how targeted training can enhance the 
    overall performance of multi-agent systems.
    \item[Scalability Solutions:] 
    The proposed methodologies will address the challenges identified by Smit et al. 
    in scaling from smaller to larger team configurations. Successful implementation of these 
    techniques will offer scalable training solutions for complex, large-scale environments.
    \item[Observation Space Transformations:] 
    Through experimenting with transformations in the agents' observation space,
    we aim to improve their situational awareness and decision-making capabilities. 
    This contribution will offer new techniques for structuring observation spaces to 
    better support complex team-based tasks.
    \item[Real-World Applications:] 
    The insights gained from this research will be applicable to various real-world scenarios 
    where heterogeneous teams need to collaborate effectively. 
    Examples include robotic swarms for search and rescue missions, 
    autonomous vehicle coordination, and complex industrial automation tasks.
    \item[Empirical Validation:] 
    The study will provide empirical validation of the proposed training extensions, 
    demonstrating their impact on the scalability and effectiveness of multi-agent systems. 
    These findings will contribute to the broader understanding and advancement of 
    heterogeneous-agent reinforcement learning.
\end{description}


\section{Contribution 3}

\subsection{Motivation}
The direction of Contribution 3 is highly dependent on the results obtained 
from the previous contributions. The general motivation is to 
continue exploring and developing more efficient methods for scalable \gls{marl}. 
The outcomes of the earlier studies will influence our perspective on the potential of 
\gls{harl} algorithms as a superior approach for achieving scalable, robust, and diverse solutions.

Given Contributions 1 and 2, the next logical step appears to be exploring auto-curricular methods. 
This research area is relatively new, with limited literature available before 2019. 
Auto-curricular methods involve dynamic adjustment of training curricula based on the agents' 
performance and learning progress, which could significantly enhance the adaptability and 
efficiency of \gls{marl} systems.

The exploration of auto-curricular methods aims to automate the process of curriculum design, 
potentially leading to more effective training regimes that adapt to the evolving capabilities of 
agents. This approach could address several challenges identified in earlier contributions, 
such as role specialization, scalability, and performance consistency across different team 
configurations.

By investigating auto-curricular methods, we aim to push the boundaries of current \gls{marl} 
research, providing a foundation for more advanced and autonomous training methodologies. 
This direction aligns with the overarching goal of developing scalable, efficient, and 
robust multi-agent systems capable of tackling complex real-world problems.

% ----------------------------
%\chapter{Results and Analysis}%
\label{ch:results}% chktex 24
\input{results.tex}

~\label{ch:discussion}

% ------------------
%\chapter{Conclusion}%
\label{ch:conclusion}% chktex 24
This dissertation addressed the challenge of training robust multi-agent 
policies in heterogeneous settings, where agents may differ in their sensing capabilities, 
actuation constraints, or behavioral roles. 
The three contributions examined complementary strategies for improving 
training efficiency, enabling parameter sharing across structurally distinct agents, 
and evaluating the relative merits of architectural versus representational 
approaches to heterogeneity. Together, these investigations advance both the 
theoretical understanding and practical application of \gls{harl}.

\section{Summary of Contributions}

The first contribution established that curriculum-based team scaling, 
where policies are pretrained on smaller teams before upsampling to target 
configurations, can meaningfully reduce training costs in cooperative \gls{marl}. 
Results across Waterworld, Multiwalker, and \gls{lbf} demonstrated that the 
effectiveness of this strategy depends critically on task structure. 
In environments with low coordination demands and symmetric agent roles, 
pretraining accelerated convergence without compromising final performance. 
In settings requiring tight coordination under sparse rewards, pretraining 
served as a stabilizing mechanism that enabled learning in configurations 
where training from scratch frequently failed. However, when dynamic role 
complexity was high, the benefits diminished, indicating that small-team 
pretraining offers limited foresight into the diverse interactions encountered 
by larger, more heterogeneous groups.

The second contribution introduced implicit indication as a representational 
framework for enabling shared policies across heterogeneous agents. 
By constructing homogenized observation spaces that span all agent-specific 
subspaces and allowing policies to condition implicitly on the pattern of 
populated observation elements, this approach eliminates the need for 
explicit agent identifiers or per-type policy networks. 
Empirical evaluation demonstrated that implicit indication matches the 
performance of heterogeneous baselines while achieving a storage footprint
of $1/|\gls{i}|$ relative to methods maintaining separate policies per agent type. 
The disjoint-span training condition yielded particularly strong relative improvements, 
suggesting that non-overlapping observation structures may reduce gradient interference 
during shared-parameter learning~\cite{zhong2024}.

The third contribution provided a direct comparison between architectural 
and representational paradigms for handling observation heterogeneity. 
The \gls{pic}~\cite{liu2020b}, which employs \glspl{gnn} to achieve 
permutation-invariant value estimation, was evaluated against implicit 
indication under controlled experimental conditions. Results demonstrated 
that the representational approach substantially outperformed the 
architectural alternative across all sensor configurations and evaluation 
conditions, including scenarios involving sensor loss, team composition 
changes, and zero-shot generalization to novel observation patterns. 
These findings indicate that when heterogeneity is structural and 
semantically decomposable, explicit observation masking provides 
more effective conditioning for policy learning than learned graph-based aggregation.

\section{Unifying Insights}

Across all three contributions, a consistent theme emerged: the manner 
in which the learning problem is represented matters as much or more 
than algorithmic sophistication in enabling effective \gls{harl}. 
Observation schema design in the first contribution, homogenized 
spaces in the second, and the comparison of masking-based versus graph-based 
approaches in the third all point toward the primacy of representational 
choices in determining learning outcomes. This finding aligns with broader 
trends in machine learning, where feature engineering and input encoding have 
historically played decisive roles even as model architectures have grown more powerful.

A second insight concerns the relationship between heterogeneity type and method 
selection. The distinction between behavioral heterogeneity, where structurally 
identical agents develop divergent behaviors through independent learning, and 
intrinsic heterogeneity, where agents differ in their observation or action spaces, 
proves consequential for determining which strategies are most effective. 
Curriculum-based scaling addresses behavioral heterogeneity by providing 
foundational coordination skills that transfer to larger teams, while 
homogenization addresses intrinsic heterogeneity by enabling a single 
policy to operate across structurally distinct agents. Practitioners 
confronting heterogeneous multi-agent systems should first characterize 
the nature of the heterogeneity present before selecting an appropriate training methodology.

A third insight concerns robustness. The implicit indication framework exhibited 
inherent tolerance to sensor dropout, team-size changes, and novel agent compositions 
without requiring explicit robustness training. This suggests that well-designed 
representations can yield deployment flexibility as a natural consequence rather 
than as an additional objective to be optimized. For autonomous systems operating 
in dynamic environments where configurations may change during operation, such 
inherent robustness properties offer practical advantages over methods requiring 
retraining or architectural modification.

\section{Significance}

The barriers to deploying autonomous multi-agent systems remain substantial~\cite{jin2025}. 
Training costs scale with team size and interaction complexity, 
heterogeneous agent configurations complicate policy design, 
and deployed systems must operate reliably under conditions that 
may differ from training. This dissertation contributes to addressing each of these barriers.

By demonstrating that reduced-size pretraining can lower computational 
costs in settings where role differentiation is limited, the first 
contribution offers a practical tool for improving training efficiency. 
By establishing implicit indication as a viable framework for shared 
learning across structurally distinct agents, the second contribution 
reduces the architectural overhead traditionally associated with heterogeneous teams. 
By showing that representational solutions can outperform more complex architectural 
alternatives, the third contribution simplifies the design space confronting 
practitioners building heterogeneous multi-agent systems.

These findings have particular relevance for domains where heterogeneity is 
common and computational resources are constrained. 
Multi-robot systems deployed for disaster response, environmental monitoring, 
or defense applications often involve agents with varying sensor suites and 
must be trained efficiently to meet operational 
timelines~\cite{mohddaud2022, kouzeghar2023}. The methods 
developed here offer paths toward more tractable training regimes 
and more flexible deployment configurations.

More broadly, this work underscores that the challenges of 
heterogeneous multi-agent learning are fundamentally representational 
before they are algorithmic. Advances in optimization and architecture 
remain valuable, but their impact is mediated by how the learning 
problem is structured. As multi-agent systems continue to grow in 
scale and diversity, attending carefully to representational design 
will be essential for realizing their potential in real-world applications.


%
%Research schedule
%	Research phase
%	Objectives
%	Deadline
%
%references


% --------------------------------- optional -----------------------------------
\appendix % This command is necessary before any appendix chapters.

\chapter{Glossary}
\printunsrtglossary[type=symbols,title=Summary of Notation]
\printglossaries%
%\printabbreviations[title=]
%\printunsrtglossary[type=symbols,style=long]


% ------------------------------- bibliography ---------------------------------
\nocite{*} % Uncomment to print all .bib entries, regardless of citation.
\clearpage\phantomsection% Creates new page and section for bibliography.
\addcontentsline{toc}{appendix}{\bibname} % Adds bib to table of contents.
\printbibliography% Prints the bibliography.

% --------------------------------- optional -----------------------------------
%\begin{vita} % Add name within brackets for multiple vitas.
%    If you are adding a description about yourself. Use the \verb|vita|
%    environment. Keep the contents to one page. If for some reason there is
%    more than one author contributing to this paper, each author should have a
%    separate page.
%\end{vita}

% ---------------------------- Standard Form 298 -------------------------------
%\sfTwoNineEight
\end{document}
