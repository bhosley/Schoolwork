\documentclass[xcolor={svgnames}]{beamer}
\usepackage[english]{babel}
\usepackage{lmodern}
\usepackage{libertinus}
\usepackage{csquotes}
\usepackage{graphicx}
\usepackage{multicol}
%\usepackage[svgnames]{xcolor} % Must pass direct to beamer class
\graphicspath{../images}

\usepackage{caption}
\usepackage{algorithm}
\usepackage{algpseudocode}
\usepackage{comment} 

\usepackage[backend=biber,style=ieee,style=authortitle]{biblatex} %authortitle authoryear
\addbibresource{../../2025Bibs/Prospectus.bib}

\usepackage{xpatch}
\xapptobibmacro{cite}{\setunit{\nametitledelim}\printfield{year}}{}{}


\title{Contribution 1 - Vector Check}
\subtitle{Preliminary Results and Research Progress}
\author{Capt. Brandon Hosley\inst{1}}
\institute[ENS]{
    \inst{1}
    Department of Operational Sciences\\
    Air Force Institute of Technology}
\date{\today}

\AtBeginSection[]{
  \begin{frame}
    \tableofcontents[currentsection]
  \end{frame}
}

\usetheme{Madrid}
\usecolortheme{beaver}
\mode<presentation>

\begin{document}

% Title slide
\frame{\titlepage}
\begin{frame}
    \tableofcontents
\end{frame}

% Slide 1: Introduction and Objectives
\section{Introduction}

\begin{frame}{Research Overview}

    %\textbf{Dissertation Theme:} 

    \textbf{Motivating Problem:}
    To enhance the usefulness of heterogeneous-agent reinforcement learning (HARL)
    via efficiency and efficacy improvements through:
    \begin{itemize}
        \item Curriculum Design
        \item Group Composition
        \item Roles
        \item Etc.
        %\item \emph{Leveraging insights from game theory and evolutionary biology to inform agent cooperation and specialization.}
    \end{itemize}
    \vspace{0.5cm}

    \textit{Objective:} Improve scalability, resource-efficiency, and adaptability
    of HARL policies that adapt to changing environments and heterogeneous agent teams.
\end{frame}

% Begin Lit review

% Evolution of RL in Games

\begin{frame}{Evolution of Reinforcement Learning in Games}
    \textbf{Key References:}
    \begin{itemize}
        \item Go \footcite{silver2016}$^,$\footcite{silver2017}
        \item Chess\footcite{silver2017a}
        \item StarCraft II\footcite{vinyals2019}
        \item DOTA\footcite{berner2019}
        \item \textbf{Challenge:} High resource demands highlight the need for more efficient 
            training methods.
    \end{itemize}
\end{frame}

% Foundations of MARL
\begin{frame}{Foundations of MARL}
    \textbf{Key References:}
    \begin{itemize}
        \item Shoham et al.\footcite{shoham2007a}, Busonui et al.\footcite{busoniu2008}, 
            and Cao et al.\footcite{cao2012}.
        \item Early concepts in multi-agent learning, distributed coordination.
        \item \textbf{Challenges:} Non-stationarity, credit assignment, 
            \textcolor{DarkBlue}{scalability issues}.
    \end{itemize}
\end{frame}

% Scalability Challenges in MARL
\begin{frame}{Scalability Challenges in MARL}
    \textbf{Key Insights}
    \begin{itemize}
        \item \textbf{Agent Complexity:} Action-state space explosion.
            \footcite{lillicrap2019}$^,$~\footcite{baker2019}$^,$~\footcite{leibo2021}
        \item \textbf{Environment Complexity:} Increased observation-action space.
            $^{11,}$~\footcite{ye2020}$^,$~\footcite{shukla2022}$^,$~\footcite{liang2024}
        \item \textbf{Robustness:} Sensitivity to disturbances and adversarial influences.
            \footcite{gleave2021}$^,$~\footcite{li2019}$^,$~\footcite{spooner2020}$^,$~\footcite{guo2022}
    \end{itemize}
    \vspace{0.5em}
\end{frame}
%(Liu et al., 2024) liu2024

% MARL Scalability Strategies

\begin{frame}{MARL Scalability Strategies}
    \textbf{Notable Strategies:}
    \begin{itemize}
        \item Value decomposition to handle dynamic populations.
            \footcite{zhang2021}$^,$~\footcite{nguyen2020}
        \item Hierarchical learning and role assignment.
            \footcite{cui2022}
        \item \textcolor{DarkBlue}{Curriculum learning} and transfer learning for adaptability.
            \footcite{shukla2022}$^,$~\footcite{shi2023}
    \end{itemize}
\end{frame}

% TODO: Add HARL Slide

\begin{frame}{Heterogeneous-Agent Reinforcement Learning (HARL)}
    \textbf{Harl Extends MARL Challenges:}
    \begin{itemize}
        \item \textbf{Coordination Complexity:} Heterogeneous roles further complicate 
            cooperation and strategic alignment.
            ~\footcite{wakilpoor2020}$^,$~\footcite{yang2020a}$^,$~\footcite{gronauer2022}
        \item \textbf{Scalability Issues:} Increased joint state-action space makes 
            learning computationally expensive.
            ~\footcite{leibo2021}$^,$~\footcite{rizk2019}
        \item \textbf{Adaptability and Robustness:} Policies must generalize across 
            changing agent compositions and dynamic environments.
            ~\footcite{yang2021a}$^,$~\footcite{koster2020}
    \end{itemize}
\end{frame}

% Distinction Between MARL and HARL Variants

\begin{frame}{Defining MARL and HARL Variants}
    \textbf{Proposed Distinctions:}
    \begin{itemize}
        \item \textbf{MARL (Multi-Agent Reinforcement Learning):}
        \begin{itemize}
            \item Agents may share identical structures or learn different behaviors.
            \item Focuses on collaborative or competitive dynamics among agents.
        \end{itemize}
        \item \textbf{Heterodox-Agent RL (Behavioral Heterogeneity):}
        \begin{itemize}
            \item \emph{Structural Homogeneity:} Agents have identical observation and action spaces.
            \item \emph{Behavioral Diversity:} Independent policies allow agents to learn unique behaviors.
            \item \textit{Example:} Identical drones taking on distinct roles in a cooperative task.
        \end{itemize}
        \item \textbf{Heterogeneous-Agent RL (Intrinsic Heterogeneity):}
        \begin{itemize}
            \item \emph{Structural Heterogeneity:} Agents differ in observation or action spaces.
            \item \emph{Behavioral Specialization:} Policies adapt to agents' intrinsic differences.
            \item \textit{Example:} Combined arms.
        \end{itemize}
    \end{itemize}
\end{frame}
% Contextual Background

\begin{frame}{Context for Contribution 1}
    \textbf{Motivation:}
    \begin{itemize}
        \item Inspired by Smit et al.'s~\footcite{smit2023} 
        work using MARL to scale training in 2v2 to playing 11v11 in a football-like game.
    \end{itemize}

    \begin{multicols}{2}
        \textbf{Method:}
        \begin{enumerate}
            \item Trained teams with 4 methods
            \begin{itemize}
                \item Naive Heuristic
                \item ``Team Work'' Heuristic
                \item 2v2 Trained Agents
                \item 11v11 Trained Agents
            \end{itemize}
            \item Evaluated through full games (11v11) against each other.
        \end{enumerate}

        \vfil

        \begin{center}
            \includegraphics[width=0.55\linewidth]{images/smit2v2.png}
            \includegraphics[width=0.55\linewidth]{images/smit11v11.png}
        \end{center}
    \end{multicols}
\end{frame}


\begin{frame}{Context for Contribution 1}
    \begin{table}
        \centering
        \includegraphics[width=0.75\linewidth]{images/smit_results.png}
        \captionsetup{width=.75\linewidth}
        \caption{\footnotesize Win rate of row teams vs column teams;
            500 games of 11v11.~\footcite{smit2023}}
        \label{table:smit_winrates}
    \end{table}

    \textbf{Key Insights from Prior Work}
    \begin{itemize}
        \item Challenges: computational limit and generalization issues.
        \item Their findings suggest that agents trained in full-team settings demonstrated 
            \emph{better differentiation of individual responsibilities}.
    \end{itemize}
\end{frame}



% Contribution 1 Summary

\begin{frame}{Contribution 1 Summary}
    \textbf{Objective:}
    \begin{itemize}
        \item Investigate if a HARL approach can improve the cost or performance outcome of a
        scaling curriculum.
        %\item Examine conditions influencing performance trade-offs compared to full-scale training.
    \end{itemize}

    \vspace{0.5cm}
    \textbf{Approach:}
    \begin{itemize}
        \item Use SISL environments to implement a scaling curriculum.
        \item Compare training efficiency and task performance over different curriculum parameters.
        \item Similar to Smit's method but using with heterodox agents.
    \end{itemize}

    %\vspace{0.5cm}
    %\textbf{Expected Outcomes}
    %\begin{itemize}
    %    \item Identify conditions where reduced agent training remains viable.
    %    \item Provide insights for scaling strategies in future MARL applications.
    %\end{itemize}
\end{frame}


\section{Methodology}

% Experimental Setup

\begin{frame}{Experimental Setup}
    \textbf{Environment(s):}
    \begin{itemize}
        \item Key attributes include:
        \begin{itemize}
            \item Agent count variability.
            \item Task complexity.
            \item Cooperation requirements.
            \item Potential for role differentiation.
        \end{itemize}
        \item The chosen environments are:
        \begin{itemize}
            \item From the SISL package\footcite{gupta2017} in Farama's PettingZoo.
            \item Continuous.
        \end{itemize}
    \end{itemize}
\end{frame}

% Environment Description
\begin{frame}{Experimental Setup}
    \begin{multicols}{2}
        \textbf{Environments:}
        \begin{itemize}
            \item Waterworld
            \begin{itemize}
                \item Collect food while avoiding negative reward obstacles (poison).
                \item Locating food provides a small reward, 
                    while collecting food provides a large reward but requires multiple agents.
            \end{itemize}
            \item Multiwalker
            \begin{itemize}
                \item A team of robots carry a box.
                \item A small reward is given for forward distance of the box.
                \item Dropping the box is a large negative reward.
            \end{itemize}
        \end{itemize}

        \vfil

        \begin{figure}
            \includegraphics[width=0.25\textwidth]{images/waterworld_screenshot.png}
            \caption{Waterworld environment}
        \end{figure}
        \vspace{-1em}
        \begin{figure}
            \includegraphics[width=0.25\textwidth]{images/multiwalker_screenshot.png}
            \caption{Multiwalker environment}
        \end{figure}
    \end{multicols}
\end{frame}

% Training

\begin{frame}{Training}
    \textbf{Algorithm:}
    \begin{itemize}
        \item PPO
        \item Independent policies
    \end{itemize}

    \textbf{Curriculum:}
    \begin{enumerate}
        \item Pre-training
        \begin{itemize}
            \item $n$ iterations
            \item With set of agents (fewer than target)
            \item Exporting checkpoints periodically
        \end{itemize}
        \item Final Training
        \begin{itemize}
            \item With target number of agents
            \item Instantiated from pre-trained checkpoints
            \item Upsampling performed without replacement; 
                randomly when target number is not a multiple of pretrained agents.
        \end{itemize}
    \end{enumerate}
\end{frame}

% Evaluation Metrics

\begin{frame}{Evaluation Metrics}
    \textbf{Performance Metrics:}
    \begin{itemize}
        \item Reward progression (training curve) as the primary metric to measure performance and convergence.
        \item Comparisons based on per-agent average rewards to account for differing agent counts.
    \end{itemize}

    \textbf{Computational Efficiency:}
    \begin{itemize}
        \item Comparisons are a ratio of time per training step
        \item Must address concerns about possible variance in cluster resource provisioning.
        \item CPU utilization metrics recorded but we considered of limited scientific value.
    \end{itemize}
\end{frame}


\section{Preliminary Results}


\begin{frame}{Relative Time Cost}
    \begin{center}
        %\colorbox{orange!40}{\small Placeholder graphic. Time axis probably isn't value added}
        %\vspace{0.5em}
        \includegraphics[width=0.7\linewidth]{../data/time_costs.png}
    \end{center}
    \textbf{Observation:} Linearity of increasing agent cost.
    %\textbf{Follow-up Question:} Can this be shown (proven) generally?
\end{frame}

% 30 iterations for baseline (pre interval)

% 30 reps of the retraining (May want more)

\begin{frame}{Relative Time Cost to Adjusted Timestep}
    \textbf{Observed Scaling:}
    \begin{itemize}
        \item Linear relationship measured with simple linear regression:
        \[ \text{time(ms)} = 324.2441 + 7004.7673n \]
        where \(n\) is the number of agents.
        \item Adjusted \(R^2=0.999\).
    \end{itemize}
    
    \textbf{Adjusted Time Steps:}
    \begin{itemize}
        \item Normalizes training duration based on equivalent cost.
        \item Example: Training 2 agents for 40 steps costs the same as 4 agents for 20.
    \end{itemize}
\end{frame}

\begin{frame}{Retraining 2 vs 4 Agents}
    \includegraphics[width=\linewidth]{../data/4task.png}
\end{frame}

\begin{frame}{Retraining 2 vs 6 Agents}
    \includegraphics[width=\linewidth]{../data/6task.png}
\end{frame}

\begin{frame}{Retraining 2 vs 8 Agents}
    \includegraphics[width=\linewidth]{../data/8task.png}
\end{frame}

% Observations

\begin{frame}{Observations from Preliminary Results}
    \begin{itemize}
        \item Convergence can be reached earlier with pre-training a subset of agents.
        \item This greater when the ratio between target number of agents to 
            pre-training agents is larger.
        \item Performance gains observed might be attributable to symmetry of agent roles; 
            further investigation is warranted.
    \end{itemize}
\end{frame}


\section{Next Steps}

% Next Steps: Investigating Task Variability

\begin{frame}{Next Steps: Investigating Task Variability}
    \begin{itemize}
        \item Assessing performance in an additional task (Miniwalker, SISL\footcite{gupta2017}) 
            featuring asymmetric agent roles.
        \item Miniwalker consists of one leader and multiple followers with distinct roles.
        \item Assess with Waterworld task requiring 3 agent coop.
    \end{itemize}
\end{frame}

% Predictive Analysis for Training Efficiency

\begin{frame}{Predictive Analysis for Training Efficiency}
    \begin{itemize}
        \item Analyze training curves of two-agent training to identify regions of improved performance.
        \item Investigate the potential to predict an optimal transition point to target tasks.
        \item Determine feasibility of utilizing predictions to optimize training time allocation.
    \end{itemize}
\end{frame}

% Contribution 2: Curriculum Design

\begin{frame}{Contribution 2: Curriculum Design}
    \textbf{Objective:}
    \begin{itemize}
        \item Apply insights from Contribution 1 to a more Intrinsically HARL task 
        (ie. Combined arms wargaming a la AtlAtl).
        %\item Investigate progressive scaling strategies for agent training 
        %    (e.g. if Smit had trained 2, 4, 7, and 11 agents).
        \item Investigate inclusion of credit assignment strategies.
        %\item Explore strategies to transition agents smoothly across different training phases.
    \end{itemize}
    \begin{figure}
        \centering
        \includegraphics[width=0.5\linewidth]{images/atlatl.png}
        \caption{AtlAtl scenario modeling.}
    \end{figure}
\end{frame}

% Exploring Evolutionary Biology Insights

\begin{comment}
    

\begin{frame}{Exploring Evolutionary Biology Insights}
    \textbf{Research Interest:}
    \begin{itemize}
        \item Investigate how evolutionary principles can inform cooperation development.
        \item Examine speciation and specialization as sub-populations find roles in complex tasks.
        \item Apply biologically-inspired mechanisms to improve agent diversity and adaptability.
    \end{itemize}
\end{frame}

\end{comment}

% Contribution 3: Long-Term Vision

\begin{frame}{Contribution 3: Long-Term Vision}
    \textbf{Potential Directions:}
    \begin{itemize}
        %\item Further investigation into cooperation dynamics in heterogeneous agent teams.
        \item Further investigation 
        into cooperation dynamics in intrinsically heterogeneous agent teams.
        %\item Scaling cooperative strategies to more complex, real-world scenarios.
        \item Integration of findings from Contributions 1 and 2 into practical applications.
    \end{itemize}
    \textbf{Feedback Requested:}
    \begin{itemize}
        \item Guidance on refining research focus for the prospectus defense.
        \item Recommendations on experimental design for future contributions.
    \end{itemize}
\end{frame}

\section{Feedback and Discussion}

\begin{frame}{Feedback and Discussion}
    \begin{itemize}
        \item Key questions for the committee:
        \begin{itemize}
            \item Is the research direction still aligned with the objectives?
            \item Are there additional considerations or alternate approaches?
        \end{itemize}
        \item Open floor for suggestions and guidance.
    \end{itemize}
\end{frame}

\begin{frame}
    \centering
    \Huge
    Questions?
\end{frame}

\section{References}

\renewcommand*{\bibfont}{\tiny}
\frame[allowframebreaks]{\printbibliography}

\end{document}
