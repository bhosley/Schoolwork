\section{Hypothesis Testing}
	\begin{tabular}{ccc}
		\((H_0)\) is... &  True & is False \\
		Accept \((H_0)\) & TN \((1-\alpha)\) & Type II \((\beta)\) \\
		Reject \((H_0)\) & Type I \((\alpha)\) & TP \((1-\beta)\)
	\end{tabular}

	\subsection{Finding Tests}
		\subsubsection{Likelihood Ratio Test}
			\[ \lambda(\underset{\sim}{x}) = \frac{\sup_{\Theta_0}L(\theta|\underset{\sim}{x})}{\sup_{\Theta}L(\theta|\underset{\sim}{x})}\]
			Given some constant \(c\) we reject \(H_0\) if \(\lambda<c\) and accept if \(\lambda\geq c\).
			
		\subsubsection{Union-Intersection Test}
			Wherein 
			\[H_0:\theta\in\bigcap_{\gamma\in\Gamma}\Theta_\gamma\]
			the rejection region is
			\[\bigcup_{\gamma\in\Gamma} \left\{T_\gamma(\mathbf{x})\in R_\gamma\right\}
			= \bigcup_{\gamma\in\Gamma} \left\{T_\gamma(\mathbf{x}) > c\right\} \]
			\[= \left\{\mathbf{x}:\underset{\gamma\in\Gamma}{\sup}T_\gamma(\mathbf{x})>c\right\} \]
		
		%\subsubsection{Intersection-Union Test}
		
		
	\subsection{Evaluating Tests}
		\subsubsection{Power Function}
			\[\beta(\theta) = P_\theta(\mathbf{X}\in R)\]
		\textbf{Size} \(\alpha\) test \(\sum_{\theta\in\Theta_0} \beta(\theta) = \alpha\)
		
		\textbf{Level} \(\alpha\) test \(\sum_{\theta\in\Theta_0} \beta(\theta) \leq \alpha\)
			
		\subsubsection{Neyman-Pearson Lemma}
			(1) \(\mathbf{x}\in R\) if \(f(x|\theta_1) > k\, f(x|\theta_0) \),	\\
			(1) \(\mathbf{x}\in R^c\) if \(f(x|\theta_1) > k\, f(x|\theta_0) \)	\\
			(2) for some \(k\geq0\) and \(\alpha=P_{\theta_0}(X\in R)\). Then;
			\begin{enumerate}
				\item Any test satisfying 1 and 2, is a UMP level \(\alpha\) test
				\item If a test exists satisfying 1 and 2 with \(k>0\),
					then every UMP level \(\alpha\) test is a size \(\alpha\) test
					and every UMP level \(\alpha\) test satisfies 1.
			\end{enumerate}
			
			Given a \(T\) sufficient statistic for \(\theta\)
			then, \(t\in S\) if \(g(t|\theta_1) > k\, g(t|\theta_0)\)
			and \(S^c\) if \(<\).
		
		\subsubsection{Monotone Likelihood Ratio}
		A real value parameter \(\theta\) has an MLR if 
		for every \(\theta_2>\theta_1\),
		\(g(t|\theta_2)/g(t|\theta_1)\)
		is a monotone function over \(t\).
		
		\textit{Note:}
		A function also has an MLR if it belongs to an exponential family and the 
		\(w(\theta)\) is a non-decreasing function.
		
		\subsubsection{Karlin-Rubin Theorem}
		Given a \(H_0:\theta\leq\theta_0\) v. \(H_1:\theta>\theta_0\);
		
		and a \(T\) sufficient statistic for \(\theta\) has an MLR;
		
		then for any \(t_0\) the test that rejects \(H_0\)
		iff \(T>t_0\) is a UMP level \(\alpha\) test, 
		where \(\alpha = P_{\theta_0}(T>t_0)\)
		
		