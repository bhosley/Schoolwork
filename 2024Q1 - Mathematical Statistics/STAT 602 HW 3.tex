\documentclass[12pt,letterpaper]{exam}
\usepackage[utf8]{inputenc}
\usepackage[T1]{fontenc}
\usepackage{alphabeta}
\usepackage[width=8.50in, height=11.00in, left=0.50in, right=0.50in, top=0.50in, bottom=0.50in]{geometry}

\usepackage{libertine}
\usepackage{multicol}
\usepackage[shortlabels]{enumitem}

\usepackage{booktabs}
\usepackage[table]{xcolor}

\usepackage{amssymb}
\usepackage{amsthm}
\usepackage{mathtools}
\usepackage{hyperref}

\usepackage{graphicx}
%\usepackage{wrapfig}
%\usepackage{capt-of}
%\usepackage{tikz}
%\usepackage{pgfplots}
%\usetikzlibrary{shapes,arrows,positioning,patterns}
%\usepackage{pythonhighlight}

%%%%%%%%%%%%%%%%%%%%%%%%%%%%%%%%%%%%%%%%%%%%%%%%%%%%%%%%%%%%%%%%%%
\newcommand{\class}{STAT 602} % This is the name of the course
\newcommand{\assignmentname}{Chapter 7 Problems} %
\newcommand{\authorname}{Maj Isaiah Warnke} %
\newcommand{\workdate}{January 2024} %
\printanswers % this includes the solutions sections
%%%%%%%%%%%%%%%%%%%%%%%%%%%%%%%%%%%%%%%%%%%%%%%%%%%%%%%%%%%%%%%%%%

\begin{document}
\pagestyle{plain}
\thispagestyle{empty}
\noindent

%%%%%%%%%%%%%%%%%%%%%%%%%%%%%%%%%%%%%%%%%%%%%%%%%%%%%%%%%%%%%%%%%%%%%%%%%%%%%%%%%%%
\noindent
\begin{tabular*}{\textwidth}{l @{\extracolsep{\fill}} r @{\extracolsep{10pt}} l}
	\textbf{\class} & \textbf{\authorname}  &\\ %Your name here instead, obviously
	\textbf{\assignmentname } & \textbf{\workdate} & \\
\end{tabular*}\\
\rule{\textwidth}{2pt}
%%%%%%%%%%%%%%%%%%%%%%%%%%%%%%%% HEADER %%%%%%%%%%%%%%%%%%%%%%%%%%%%%%%%%%%%%%%%%%%

\begin{questions}

\question One observation is taken on a discrete random variable $X$ with pmf $f(x|\theta)$, where $\theta \in \{1,2,3\}$. Find the MLE of $\theta$.
	$$\begin{array}{cccc}
	\hline
	x & f(x|1) & f(x|2) & f(x|3) \\
	\hline
	0 & 1/3 & 1/4 & 0 \\
	1 & 1/3 & 1/4 & 0 \\
	2 & 0 & 1/4 & 1/4 \\
	3 & 1/6 & 1/4 & 1/2 \\
	4 & 1/6 & 0 &1/4 \\
	\hline
	\end{array}$$

	\begin{solution}
		Here is the solution
	\end{solution}

\renewcommand{\thequestion}{6}
\question Let $X_1,...,X_n$ be a random sample from the pdf $$f(x|\theta) = \theta x^{-2}, \quad 0 < \theta \le x < \infty.$$
	\begin{parts}
		\part What is a sufficient statistic for $\theta$?
		\part Find the MLE of $\theta$.
		\part Find the method of moments estimator of $\theta$.
	\end{parts}

	\begin{solution}
		Here is the next solution
	\end{solution}

\renewcommand{\thequestion}{9}
\question Let $X_1,...,X_n$ be iid with pdf $$f(x|\theta) = \frac{1}{\theta}, \quad 0 < x < \theta, \quad \theta > 0.$$
Estimate $\theta$ using both the method of moments and maximum likelihood. Calculate the means and variances of the two estimators. Which one should be preferred and why?
	\begin{solution}
		Here is the next solution
	\end{solution}

\renewcommand{\thequestion}{10}
\question The independent random variables $X_1,...,X_n$ have the common distribution
$$P(X_i \le x | \alpha, \beta) =
\begin{cases}
0 & , x< 0\\
(x/\beta)^\alpha & , 0 \le x \le \beta \\
1 & ,x > \beta
\end{cases}$$
where the parameters $\alpha$ and $\beta$ are positive.

	\begin{parts}
		\part Find a two-dimensional sufficient statistic for $(\alpha, \beta)$.
		\part Find the MLEs fo $\alpha$ and $\beta$.
		\part The length (in millimeters) of cuckoos' eggs found in hedge sparrow nests can be modeled with this distribution. For the data
		$$22.0, 23.9, 20.9, 23.8, 25.0, 24.0, 21.7, 23.8, 22.8, 23.1, 23.1, 23.5, 23.0, 23.0$$
		Find the MLEs of $\alpha$ and $\beta$.
	\end{parts}

	\begin{solution}
		Here is the next solution
	\end{solution}

\renewcommand{\thequestion}{12}
\question  Let $X_1,...,X_n$ be a random sample from the population with pmf
$$P_\theta (X = x) = \theta^x (1-\theta)^{1-x}, \quad x = 0 \text{ or } 1, \quad 0 \le \theta \le \frac{1}{2}.$$
	\begin{parts}
	\part Find the method of moments estimator and MLE of $\theta$.
	\part Find the mean squared errors of each of the estimators.
	\part Which estimator is preferred? Justify your choice
	\end{parts}

	\begin{solution}
		Here is the next solution
	\end{solution}

\renewcommand{\thequestion}{38}
\question For each of the following distributions, let $X_1,...,X_n$ be a random sample. Is there a function of $\theta$, say $g(\theta)$, for which there exists an unbiased estimator whose variance attains the Cramer-Rao lower bound? If so, find it. If not, show why not.
\begin{parts}
\part $f(x|\theta) = \theta x^{\theta -1 }, \quad 0 < x < 1, \quad \theta > 0$
\part $f(x|\theta) = \frac{ \log(\theta) }{ \theta - 1} \theta^x, \quad 0 < x < 1, \quad \theta > 1$
\end{parts}
	\begin{solution}
		Here is the next solution
	\end{solution}

\renewcommand{\thequestion}{40}
\question Let $X_1,...,X_n$ be iid Bernoulli($p$). Show that the variance of $\bar{X}$ attains the Carmer-Rao Lower bound, and hence $\bar{X}$ is the best unbiased estimator of $p$.
	\begin{solution}
		Here is the next solution
	\end{solution}

\renewcommand{\thequestion}{46}
\question Let $X_1, X_2,$ and $X_3$ be a random sample of size 3 from a uniform($\theta, 2\theta$) distribution, where $\theta >0$.
\begin{parts}
\part Find the method of moments estimator of $\theta$
\part Find the MLE, $\hat{\theta}$, and find a constant $k$ such that $E_\theta (k \hat{\theta}) = \theta$
\part Which of the two estimators can be improved by using sufficiency? How?
\part Find the method of moments estimate and the MLE of $\theta$ based on the data $$1.29, 0.86, 1.33,$$ three observations of average berry sizes (in centimeters) of wine grapes.
\end{parts}
	\begin{solution}
		Here is the next solution
	\end{solution}

\renewcommand{\thequestion}{49}
\question Let $X_1,...,X_n$ be iid exponential($\lambda$).
\begin{parts}
\part Find an unbiased estimator of $\lambda$ based only on $Y = \min \{X_1,...,X_n\}$
\part Find a better estimator than the one in part (a). Prove that it is better.
\part The following data are high-stress failure times (in hours) of Kevlar/epoxy spherical vessels used in a sustained pressure environment on the space shuttle:
$$ 50.1, 70.1, 137.0, 166.9, 170.5, 152.8, 80.5, 123.5, 112.6, 148.5, 160.0, 125.4.$$
Failure tiems are often modeled with the exponential distribution. Estimate the mean failure time using the estimators from parts (a) and (b).
\end{parts}
	\begin{solution}
		Here is the next solution
	\end{solution}

\end{questions}
\end{document}
