\documentclass[12pt,letterpaper]{exam}
\usepackage[utf8]{inputenc}
\usepackage[T1]{fontenc}
\usepackage{alphabeta}
\usepackage[width=8.50in, height=11.00in, left=0.50in, right=0.50in, top=0.50in, bottom=0.50in]{geometry}

\usepackage{libertine}
\usepackage{multicol}
\usepackage[shortlabels]{enumitem}

\usepackage{booktabs}
\usepackage[table]{xcolor}

\usepackage{amssymb}
\usepackage{amsthm}
\usepackage{mathtools}
\usepackage{hyperref}

\usepackage{hyperref}
\usepackage{graphicx}
%\usepackage{wrapfig}
%\usepackage{capt-of}
%\usepackage{tikz}
%\usepackage{pgfplots}
%\usetikzlibrary{shapes,arrows,positioning,patterns}
%\usepackage{pythonhighlight}

\newcommand\chapter{8}
\renewcommand{\thequestion}{\textbf{\chapter.\arabic{question}}}
\renewcommand{\questionlabel}{\thequestion}

%%%%%%%%%%%%%%%%%%%%%%%%%%%%%%%%%%%%%%%%%%%%%%%%%%%%%%%%%%%%%%%%%%
\newcommand{\class}{STAT 602} % This is the name of the course
\newcommand{\assignmentname}{Homework \# \chapter} % 
\newcommand{\authorname}{Hosley, Brandon} % 
\newcommand{\workdate}{\today} % 
\printanswers % this includes the solutions sections
%%%%%%%%%%%%%%%%%%%%%%%%%%%%%%%%%%%%%%%%%%%%%%%%%%%%%%%%%%%%%%%%%%

\begin{document}
\pagestyle{plain}
\thispagestyle{empty}
\noindent

%%%%%%%%%%%%%%%%%%%%%%%%%%%%%%%%%%%%%%%%%%%%%%%%%%%%%%%%%%%%%%%%%%%%%%%%%%%%%%%%%%%
\noindent
\begin{tabular*}{\textwidth}{l @{\extracolsep{\fill}} r @{\extracolsep{10pt}} l}
	\textbf{\class} & \textbf{\authorname}  &\\ %Your name here instead, obviously
	\textbf{\assignmentname } & \textbf{\workdate} & \\
\end{tabular*}\\
\rule{\textwidth}{2pt}
%%%%%%%%%%%%%%%%%%%%%%%%%%%%%%%% HEADER %%%%%%%%%%%%%%%%%%%%%%%%%%%%%%%%%%%%%%%%%%%

\begin{questions}
	
	\setcounter{question}{3-1}
	
	\question 
	The independent random variabels $X_1, ..., X_n$ have the common distribution
	$$P(X_i \le x) =
	\begin{cases}
	0 & , x \le 0\\
	(x/\beta)^\alpha &, - < x < \beta \\
	1 & x \ge \beta.\\
	\end{cases}$$
	\begin{parts}
		\part In Exercise 7.10, the MLEs of $\alpha$ and $\beta$ were found. If $\alpha$ is a known constant, $\alpha_0$, find an upper confidence limit for $\beta$ with confidence coefficient .95.
		\part Use the data of Exercise 7.10 to construct an interval estimate for $\beta$. Assume that $\alpha$ is know an equal to its MLE.
	\end{parts}
	\begin{solution}
		Here is the next solution
	\end{solution}
	%%%%%%%%%%%%%%%%%%%%%%%%%%%%%%%%%%%%%%%%%%%%%%%%%%%%%%%%%%%%%
	
	\setcounter{question}{6-1}
	
	\question 
	\begin{parts}
		\part Derive a confidence interval for a binomal $p$ by inverting the LRT of $H_0: p = p_0$ versus $H_1: p \ne p_0$.
	\end{parts}

	\begin{solution}
		\begin{parts}
			\part Here is the next solution
		\end{parts}
	\end{solution}
	%%%%%%%%%%%%%%%%%%%%%%%%%%%%%%%%%%%%%%%%%%%%%%%%%%%%%%%%%%%%%
	
	\setcounter{question}{8-1}
	
	\question 
	Given a sample $X_1, ..., X_n$ from a pdf of the form $\frac{1}{\sigma}f((x-\theta)/\sigma)$, 
	list at least five different pivotal quantities.
	\begin{solution}
		Here is the next solution
	\end{solution}
	%%%%%%%%%%%%%%%%%%%%%%%%%%%%%%%%%%%%%%%%%%%%%%%%%%%%%%%%%%%%%
	
	\setcounter{question}{13-1}

	\question 
	Let $X$ be a single observation from the beta($\theta, 1$) pdf.
	\begin{parts}
		\part Let $Y = -(log X)^{-1}$. Evaluate the confidence coefficient of the set $[y/2, y]$.
	\end{parts}
	\begin{solution}
		\begin{parts}
			\part Here is the next solution
		\end{parts}
		
	\end{solution}
	%%%%%%%%%%%%%%%%%%%%%%%%%%%%%%%%%%%%%%%%%%%%%%%%%%%%%%%%%%%%%
	
	\setcounter{question}{16-1}

	\question 
	Let $X_1, ..., X_n$ be iid n($\theta, \sigma^2$), where $\sigma^2$ is known. 
	For each of the following hypotheses, write out the acceptance region of a level $\alpha$ 
	test and the $1-\alpha$ confidence interval that results from inverting the test.
	\begin{parts}
		\part 
		$H_0: \theta = \theta_0 \text{ versus } H_1: \theta \ne \theta_0$
		\part 
		$H_0: \theta \ge \theta_0 \text{ versus } H_1: \theta < \theta_0$
		\part 
		$H_0: \theta \le \theta_0 \text{ versus } H_1: \theta > \theta_0$
	\end{parts}
	\begin{solution}
		\begin{parts}
			\part
			
			
			\part
			
			
			\part
			
			
		\end{parts}
	\end{solution}

\end{questions}
\end{document}
