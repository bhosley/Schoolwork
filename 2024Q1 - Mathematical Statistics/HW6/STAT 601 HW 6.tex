\documentclass[12pt,letterpaper]{exam}
\usepackage[utf8]{inputenc}
\usepackage[T1]{fontenc}
\usepackage[width=8.50in, height=11.00in, left=0.50in, right=0.50in, top=0.50in, bottom=0.50in]{geometry}

\usepackage{libertine}
\usepackage{multicol}
\usepackage[shortlabels]{enumitem}

\usepackage{booktabs}
\usepackage[table]{xcolor}

\usepackage{amssymb}
\usepackage{amsthm}
\usepackage{mathtools}
\usepackage{bbm}

\usepackage{hyperref}
\usepackage{graphicx}
%\usepackage{wrapfig}
%\usepackage{capt-of}
%\usepackage{tikz}
%\usepackage{pgfplots}
%\usetikzlibrary{shapes,arrows,positioning,patterns}
%\usepackage{pythonhighlight}

\newcommand\chapter{6}
\renewcommand{\thequestion}{\textbf{\chapter.\arabic{question}}}
\renewcommand{\questionlabel}{\thequestion}

%%%%%%%%%%%%%%%%%%%%%%%%%%%%%%%%%%%%%%%%%%%%%%%%%%%%%%%%%%%%%%%%%%
\newcommand{\class}{STAT 601} % This is the name of the course 
\newcommand{\assignmentname}{Homework \# \chapter} % 
\newcommand{\authorname}{Hosley, Brandon} % 
\newcommand{\workdate}{\today} % 
\printanswers % this includes the solutions sections
%%%%%%%%%%%%%%%%%%%%%%%%%%%%%%%%%%%%%%%%%%%%%%%%%%%%%%%%%%%%%%%%%%



\begin{document}
\pagestyle{plain}
\thispagestyle{empty}
\noindent
 
%%%%%%%%%%%%%%%%%%%%%%%%%%%%%%%%%%%%%%%%%%%%%%%%%%%%%%%%%%%%%%%%%%%%%%%%%%%%%%%%%%%
\noindent
\begin{tabular*}{\textwidth}{l @{\extracolsep{\fill}} r @{\extracolsep{10pt}} l}
	\textbf{\class} & \textbf{\authorname}  &\\ %Your name here instead, obviously 
	\textbf{\assignmentname } & \textbf{\workdate} & \\
\end{tabular*}\\ 
\rule{\textwidth}{2pt}
%%%%%%%%%%%%%%%%%%%%%%%%%%%%%%%% HEADER %%%%%%%%%%%%%%%%%%%%%%%%%%%%%%%%%%%%%%%%%%%

\begin{questions}
	
	% 3
	\setcounter{question}{2}
	\question 
	Let \(X_1, \ldots, X_n\) be a random sample from the pdf
	\[
		f(x|\mu,\sigma) = \frac{1}{\sigma}e^{-(x-\mu)/\sigma}\,,\ \mu<x<\infty,\ 0<\sigma<\infty.
	\]
	Find a two-dimensional sufficient statistic for \((\mu, \sigma)\).
	
	\begin{solution}
		
	\end{solution}
	%%%%%%%%%%%%%%%%%%%%%%%%%%%%%%%%%%%%%%%%%%%%%%%%%%%%%%%%%%%%%
	
	% 9
	\setcounter{question}{8}
	\question 
	For each of the following distributions let \(X_1, \ldots, X_n\) be a random sample. 
	Find a minimal sufficient statistic for \(\theta\).
	\begin{parts}
		\part
		\(f(x|\theta) = \frac{1}{\sqrt{2\pi}}e^{-(x-\theta)^2/2}
		,\quad -\infty<x<\infty ,\quad -\infty<\theta<\infty \hfill\text{(normal)}\)
		\part
		\(f(x|\theta) = 
		,\quad \theta<x<\infty ,\quad -\infty<\theta<\infty \hfill\text{(location exponential)}\)
		\part
		\(f(x|\theta) = 
		,\quad -\infty<x<\infty ,\quad -\infty<x<\infty \hfill\text{(logistic)}\)
		\part
		\(f(x|\theta) = 
		,\quad -\infty<x<\infty ,\quad -\infty<x<\infty \hfill\text{(Cauchy)}\)
		\part
		\(f(x|\theta) = 
		,\quad -\infty<x<\infty ,\quad -\infty<x<\infty \hfill\text{(double exponential)}\)
	\end{parts}

	\begin{solution}
		\begin{parts}
			\part
			\part
			\part
			\part
			\part
		\end{parts}
	\end{solution}
	%%%%%%%%%%%%%%%%%%%%%%%%%%%%%%%%%%%%%%%%%%%%%%%%%%%%%%%%%%%%%
	
	% 12
	\setcounter{question}{11}
	\question 
	A natural ancillary statistic in most problems is the sample size. 
	For example, let \(N\) be a random variable taking values \(1, 2,\ldots\) 
	with known probabilities \(p_1, p_2, \ldots,\) where \(\sum p_i=1\).
	Having observed \(N = n\), perform \(n\) Bernoulli trials with success probability
	\(\theta\), getting \(X\) successes.
	\begin{parts}
		\part
		Prove that the pair (X, N) is minimal sufficient and N is ancillary for θ. (Note
		the similarity to some of the hierarchical models discussed in Section 4.4.)
		\part
		Prove that the estimator \(X/N\) is unbiased for \(\theta\) 
		and has variance \(\theta(1-\theta)E(1/N)\).
	\end{parts}
	
	\begin{solution}
		\begin{parts}
			\part
			\part
		\end{parts}
	\end{solution}
	%%%%%%%%%%%%%%%%%%%%%%%%%%%%%%%%%%%%%%%%%%%%%%%%%%%%%%%%%%%%%
	
	% 17
	\setcounter{question}{16}
	\question 
	
	\begin{solution}
		
	\end{solution}
	%%%%%%%%%%%%%%%%%%%%%%%%%%%%%%%%%%%%%%%%%%%%%%%%%%%%%%%%%%%%%
	
	% 20
	\setcounter{question}{19}
	\question 
	
	\begin{solution}
		
	\end{solution}
	%%%%%%%%%%%%%%%%%%%%%%%%%%%%%%%%%%%%%%%%%%%%%%%%%%%%%%%%%%%%%
	
	% 23
	\setcounter{question}{22}
	\question 
	
	\begin{solution}
		
	\end{solution}
	%%%%%%%%%%%%%%%%%%%%%%%%%%%%%%%%%%%%%%%%%%%%%%%%%%%%%%%%%%%%%

\end{questions}
\end{document}
