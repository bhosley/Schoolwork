\documentclass[12pt,letterpaper]{exam}
\usepackage[utf8]{inputenc}
\usepackage[T1]{fontenc}
\usepackage[width=8.50in, height=11.00in, left=0.50in, right=0.50in, top=0.50in, bottom=0.50in]{geometry}

\usepackage{libertine}
\usepackage{multicol}
\usepackage[shortlabels]{enumitem}

\usepackage{booktabs}
\usepackage[table]{xcolor}

\usepackage{amssymb}
\usepackage{amsthm}
\usepackage{mathtools}
\usepackage{bbm}

\usepackage{hyperref}
\usepackage{graphicx}
%\usepackage{wrapfig}
%\usepackage{capt-of}
%\usepackage{tikz}
%\usepackage{pgfplots}
%\usetikzlibrary{shapes,arrows,positioning,patterns}
%\usepackage{pythonhighlight}

\newcommand\chapter{ 2 }
%\renewcommand{\thequestion}{\textbf{\chapter.\arabic{question}}}
%\renewcommand{\questionlabel}{\thequestion}

%%%%%%%%%%%%%%%%%%%%%%%%%%%%%%%%%%%%%%%%%%%%%%%%%%%%%%%%%%%%%%%%%%
\newcommand{\class}{  } % This is the name of the course 
\newcommand{\assignmentname}{Quiz \chapter Redux} % 
\newcommand{\authorname}{Hosley, Brandon} % 
\newcommand{\workdate}{\today} % 
\printanswers % this includes the solutions sections
%%%%%%%%%%%%%%%%%%%%%%%%%%%%%%%%%%%%%%%%%%%%%%%%%%%%%%%%%%%%%%%%%%



\begin{document}
\pagestyle{plain}
\thispagestyle{empty}
\noindent
 
%%%%%%%%%%%%%%%%%%%%%%%%%%%%%%%%%%%%%%%%%%%%%%%%%%%%%%%%%%%%%%%%%%%%%%%%%%%%%%%%%%%
\noindent
\begin{tabular*}{\textwidth}{l @{\extracolsep{\fill}} r @{\extracolsep{10pt}} l}
	\textbf{\class} & \textbf{\authorname}  &\\ %Your name here instead, obviously 
	\textbf{\assignmentname } & \textbf{\workdate} & \\
\end{tabular*}\\ 
\rule{\textwidth}{2pt}
%%%%%%%%%%%%%%%%%%%%%%%%%%%%%%%% HEADER %%%%%%%%%%%%%%%%%%%%%%%%%%%%%%%%%%%%%%%%%%%

\begin{questions}
	\question[40] 
	Let \(X_1,X_2,\ldots,X_5\) be a random sample from a \(Ber(p)\) distribution.
	Let \(H_0:p = \frac{1}{2}\) and \(H_1:p = \frac{1}{4}\).
	\begin{parts}
		\part[15]
		Show that the test \(Y=\sum_{i=1}^{5}X_i\leq1\) is UMP of size \(\alpha=\frac{3}{16}\).
			
		\part[15]
		Give the expression for the \textit{power function} of the test in part a.
		What is the \textit{power} of this test?
		
		\part[10]
		Is the test in part a) UMP level \(\alpha\) for the hypotheses \(H_0p\geq\frac{1}{2}\)
		versus \(H_1:p<\frac{1}{2}\)? Justify your answer.
		
	\end{parts}
	\begin{solution}
		\begin{parts}
			\part
			First, we note that the sum of Bernoulli trials is a binomial distribution. Then,
			
			\begin{align*}
				\beta(\theta)
				= P_\theta(Y\leq1)
				&= \binom{5}{0} (1-p)^5 + \binom{5}{1} p(1-p)^4 \\
				&= (1-p)^5 + 5 p(1-p)^4 \\
				&= (1-p)^4 (1+4p),
			\end{align*}
			in general. And in this case,
			\begin{align*}
				\beta(\theta=p_0)
				= \left(\frac{1}{4}\right)^4(3)
				= \frac{3}{16}
			\end{align*}
			and thus we conclude that \(Y\) is a test of size \(\alpha=\frac{3}{16}\)
			
			Next, we need to prove that it is also UMP.
			
			
			
			\part
			The power function is,
			\begin{align*}
				\beta(\theta)
				= 1 - P_\theta(Y\leq k)
				= \sum_{}^{} \binom{5}{y} \theta^{y} (1-\theta)^{5-y}
			\end{align*}
			
			Need to specify the values of y
			
			need to compute the value of the power
			
			\part
			
			Can use Karlin-Ruben; but will need to demonstrate the MLR property.
			
		\end{parts}
	\end{solution}
	%%%%%%%%%%%%%%%%%%%%%%%%%%%%%%%%%%%%%%%%%%%%%%%%%%%%%%%%%%%%%
	
\end{questions}
\end{document}
