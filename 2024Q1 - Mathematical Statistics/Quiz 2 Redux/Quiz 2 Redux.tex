\documentclass[12pt,letterpaper]{exam}
\usepackage[utf8]{inputenc}
\usepackage[T1]{fontenc}
\usepackage[width=8.50in, height=11.00in, left=0.50in, right=0.50in, top=0.50in, bottom=0.50in]{geometry}

\usepackage{libertine}
\usepackage{multicol}
\usepackage[shortlabels]{enumitem}

\usepackage{booktabs}
\usepackage[table]{xcolor}

\usepackage{amssymb}
\usepackage{amsthm}
\usepackage{mathtools}
\usepackage{bbm}

\usepackage{hyperref}
\usepackage{graphicx}
%\usepackage{wrapfig}
%\usepackage{capt-of}
%\usepackage{tikz}
%\usepackage{pgfplots}
%\usetikzlibrary{shapes,arrows,positioning,patterns}
%\usepackage{pythonhighlight}

\newcommand\chapter{ 2 }
%\renewcommand{\thequestion}{\textbf{\chapter.\arabic{question}}}
%\renewcommand{\questionlabel}{\thequestion}

%%%%%%%%%%%%%%%%%%%%%%%%%%%%%%%%%%%%%%%%%%%%%%%%%%%%%%%%%%%%%%%%%%
\newcommand{\class}{  } % This is the name of the course 
\newcommand{\assignmentname}{Quiz \chapter Redux} % 
\newcommand{\authorname}{Hosley, Brandon} % 
\newcommand{\workdate}{\today} % 
\printanswers % this includes the solutions sections
%%%%%%%%%%%%%%%%%%%%%%%%%%%%%%%%%%%%%%%%%%%%%%%%%%%%%%%%%%%%%%%%%%



\begin{document}
\pagestyle{plain}
\thispagestyle{empty}
\noindent
 
%%%%%%%%%%%%%%%%%%%%%%%%%%%%%%%%%%%%%%%%%%%%%%%%%%%%%%%%%%%%%%%%%%%%%%%%%%%%%%%%%%%
\noindent
\begin{tabular*}{\textwidth}{l @{\extracolsep{\fill}} r @{\extracolsep{10pt}} l}
	\textbf{\class} & \textbf{\authorname}  &\\ %Your name here instead, obviously 
	\textbf{\assignmentname } & \textbf{\workdate} & \\
\end{tabular*}\\ 
\rule{\textwidth}{2pt}
%%%%%%%%%%%%%%%%%%%%%%%%%%%%%%%% HEADER %%%%%%%%%%%%%%%%%%%%%%%%%%%%%%%%%%%%%%%%%%%

\begin{questions}
	\question[40] 
	Let \(X_1,X_2,\ldots,X_5\) be a random sample from a \(Ber(p)\) distribution.
	Let \(H_0:p = \frac{1}{2}\) and \(H_1:p = \frac{1}{4}\).
	\begin{parts}
		\part[15]
		Show that the test \(Y=\sum_{i=1}^{5}X_i\leq1\) is UMP of size \(\alpha=\frac{3}{16}\).
			
		\part[15]
		Give the expression for the \textit{power function} of the test in part a.
		What is the \textit{power} of this test?
		
		\part[10]
		Is the test in part a) UMP level \(\alpha\) for the hypotheses \(H_0:p\geq\frac{1}{2}\)
		versus \(H_1:p<\frac{1}{2}\)? Justify your answer.
		
	\end{parts}
	\begin{solution}
		\begin{parts}
			\part
			First, we note that the sum of Bernoulli trials is a binomial distribution. Then,
			%		
			\begin{align*}
				\alpha
				= P_{\theta_0}(Y\leq1)
				&= \binom{5}{0} (1-p)^5 + \binom{5}{1} p(1-p)^4 \\
				&= (1-p)^5 + 5 p(1-p)^4 \\
				&= (1-p)^4 (1+4p),
			\end{align*}
			%
			in general. And in this case,
			%
			\begin{align*}
				P_{}\left(Y\leq1\left|p=\frac{1}{2}\right.\right)
				= \left(\frac{1}{2}\right)^4(3)
				= \frac{3}{16}
			\end{align*}
			%
			and thus we conclude that \(Y\) is a test of size \(\alpha=\frac{3}{16}\)
			Next, we look at the likelihood ratio;
			%
			\begin{align*}
				\prod_{i=1}^{5}\frac{\left(\frac{1}{4}\right)^{x_i} \left(\frac{3}{4}\right)^{1-x_i}}{\left(\frac{1}{2}\right)^{x_i} \left(\frac{1}{2}\right)^{1-x_i}} &\leq k  \\
				%
				\prod_{i=1}^{5}\frac{\left(\frac{1}{4}\right)^{x_i} \left(\frac{3}{4}\right)^{-x_i} \left(\frac{3}{4}\right)^{}}{ \left(\frac{1}{2}\right)^{}} &\leq k  \\
				%
				\prod_{i=1}^{5} \left(\frac{1}{3}\right)^{x_i} \left(\frac{3}{2}\right)^{} &\leq k  \\
				%
				\prod_{i=1}^{5} \left(\frac{1}{3}\right)^{x_i} &< k \left(\frac{2}{3}\right)^{5} \\
				%
				\ln\left(\frac{1}{3}\right)\sum_{i=1}^{5}x_i &< \ln k + 5\ln\left(\frac{2}{3}\right)^{} \\
				%
				\sum_{i=1}^{5}x_i &\geq k^*
			\end{align*}
			%
			We showed in the first part that this \(k^*=1\) for a level \(\alpha\) test.
			And by the Neyman-Pearson Lemma we conclude that \(Y=\sum_{i=1}^{5}X_i\leq1\) is UMP for this level.
			
			\part
			The power function is,
			\begin{align*}
				\beta(\theta)
				= P_{\theta_1}(Y> k)
				= 1 - P_{\theta_1}(Y\leq k)
				= 1 - \left(\frac{3}{4}\right)^4 (2)
				= 1 - \left(\frac{3}{4}\right)^4 (2)
			\end{align*}
			%
			In this case we can leverage the general expression in part a to help calculate,
			the power of this test is;
			%
			\begin{align*}
				1 - \sum_{y=0}^{1} \binom{5}{y} (1/4)^{y} (3/4)^{5-y}
				%
				&= 1 - (3/4)^{4} (2) \\
				%
				&= 1 - \frac{3^4}{2^7} \\
				&= 1 - \frac{81}{128} \\
				&= \frac{47}{128}
			\end{align*}
			
			\part
			
			Using the Karlin-Rubin theorem, given that \(Y\) is sufficient for \(p\) then if it has an MLR
			we can conclude that this test is UMP for the new set of hypotheses as well.
			\begin{align*}
				\frac{\mathcal{L}(\theta_2|x)}{\mathcal{L}(\theta_1|x)}
				%
				&= \prod_{i=1}^{5}\frac{\left(\theta_2\right)^{x_i} \left(1-\theta_2\right)^{1-x_i}}{\left(\theta_1\right)^{x_i} \left(1-\theta_1\right)^{1-x_i}} \\
				%
				&= \frac{\left(\theta_2\right)^{\sum_{i=1}^{5} x_i} \left(1-\theta_2\right)^{\sum_{i=1}^{5}(1-x_i)}} {\left(\theta_1\right)^{\sum_{i=1}^{5}x_i} \left(1-\theta_1\right)^{\sum_{i=1}^{5}(1-x_i)}} \\
				%
				&= \ln(\theta_2){\sum_{i=1}^{5} x_i} + \ln\left(1-\theta_2\right){\sum_{i=1}^{5}(1-x_i)} \\
				&\quad- \ln\left(\theta_1\right){\sum_{i=1}^{5}x_i} - \ln\left(1-\theta_1\right){\sum_{i=1}^{5}(1-x_i)} \\
				%
				&= \ln\left(\frac{\theta_2}{\theta_1}\right){\sum_{i=1}^{5} x_i} + \ln\left(\frac{1-\theta_2}{1-\theta_1}\right){\sum_{i=1}^{5}(1-x_i)} \\
				%
				&= {\sum_{i=1}^{5}x_i} \left(\ln\frac{\theta_2}{\theta_1} - \ln\left(\frac{1-\theta_2}{1-\theta_1}\right)\right) + 5\ln\left(\frac{1-\theta_2}{1-\theta_1}\right) \\
			\end{align*}
			When \(\theta_2>\theta_1\) then \(\ln(\theta_2/\theta_1) > 0\) and \(\ln(1-\theta_2\ /\ 1-\theta_1) < 0\)
			thus this function has an MLR. And further we conclude that this test is also UMP level \(\alpha\) 
			for the hypotheses \(H_0:p\geq\frac{1}{2}\)	versus \(H_1:p<\frac{1}{2}\).
			
		\end{parts}
	\end{solution}
	%%%%%%%%%%%%%%%%%%%%%%%%%%%%%%%%%%%%%%%%%%%%%%%%%%%%%%%%%%%%%
	
\end{questions}
\end{document}
