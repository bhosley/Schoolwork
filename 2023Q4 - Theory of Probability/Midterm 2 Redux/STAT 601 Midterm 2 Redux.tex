\documentclass[12pt,letterpaper]{exam}
\usepackage[utf8]{inputenc}
\usepackage[T1]{fontenc}
\usepackage[width=8.50in, height=11.00in, left=0.50in, right=0.50in, top=0.50in, bottom=0.50in]{geometry}

\usepackage{libertine}
\usepackage{multicol}
\usepackage[shortlabels]{enumitem}

\usepackage{booktabs}
\usepackage[table]{xcolor}

\usepackage{amssymb}
\usepackage{amsthm}
\usepackage{mathtools}
\usepackage{bbm}

\usepackage{hyperref}
\usepackage{graphicx}
%\usepackage{wrapfig}
%\usepackage{capt-of}
%\usepackage{tikz}
%\usepackage{pgfplots}
%\usetikzlibrary{shapes,arrows,positioning,patterns}
%\usepackage{pythonhighlight}

%\newcommand\chapter{5}
%\renewcommand{\thequestion}{\textbf{\chapter.\arabic{question}}}
\renewcommand{\questionlabel}{\textbf{\thequestion}.}

%%%%%%%%%%%%%%%%%%%%%%%%%%%%%%%%%%%%%%%%%%%%%%%%%%%%%%%%%%%%%%%%%%
\newcommand{\class}{STAT 601} % This is the name of the course 
\newcommand{\assignmentname}{Midterm 2 Redux} % 
\newcommand{\authorname}{Hosley, Brandon} % 
\newcommand{\workdate}{\today} % 
\printanswers % this includes the solutions sections
%%%%%%%%%%%%%%%%%%%%%%%%%%%%%%%%%%%%%%%%%%%%%%%%%%%%%%%%%%%%%%%%%%



\begin{document}
\pagestyle{plain}
\thispagestyle{empty}
\noindent
 
%%%%%%%%%%%%%%%%%%%%%%%%%%%%%%%%%%%%%%%%%%%%%%%%%%%%%%%%%%%%%%%%%%%%%%%%%%%%%%%%%%%
\noindent
\begin{tabular*}{\textwidth}{l @{\extracolsep{\fill}} r @{\extracolsep{10pt}} l}
	\textbf{\class} & \textbf{\authorname}  &\\ %Your name here instead, obviously 
	\textbf{\assignmentname } & \textbf{\workdate} & \\
\end{tabular*}\\ 
\rule{\textwidth}{2pt}
%%%%%%%%%%%%%%%%%%%%%%%%%%%%%%%% HEADER %%%%%%%%%%%%%%%%%%%%%%%%%%%%%%%%%%%%%%%%%%%

\begin{questions}

	\setcounter{question}{0}
	\question 
	\textbf{(30 points)}  \(f_{Y|X}(y|x) = \frac{2y}{x^2}\mathbbm{1}_{(0,x)}(y) \) and \(f_{X}(x) = 5x^4\mathbbm{1}_{(0,1)}(x) \)
	
	\begin{parts}
	\setcounter{partno}{3}
		\part Given \(X\) is less than \(\frac{1}{2}\), what is the probability that \(Y\) is less than \(\frac{1}{4}\)?
	\end{parts}
	
	

	\begin{solution}
	
	\end{solution}
	%%%%%%%%%%%%%%%%%%%%%%%%%%%%%%%%%%%%%%%%%%%%%%%%%%%%%%%%%%%%%

	\question 
	\textbf{(30 points)}  Let both \(X_1, X_2\) be independent \(POI(\lambda)\) random variables. Define \(Y = X_1 + X_2\).
	
	\begin{parts}
		\part Derive and identify the distribution of \(Y\). Give the form of its pdf, \(f_Y(y)\)
	\end{parts}
	
	\begin{solution}
		
	\end{solution}
	%%%%%%%%%%%%%%%%%%%%%%%%%%%%%%%%%%%%%%%%%%%%%%%%%%%%%%%%%%%%%
	
	\question 
	\textbf{(30 points)} : Let \(X\) be the number obtained from a single roll of a fair 8 sided die.
	Given the value of \(X\), roll a second fair die with \(x\) sides, numbered \(1,2,\ldots,x\).
	Let \(Y\) denote the number obtained on the roll of the second die.
	The joint pmf is:
	\begin{flalign*} \qquad
		f_{XY}(x,y) = \begin{cases}
			\frac{1}{8x} & x= 1,2,\ldots,8;\ y = 1,2,\ldots,x \\
			0 & \text{otherwise.}
		\end{cases} &&
	\end{flalign*}
	
	\begin{parts}
		\part Evaluate \(P(Y>X-3)\).
	\end{parts}
	
	\begin{solution}
		
	\end{solution}
	%%%%%%%%%%%%%%%%%%%%%%%%%%%%%%%%%%%%%%%%%%%%%%%%%%%%%%%%%%%%%
		

\end{questions}
\end{document}
