\documentclass[12pt]{amsart}
\usepackage[left=0.5in, right=0.5in, bottom=0.75in, top=0.75in]{geometry}
\usepackage[english]{babel}
\usepackage[utf8x]{inputenc}
\usepackage{amsmath,amssymb,amsthm}
\usepackage{enumerate}
\usepackage{graphicx}

\usepackage[table,xcdraw,dvipsnames]{xcolor}
\usepackage{tikz}
\usepackage{pgfplots}
\usepgfplotslibrary{fillbetween}
\usepackage{booktabs}

\renewcommand{\thesection}{\arabic{section}}
\renewcommand{\thesubsection}{\arabic{section}.\arabic{subsection}}
\renewcommand{\thesubsubsection}{\quad(\alph{subsubsection})}

\begin{document}
\raggedbottom

\noindent{\large STAT 601 - Theory of Probability %
	- Homework 1 }
\hspace{\fill} {\large B. Hosley}
\bigskip


%%%%%%%%%%%%%%%%%%%%%%%
\setcounter{section}{1}
\setcounter{subsection}{4}
\subsection{} % 1.5
\textit{Approximately one-third of all human twins are identical (one-egg) and two-thirds are fraternal (two-egg) twins. 
	Identical twins are necessarily the same sex, with male and female being equally likely. 
	Among fraternal twins, approximately one-fourth are both female, one-fourth are both male, andhalf are one male andone female. 
	Finally, among all U.S. births, approximately 1 in 90 is a twin birth. Define the following events:}
	\begin{align*}
		A &= \{\text{a U.S. birth results in twin females}\} \\
		B &= \{\text{a U.S. birth results in identical twins}\} \\
		C &= \{\text{a U.S. birth results in twins}\}
	\end{align*}
	\subsubsection{} State, in words, the event \(A\cap B\cap C\).
	\subsubsection{} Find \(P(A\cap B\cap C)\).


\subsection{} % 1.6
\textit{Two pennies, one with \(P\)(head)\( = u\) and one with \(P\)(head) \(= w\), are to be tossed together independently. Define}
\begin{align*}
	p_0 &= P(\text{0 heads occurs}), \\
	p_1 &= P(\text{1 heads occurs}), \\
	p_2 &= P(\text{2 heads occurs}). \\
\end{align*}
\textit{Can \(u\) and \(w\) be chosen such that \(p_0 = p_1 = p_2\)? Prove your answer.}

\setcounter{subsection}{32}
\subsection{} % 1.33
\textit{Suppose that 5\% of men and .25\% of women are color-blind. A person is chosen at random and that person is color-blind. 
	What is the probability that the person is male? (Assume males and females to be in equal numbers.)}

\setcounter{subsection}{34}
\subsection{} % 1.35
\textit{Prove that if \(P(\cdot)\) is a legitimate probability function and \(B\) is a set with \(P (B) > 0\),
	then \(P (\cdot|B)\) also satisfies Kolmogorov’s Axioms.}

\setcounter{subsection}{36}
\subsection{} % 1.37
\textit{Here we look at some variations of Example 1.3.4}
	\subsubsection{}
	\textit{ In the warden’s calculation of Example 1.3.4 it was assumed that if \(A\) were to be
		pardoned, then with equal probability the warden would tell A that either \(B\) or \(C\)
		would die. However, this need not be the case. The warden can assign probabilities
		\(\gamma\) and \(1 − \gamma\) to these events, as shown here:} \\
	\begin{center}
		\begin{tabular}{ccc}
			\toprule
			Prisoner pardoned & Warden tells A & \\
			\midrule
			A & B dies & with probability \(\gamma\) \\
			A & C dies & with probability \(1-\gamma\) \\
			B & C dies & \\
			C & B dies & \\
			\bottomrule
		\end{tabular} \\[1.5em]
	\end{center}
	\textit{Calculate \(P(A|W)\) as a function of \(\gamma\). For what values of \(\gamma\) is \(P(A|W)\) less than,
		equal to, or greater than \(\frac 1 3\)?}
	\subsubsection{}
	\textit{Suppose again that \(\gamma = \frac 1 2\), as in the example. After the warden tells \(A\) that \(B\)
		will die, \(A\) thinks for a while and realizes that his original calculation was false.
		However, \(A\) then gets a bright idea. \(A\) asks the warden if he can swap fates with \(C\).
		The warden, thinking that no information has been passed, agrees to this. Prove
		that \(A\)’s reasoning is now correct and that his probability of survival has jumped to \(\frac 2 3\)!} \\
	
\textit{A similar, but somewhat more complicated, problem, the “Monte Hall problem” is discussed by Selvin (1975). 
	The problem in this guise gained a fair amount of notoriety when it appearedin a Sunday magazine (vos Savant 1990) along with a correct
	answer but with questionable explanation. The ensuing debate was even reported on the front page of the Sunday New York Times (Tierney 1991). 
	A complete andsomewhat amusing treatment is given by Morgan et al. (1991) [see also the response by vos Savant 1991]. 
	Chun (1999) pretty much exhausts the problem with a very thorough analysis.}

\subsection{} % 1.38
\textit{Prove each of the following statements. (Assume that any conditioning event has positive probability.)}
	\subsubsection{} \textit{If \(P(B) = 1\), then \(P(A|B) = P(A)\) for any \(A\).}
	\subsubsection{} \textit{If \(A \subset B\), then \(P(B|A) = 1\) and \(P(A|B) = P(A)/P(B)\).}
	\subsubsection{} \textit{If \(A\) and \(B\) are mutually exclusive, then}
		\[ P(A|A\cup B) = \frac{P(A)}{P(A) + P(B)}. \]
	\subsubsection{} \(P(A\cap B\cap C) = P(A|B\cap C) P(B|C) P(C)\).

\subsection{} % 1.39
\textit{A pair of events \(A\) and \(B\) cannot be simultaneously mutually exclusive and independent.
	Prove that if \(P(A) > 0\) and \(P(B) > 0\), then:}
	\subsubsection{} \textit{If \(A\) and \(B\) are mutually exclusive, they cannot be independent.}
	\subsubsection{} \textit{If \(A\) and \(B\) are independent, they cannot be mutually exclusive.}

\setcounter{subsection}{46}
\subsection{} % 1.47
\textit{Prove that the following functions are cdfs.}
	\subsubsection{} \( \frac{1}{2}+\frac{1}{\pi} \tan^{-1}(x),\, x\in(-\infty,\infty) \)
	\subsubsection{} \( (1+e^{-x})^{-1}\, x\in(-\infty,\infty) \)
	\subsubsection{} \( e^{-e^{-x}}\, x\in(-\infty,\infty) \)
	\subsubsection{} \( 1-e^{-x}\, x\in(-\infty,\infty) \)
	\subsubsection{} \textit{the function defined in (1.5.6)}

\setcounter{subsection}{50}
\subsection{} % 1.51
\textit{An appliance store receives a shipment of 30 microwave ovens, 5 of which are (unknown
	to the manager) defective. The store manager selects 4 ovens at random, without
	replacement, and tests to see if they are defective. Let \(X =\) number of defectives
	found. Calculate the pmf and cdf of \(X\) and plot the cdf.}


\subsection{} % 1.52
\textit{Let \(X\) be a continuous random variable with pdf \(f(x)\) and cdf \(F(x)\). For a fixed number \(x_0\), define the function
	\[g(x) = \begin{cases}
		f(x)/[1 − F(x_0)] & x\geq x_0 \\
		0 & x < x_0.
	\end{cases} \]
	Prove that \(g(x)\) is a pdf. (Assume that \(F(x0) < 1\).)}


\subsection{} % 1.53
\textit{A certain river floods every year. Suppose that the low-water mark is set at \(1\) and the
	high-water mark \(Y\) has distribution function}
	\[ F_Y = P(Y\leq y) = 1-\frac{1}{y^2}, \quad 1\leq y<\infty. \]
	\subsubsection{} Verify \(F_Y(y)\) is a cdf.
	
	\subsubsection{} Find \(f_Y(y)\), the pdf of \(Y\).
	
	\subsubsection{} If the low-water mark is reset at \(0\) and we use a unit of measurement that is $\frac{1}{10}$ of
		that given previously, the high-water mark becomes \(Z = 10(Y − 1)\). Find \(F_Z(z)\).
		


\subsection{} % 1.54
\textit{For each of the following, determine the value of $c$ that makes $f(x)$ a pdf.}
	\subsubsection{}
	\( f(x) = c\sin x,\, 0<x<\pi/2 \)
	
	\subsubsection{}
	\( f(x) = ce^{-x|x|},\, -\infty<x<\infty \)
	


\subsection{} % 1.55
\textit{An electronic device has lifetime denoted by $T$. The device has value $V = 5$ if it fails
	before time $t = 3$; otherwise, it has value $V = 2T$. Find the cdf of $V$, if $T$ has pdf}
	\[ f_T(t) = \frac{1}{1.5}e^{-t/(1.5)},\quad t>0.\]


\end{document}