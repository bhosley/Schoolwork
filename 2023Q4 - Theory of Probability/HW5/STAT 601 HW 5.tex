\documentclass[12pt,letterpaper]{exam}
\usepackage[utf8]{inputenc}
\usepackage[T1]{fontenc}
\usepackage[width=8.50in, height=11.00in, left=0.50in, right=0.50in, top=0.50in, bottom=0.50in]{geometry}

\usepackage{libertine}
\usepackage{multicol}
\usepackage[shortlabels]{enumitem}

\usepackage{booktabs}
\usepackage[table]{xcolor}

\usepackage{amssymb}
\usepackage{amsthm}
\usepackage{mathtools}
\usepackage{bbm}

\usepackage{hyperref}
\usepackage{graphicx}
%\usepackage{wrapfig}
%\usepackage{capt-of}
%\usepackage{tikz}
%\usepackage{pgfplots}
%\usetikzlibrary{shapes,arrows,positioning,patterns}
%\usepackage{pythonhighlight}

\newcommand\chapter{5}
\renewcommand{\thequestion}{\textbf{\chapter.\arabic{question}}}
\renewcommand{\questionlabel}{\thequestion}

%%%%%%%%%%%%%%%%%%%%%%%%%%%%%%%%%%%%%%%%%%%%%%%%%%%%%%%%%%%%%%%%%%
\newcommand{\class}{STAT 601} % This is the name of the course 
\newcommand{\assignmentname}{Homework \# \chapter} % 
\newcommand{\authorname}{Hosley, Brandon} % 
\newcommand{\workdate}{\today} % 
\printanswers % this includes the solutions sections
%%%%%%%%%%%%%%%%%%%%%%%%%%%%%%%%%%%%%%%%%%%%%%%%%%%%%%%%%%%%%%%%%%



\begin{document}
\pagestyle{plain}
\thispagestyle{empty}
\noindent
 
%%%%%%%%%%%%%%%%%%%%%%%%%%%%%%%%%%%%%%%%%%%%%%%%%%%%%%%%%%%%%%%%%%%%%%%%%%%%%%%%%%%
\noindent
\begin{tabular*}{\textwidth}{l @{\extracolsep{\fill}} r @{\extracolsep{10pt}} l}
	\textbf{\class} & \textbf{\authorname}  &\\ %Your name here instead, obviously 
	\textbf{\assignmentname } & \textbf{\workdate} & \\
\end{tabular*}\\ 
\rule{\textwidth}{2pt}
%%%%%%%%%%%%%%%%%%%%%%%%%%%%%%%% HEADER %%%%%%%%%%%%%%%%%%%%%%%%%%%%%%%%%%%%%%%%%%%

\begin{questions}
	
	%5.3
	\setcounter{question}{2}
	\question 
	Let \(X_1,\ldots, X_n\) be iid random variables with continuous cdf \(F_X\), 
	and suppose \(E\,X_i = \mu\).
	Define the random variables \(Y_1,\ldots, Y_n\) by
	
	\[ 
	Y_i = 
	\begin{cases}
		1 & \text{if } X_i > \mu \\
		0 & \text{if } X_i \leq\mu.
	\end{cases}
	\]
	
	Find the distribution of \(\sum^{n}_{i=1}Y_i\).
	
	\begin{solution}
		The distribution can be calculated if \(F_X\) is known, the relationship between the distributions is,
		\[\sum^{n}_{i=1}Y_i \sim \text{binom}(n,p = P(X>\mu)).\]
	\end{solution}
	%%%%%%%%%%%%%%%%%%%%%%%%%%%%%%%%%%%%%%%%%%%%%%%%%%%%%%%%%%%%%
	
	%5.6
	\setcounter{question}{5}
	\question 
	If \(X\) has pdf \(f_X(x)\) and \(Y\), independent of \(X\), has pdf \(f_Y(y)\), 
	establish formulas, similar to (5.2.3), for the random variable \(Z\) 
	in each of the following situations.
	\begin{parts}
		\part \(Z = X - Y\)
		\part \(Z = XY\)
		\part \(Z = X/Y\)
	\end{parts}
	
	\begin{solution}
		
	\end{solution}
	%%%%%%%%%%%%%%%%%%%%%%%%%%%%%%%%%%%%%%%%%%%%%%%%%%%%%%%%%%%%%
	
	%5.11
	\setcounter{question}{10}
	\question 
	Suppose \(\bar{X}\) and \(S^2\) are calculated from a random sample \(X_1,\ldots, X_n\) 
	drawn from a population with finite variance \(\sigma^2\). We know that \(E\,S^2 = \sigma^2\). 
	Prove that \(E\,S \leq \sigma\), and
	if \(\sigma^2 > 0\), then \(E\,S < \sigma\).
	
	\begin{solution}
		
	\end{solution}
	%%%%%%%%%%%%%%%%%%%%%%%%%%%%%%%%%%%%%%%%%%%%%%%%%%%%%%%%%%%%%
	
	%5.15
	\setcounter{question}{14}
	\question 
	Establish the following recursion relations for means and variances. 
	Let \(\bar{X}_n\) and \(S_n^2\) be the mean and variance, respectively, of \(X_1,\ldots,X_n\). 
	Then suppose another observation, \(X_{n+1}\), becomes available. Show that
	
	\begin{parts}
		\part \(\bar{X}_{n+1} = \frac{X_{n+1}  + n\bar{X}_n}{n + 1} \).
		\part \(nS_{n+1}^2 = (n-1)S_n^2 + \left( \frac{n}{n+1} \right) (X_{n+1}-\bar{X}_n)^2\).
	\end{parts}
	
	\begin{solution}
		\begin{parts}
			\part
			\[
			\bar{X}_{n+1} = \frac{\sum_{i=1}^{n+1}X_i}{n+1} = \frac{X_{n+1} + \sum_{i=1}^{n}X_i}{n+1} = \frac{X_{n+1} + n\bar X_n}{n+1}
			\]
			
			\part
			\[
			nS_{n+1}^2 = \sum_{i=1}^{n+1} \frac{\left(X_i - \bar{X}_{n+1} \right)^2}{(n+1)-1}
			\]
			
		\end{parts}
	\end{solution}
	%%%%%%%%%%%%%%%%%%%%%%%%%%%%%%%%%%%%%%%%%%%%%%%%%%%%%%%%%%%%%
	
	%5.16
	\question 
	Let \(X_i,i = 1, 2, 3\), be independent with n\((i,i^2)\) distributions. 
	For each of the following situations, use the \(X_is\) to construct a statistic with the indicated distribution.
	
	\begin{parts}
	\part chi squared with 3 degrees of freedom
	\part \(t\) distribution with 2 degrees of freedom
	\part \(F\) distribution with 1 and 2 degrees of freedom
	\end{parts}
	
	
	\begin{solution}
		
	\end{solution}
	%%%%%%%%%%%%%%%%%%%%%%%%%%%%%%%%%%%%%%%%%%%%%%%%%%%%%%%%%%%%%
	
	%5.17
	\question 
	Let \(X\) be a random variable with an \(F_{p,q}\) distribution.
	
	\begin{parts}
		\part Derive the pdf of \(X\).
		\part Derive the mean and variance of \(X\).
		\part Show that \(1/X\) has an \(F_{q,p}\) distribution.
		\part Show that \((p/q)X/[1 + (p/q)X]\) has a beta distribution with parameters \(p/2\) and \(q/2\).
	\end{parts}
	
	\begin{solution}
		
	\end{solution}
	%%%%%%%%%%%%%%%%%%%%%%%%%%%%%%%%%%%%%%%%%%%%%%%%%%%%%%%%%%%%%
	
	%5.22
	\setcounter{question}{21}
	\question 
	Let \(X\) and \(Y\) be iid n\((0, 1)\) random variables, 
	and define \(Z = \min(X,Y)\). Prove that \(Z^2 \sim \chi^2_1 \).
	
	\begin{solution}
		
	\end{solution}
	%%%%%%%%%%%%%%%%%%%%%%%%%%%%%%%%%%%%%%%%%%%%%%%%%%%%%%%%%%%%%

	%5.24
	\setcounter{question}{23}
	\question 
	Let \(X_1,\ldots, X_n\) be a random sample from a population with pdf
	
	\[
	f_X(x)=
	\begin{cases}
		1/\theta & \text{if } 0 < x < \theta \\
		0 	& \text{otherwise}.
	\end{cases}
	\]
	
	Let \(X_{(1)} < \ldots < X_{(n)}\) be the order statistics. 
	Show that \(X_{(1)}/X_{(n)}\) and \(X_{(n)}\) are independent random variables.
	
	\begin{solution}
		
	\end{solution}
	%%%%%%%%%%%%%%%%%%%%%%%%%%%%%%%%%%%%%%%%%%%%%%%%%%%%%%%%%%%%%

	%5.31
	\setcounter{question}{30}
	\question
	Suppose \(\bar{X}\) is the mean of 100 observations from a population with mean \(\mu\) and variance \(\sigma^2 = 9\). 
	Find limits between which \(\bar{X}-\mu\) will lie with probability at least \(.90\). 
	Use both Chebychev’s Inequality and the Central Limit Theorem, and comment on each. 
	
	\begin{solution}
		
	\end{solution}
	%%%%%%%%%%%%%%%%%%%%%%%%%%%%%%%%%%%%%%%%%%%%%%%%%%%%%%%%%%%%%
		

\end{questions}
\end{document}
