\documentclass[12pt,letterpaper]{exam}
\usepackage[utf8]{inputenc}
\usepackage[T1]{fontenc}
\usepackage[width=8.50in, height=11.00in, left=0.50in, right=0.50in, top=0.50in, bottom=0.50in]{geometry}

\usepackage{libertine}
\usepackage{multicol}
\usepackage[shortlabels]{enumitem}

\usepackage{booktabs}
\usepackage[table]{xcolor}

\usepackage{amssymb}
\usepackage{amsthm}
\usepackage{mathtools}
\usepackage{bbm}

\usepackage{hyperref}
\usepackage{graphicx}
%\usepackage{wrapfig}
%\usepackage{capt-of}
%\usepackage{tikz}
%\usepackage{pgfplots}
%\usetikzlibrary{shapes,arrows,positioning,patterns}
%\usepackage{pythonhighlight}

%\newcommand\chapter{5}
%\renewcommand{\thequestion}{\textbf{\chapter.\arabic{question}}}
\renewcommand{\questionlabel}{\textbf{\thequestion}.}

%%%%%%%%%%%%%%%%%%%%%%%%%%%%%%%%%%%%%%%%%%%%%%%%%%%%%%%%%%%%%%%%%%
\newcommand{\class}{STAT 601} % This is the name of the course 
\newcommand{\assignmentname}{Midterm 1 Redux} % 
\newcommand{\authorname}{Hosley, Brandon} % 
\newcommand{\workdate}{\today} % 
\printanswers % this includes the solutions sections
%%%%%%%%%%%%%%%%%%%%%%%%%%%%%%%%%%%%%%%%%%%%%%%%%%%%%%%%%%%%%%%%%%



\begin{document}
\pagestyle{plain}
\thispagestyle{empty}
\noindent
 
%%%%%%%%%%%%%%%%%%%%%%%%%%%%%%%%%%%%%%%%%%%%%%%%%%%%%%%%%%%%%%%%%%%%%%%%%%%%%%%%%%%
\noindent
\begin{tabular*}{\textwidth}{l @{\extracolsep{\fill}} r @{\extracolsep{10pt}} l}
	\textbf{\class} & \textbf{\authorname}  &\\ %Your name here instead, obviously 
	\textbf{\assignmentname } & \textbf{\workdate} & \\
\end{tabular*}\\ 
\rule{\textwidth}{2pt}
%%%%%%%%%%%%%%%%%%%%%%%%%%%%%%%% HEADER %%%%%%%%%%%%%%%%%%%%%%%%%%%%%%%%%%%%%%%%%%%

\begin{questions}

	\setcounter{question}{0}
	\question 
	
	\textbf{(40 points) Let Y be a random variable with pdf given by:}
	
	\[f_\mathbf{Y}(y) = e^ae^{-y} I_{[a,\infty)}(y)\]
	
	\begin{parts}
		\part Find the Moment Generating Function of Y.
		\part Find E[Y] and Var[Y].
		\setcounter{partno}{3}
		\part Find the pdf and CDF of X where \(X = e^{(Y-a)}\).
	\end{parts}

	\begin{solution}
	\begin{parts}
		\part 
		\begin{align*}
			M_Y(y)
			&= \int_{-\infty}^{\infty} e^{ty}e^ae^{-y} \mathbb{I}_{[a,\infty)} \,dy \\
			&= \mathbb{I}_{[a,\infty)}e^a \int_{-\infty}^{\infty} e^{y(t-1)} \,dy \\
			&= e^a \int_{a}^{\infty} e^{y(t-1)} \,dy \\
			&= e^a \left. \frac{1}{t-1}e^{y(t-1)} \right|_a^\infty \\
			&= \left. \frac{1}{t-1}e^{y(t-1)+a} \right|_a^\infty \\
			&= 0 - \frac{1}{t-1}e^{a(t-1)+a} \\
			&= - \frac{e^{at}}{t-1} \\
		\end{align*}
		
		\part 
		\begin{align*}
			E[Y] 
			&= \frac{d}{dt} \frac{-e^{at}}{t-1} \\
			&= \left. \frac{-ae^{at}(t-1) + e^{at} }{(t-1)^2} \right|_{t=0} \\
			&=  \frac{a(1)(-1) + 0 }{(-1)^2}  \\ &= a
		\end{align*}
		
		\begin{align*}
			E[Y^2] 
			&= \frac{d^2}{dt^2} \frac{-e^{at}}{t-1} \\
			&= \frac{d}{dt} \frac{-ae^{at}(t-1) + e^{at} }{(t-1)^2} \\
			&= \frac{d}{dt} \frac{-ae^{at}}{(t-1)} + \frac{d}{dt} \frac{e^{at} }{(t-1)^2} \\
			&= \left. \frac{-a^2e^{at}(t-1) + ae^{at} }{(t-1)^2} + \frac{ae^{at}(t-1)^2 - e^{at}(2t-2) }{(t-1)^4} \right|_{t=0} \\
			&= \frac{-a^2(1)(-1) + a(1) }{1} + \frac{a(1)(-1)^2 - (1)(-2) }{1} \\
			&= a^2 + 2a + 2
		\end{align*}
		
		\begin{align*}
			\text{Var}[Y] 
			&= E[Y^2] -E[Y]^2 \\
			&= a^2 + 2a + 2 - a^2 \\
			&= 2a - 2
		\end{align*}
		
		\setcounter{partno}{3}
		\part 
		\begin{align*}
			g_X(x)
			&= P(X=x) \\
			&= P(e^{Y-a}=x) \\
			&= P(Y=\ln x +a) \\
			&= e^ae^{-(\ln x +a)} \\
			&= e^{-\ln x} \\
			&= \frac{1}{x} \\
		\end{align*}
		Thus,
		\[G_x(x) = \ln x.\]
		And the support will be
		\begin{align*}
			a <\ &y < \infty \\
			a < \ln x& +a < \infty \\
			0 < \ln\,&x < \infty \\
			1 <\ &x < \infty .
		\end{align*}
		
	\end{parts}	
	\end{solution}
	%%%%%%%%%%%%%%%%%%%%%%%%%%%%%%%%%%%%%%%%%%%%%%%%%%%%%%%%%%%%%
	\clearpage
	\setcounter{question}{2}
	\question 
	
	\textbf{(30 points) Let X be a random variable with pdf given by}
	
	\[
	f_X(x) = \begin{cases}
		\frac{x^2}{81} & \text{for } -6<x<3 \\
		0 & \text{otherwise}
	\end{cases}
	\]
	Find the pdf of \(Y=X^2\)
	
	\begin{solution}
		\begin{align*}
			g_Y(y)
			&= P(Y=y) \\
			&= P(X^2=y) \\
			&= P(X=\sqrt x) \\
			&= \frac{y}{81} \\
		\end{align*}
		On the support of
		\begin{align*}
			-6<\ &x<3 \\
			-6<\ &\sqrt y<3 \\
			(-6)^2<\ &y<(3)^2 \\
		\intertext{However, this is an invalid operation on the lower bound, thus,}
			0<\ &y<9 \\
		\end{align*}
		
	\end{solution}
%%%%%%%%%%%%%%%%%%%%%%%%%%%%%%%%%%%%%%%%%%%%%%%%%%%%%%%%%%%%%
		

\end{questions}
\end{document}
