\documentclass[12pt,letterpaper]{exam}
\usepackage[utf8]{inputenc}
\usepackage[T1]{fontenc}
\usepackage[width=8.50in, height=11.00in, left=0.50in, right=0.50in, top=0.50in, bottom=0.50in]{geometry}

\usepackage{libertine}
\usepackage{multicol}
\usepackage[shortlabels]{enumitem}

\usepackage{booktabs}
\usepackage[table]{xcolor}

\usepackage{amssymb}
\usepackage{amsthm}
\usepackage{mathtools}
\usepackage{hyperref}

\usepackage{graphicx}
%\usepackage{wrapfig}
%\usepackage{capt-of}
%\usepackage{tikz}
%\usepackage{pgfplots}
%\usetikzlibrary{shapes,arrows,positioning,patterns}
%\usepackage{pythonhighlight}

\newcommand\chapter{2}
\renewcommand{\thequestion}{\textbf{\chapter.\arabic{question}}}
\renewcommand{\questionlabel}{\thequestion}

%%%%%%%%%%%%%%%%%%%%%%%%%%%%%%%%%%%%%%%%%%%%%%%%%%%%%%%%%%%%%%%%%%
\newcommand{\class}{STAT 601} % This is the name of the course 
\newcommand{\assignmentname}{Homework \# \chapter} % 
\newcommand{\authorname}{Hosley, Brandon} % 
\newcommand{\workdate}{October 2023} % 
\printanswers % this includes the solutions sections
%%%%%%%%%%%%%%%%%%%%%%%%%%%%%%%%%%%%%%%%%%%%%%%%%%%%%%%%%%%%%%%%%%



\begin{document}
\pagestyle{plain}
\thispagestyle{empty}
\noindent
 
%%%%%%%%%%%%%%%%%%%%%%%%%%%%%%%%%%%%%%%%%%%%%%%%%%%%%%%%%%%%%%%%%%%%%%%%%%%%%%%%%%%
\noindent
\begin{tabular*}{\textwidth}{l @{\extracolsep{\fill}} r @{\extracolsep{10pt}} l}
	\textbf{\class} & \textbf{\authorname}  &\\ %Your name here instead, obviously 
	\textbf{\assignmentname } & \textbf{\workdate} & \\
\end{tabular*}\\ 
\rule{\textwidth}{2pt}
%%%%%%%%%%%%%%%%%%%%%%%%%%%%%%%% HEADER %%%%%%%%%%%%%%%%%%%%%%%%%%%%%%%%%%%%%%%%%%%

\begin{questions}

	\question In each of the following, find the bdf of $Y$. Show that the pdf integrates to 1.
	\begin{parts}
		\part $Y=X^3$ and $f_X(x) = 42 x^5 (1-x), \ 0 < x < 1$
		\part $Y= 4X + 3$ and $f_X(x) = 7e^{-7x}, \ 0 < x <\infty$
		\part $Y=X^2$ and $f_X(x) = 30 x^2 (1-x)^2, \ 0 < x < 1$
	\end{parts}
	(See Example A.0.2 in Appendix A.)
	
	\begin{solution}\\
		(a)
		(b)
		(c)
	\end{solution}
	%%%%%%%%%%%%%%%%%%%%%%%%%%%%%%%%%%%%%%%%%%%%%%%%%%%%%%%%%%%%%
	
	\setcounter{question}{2}
	\question Suppose $X$ has the geometric pmf $f_X(x) = \frac{1}{3} \left(\frac{2}{3}\right)^{x}, \ x = 0,1,2,...$. Determine the probability distribution of $Y= X/(X+1)$. Noe that here both $X$ and $Y$ are discrete random variables. To specify the probability distribution of $Y$, specify its pmf.
	
	\begin{solution}
		Here is the next solution
	\end{solution}
	%%%%%%%%%%%%%%%%%%%%%%%%%%%%%%%%%%%%%%%%%%%%%%%%%%%%%%%%%%%%%

	\question Let $\lambda$ be a fixed positive constant, and define the function $f(x)$ by $f(x) = \frac{1}{2} \lambda e^{-\lambda x}$ if $x \ge 0$ and $f(x) = \frac{1}{2} \lambda e^{\lambda x}$ if $x < 0$. 
	
	\begin{parts}
		\part Verify that $f(x)$ is a pdf. 
		\part If $X$ is a random variable with pdf given by $f(x)$, find $P(X < t)$ for all $t$. Evaluate all integrals.
		\part Find $P(|X| <t)$ for all $t$. Evaluate all integrals.
	\end{parts}
	
	\begin{solution}
		Here is the solution
	\end{solution}
	%%%%%%%%%%%%%%%%%%%%%%%%%%%%%%%%%%%%%%%%%%%%%%%%%%%%%%%%%%%%%
	
	\setcounter{question}{8}
	\question  (Casella Ex 2.9 ) If the random variable $X$ has pdf $$f(X) = \begin{cases}\frac{x-1}{2} & 1<x<3 \\ 0 & \text{otherwise} \end{cases}$$
	find a monotone function $u(x)$ such that the random variabel $Y = u(X)$ has a uniform(0,1) distribution.
	\begin{solution}
		Here is the next solution
	\end{solution}
	%%%%%%%%%%%%%%%%%%%%%%%%%%%%%%%%%%%%%%%%%%%%%%%%%%%%%%%%%%%%%	

	\setcounter{question}{11}
	\question (Casella Ex 2.12 ) A random right triangle can be constructed in teh following manner. Let $X$ be a random angle whose distribution is uniform on $(0, \pi / 2)$. For each $X$, construct a triangle as pictured below. Here, $Y =$ height of the random triangle. For a fixed constant $d$, find the distribution of $Y$ and $E(Y)$. 
	\begin{center}
	%\includegraphics[width=0.3\linewidth]{img/prob2.12}
	\end{center}
	
	\begin{solution}
		Here is the solution
	\end{solution}
	%%%%%%%%%%%%%%%%%%%%%%%%%%%%%%%%%%%%%%%%%%%%%%%%%%%%%%%%%%%%%
	
	\setcounter{question}{16}
	\question  (Casella Ex 2.17 ) A median of a distribution is a value $m$ such that $P(X \le m) \ge \frac{1}{2}$ and $P(X \ge m) \ge \frac{1}{2}$.
	If $X$ is continuous, $m$ satisfies $\int_{-\infty}^{m} f(x) dx = \int_{m}^{\infty} f(x) dx = \frac{1}{2}$. Find the median of the following distributions.
	\begin{parts}
		\part $f(x) = 3x^2 , \ 0 < x < 1$
		\part $f(x) = \frac{1}{\pi (1+ x^2)}, \ \infty < x < \infty$
	\end{parts}
	
	\begin{solution}
		Here is the next solution
	\end{solution}
	%%%%%%%%%%%%%%%%%%%%%%%%%%%%%%%%%%%%%%%%%%%%%%%%%%%%%%%%%%%%%

	\setcounter{question}{21}
	\question (Casella Ex 2.22 ) Let $X$ have the pdf $$ f(x) = \frac{4}{\beta^3\sqrt{\pi} x^2 e^{-x/\beta^2}}, 0 < x < \infty, \beta > 0.$$
	\begin{parts}
		\part Verify that $f(x)$ is a pdf.
		\part Find $E(X)$ and $Var(X)$
	\end{parts}
	
	\begin{solution}
		Here is the solution
	\end{solution}
	%%%%%%%%%%%%%%%%%%%%%%%%%%%%%%%%%%%%%%%%%%%%%%%%%%%%%%%%%%%%%
	
	\setcounter{question}{23}
	\question  (Casella Ex 2.24 ) Compute $E(X)$ and $Var(X)$ for each of the following probability distributions.
	\begin{parts}
		\part $f_X(x) = ax^{a-1}, \ 0<x<1, \ a > 0$
		\part $f_X(x) = \frac{1}{n}, \ x=1,2,..,n \ n > 0$ an integer
		\part $f_X(x) = \frac{3}{2}(x-1)^{2}, \ 0<x<2$
	\end{parts}
	
	\begin{solution}
		Here is the next solution
	\end{solution}
	%%%%%%%%%%%%%%%%%%%%%%%%%%%%%%%%%%%%%%%%%%%%%%%%%%%%%%%%%%%%%

	\question (Casella Ex 2.25 ) Suppose the pdf  $f_X(x)$ of a random variable $X$ is an even function. ($f_X(x)$ is an even function if$f_X(x) = f_X(-x)$ for every $x$.) Show that 
	\begin{parts}
		\part $X$ and $-X$ are identically distributed.
		\part $M_X(t)$ is symmetric about 0.
	\end{parts}
	
	\begin{solution}
		Here is the solution
	\end{solution}
	%%%%%%%%%%%%%%%%%%%%%%%%%%%%%%%%%%%%%%%%%%%%%%%%%%%%%%%%%%%%%
	
	\question  (Casella Ex 2.26 ) Let $f(x)$ be a pdf and let $a$ be a number such that, for all $\epsilon > 0, \ f(x+\epsilon) = f(a-\epsilon)$. Such a pdf is said to be symmetric about the point $a$.
	
	\begin{parts}
		\part Give three examples of symmetric pdfs.
		\part Show that if $X \sim f(x)$, symmetric, then the median of $X$ is the number $a$.
		\part Show that if $X \sim f(x)$, symmetric, and $E(X)$ exists, then $E(X) = a$.
		\part Show that $f(X) = e^{-x}, \ x \ge 0$, is not a symmetric pdf.
		\part Show that in the pdf in part (d), the median is less than the mean.
	\end{parts}
	\begin{solution}
		Here is the next solution
	\end{solution}
	%%%%%%%%%%%%%%%%%%%%%%%%%%%%%%%%%%%%%%%%%%%%%%%%%%%%%%%%%%%%%

	\setcounter{question}{28}
	\question (Casella Ex 2.29a,b ) To calculate moments of discrete distributions, it is often easier to work with the factorial moments (see Miscellanea 2.6.2). 
	\begin{parts}
		\part Calculate the factorial moment $E[X(X-1)]$ for the binomial and Poisson distributions.
		\part Use the results of part (a) to calculate the variances of the binomial and Poisson distributions. 
	\end{parts}
	
	\begin{solution}
		Here is the solution
	\end{solution}
	%%%%%%%%%%%%%%%%%%%%%%%%%%%%%%%%%%%%%%%%%%%%%%%%%%%%%%%%%%%%%
	
	\setcounter{question}{31}
	\question  (Casella Ex 32 ) Let $M_X(t)$ be the moment generating function of $X$, and define $S(t) = \log{(M_X(t))}$. Show that
	$$ \frac{d}{dt} \left. S(t) \right|_{t=0} = E(X) \text{ and }  \frac{d^2}{dt^2} \left. S(t) \right|_{t=0} = Var(X)$$
	\begin{solution}
		Here is the next solution
	\end{solution}
	%%%%%%%%%%%%%%%%%%%%%%%%%%%%%%%%%%%%%%%%%%%%%%%%%%%%%%%%%%%%%	
	
	\question  (Casella Ex 33 ) In each of the following cases verify the expression given for the moment generating funciton, and in each case use the mgf to calculate $E(X)$ and $Var(X)$.
	\begin{parts}
		\part $P(X=x)=\frac{e^{-\lambda} \lambda^x}{x}, \quad M_X(t)=e^{\lambda\left(e^t-1\right)}, \quad x=0,1, \ldots ; \quad \lambda>0$
		\part $P(X=x)=p(1-p)^x, \quad M_X(t)=\frac{p}{1-(1-p) e^t}, \quad x=0,1, \ldots ; \quad 0<p<1$
		\part $f_X(x)=\frac{e^{-(x-\mu)^2 /\left(2 \sigma^2\right)}}{\sqrt{2 \pi} \sigma}, M_X(t)=e^{\mu t+\sigma^2 t^2 / 2},-\infty<x<\infty ;-\infty<\mu<\infty, \  \sigma>0$
	\end{parts}
	
	\begin{solution}
		Here is the next solution
	\end{solution}
	%%%%%%%%%%%%%%%%%%%%%%%%%%%%%%%%%%%%%%%%%%%%%%%%%%%%%%%%%%%%%	
	
	\setcounter{question}{37}
	\question  (Casella Ex 38 ) Let $X$ have the negative binomial distribution with pmf
	$$ f(x) = { r + x -1 \choose x} p^r (1-p)^x, \quad x= 0,1,2,...,$$
	where $0 < p< 1$ and $r >0$ is an integer.
	\begin{parts}
		\part Calculate the mgf of $X$.
		\part Define a new random variable by $Y = 2pX$. Show that, as $p \downarrow 0$, the mgf of $Y$ converges to that of a chi squared random variable with $2r$ degrees of freedom by showing that $$\lim_{p \rightarrow 0} M_Y(t) = \left(\frac{1}{1-2t}\right)^r , \quad |t| < \frac{1}{2}.$$
	\end{parts}
	\begin{solution}
		Here is the next solution
	\end{solution}
	%%%%%%%%%%%%%%%%%%%%%%%%%%%%%%%%%%%%%%%%%%%%%%%%%%%%%%%%%%%%%			

\end{questions}
\end{document}
