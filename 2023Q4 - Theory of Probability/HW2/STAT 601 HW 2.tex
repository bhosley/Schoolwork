\documentclass[12pt,letterpaper]{exam}
\usepackage[utf8]{inputenc}
\usepackage[T1]{fontenc}
\usepackage[width=8.50in, height=11.00in, left=0.50in, right=0.50in, top=0.50in, bottom=0.50in]{geometry}

\usepackage{libertine}
\usepackage{multicol}
\usepackage[shortlabels]{enumitem}

\usepackage{booktabs}
\usepackage[table]{xcolor}

\usepackage{amssymb}
\usepackage{amsthm}
\usepackage{mathtools}
\usepackage{bbm}

\usepackage{hyperref}
\usepackage{graphicx}
%\usepackage{wrapfig}
%\usepackage{capt-of}
%\usepackage{tikz}
%\usepackage{pgfplots}
%\usetikzlibrary{shapes,arrows,positioning,patterns}
%\usepackage{pythonhighlight}

\newcommand\chapter{2}
\renewcommand{\thequestion}{\textbf{\chapter.\arabic{question}}}
\renewcommand{\questionlabel}{\thequestion}

%%%%%%%%%%%%%%%%%%%%%%%%%%%%%%%%%%%%%%%%%%%%%%%%%%%%%%%%%%%%%%%%%%
\newcommand{\class}{STAT 601} % This is the name of the course 
\newcommand{\assignmentname}{Homework \# \chapter} % 
\newcommand{\authorname}{Hosley, Brandon} % 
\newcommand{\workdate}{October 2023} % 
\printanswers % this includes the solutions sections
%%%%%%%%%%%%%%%%%%%%%%%%%%%%%%%%%%%%%%%%%%%%%%%%%%%%%%%%%%%%%%%%%%



\begin{document}
\pagestyle{plain}
\thispagestyle{empty}
\noindent
 
%%%%%%%%%%%%%%%%%%%%%%%%%%%%%%%%%%%%%%%%%%%%%%%%%%%%%%%%%%%%%%%%%%%%%%%%%%%%%%%%%%%
\noindent
\begin{tabular*}{\textwidth}{l @{\extracolsep{\fill}} r @{\extracolsep{10pt}} l}
	\textbf{\class} & \textbf{\authorname}  &\\ %Your name here instead, obviously 
	\textbf{\assignmentname } & \textbf{\workdate} & \\
\end{tabular*}\\ 
\rule{\textwidth}{2pt}
%%%%%%%%%%%%%%%%%%%%%%%%%%%%%%%% HEADER %%%%%%%%%%%%%%%%%%%%%%%%%%%%%%%%%%%%%%%%%%%

\begin{questions}

	\question In each of the following, find the pdf of $Y$. Show that the pdf integrates to 1.
	\begin{parts}
		\part $Y=X^3$ and $f_X(x) = 42 x^5 (1-x), \ 0 < x < 1$
		\part $Y= 4X + 3$ and $f_X(x) = 7e^{-7x}, \ 0 < x <\infty$
		\part $Y=X^2$ and $f_X(x) = 30 x^2 (1-x)^2, \ 0 < x < 1$
	\end{parts}
	(See Example A.0.2 in Appendix A.)
	
	\begin{solution}\\
		(a)
		The PDF is
		\begin{align*}
			f_Y(y) &= f_X(g^{-1}(y)) \left|\frac{d}{dy}g^{-1}(y) \right| \\
			&= f_X(y^{1/3}) \left|\frac{d}{dy}y^{1/3} \right| \\
			&=  42y^{5/3} \left(1-y^{1/3}\right)  \left|\frac{1}{3}y^{-2/3} \right| \\
			&=  14y\left(1-y^{1/3}\right). \\
		\end{align*}
		The range is the same, shown by $ 0^{3} < y < 1^{3} = 0 < y < 1$. \\
		Integration is equal to 1:
		\begin{align*}
			\int_{0}^{1}f_Y(y) \,dy
			&= \int_{0}^{1} 14y-14y^{4/3} \,dy \\
			&= \left. 7y^2-6y^{7/3} \right|_0^1 \\
			&= (7-6) - (0-0) \\
			&= 1.
		\end{align*}
		 
		
		(b)
		The PDF is
		\begin{align*}
			f_Y(y) &= f_X(g^{-1}(y)) \left|\frac{d}{dy}g^{-1}(y) \right| \\
			&= f_X( \frac{y-3}{4} ) \left|\frac{d}{dy}\,\frac{y-3}{4}  \right| \\
			&= \frac{7}{4}e^{-\frac{7}{4}(y-3)}
		\end{align*}
		The range is $ \left[4(0)+3\right] < y < \left[4(\infty)+3\right] = 3 < y < \infty$. \\
		Integration is equal to 1:
		\begin{align*}
			\int_{3}^{\infty}f_Y(y) \,dy
			&= \int_{3}^{\infty} \frac{7}{4}e^{-\frac{7}{4}(y-3)} \,dy \\
			&= \left. -e^{-\frac{7}{4}(y-3)} \right|_3^\infty \\
			&= (-e^{-\infty}) - (-e^0)
			= 0 + 1 \\
			&= 1.
		\end{align*}
		
		(c)
		The PDF is
		\begin{align*}
			f_Y(y) &= f_X(g^{-1}(y)) \left|\frac{d}{dy}g^{-1}(y) \right| \\
			&= f_X( y^{1/2} ) \left|\frac{d}{dy} y^{1/2} \right| \\
			&=  30(y^{1/2})^2(1-y^{1/2})^2  \left|\frac{1}{2} y^{-1/2} \right| \\
			&= \frac{ 30y(1-2y^{1/2} + y) }{2y^{1/2}} \\
			&= 15y^{1/2}(1-2y^{1/2} + y) \\
			&= 15(y^{1/2}-2y + y^{3/2})
		\end{align*}
		The range is the same, shown by $ 0^{2} < y < 1^{2} = 0 < y < 1$. \\
		Integration is equal to 1:
		\begin{align*}
			\int_{0}^{1}f_Y(y) \,dy
			&= \int_{0}^{1} 15(y^{1/2}-2y + y^{3/2}) \,dy \\
			&= \left. 10y^{3/2}-15y^2 + 6y^{5/2}) \right|_0^1 \\
			&= (10-15+6) - (0-0+0) \\
			&= 1.
		\end{align*}
		
	\end{solution}
	%%%%%%%%%%%%%%%%%%%%%%%%%%%%%%%%%%%%%%%%%%%%%%%%%%%%%%%%%%%%%
	
	\setcounter{question}{2}
	\question Suppose $X$ has the geometric pmf $f_X(x) = \frac{1}{3} \left(\frac{2}{3}\right)^{x}, \ x = 0,1,2,...$. Determine the probability distribution of $Y= X/(X+1)$. 
	Note that here both $X$ and $Y$ are discrete random variables. To specify the probability distribution of $Y$, specify its pmf.
	
	\begin{solution}
		First we identify a $g^{-1}(\cdot)$ function,
		\begin{align*}
			Y &= \frac{X}{X+1} \\
			Y(X+1) &= X \\
			YX+Y &= X \\
			Y &= X- XY \\
			Y &= X(1-Y) \\
			\frac{Y}{(1-Y)} &= X. \\
		\end{align*}
		
		Then we may use this function in conjunction with the original PMF to find
		$f_Y(y) = \frac{1}{3}\left(\frac{2}{3}\right)^\frac{y}{(1-y)}$. \\
		Additionally we will use the original ${Y} = \frac{X}{X+1}$ to transform the domain
		to be ${y_i} = \frac{x_i}{x_i+1}, \,\ \forall\ x_i\in X$ or
		$Y = \{0,\frac{1}{2},\frac{2}{3},\ldots\}$. \\
		
	\end{solution}
	%%%%%%%%%%%%%%%%%%%%%%%%%%%%%%%%%%%%%%%%%%%%%%%%%%%%%%%%%%%%%
	\clearpage
	
	\question Let $\lambda$ be a fixed positive constant, and define the function $f(x)$ by $f(x) = \frac{1}{2} \lambda e^{-\lambda x}$ if $x \ge 0$ and $f(x) = \frac{1}{2} \lambda e^{\lambda x}$ if $x < 0$. 
	
	\begin{parts}
		\part Verify that $f(x)$ is a pdf. 
		\part If $X$ is a random variable with pdf given by $f(x)$, find $P(X < t)$ for all $t$. Evaluate all integrals.
		\part Find $P(|X| <t)$ for all $t$. Evaluate all integrals.
	\end{parts}
	
	\begin{solution}
		(a)
		Because $\lambda$ is assumed to be non-negative, then we can see that $f(x)$ is non-negative for all $x$ as a result of $e^a$ always being positive.
		Next we will evaluate integration to 1.
		\begin{align*}
			f(x)
			&= \int_{0}^{\infty} \frac{1}{2}\lambda e^{-\lambda x} \,dx + \int_{-\infty}^{0} \frac{1}{2}\lambda e^{\lambda x} \,dx \\
			&= \left. -\frac{1}{2}\lambda e^{-\lambda x} \right|_{0}^{\infty} + \left. \frac{1}{2}\lambda e^{\lambda x} \right|_{-\infty}^{0} \\
			&= (0-(-\frac{1}{2})) + (\frac{1}{2} - 0) \\
			&= 1.
		\end{align*}
		Thus we conclude that this $f(x)$ is a valid PDF.
		
		(b)
		\begin{align*}
			P(X<t)
			&= \int_{-\infty}^{\min(t,0)} \frac{1}{2}\lambda e^{\lambda x} \,dx 
			+ \text{\LARGE $\mathbbm{1}$}_{t\geq0} \left( \int_{0}^{t} \frac{1}{2}\lambda e^{-\lambda x} \,dx\right)  \\
			&=  \left( \frac{1}{2} e^{\lambda t} - 0 \right)
			- \text{\LARGE $\mathbbm{1}$}_{t\geq0} \left( \frac{1}{2} e^{-\lambda t} + \frac{1}{2} \right)  \\
			&=  \frac{1}{2} e^{\lambda t}
			- \text{\LARGE $\mathbbm{1}$}_{t\geq0}  \left( \frac{1}{2} (e^{-\lambda t} + 1) \right)  \\
		\end{align*}
		
		(c)
		\begin{align*}
			P(X<t)
			&= \int_{-t}^{0} \frac{1}{2}\lambda e^{\lambda x} \,dx + \int_{0}^{t} \frac{1}{2}\lambda e^{-\lambda x} \,dx  \\
			&= \left( \frac{1}{2} - \frac{1}{2} e^{-\lambda t} \right) + \left( -\frac{1}{2} e^{-\lambda t} + \frac{1}{2} \right)  \\
			&= 1 - e^{-\lambda t}
		\end{align*}
		
	\end{solution}
	%%%%%%%%%%%%%%%%%%%%%%%%%%%%%%%%%%%%%%%%%%%%%%%%%%%%%%%%%%%%%
	\clearpage
	
	\setcounter{question}{8}
	\question  If the random variable $X$ has pdf $$f(X) = \begin{cases}\frac{x-1}{2} & 1<x<3 \\ 0 & \text{otherwise} \end{cases}$$
	find a monotone function $u(x)$ such that the random variable $Y = u(X)$ has a uniform(0,1) distribution.
	\begin{solution}
		
			Using theorem 2.1.10:
		\begin{align*}
			u(x) &= F_{\mathbf X}(x) \\
			&= \int_{1}^{3} f_\mathbf{X}(x) \,dx \\
			&= \int_{1}^{3} \frac{x-1}{2} \,dx \\
			&= \int_{1}^{3} \frac{x}{2}-\frac{1}{2} \,dx \\
			&= \frac{x^2}{4}-\frac{x}{2} \\
			&= \left. \frac{x^2-2x}{4} \right|_1^3 \\ \\
			&\Rightarrow \text{\LARGE $\mathbbm{1}$}_{1<x<3} \left( \frac{x(x-2)}{4} \right)
		\end{align*}
	\end{solution}
	%%%%%%%%%%%%%%%%%%%%%%%%%%%%%%%%%%%%%%%%%%%%%%%%%%%%%%%%%%%%%	

	\setcounter{question}{11}
	\question A random right triangle can be constructed in the following manner. Let $X$ be a random angle whose distribution is uniform on $(0, \pi / 2)$. For each $X$, construct a triangle as pictured below. Here, $Y =$ height of the random triangle. For a fixed constant $d$, find the distribution of $Y$ and $E[Y]$. 
	\begin{center}
	\includegraphics[width=0.3\linewidth]{prob12}
	\end{center}
	
	\begin{solution} \\
		First we identify a relationship between $x$ and $y$, $\tan x = \frac{y}{d}$.
		
		Which allows us to determine that $x = \tan^{-1}\left(\frac{y}{d}\right)$.
		
		Applied to the domain of $x$  we find that $0<y<\infty$.
		
		Next,
		\begin{align*}
			F_\mathbf{X}(x) 
			&= \frac{x-0}{\frac{\pi}{2}-0} &\sim\text{U}(0, \pi / 2) \\
			&= \frac{2x}{\pi} \\
			\Rightarrow F_\mathbf{Y}(y)
			&= \frac{2}{\pi}\tan^{-1}\left(\frac{y}{d}\right).
		\end{align*}
		Which allows us to find the distribution
		\begin{align*}
			f_\mathbf{Y}(y)
			&= \frac{d}{dy} F_\mathbf{Y}(y) \\
			&= \frac{2}{\pi}\tan^{-1}\left(\frac{y}{d}\right) \ \frac{d}{dy} \\
			&= \left(\frac{2}{\pi}\right)\left(\frac{1}{1+(\frac{y}{d})^2}\right).
		\end{align*}
		\begin{align*}
			E[Y] 
			&= \int_{-\infty}^{\infty} y f_\mathbf{Y}(y) \,dy \\
			&= \int_{0}^{\infty} y \left(\frac{2}{\pi}\right)\left(\frac{1}{1+(\frac{y}{d})^2}\right) \,dy \\
			&\Rightarrow \infty - 0 = \infty.
		\end{align*}
		
	\end{solution}
	%%%%%%%%%%%%%%%%%%%%%%%%%%%%%%%%%%%%%%%%%%%%%%%%%%%%%%%%%%%%%
	
	\setcounter{question}{16}
	\question  A median of a distribution is a value $m$ such that $P(X \le m) \ge \frac{1}{2}$ and $P(X \ge m) \ge \frac{1}{2}$.
	If $X$ is continuous, $m$ satisfies $\int_{-\infty}^{m} f(x) dx = \int_{m}^{\infty} f(x) dx = \frac{1}{2}$. Find the median of the following distributions.
	\begin{parts}
		\part $f(x) = 3x^2 , \ 0 < x < 1$
		\part $f(x) = \frac{1}{\pi (1+ x^2)}, \ -\infty < x < \infty$
	\end{parts}
	
	\begin{solution} \\
		(a)
		\begin{align*}
			\int_{-\infty}^{m} 3x^2 dx &= \int_{m}^{\infty} 3x^2 dx = \frac{1}{2} \\
			\left. x^3 \right|_{0}^{m} &= \left. x^3 \right|_{m}^{2} = \frac{1}{2} \\
			m^3 &= 1-m^3 = \frac{1}{2} \\
			m &= \sqrt[3]{\frac{1}{2}}
		\end{align*}
		
		(b)
		\begin{align*}
			\int_{-\infty}^{m} \frac{1}{\pi(1+x^2)} dx &= \int_{m}^{\infty} \frac{1}{\pi(1+x^2)} dx = \frac{1}{2} \\
		\end{align*}
		Because $x^2$ makes this function symmetric and in particular around $0$
		we infer that $m=0$.
		
	\end{solution}
	%%%%%%%%%%%%%%%%%%%%%%%%%%%%%%%%%%%%%%%%%%%%%%%%%%%%%%%%%%%%%

	\setcounter{question}{21}
	\question Let $X$ have the pdf $$ f(x) = \frac{4}{\beta^3\sqrt{\pi}} x^2 e^{-x/\beta^2},\quad 0 < x < \infty,\quad \beta > 0.$$
	\begin{parts}
		\part Verify that $f(x)$ is a pdf.
		\part Find $E(X)$ and $Var(X)$
	\end{parts}
	
	\begin{solution}
		Here is the solution
	\end{solution}
	%%%%%%%%%%%%%%%%%%%%%%%%%%%%%%%%%%%%%%%%%%%%%%%%%%%%%%%%%%%%%
	
	\setcounter{question}{23}
	\question  Compute $E(X)$ and $Var(X)$ for each of the following probability distributions.
	\begin{parts}
		\part $f_X(x) = ax^{a-1}, \ 0<x<1, \ a > 0$
		\part $f_X(x) = \frac{1}{n}, \ x=1,2,..,n, \ n > 0$ an integer
		\part $f_X(x) = \frac{3}{2}(x-1)^{2}, \ 0<x<2$
	\end{parts}
	
	\begin{solution}\\
		(a)
		\begin{align*}
			E[X]&
			= \int_0^1 x\, ax^{a-1}\,dx 
			= \int_0^1 ax^{a}\,dx 
			= \left. \frac{ax^{a+1}}{a+1}\right|_0^1 
			= \frac{a1^{a+1}}{a+1} - 0
			= \frac{a}{a+1}
			\intertext{}
			E[X^2]&
			= \int_0^1 x^2\, ax^{a-1}\,dx 
			= \int_0^1 ax^{a+1}\,dx 
			= \left. \frac{ax^{a+2}}{a+2}\right|_0^1 
			= \frac{a1^{a+2}}{a+2} - 0
			= \frac{a}{a+2}
			\intertext{} 
			\text{Var}[X]&
			= E[X^2] - E[X]^2
			= \frac{a}{a+2} - \left( \frac{a}{a+1} \right)^2
			= \frac{a}{a+2} - \frac{a^2}{(a+1)^2}
			\\ &
			= \frac{ a(a+1)^2-a^2(a+2) }{ (a+2)(a+1)^2 }
			= \frac{(a^3+2a^2+a)-(a^3+2a^2) }{(a+2)(a+1)^2}
			= \frac{a}{(a+2)(a+1)^2}
		\end{align*}
		
		(b)
		\begin{align*}
			E[X]&
			= \sum_{x=1}^n x \frac{1}{n} 
			= \frac{1}{n} \sum_{x=1}^n x 
			= \frac{1}{n}\left( \frac{n(n+1)}{2} \right)
			= \frac{n+1}{2}
			\intertext{}
			E[X^2]&
			= \sum_{x=1}^n x^2 \frac{1}{n} 
			= \frac{1}{n} \sum_{x=1}^n x^2 
			= \frac{1}{n}\left( \frac{n(n+1)(2n+1)}{6} \right)
			= \frac{(n+1)(2n+1)}{6}
			\intertext{} 
			\text{Var}[X]&
			= E[X^2] - E[X]^2
			= \frac{(n+1)(2n+1)}{6} - \left( \frac{n+1}{2} \right)^2
			= \frac{2(n+1)(2n+1)-3(n+1)^2}{12}
			\\ &
			= \frac{(n+1)\big(2(2n+1)-3(n+1)\big)}{12}
			= \frac{(n+1)(n-1)}{12}
		\end{align*}
		
		(c)
		\begin{align*}
			E[X]&
			= \int_0^1 x\, \frac{3}{2}(x-1)^2\,dx 
			= \frac{3}{2} \int_0^1 x\, (x-1)^2\,dx 
			= \frac{3}{2} \int_0^1 x^3 - 2x^2 + x \,dx 
			\\&
			= \frac{3}{2} \left. \left( \frac{x^4}{4} - \frac{2x^3}{3} + \frac{x^2}{2} \right) \right|_0^1 
			= \frac{3}{2} \left( 4 - \frac{16}{3} + 2 \right)
			= 6-8+3=1
			\intertext{}
			E[X^2]&
			= \int_0^1 x^2\, \frac{3}{2}(x-1)^2\,dx 
			= \frac{3}{2} \int_0^1 x^2\, (x-1)^2\,dx 
			= \frac{3}{2} \int_0^1 x^4 - 2x^3 + x^2 \,dx 
			\\&
			= \frac{3}{2} \left. \left( \frac{x^5}{5} - \frac{x^4}{2} + \frac{x^3}{3} \right) \right|_0^1 
			= \frac{3}{2} \left( \frac{32}{5} - 8 + \frac{8}{3} \right) - 0
			= \frac{96}{10}-12+4
			= 1\frac{3}{5}
			= \frac{8}{5}
			\intertext{} 
			\text{Var}[X]&
			%
			= E[X^2] - E[X]^2
			= \frac{a}{a+2} - \left( \frac{a}{a+1} \right)^2
			= \frac{a}{a+2} - \frac{a^2}{(a+1)^2}
			= \frac{8}{5} - 1^2
			= \frac{3}{5}
		\end{align*}
	\end{solution}
	%%%%%%%%%%%%%%%%%%%%%%%%%%%%%%%%%%%%%%%%%%%%%%%%%%%%%%%%%%%%%

	\question Suppose the pdf  $f_X(x)$ of a random variable $X$ is an even function. ($f_X(x)$ is an even function if $f_X(x) = f_X(-x)$ for every $x$.) Show that 
	\begin{parts}
		\part $X$ and $-X$ are identically distributed.
		\part $M_X(t)$ is symmetric about 0.
	\end{parts}
	
	\begin{solution}\\
		(a)
		
		(b)
		
	\end{solution}
	%%%%%%%%%%%%%%%%%%%%%%%%%%%%%%%%%%%%%%%%%%%%%%%%%%%%%%%%%%%%%
	
	\question  Let $f(x)$ be a pdf and let $a$ be a number such that, for all $\epsilon > 0, \ f(x+\epsilon) = f(a-\epsilon)$. Such a pdf is said to be symmetric about the point $a$.
	
	\begin{parts}
		\part Give three examples of symmetric pdfs.
		\part Show that if $X \sim f(x)$, symmetric, then the median of $X$ is the number $a$.
		\part Show that if $X \sim f(x)$, symmetric, and $E(X)$ exists, then $E(X) = a$.
		\part Show that $f(X) = e^{-x}, \ x \ge 0$, is not a symmetric pdf.
		\part Show that in the pdf in part (d), the median is less than the mean.
	\end{parts}
	\begin{solution}\\
		(a)
		\begin{enumerate}
			\item $f(x) = x^2$, or more generally $f(x) = x^{2i}$, is symmetric around $a=0$ for all $i\in\mathbb{N}_1$.
			\item $\sim U(i,j) \Rightarrow f(x) = \frac{1}{j-i}$ is symmetric around $a=\frac{i+j}{2}$
			\item A normal distribution $\sim N(\mu,\sigma^2)\ $ or $\ f(x) = \frac{1}{\sigma\sqrt{2\pi}}e^{-\frac{1}{2}\left(\frac{x-\mu}{\sigma}\right)^2}$
			is symmetric around $a=\mu$.
		\end{enumerate}
		
		(b)
		
		
		(c)
		
		(d)
		
		(e)
		
		
	\end{solution}
	%%%%%%%%%%%%%%%%%%%%%%%%%%%%%%%%%%%%%%%%%%%%%%%%%%%%%%%%%%%%%

	\setcounter{question}{28}
	\question To calculate moments of discrete distributions, it is often easier to work with the factorial moments (see Miscellanea 2.6.2). 
	\begin{parts}
		\part Calculate the factorial moment $E[X(X-1)]$ for the binomial and Poisson distributions.
		\part Use the results of part (a) to calculate the variances of the binomial and Poisson distributions. 
	\end{parts}
	
	\begin{solution}
		Here is the solution
	\end{solution}
	%%%%%%%%%%%%%%%%%%%%%%%%%%%%%%%%%%%%%%%%%%%%%%%%%%%%%%%%%%%%%
	
	\setcounter{question}{31}
	\question  Let $M_X(t)$ be the moment generating function of $X$, and define $S(t) = \log{(M_X(t))}$. Show that
	$$ \frac{d}{dt} \left. S(t) \right|_{t=0} = E(X) \text{ and }  \frac{d^2}{dt^2} \left. S(t) \right|_{t=0} = Var(X)$$
	\begin{solution}
		Here is the next solution
	\end{solution}
	%%%%%%%%%%%%%%%%%%%%%%%%%%%%%%%%%%%%%%%%%%%%%%%%%%%%%%%%%%%%%	
	
	\question  In each of the following cases verify the expression given for the moment generating function, and in each case use the mgf to calculate $E(X)$ and $Var(X)$.
	\begin{parts}
		\part $P(X=x)=\frac{e^{-\lambda} \lambda^x}{x}, \quad M_X(t)=e^{\lambda\left(e^t-1\right)}, \quad x=0,1, \ldots ; \quad \lambda>0$
		\part $P(X=x)=p(1-p)^x, \quad M_X(t)=\frac{p}{1-(1-p) e^t}, \quad x=0,1, \ldots ; \quad 0<p<1$
		\part $f_X(x)=\frac{e^{-(x-\mu)^2 /\left(2 \sigma^2\right)}}{\sqrt{2 \pi} \sigma}, M_X(t)=e^{\mu t+\sigma^2 t^2 / 2},-\infty<x<\infty ;-\infty<\mu<\infty, \  \sigma>0$
	\end{parts}
	
	\begin{solution}
		(a)
		
	\end{solution}
	%%%%%%%%%%%%%%%%%%%%%%%%%%%%%%%%%%%%%%%%%%%%%%%%%%%%%%%%%%%%%	
	
	\setcounter{question}{37}
	\question  Let $X$ have the negative binomial distribution with pmf
	$$ f(x) = { r + x -1 \choose x} p^r (1-p)^x, \quad x= 0,1,2,...,$$
	where $0 < p< 1$ and $r >0$ is an integer.
	\begin{parts}
		\part Calculate the mgf of $X$.
		\part Define a new random variable by $Y = 2pX$. Show that, as $p \downarrow 0$, the mgf of $Y$ converges to that of a chi squared random variable with $2r$ degrees of freedom by showing that $$\lim_{p \rightarrow 0} M_Y(t) = \left(\frac{1}{1-2t}\right)^r , \quad |t| < \frac{1}{2}.$$
	\end{parts}
	\begin{solution}
		Here is the next solution
	\end{solution}
	%%%%%%%%%%%%%%%%%%%%%%%%%%%%%%%%%%%%%%%%%%%%%%%%%%%%%%%%%%%%%			

\end{questions}
\end{document}
