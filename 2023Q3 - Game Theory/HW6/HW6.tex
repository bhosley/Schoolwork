\documentclass[12pt]{amsart}
\usepackage[left=0.5in, right=0.5in, bottom=0.75in, top=0.75in]{geometry}
\usepackage[english]{babel}
\usepackage[utf8x]{inputenc}
\usepackage{amsmath,amssymb,amsthm}
\usepackage{enumerate}
\usepackage{graphicx}


\usepackage{xcolor}
\usepackage{xparse}

\renewcommand{\thesection}{}
\renewcommand{\thesubsection}{\arabic{subsection}}
\renewcommand{\thesubsubsection}{\quad(\alph{subsubsection})}

\begin{document}
\raggedbottom

\noindent{\large OPER 618 - Game Theory and Math Programming %
	- Homework 5 }
\hspace{\fill} {\large B. Hosley}
\bigskip


%%%%%%%%%%%%%%%%%%%%%%%
\setcounter{subsection}{0}
\subsection{}
\textbf{Network Interdiction for the Shortest Path Problem (SPP)}.

\subsubsection{}
\textit{Adapt the network interdiction formulation(s) by Wood (1993) to model an attacker
	maximizing the shortest path from node $s$ to node $t$ in a directed network $G(N, A)$
	having arc lengths $c_{ij}, ∀ (i, j) \in A$.} \\

	Let,
	\begin{description}
		\item[$N$] Is the set of all nodes in the system
		\item[$A$] The set of all $(i,j)$ where an arc exists between nodes $i$ and $j$
		\item[$c_{ij}$] Arc length between nodes $i$ and $j$.
		\item[$x_{ij}$] A binary value where $1$ indicates the use of path $i$ to $j$
		\item[$\gamma_{ij}$] A binary value where $1$ indicates an elimination of path $i$ to $j$
		\item[$s$]  A representation of the starting/source node
		\item[$t$]  A representation of the ending/sink node
	\end{description}

	Then the shortest path maxmin problem can be represented as,
	\begin{align*}
		\max_\gamma\min_x \hspace{8em} \sum_{(i,j)\in A} c_{ij}x_{ij}& \\
		\text{s.t.}\quad
		\sum_{j:(i,j)\in A}x_{ij} - \sum_{j:(j,i)\in A}x_{ji} -1 &= 0, \qquad\qquad i=s, \\
		\sum_{j:(i,j)\in A}x_{ij} - \sum_{j:(j,i)\in A}x_{ji}\ \quad\ &= 0, \qquad\qquad i\in N\backslash \{s,t\}, \\
		\sum_{j:(i,j)\in A}x_{ij} - \sum_{j:(j,i)\in A}x_{ji} +1 &= 0, \qquad\qquad i=t, \\
		%\{s,t\}
		x_{ij} &\leq (1- \gamma_{ij}), \quad \forall(i,j) \in A, \\
		x_{ij} &\in \{0,1\}, \hspace{2.5em} \forall(i,j)\in A\cup \{s,t\}, \\
		%
		\sum_{(i,j)\in A} r_{ij}\gamma_{ij} &\leq R, \\
		\gamma_{ij} &\in \{0,1\}\ \qquad\quad \forall(i,j) \in A,
	\end{align*}

\subsubsection{}
\textit{Take the dual of the inner (i.e., lower level) problem to identify a mixed-integer
	nonlinear single-level formulation.}

	\begin{align*}
		\max_\gamma\max_{\alpha,\theta} \qquad \sum_{(i,j)\in A} \theta_{ij}(1-\gamma_{ij})& \\
		\text{s.t.}\qquad\quad
		\alpha_i - \alpha_j + \theta_{ij} &\geq 0,\hspace{5em} \forall(i,j) \in A, \\
		\alpha_t - \alpha_s &\geq \sum_{(i,j)\in A} c_{ij}, \\
		\theta_{ij} &\geq 0, \hspace{5em} \forall(i,j) \in A,  \\
		\alpha_i &\text{: unrestricted}, \quad i\in N, \\
		%
		\sum_{(i,j)\in A} r_{ij}\gamma_{ij} &\leq R, \\
		\gamma_{ij} &\in \{0,1\}\hspace{3.5em} \forall(i,j) \in A
	\end{align*}

\subsection{}
\textbf{Applications of Network Interdiction for the Shortest Path Problem (SPP)}.
\textit{Describe a practical application for this model.} \\

	While this type of problem is usually intuited via transportation networks of some flavor, 
	as they are tangible and easily grasped. There is a lot of fuzziness in the translation between
	the mathematical model and the actual implementation. Leaving a (usually acceptable) amount of 
	irreducible uncertainty.
	
	A similar application that typically feels more abstract is in digital communications networks.
	The scales at which these networks operate lend a difficulty in conveying the effects to a network
	communications lay-person. However, the situation is much more likely to adhere to the assertions made
	in the mathematical model.
	
	Simply knowing the clock speed of the network devices that constitute the nodes can provide an extremely
	accurate measurement of throughput via various routes in the network.
	This measurement can be made even more accurate with knowledge of the connection medium.
	
	The objectives can be the same, such that degrading certain routes saturates nodes in a bottleneck fashion.
	It is also possible that this type of analysis may be useful when attempting to divert more traffic to a particular
	route that has a tap for intel collection.


\section{}
\setcounter{subsection}{2}

\textbf{Zero-sum interdiction for the Lower-level Problem with (Binary) Integer Restrictions.} 
\textit{For Questions 3, 4, \& 5, consider the following formulation of the maximal covering 
location problem (MCLP).} \\

\begin{center}
\begin{minipage}{0.75\linewidth}
	
	\underline{Sets}
		\begin{description}
			\item[$I$] the set of demand node locations, indexed on $i$
			\item[$J$] the set of possible facility sites, indexed on $j$
		\end{description}
		
	\underline{Parameters}
		\begin{description}
			\item[$p$]  the number of facilities available for location
			\item[$v_i$] the value attained by covering facility $i$
			\item[$a_{ij}= 1$] if a facility emplaced at site $j$ can cover a demand at location $i$
		\end{description}
		
	\underline{Decision Variables}
		\begin{description}
			\item[$x_j$] a binary decision variable equal to 1 if a facility is emplaced at location $i$, and 0 otherwise
			\item[$y_i$] a binary decision variable equal to 1 if the demand at location $i$ is covered by an emplaced
			facility within range, and 0 otherwise
		\end{description}
	
	\underline{Formulation}
		\begin{align*}
			\max_{x,y} \qquad\qquad \sum_{i\in I} v_iy_i& \\
			\text{s.t.}\quad \sum_{j\in J} a_{ij}x_j \geq y_i,& \quad \forall i\in I, \\
			\sum_{j\in J} x_j \leq p,& \\
			x_j \in \{0,1\},& \quad \forall j\in J, \\
			y_i \in \{0,1\},& \quad \forall i\in I. \\
		\end{align*}
	
\end{minipage}
\end{center}

\begin{center}
\begin{minipage}{0.75\linewidth}
		
\textit{Now, consider a zero-sum bilevel formulation wherein an interdictor can block up to $q$ sites
	in $J$ from being used, wherein a decision to “block” corresponds with the binary decision
	variable $z_j = 1$. (For ease of discrimination, the upper-level player’s objective and
	constraints are indicated in {\color{blue} blue-colored font}.)} \\
	
	\underline{Zero-sum Formulation}
	\begin{align*}
		{\color{blue}\min_z}\max_{x,y} \qquad\qquad \sum_{i\in I} v_iy_i& \\
		\text{s.t.}\quad
		{\color{blue} x_j+z_j \leq 1,}& \quad {\color{blue}\forall j\in J,}\\
		{\color{blue} \sum_{j\in J}z_j\leq q,}&  \\
		{\color{blue} z_j\in \{0,1\},}& {\color{blue}\quad\forall j\in J}.  \\
		\sum_{j\in J} a_{ij}x_j \geq y_i,& \quad \forall i\in I, \\
		\sum_{j\in J} x_j \leq p,& \\
		x_j \in \{0,1\},& \quad \forall j\in J, \\
		y_i \in \{0,1\},& \quad \forall i\in I. \\
	\end{align*}
	
\textit{To take the dual of the lower-level problem and attain a single-level reformulation (e.g.,
	Wood (1993), Lessin et al. (2018)), one would have to relax the binary integer restrictions
	on $x_i$ and $y_i$ decision variables. The constraint matrix for a covering problem is not totally
	unimodular, so this integer-relaxation may have consequences on the solution(s) attained.} \\

\end{minipage}
\end{center}		

\subsection{}
\textit{Would solving the resulting single-level formulation yield an upper bound or a lower bound
on the optimal solution to the zero-sum bilevel formulation? Explain.} \\

Solving the first, single-level formulation will provide an upper bound. 
Implicit in the single-level is that player one will make their best
possible move without any interference, they cannot perform better.
In other words, one can imagine the second player being present but is either unable to make a move, 
or makes the worst possible move relative to their own interest (assuming zero-sum). \\


\subsection{}
\textit{Consider the following solution procedure. First, solve the reformulated, single-level
	problem for the integer-relaxed lower-level problem. Take the binary-valued decisions $z_j$
	for the upper-level player and affix them. Let’s call them $\bar z_j$. Now, solve the following
	problem to attain values $\bar x_j$ and $\bar y_i$ and a corresponding objective function 
	value $\sum_{i\in I}v_i\bar y_i$.}

	\begin{align*}
		{\color{blue}\min_z}\max_{x,y} \qquad\qquad \sum_{i\in I} v_iy_i& \\
		\text{s.t.}\quad
		{\color{blue} x_j \leq 1-\bar z_j,}& \quad {\color{blue}\forall j\in J,}\\
		\sum_{j\in J} a_{ij}x_j \geq y_i,& \quad \forall i\in I, \\
		\sum_{j\in J} x_j \leq p,& \\
		x_j \in \{0,1\},& \quad \forall j\in J, \\
		y_i \in \{0,1\},& \quad \forall i\in I. \\
	\end{align*}

\textit{Does the solution to this problem provide an upper bound or a lower bound on the optimal
	solution to the zero-sum bilevel formulation? How does it relate to the objective function
	value attained via the procedure described in Questions 2 and 3? Explain.} \\
	
	\begin{align*} 
		\bar x_j &\leq 1 - \bar z_j \hspace{5.5em} \forall j\in J \\
		\bar y_i &\leq \sum_{j\in J} a_{ij}(1-\bar z_{i}) \hspace{2em}  \forall i\in I \\
		\max_{x,y}& \sum_{i\in I} \sum_{j\in J} v_i a_{ij}(1-\bar z_{i}) \\
	\end{align*}


	This formulation gives a lower bound. The relaxation removes the $q$ boundary limiting what $z_j$
	can be. By eliminating this, the formulation demonstrates what the solution would be
	given no limit on $z$. Thus we can identify what the minimum value for player 1 could be
	given unlimited action by player 2, such that they cannot remove anything more from player 1. \\

\subsection{}
\textit{Would it be reasonable for the upper-level decision-maker to implement a “blocking
	solution” attained by the solution procedure described by Question 4? Why or why not?} \\

	This is not likely a reasonable solution. As alluded to in the previous response,
	the elimination of the $q$ constraint represents unlimited resources on the interdictor's part.
	While this would be ideal from the perspective of player 2, it is seldom the case that this would be true.
	Even if it were practically true, wherein player 2's resources so outmatch player 1's as to appear
	unlimited, it would probably be seen as poor stewardship of resources to invest so far beyond a 
	point of diminishing returns as to maximally dominate the zero sum game.


\end{document}