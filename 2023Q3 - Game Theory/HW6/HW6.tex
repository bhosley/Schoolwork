\documentclass[12pt]{amsart}
\usepackage[left=0.5in, right=0.5in, bottom=0.75in, top=0.75in]{geometry}
\usepackage[english]{babel}
\usepackage[utf8x]{inputenc}
\usepackage{amsmath,amssymb,amsthm}
\usepackage{enumerate}
\usepackage{graphicx}


\usepackage{xcolor}
\usepackage{xparse}

\renewcommand{\thesection}{}
\renewcommand{\thesubsection}{\arabic{subsection}}
\renewcommand{\thesubsubsection}{\quad(\alph{subsubsection})}

\begin{document}
\raggedbottom

\noindent{\large OPER 618 - Game Theory and Math Programming %
	- Homework 6 }
\hspace{\fill} {\large B. Hosley}
\bigskip


%%%%%%%%%%%%%%%%%%%%%%%
\setcounter{subsection}{0}
\subsection{}
\textbf{Bilevel Programming Problems.} 
\textit{Consider the following linear BLPP.}
\begin{align*}
	\min_x\	\ x &+ 3y \\
	\text{s.t. }\	1 &\leq x \leq 6 \\
	y \in \arg\min \{ -y &: x + y \leq 8, x + 4y \leq 8, x + 2y \leq 13 \}
\end{align*}

%
% neg gradient is the direction of greatest improvement for the minimize problem
%

\subsubsection{}
\textit{Draw the relaxed feasible region $\Omega$.}

\subsubsection{}
\textit{Somewhere on the plot, depict $-\nabla f(x,y)$, the negative of the gradient for the follower’s objective function.}

\subsubsection{}
\textit{Based on your answers to (a) and (b), draw the inducible region IR.}

\subsubsection{}
\textit{Somewhere on the plot, depict $-\nabla F(x,y)$, the negative of the gradient for the leader’s objective function.}

\subsubsection{}
\textit{Based on your answers on (c) and (d), what is the optimal decision for the leader?}




\subsection{}
\textbf{Bilevel Programming Problems.} 
\textit{Discuss how the complexity of solving the BLPP in Problem 1 would change under the following conditions. A written description is necessary, and a manual sketch to illustrate any points might be helpful.}

% 
% opt/pessimistic occurs when follower has multiple optima
% 

\subsubsection{}
\textit{The formulation included nonlinear constraints and/or nonlinear objective functions.}

\subsubsection{}
\textit{The formulation included integer restrictions on the leader and/or follower’s decision variables.}



\subsection{}
\textbf{BLPPs – Optimistic and Pessimistic Solutions.} 
\textit{Consider the following BLPP, for which $y\in R(x)$ is not always a singleton.}

\begin{align*}
	\min_x	-x-y
	\text{s.t.}	x \leq 5
	\min_y	(-\|y\|)
	\text{s.t.}	x^2 + y^2 \leq 16
\end{align*}

% 
% draw inducible region/ and the follower's neg gradient
% 

\subsubsection{}
\textit{Draw the relaxed feasible region $\Omega$.}

\subsubsection{}
\textit{Based on your answers to (a) and (b), draw the inducible region IR.}

\subsubsection{}
\textit{Somewhere on the plot, depict $-\nabla F(x,y)$, the negative of the gradient for the leader’s objective function.}

\subsubsection{}
\textit{Identify the optimal optimistic solution to the BLPP and depict it on the plot.}

\subsubsection{}
\textit{Identify the optimal pessimistic solution to the BLPP and depict it on the plot.}



\subsection{}
\textbf{Exact Solution Methods for BLPPs.} 
\textit{The branch-and-bound procedure described by Gümüş and Floudas (2001) requires the identification of a global optimal solution to a mixedinteger nonlinear problem in Step 8. The authors use the(ir) \textbf{GMIN-}$\mathbf\alpha$\textbf{BB} algorithm. However, a suitable commercial solver could be used instead. Research and identify three commercial solvers that would be suitable for use in this step.} \\

\textit{\underline{Note}: Answers identifying commercial solvers guaranteed to find a global optimum are guaranteed to receive full credit. Answers identifying commercial solvers reliant upon metaheurstics (e.g., particle swarm optimization, genetic algorithms) can be expected -- much like metaheurstics -- to receive a great deal of credit but with no guarantee of the
maximum.}



\subsection{}
\textbf{Limitations on Exact Solution Methods for BLPPs.} 
\textit{The branch-and-bound procedure described by Gümüş and Floudas (2001) requires the objective function and each constraint in the lower-level problem to be continuous and twice differentiable. Formulate a bilevel program motivated by a real-world application that does not meet these criteria.}



\end{document}