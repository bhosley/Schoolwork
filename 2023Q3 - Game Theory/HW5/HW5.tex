\documentclass[12pt]{amsart}
\usepackage[left=0.5in, right=0.5in, bottom=0.75in, top=0.75in]{geometry}
\usepackage[english]{babel}
\usepackage[utf8x]{inputenc}
\usepackage{amsmath,amssymb,amsthm}
\usepackage{enumerate}
\usepackage{graphicx}


\usepackage{xcolor}
\usepackage{xparse}

\renewcommand{\thesection}{}
\renewcommand{\thesubsection}{\arabic{subsection}}
\renewcommand{\thesubsubsection}{\quad(\alph{subsubsection})}

\begin{document}
\raggedbottom

\noindent{\large OPER 618 - Game Theory and Math Programming %
	- Homework 5 }
\hspace{\fill} {\large B. Hosley}
\bigskip


%%%%%%%%%%%%%%%%%%%%%%%
\setcounter{subsection}{0}
\subsection{}
\textbf{Network Interdiction for the Shortest Path Problem (SPP)}.

\subsubsection{}
\textit{Adapt the network interdiction formulation(s) by Wood (1993) to model an attacker
	maximizing the shortest path from node $s$ to node $t$ in a directed network $G(N, A)$
	having arc lengths $c_i, ∀ (i, j) \in A$.}


\subsubsection{}
\textit{Take the dual of the inner (i.e., lower level) problem to identify a mixed-integer
	nonlinear single-level formulation.}



\subsection{}
\textbf{Applications of Network Interdiction for the Shortest Path Problem (SPP)}.
\textit{Describe a practical application for this model.}



\section{}
\setcounter{subsection}{2}

\textbf{Zero-sum interdiction for the Lower-level Problem with (Binary) Integer Restrictions.} 
\textit{For Questions 3, 4, \& 5, consider the following formulation of the maximal covering 
location problem (MCLP).} \\

\begin{center}
\begin{minipage}{0.75\linewidth}
	
	\underline{Sets}
		\begin{description}
			\item[$I$] the set of demand node locations, indexed on $i$
			\item[$J$] the set of possible facility sites, indexed on $j$
		\end{description}
		
	\underline{Parameters}
		\begin{description}
			\item[$p$]  the number of facilities available for location
			\item[$v_i$] the value attained by covering facility $i$
			\item[$a_{ij}= 1$] if a facility emplaced at site $j$ can cover a demand at location $i$
		\end{description}
		
	\underline{Decision Variables}
		\begin{description}
			\item[$x_j$] a binary decision variable equal to 1 if a facility is emplaced at location $i$, and 0 otherwise
			\item[$y_i$] a binary decision variable equal to 1 if the demand at location $i$ is covered by an emplaced
			facility within range, and 0 otherwise
		\end{description}
		
	\underline{Formulation}
		\begin{align*}
			\max_{x,y} \qquad\qquad \sum_{i\in I} v_iy_i& \\
			\text{s.t.}\quad \sum_{j\in J} a_{ij}x_j \geq y_i,& \quad \forall i\in I, \\
			\sum_{j\in J} x_j \leq p,& \\
			x_j \in \{0,1\},& \quad \forall j\in J, \\
			y_i \in \{0,1\},& \quad \forall i\in I. \\
		\end{align*}
	
\textit{Now, consider a zero-sum bilevel formulation wherein an interdictor can block up to $q$ sites
	in $J$ from being used, wherein a decision to “block” corresponds with the binary decision
	variable $z_j = 1$. (For ease of discrimination, the upper-level player’s objective and
	constraints are indicated in {\color{blue} blue-colored font}.)}
	
\end{minipage}
\end{center}

\begin{center}
\begin{minipage}{0.75\linewidth}
	
	\underline{Zero-sum Formulation}
	\begin{align*}
		{\color{blue}\min_z}\max_{x,y} \qquad\qquad \sum_{i\in I} v_iy_i& \\
		\text{s.t.}\quad
		{\color{blue} x_j+z_j \leq 1,}& \quad {\color{blue}\forall j\in J,}\\
		{\color{blue} \sum_{j\in J}z_j\leq q,}&  \\
		{\color{blue} z_j\in \{0,1\},}& {\color{blue}\quad\forall j\in J}.  \\
		\sum_{j\in J} a_{ij}x_j \geq y_i,& \quad \forall i\in I, \\
		\sum_{j\in J} x_j \leq p,& \\
		x_j \in \{0,1\},& \quad \forall j\in J, \\
		y_i \in \{0,1\},& \quad \forall i\in I.
	\end{align*}
	
\textit{To take the dual of the lower-level problem and attain a single-level reformulation (e.g.,
	Wood (1993), Lessin et al. (2018)), one would have to relax the binary integer restrictions
	on $x_i$ and $y_i$ decision variables. The constraint matrix for a covering problem is not totally
	unimodular, so this integer-relaxation may have consequences on the solution(s) attained.}

\end{minipage}
\end{center}		

\subsection{}
\textit{Would solving the resulting single-level formulation yield an upper bound or a lower bound
on the optimal solution to the zero-sum bilevel formulation? Explain.}


\subsection{}
\textit{Consider the following solution procedure. First, solve the reformulated, single-level
	problem for the integer-relaxed lower-level problem. Take the binary-valued decisions $z_i$
	for the upper-level player and affix them. Let’s call them $z_i$. Now, solve the following
	problem to attain values $\bar x_j$ and $\bar y_i$ and a corresponding objective function 
	value $\sum_{i\in I}v_i\bar y_i$.}

	\begin{align*}
		{\color{blue}\min_z}\max_{x,y} \qquad\qquad \sum_{i\in I} v_iy_i& \\
		\text{s.t.}\quad
		{\color{blue} x_j \leq 1-z_j,}& \quad {\color{blue}\forall j\in J,}\\
		\sum_{j\in J} a_{ij}x_j \geq y_i,& \quad \forall i\in I, \\
		\sum_{j\in J} x_j \leq p,& \\
		x_j \in \{0,1\},& \quad \forall j\in J, \\
		y_i \in \{0,1\},& \quad \forall i\in I.
	\end{align*}

\textit{Does the solution to this problem provide an upper bound or a lower bound on the optimal
	solution to the zero-sum bilevel formulation? How does it relate to the objective function
	value attained via the procedure described in Questions 2 and 3? Explain.}
	

\subsection{}
\textit{Would it be reasonable for the upper-level decision-maker to implement a “blocking
	solution” attained by the solution procedure described by Question 4? Why or why not?}


\end{document}