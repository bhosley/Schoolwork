\documentclass[12pt]{amsart}
\usepackage[left=0.5in, right=0.5in, bottom=0.75in, top=0.75in]{geometry}
\usepackage[english]{babel}
\usepackage[utf8x]{inputenc}
\usepackage{amsmath,amssymb,amsthm}
\usepackage{enumerate}
\usepackage{graphicx}

\usepackage{comment}
\usepackage[dvipsnames]{xcolor}
\usepackage[edges]{forest}
\usepackage{caption}
\usepackage{wrapfig}
\newcommand\modulo[2]{\@tempcnta=#1
	\divide\@tempcnta by #2
	\multiply\@tempcnta by #2
	\multiply\@tempcnta by -1
	\advance\@tempcnta by #1\relax
	\the\@tempcnta}

\renewcommand{\thesubsection}{\arabic{subsection}}
\renewcommand{\thesubsubsection}{\quad(\alph{subsubsection})}

\begin{document}
\raggedbottom

\noindent{\large OPER 618 - Game Theory and Math Programming %
	- Homework 4 }
\hspace{\fill} {\large B. Hosley}
\bigskip


%%%%%%%%%%%%%%%%%%%%%%%
%\setcounter{section}{}
%\setcounter{subsection}{}

%%%%%%%%%%%%%%%%%%%%%%%%%%%%%%%
%	\begin{ Question 1 }	  %
%%%%%%%%%%%%%%%%%%%%%%%%%%%%%%%
\subsection{}
\textbf{Perfect-Information Extensive Form Games – Advanced Payment Game.}\textit{ Consider
	internet shopping. There is one seller and one buyer of a good. The seller gets a payoff of
	300 if she does not sell the good and consumes it herself. If the buyer obtains the good and
	consumes it, he gets a payoff of 1000. Currently, the seller is posting the price of 500. Each
	player’s total payoff is the sum of the consumption payoff and the monetary payoff.} \\

\textit{First, the buyer chooses whether to send the payment of 500 or not. If he does not send
	the money, the game ends. If he does, then the seller chooses whether to ship the good or
	not and the game ends. If she ships the good, the buyer consumes the good; otherwise, the
	seller consumes the good.}

\subsubsection{}
\textit{Draw the game tree of this game, specifying the payoff combinations at each terminal
	node, and find the solution by backward induction.} \\
	
	\begin{center}
		\forestset{default preamble={ for tree={l sep=7.5mm, s sep=7.5mm, edge=ultra thick} }} 
		\begin{forest}
			[\textcolor{white}{1}, circle, fill=ForestGreen, draw, font={\bfseries\large}
				[{(500,300)}, edge label={node[midway,left,font=\scriptsize]{Do Not Send Money}} ]
				[\textcolor{white}{2}, edge label={node[midway,right,font=\scriptsize]{Send Money}}, 
				circle, draw, fill=Maroon, font={\bfseries\large} 
					[{(0,800)}, edge label={node[midway,left,font=\scriptsize]{Do Not Ship Good}} ]
					[{(1000,500)}, edge label={node[midway,right,font=\scriptsize]{Ship Good}} ]
				]
			]
		\end{forest}
	\end{center}
	
	Ignoring externalities and assuming self interest, 
	it is assumed that player 2 (the seller) will keep both the money and the good. 
	With this in mind player 1 will not send the money.\\

\subsubsection{}
\textit{Let’s consider a modification to the game structure. If the seller does not ship the good
	after the buyer sent the money, the buyer informs the Consumer Authority and the
	authority fines the seller. Half of the fine will be paid to the buyer, and the other half
	will be taken by the authority as a fee. Let 𝑥𝑥 be the fine. Modify the payoff
	combinations of the game tree you made in (a) and identify the minimum level of $x$ that
	makes (send money, ship the good) the outcome of a solution via backward
	induction.} \\
	
	\begin{center}
		\forestset{default preamble={ for tree={l sep=7.5mm, s sep=7.5mm, edge=ultra thick} }} 
		\begin{forest}
			[\textcolor{white}{1}, circle, fill=ForestGreen, draw, font={\bfseries\large}
				[{(500,300)}, edge label={node[midway,left,font=\scriptsize]{Do Not Send Money}} ]
					[\textcolor{white}{2}, edge label={node[midway,right,font=\scriptsize]{Send Money}}, 
				circle, draw, fill=Maroon, font={\bfseries\large} 
					[{($x/2$,800-$x$)}, edge label={node[midway,left,font=\scriptsize]{Do Not Ship Good}} ]
					[{(1000,500)}, edge label={node[midway,right,font=\scriptsize]{Ship Good}} ]
				]
			]
		\end{forest}
	\end{center}
	
	Using backwards induction we can see that the value of $x$ needs only to be strictly larger than the difference in 
	utility between player 2's choices. In this case, any value $x>300$ fulfills the intent of the Consumer Authority.


\clearpage
%%%%%%%%%%%%%%%%%%%%%%%%%%%%%%%
%	\begin{ Question 2 }	  %
%%%%%%%%%%%%%%%%%%%%%%%%%%%%%%%
\subsection{}
\textbf{Perfect-Information Extensive Form Games – Marienbad.}\textit{ There are two piles of matches
	and two players. The game starts with Player 1 and thereafter the players take turns.
	When it is a player’s turn, he can remove any number of matches from either pile. Each
	player is required to remove some number of matches if either pile has matches remaining,
	and he can only remove matches from one pile at a time. In Marienbad, the player who
	removes the last match loses the game.}

\subsubsection{}
\textit{If the game begins with one match in each pile – i.e., $(1,1)$ – who will win? Show it via
	game tree mapping and backward induction.}
	
	\begin{center}
		\forestset{default preamble={ for tree={l sep=7.5mm, s sep=7.5mm, edge=ultra thick} }} 
		\begin{forest}
			[\textcolor{white}{(1,1)}, circle, fill=ForestGreen, draw, font={\bfseries\scriptsize}
				[\textcolor{white}{(1,0)}, circle, fill=Maroon, draw, font={\bfseries\scriptsize}, 
						edge label={node[midway,right]{$p_1$(-1)}} 
					[\textcolor{white}{(0,0)}, circle, fill=Gray, draw, font={\bfseries\scriptsize}, 
						edge label={node[midway,right]{$p_2$(-1)}}
					]
				]
			]
		\end{forest}
	\end{center}
	
	For this, and the following trees we will treat the piles as distinct but unlabeled.
	I.e. $(1,0)$ represents one toothpick in either pile.
	
	In this game the second player always loses.

\subsubsection{}
\textit{If the game begins with an equal number of matches in each pile, and with at least two
	matches in each pile, who will win? Show it via game tree mapping and backward
	induction for the case of $(2,2)$ and argue the validity of the general result for $(x, x)$.}
	
	\begin{wrapfigure}{l}{0.6\textwidth}
		\centering
		\forestset{default preamble={ for tree={l sep=7.5mm, s sep=17.5mm, edge=ultra thick} }} 
		\begin{forest}
			[\textcolor{white}{(2,2)}, circle, fill=ForestGreen, draw, font={\bfseries\scriptsize},
					delay={for children={circle,fill=Maroon,font={\bfseries\scriptsize}}}
				[\textcolor{white}{(2,0)}, draw, edge label={node[midway,right]{$p_1$(-2)}},
						delay={for children={circle,fill=ForestGreen,font={\bfseries\scriptsize}}}
					[\textcolor{white}{(1,0)}, draw, edge label={node[midway,right]{$p_2$(-1)}}, name=A,
							delay={for children={circle,fill=Gray,font={\bfseries\scriptsize}}}
						[\textcolor{white}{(0,0)}, draw, edge label={node[midway,right]{$p_1$(-1)}}]
					]
				]
				[\textcolor{white}{(2,1)}, draw, edge label={node[midway,right]{$p_1$(-1)}},
						delay={for children={circle,fill=ForestGreen,font={\bfseries\scriptsize}}}
					[\textcolor{white}{(1,0)}, draw, edge label={node[midway,right]{$p_2$(-2)}},
							delay={for children={circle,fill=Gray,font={\bfseries\scriptsize}}}
						[\textcolor{white}{(0,0)}, draw, edge label={node[midway,right]{$p_1$(-1)}}]
					]
					[\textcolor{white}{(1,1)}, draw, edge label={node[midway,right]{$p_2$(-1)}},
							delay={for children={circle,fill=Maroon,font={\bfseries\scriptsize}}}
						[\textcolor{white}{(1,0)}, draw, edge label={node[midway,right]{$p_1$(-1)}},
								delay={for children={circle,fill=Gray,font={\bfseries\scriptsize}}}
							[\textcolor{white}{(0,0)}, draw, edge label={node[midway,right]{$p_2$(-1)}}]
						]
					]
					[\textcolor{white}{(2,0)}, draw, edge label={node[midway,right]{$p_2$(-1)}},
							delay={for children={circle,fill=Maroon,font={\bfseries\scriptsize}}}
						[\textcolor{white}{(1,0)}, draw, edge label={node[midway,right]{$p_1$(-1)}},
							delay={for children={circle,fill=Gray,font={\bfseries\scriptsize}}}
						[\textcolor{white}{(0,0)}, draw, edge label={node[midway,right]{$p_2$(-1)}}]]
					]
				]
			]
		\end{forest}
		\caption{$(2,2)$ Marienbad full tree.}
		\label{fig:2b.1}
	\end{wrapfigure}
	
	\phantom{}\\
	Figure \ref{fig:2b.1} shows the game with 2 in both piles. 
	Regardless of player 1's move, player two can win.
	From this we show that leaving the opponent with only 1 toothpick in 
	either pile while the other is empty is equivalent to the win condition,
	thus the final move will be omitted in future diagrams.
	
	For the next part, using the patterns observed in figure \ref{fig:2b.1}
	to build another tree backward to generalize a strategy.
	Figure \ref{fig:2b.2} represents a generalized version of the same tree.
	In this tree let $2\leq\alpha,\beta<x$. 
	In this tree we can see that player 2 can always prolong the game until an eventual victory.
	
	Note: that $(x,x)$ is somewhat inaccurate, it may be more accurate to say, $(x,x^\prime)$ 
	wherein all of the $x$ that follow that root node may arbitrarily be either. \\

\begin{figure}{}
	\centering
	\newcommand{\labPos}{above left}
	\begin{forest}
		forked edges,
		for tree={%
			%grow'=180,
			grow'=0,
			draw,
			circle,
			font={\bfseries\scriptsize\color{White}},
			l sep=21mm,
		},
		where= {level()==0 || level()==2 || level()==4 || level()==6 }%
		{fill=ForestGreen}%
		{fill=Maroon},
		[{(x,x)}, name=root
			[{(x,0)}, edge label={node[\labPos]{$p_i$($x$)} },
				[{(1,0)}, edge label={node[\labPos]{$p_j$($x-1$)} }, name=node10]
			]
			[{(x,1)}, edge label={node[\labPos]{$p_i$($x-1$)} },
				[{(1,0)}, edge label={node[\labPos]{$p_j$($x$)} }, name=node01, [w]]
			]
			[{(x,$\boldsymbol{\alpha}$)}, edge label={node[\labPos]{$p_i$($x-\alpha$)} },
				[{(x,0)}, edge label={node[\labPos]{$p_j$($\alpha$)} },
					[{(1,0)}, edge label={node[\labPos]{$p_i$($x-1$)} }, name=node33]
				]
				[{($\boldsymbol{\alpha}$,0)}, edge label={node[\labPos]{$p_j$($x$)} },
					[{(1,0)}, edge label={node[\labPos]{$p_i$($\alpha-1$)} },name=node30, [w]]
				]
				[{($\boldsymbol{\alpha}$,1)}, edge label={node[\labPos]{$p_j$($x-1$)} },
					[{(1,0)}, edge label={node[\labPos]{$p_i$($\alpha$)} },name=node03]
				]
				[{($\boldsymbol{\alpha}$,$\boldsymbol\beta$)},edge label={node[\labPos]{$p_j$($x-\beta$)}},name=end
				]
			]
		]
		\draw[dashed,thick] (node01) -- node[] {} (node10);
		\draw[dashed,thick] (node03) -- node[] {} (node30);
		\draw[dashed,thick] (node33) -- node[] {} (node30);
		\draw[->,dashed] (end) to[out=south west,in=south] (root);
	\end{forest}
	\caption{$(x,x)$ Marienbad abridged tree.}
	\label{fig:2b.2}
\end{figure} 



\subsubsection{}
\textit{If the game begins with an unequal number of matches in each pile, and with at least
	one match in the smaller pile, who will win? Show it via game tree mapping and
	backward induction for the case of $(2,1)$ and argue the validity of the general result for
	$(x, m)$. \underline{Hint}: the results for (a) and (b) might help with this argument.} \\
	
	The case for $(2,1)$ can be seen directly in figure \ref{fig:2b.1}. 
	For every other value where $m>2$ it may operate just like the $x^\prime$
	in part b.

\clearpage
%%%%%%%%%%%%%%%%%%%%%%%%%%%%%%%
%	\begin{ Question 3 }	  %
%%%%%%%%%%%%%%%%%%%%%%%%%%%%%%%
\subsection{}
\textbf{Game Tree Search – Alpha Beta Pruning.} 
\textit{Consider the game tree shown below for a two-player, two-stage, 
	zero-sum extensive form game with perfect information, wherein Player
	1 seeks to maximize the utility at a leaf node, and Player 2 seeks to minimize it.}
	
\subsubsection{}
\textit{Apply the Alpha Beta Pruning algorithm, searching the tree \underline{from left-to-right}. Indicate
	your values for $(\alpha,\beta)$ at each node upon the conclusion of the algorithm, and indicate
	branches that can be fathomed (e.g., with an \textcolor{red}{X}). \underline{Hint}: remember to fathom whenever
	$(\alpha\geq\beta)$.}
	
	\begin{center}
	\forestset{default preamble={ for tree={l sep=30mm, s sep=7.5mm, edge=ultra thick} }} 
	\begin{forest}
		[\textcolor{white}{1}, circle, fill=ForestGreen, draw, font={\bfseries\large},
			label={[below right, xshift=3ex, align=left]$\alpha=14$\\
				$\beta=-\infty$},
			delay={for children={fill=Maroon,font={\bfseries\large} }}
			[\textcolor{white}2, circle, draw,
				label={[below right, xshift=3ex, align=left]$\alpha=\infty$ \\ $\beta=12$}
				[15]
				[33]
				[12]]
			[\textcolor{white}2, circle, draw,
				label={[below right, xshift=3ex, align=left]$\alpha=12$ \\ $\beta=10$}
				[15]
				[10]
				[18, edge label={node[midway]{\Huge\textcolor{red}{X}}} ]]
			[\textcolor{white}2, circle, draw,
				label={[below right, xshift=3ex, align=left]$\alpha=14$ \\ $\beta=12$}
				[16]
				[14]
				[21]]
			[\textcolor{white}2, circle, draw,
				label={[below right, xshift=3ex, align=left]$\alpha=14$ \\ $\beta=9$}
				[9]
				[18, edge label={node[midway]{\Huge\textcolor{red}{X}}}]
				[24, edge label={node[midway]{\Huge\textcolor{red}{X}}}]]
		]
	\end{forest}
	\end{center}
	
	
\subsubsection{}
\textit{Apply the Alpha Beta Pruning algorithm, searching the tree from \underline{right-to-left}. Indicate
	your values for $(\alpha,\beta)$ at each node upon the conclusion of the algorithm, and indicate
	branches that can be fathomed (e.g., with an \textcolor{red}{X}).	}
	
	\begin{center}
	\forestset{default preamble={ for tree={l sep=30mm, s sep=7.5mm, edge=ultra thick} }} 
	\begin{forest}
		[\textcolor{white}{1}, circle, fill=ForestGreen, draw, font={\bfseries\large},
			label={[below right, xshift=3ex, align=left]$\alpha=14$\\
				$\beta=-\infty$},
			delay={for children={fill=Maroon,font={\bfseries\large} }}
			[\textcolor{white}2, circle, draw,
				label={[below right, xshift=3ex, align=left]$\alpha=$ \\ $\beta=12$}
				[15, edge label={node[midway]{\Huge\textcolor{red}{X}}}]
				[33, edge label={node[midway]{\Huge\textcolor{red}{X}}}]
				[12]]
			[\textcolor{white}2, circle, draw,
				label={[below right, xshift=3ex, align=left]$\alpha=14$ \\ $\beta=10$}
				[15, edge label={node[midway]{\Huge\textcolor{red}{X}}}]
				[10]
				[18]]
			[\textcolor{white}2, circle, draw,
				label={[below right, xshift=3ex, align=left]$\alpha=14$ \\ $\beta=9$}
				[16]
				[14]
				[21]]
			[\textcolor{white}2, circle, draw,
				label={[below right, xshift=3ex, align=left]$\alpha=\infty$ \\ $\beta=9$}
				[9]
				[18]
				[24]]
		]
	\end{forest}
	\end{center}
	
\clearpage	
%%%%%%%%%%%%%%%%%%%%%%%%%%%%%%%
%	\begin{ Question 4 }	  %
%%%%%%%%%%%%%%%%%%%%%%%%%%%%%%%	
\subsection{}
\textbf{Imperfect Information Extensive Form Games.} 
\textit{Describe a real-world example of such a game. It should have at least two players 
	and a finite number of stages. Depict the game tree for your example.}



%%%%%%%%%%%%%%%%%%%%%%%%%%%%%%%
%	\begin{ Question 5 }	  %
%%%%%%%%%%%%%%%%%%%%%%%%%%%%%%%
\subsection{}
\textbf{Imperfect Information Extensive Form Games with Perfect Recall.}

\subsubsection{}
\textit{What is an advantage of a sequence-form representation of an IIEF game in lieu of a
	matrix form representation?} \\
	
	The sequential form is generally more understandable for conveying information about games over a certain size.
	A matrix representation requires that every distinct pathway to an outcome be enumerated. 
	Additionally, in the sequential representation decisions that a player would never make can terminate, 
	not needing any further enumeration along that pathway. 
	Sequential representation is also more amenable to equivalent game states from different pathways.

\subsubsection{}
\textit{What is an advantage of representing a realization plan in $r_i$ (i.e., as per Definition 5.2.9)
	instead of in $\beta_i$ (i.e., as per Definition 5.2.8).} \\

	One of the largest benefits of this style of representation is that it is algorithmically far more efficient.
	A way of thinking about this is that it acts in a manner similar to Alpha-beta pruning, but preemptively.
	Ignoring branches that are impossible to reach via the player's own decisions saves the time of pointless
	guessing, which is especially valuable in an imperfect information environment.

\end{document}