\documentclass[12pt]{amsart}
\usepackage[left=0.5in, right=0.5in, bottom=0.75in, top=0.75in]{geometry}
\usepackage[english]{babel}
\usepackage[utf8x]{inputenc}
\usepackage{amsmath,amssymb,amsthm}
\usepackage{enumerate}
\usepackage{graphicx}
\usepackage[dvipsnames]{xcolor}
\usepackage[]{forest}

\renewcommand{\thesubsection}{\arabic{subsection}}
\renewcommand{\thesubsubsection}{\quad(\alph{subsubsection})}

\begin{document}
\raggedbottom

\noindent{\large OPER 618 - Game Theory and Math Programming %
	- Homework 4 }
\hspace{\fill} {\large B. Hosley}
\bigskip


%%%%%%%%%%%%%%%%%%%%%%%
%\setcounter{section}{}
%\setcounter{subsection}{}

%%%%%%%%%%%%%%%%%%%%%%%%%%%%%%%
%	\begin{ Question 1 }	  %
%%%%%%%%%%%%%%%%%%%%%%%%%%%%%%%
\subsection{}
\textbf{Perfect-Information Extensive Form Games – Advanced Payment Game.}\textit{ Consider
	internet shopping. There is one seller and one buyer of a good. The seller gets a payoff of
	300 if she does not sell the good and consumes it herself. If the buyer obtains the good and
	consumes it, he gets a payoff of 1000. Currently, the seller is posting the price of 500. Each
	player’s total payoff is the sum of the consumption payoff and the monetary payoff.} \\

\textit{First, the buyer chooses whether to send the payment of 500 or not. If he does not send
	the money, the game ends. If he does, then the seller chooses whether to ship the good or
	not and the game ends. If she ships the good, the buyer consumes the good; otherwise, the
	seller consumes the good.}

\subsubsection{}
\textit{Draw the game tree of this game, specifying the payoff combinations at each terminal
	node, and find the solution by backward induction.}

\subsubsection{}
\textit{Let’s consider a modification to the game structure. If the seller does not ship the good
	after the buyer sent the money, the buyer informs the Consumer Authority and the
	authority fines the seller. Half of the fine will be paid to the buyer, and the other half
	will be taken by the authority as a fee. Let 𝑥𝑥 be the fine. Modify the payoff
	combinations of the game tree you made in (a) and identify the minimum level of 𝑥𝑥 that
	makes (send money, ship the good) the outcome of a solution via backward
	induction.}


%%%%%%%%%%%%%%%%%%%%%%%%%%%%%%%
%	\begin{ Question 2 }	  %
%%%%%%%%%%%%%%%%%%%%%%%%%%%%%%%
\subsection{}
\textbf{Perfect-Information Extensive Form Games – Marienbad.}\textit{ There are two piles of matches
	and two players. The game starts with Player 1 and thereafter the players take turns.
	When it is a player’s turn, he can remove any number of matches from either pile. Each
	player is required to remove some number of matches if either pile has matches remaining,
	and he can only remove matches from one pile at a time. In Marienbad, the player who
	removes the last match loses the game.}

\subsubsection{}
\textit{If the game begins with one match in each pile – i.e., $(1,1)$ – who will win? Show it via
	game tree mapping and backward induction.}

\subsubsection{}
\textit{If the game begins with an equal number of matches in each pile, and with at least two
	matches in each pile, who will win? Show it via game tree mapping and backward
	induction for the case of $(2,2)$ and argue the validity of the general result for $(x, x)$.}

\subsubsection{}
\textit{If the game begins with an unequal number of matches in each pile, and with at least
	one match in the smaller pile, who will win? Show it via game tree mapping and
	backward induction for the case of $(2,1)$ and argue the validity of the general result for
	$(x, m)$. \underline{Hint}: the results for (a) and (b) might help with this argument.}


%%%%%%%%%%%%%%%%%%%%%%%%%%%%%%%
%	\begin{ Question 3 }	  %
%%%%%%%%%%%%%%%%%%%%%%%%%%%%%%%
\subsection{}
\textbf{Game Tree Search – Alpha Beta Pruning.} 
\textit{Consider the game tree shown below for a two-player, two-stage, 
	zero-sum extensive form game with perfect information, wherein Player
	1 seeks to maximize the utility at a leaf node, and Player 2 seeks to minimize it.}
	\begin{center}
		%delay={font=\bfseries}
	\forestset{default preamble={ for tree={l=20mm,font={\bfseries\large}} }} 
	\begin{forest}
		[\textcolor{white}{1}, circle, fill=ForestGreen, draw, delay={for children={fill=Maroon}}
			[\textcolor{white}2, circle, draw
				[15][33][12]]
			[\textcolor{white}2, circle, draw
				[15][10][18]]
			[\textcolor{white}2, circle, draw
				[16][14][21]]
			[\textcolor{white}2, circle, draw
				[9][18][24]]
		]
	\end{forest}
	\end{center}
	
\subsubsection{}
\textit{Apply the Alpha Beta Pruning algorithm, searching the tree \underline{from left-to-right}. Indicate
	your values for $(\alpha,\beta)$ at each node upon the conclusion of the algorithm, and indicate
	branches that can be fathomed (e.g., with an \textcolor{red}{X}). \underline{Hint}: remember to fathom whenever
	$(\alpha\geq\beta)$.}
	
\subsubsection{}
\textit{Apply the Alpha Beta Pruning algorithm, searching the tree from right-to-left. Indicate
	your values for $(\alpha,\beta)$ at each node upon the conclusion of the algorithm, and indicate
	branches that can be fathomed (e.g., with an \textcolor{red}{X}).	}
	
	
%%%%%%%%%%%%%%%%%%%%%%%%%%%%%%%
%	\begin{ Question 4 }	  %
%%%%%%%%%%%%%%%%%%%%%%%%%%%%%%%	
\subsection{}
\textbf{Imperfect Information Extensive Form Games.} 
\textit{Describe a real-world example of such a game. It should have at least two players 
	and a finite number of stages. Depict the game tree for your example.}



%%%%%%%%%%%%%%%%%%%%%%%%%%%%%%%
%	\begin{ Question 5 }	  %
%%%%%%%%%%%%%%%%%%%%%%%%%%%%%%%
\subsection{}
\textbf{Imperfect Information Extensive Form Games with Perfect Recall.}

\subsubsection{}
\textit{What is an advantage of a sequence-form representation of an IIEF game in lieu of a
	matrix form representation?}

\subsubsection{}
\textit{What is an advantage of representing a realization plan in $r_i$ (i.e., as per Definition 5.2.9)
	instead of in $\beta_i$ (i.e., as per Definition 5.2.8).}


\end{document}