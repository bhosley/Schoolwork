\documentclass[12pt]{amsart}
\usepackage[left=0.5in, right=0.5in, bottom=0.75in, top=0.75in]{geometry}
\usepackage[english]{babel}
\usepackage[utf8x]{inputenc}
\usepackage{amsmath,amssymb,amsthm}
\usepackage{enumerate}
\usepackage{graphicx}
\usepackage[table,xcdraw]{xcolor}


\usepackage{xcolor}
\usepackage{xparse}

\renewcommand{\thesection}{}
\renewcommand{\thesubsection}{\arabic{subsection}}
\renewcommand{\thesubsubsection}{\quad(\alph{subsubsection})}

\begin{document}
\raggedbottom

\noindent{\large OPER 618 - Game Theory and Math Programming %
	- Homework 8 }
\hspace{\fill} {\large B. Hosley}
\bigskip


%%%%%%%%%%%%%%%%%%%%%%%
\setcounter{subsection}{0}
For Questions 1-5, consider the following coalition game:

\begin{center}
	{\renewcommand{\arraystretch}{1.2}
	\begin{tabular}{|c|c|}
		\hline
		\rowcolor[HTML]{C0C0C0} 
		\(S\) 		& \(v(S)\) 	\\ \hline
		$\varnothing$ & $0$ 	\\ \hline
		$\{1\}$ 	& $\alpha$ 	\\ \hline
		$\{2\}$ 	& $0.2$ 	\\ \hline
		$\{3\}$ 	& $0.3$ 	\\ \hline
		$\{1,2\}$ 	& $0.4$ 	\\ \hline
		$\{1,3\}$ 	& $\beta$ 	\\ \hline
		$\{2,3\}$ 	& $0.5$ 	\\ \hline
		$N=\{1,2,3\}$ & $1$ 	\\ \hline
	\end{tabular}}
\end{center}

\subsection{}
\textbf{Game Classification.} 
\textit{For what values of $\alpha$ and $\beta$ is the game superadditive? (Provide a plot of $(\alpha,\beta)$ with the relevant domain shaded.)}

\subsection{}
\textbf{Game Classification.} 
\textit{What additional restrictions, if any, are required on $\alpha$ and/or $\beta$ to ensure the game is convex? (Provide a plot of $(\alpha,\beta)$ with the relevant domain shaded.)}

\subsection{}
\textbf{Shapley Value Allocations.} 
\textit{Calculate each the player’s Shapley value allocations as a function of $\alpha$ and $\beta$. For a convex game (i.e., given your answer to Question 2 above), identify the respective values of $(\alpha,\beta)$ that would maximize each player’s allocation in a grand coalition.}

\subsection{}
\textbf{The Core.} 
Construct a two-player coalition game with payoffs $v(s),\ \in S \subseteq N$ with an empty core. Show the core is empty. For the sake of this exercise, ensure $v(\{1\}) \neq v(\{2\})$.

\subsection{}
\textbf{The $\boldsymbol\epsilon$-core.} 
\textit{For the game you created for Question 4, identify the $\epsilon$-core payoff. Discuss the significance of your answer in the context of the game.}



\end{document}