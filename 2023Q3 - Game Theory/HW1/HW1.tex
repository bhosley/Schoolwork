\documentclass[12pt]{amsart}
\usepackage[left=0.5in, right=0.5in, bottom=0.75in, top=0.75in]{geometry}
\usepackage[english]{babel}
\usepackage[utf8x]{inputenc}
\usepackage{amsmath,amssymb,amsthm}
\usepackage{enumerate}
\usepackage{graphicx}
\usepackage[table,xcdraw]{xcolor}


\renewcommand{\thesubsection}{\arabic{subsection}}
\renewcommand\thesubsubsection{\quad(\alph{subsubsection})}


\begin{document}
\raggedbottom

\noindent{\large OPER 618 - Game Theory and Math Programming %
	- Homework 1 }
\hspace{\fill} {\large B. Hosley}
\bigskip


%%%%%%%%%%%%%%%%%%%%%%%
%\setcounter{section}{1}
%\setcounter{subsection}{}
\subsection{}
\emph{ Give three examples of game-like situations from your everyday life. Be sure in each case to
	identify the players, the nature of the interaction, the actions available to each player, and
	each player’s preferences over the possible outcomes (i.e., the action tuples).}

\begin{enumerate}
	\item 
	Handling a dirty dish. This is not a normal form game. The action tuple is to {Place the dish in the sink, rinse the dish and put it in the dishwasher, or hand wash the dish}. Placing the dish in the sink minimizes the current negative utility but increases the value of the negative utility 'jackpot'. Rinsing and placing in the dishwasher has a slightly higher negative utility, but has a finite capacity and has a risk of negative utility associated with emptying a full and clean dishwasher. Washing the dish immediately has the highest negative utility associated with the initial utility. The sink 'jackpot' can be emptied to reduce the negative value, but must be emptied before it reaches capacity.
	\item 
	The choice for two members of an OPER cohort deciding to pack a lunch or not. 
	If they bring lunch they eat in the break room, otherwise they go to a short order restaurant.
	Utility is higher in general for packing a lunch over going out to eat.
	The utility is higher if they eat together. 
	The highest utility for both is if they go to eat at the restaurant together.
	(Really just the common-payoff game)
	\item 
	I have two actions; to grease the pole or not. 
	To do so comes at the cost of time and about \$0.02 worth of crisco.
	The squirrel can attempt to climb the pole to access delicious bird seed, or not.
	The squirrel pays a small cost in the form of energy expended by trying to climb the pole. 
	If the pole is greased the squirrel will fail to access the bird seed; 
	resulting in a greater number of birds at the feeder, increasing both my and my cat's utility.
\end{enumerate}

\phantom{} \\
\hrule

\subsection{}
\emph{Construct a matrix representation for each of the following games:}

\subsubsection{}
\emph{There are two salespeople, Ms. A and Mr. B. They have the same abilities, and their
	sales performance depends only on their effort. They both have the same feasible
	strategies: to make an effort (strategy $E$), or to not make an effort (strategy $N$). They
	chose one of these strategies simultaneously. Their payoffs are the following “points”,
	which are based on their relative performance. \\
	If both players pick the same strategy, then the sales figures are the same, and
	therefore each player gets $1$ point. If one player makes an effort while the other player
	does not, then the player who makes an effort has the higher sales figure, so that (s)he
	gets $3$ points and the other gets 0 points.}\\
	

	This game is represented by the matrix below. Because the results of their choices are 
	symmetric the player assignment on axes are arbitrary. 
	For both players, making an effort is the weakly dominant strategy. \\ 
	
	\begin{center}	
		\begin{tabular}{|
				>{\columncolor[HTML]{EFEFEF}}l |l|l|}
			\hline
			& \cellcolor[HTML]{EFEFEF}E & \cellcolor[HTML]{EFEFEF}N \\ \hline
			E & (1,1)                     & (3,0)                     \\ \hline
			N & (0,3)                     & (1,1)                     \\ \hline
		\end{tabular}
	\end{center}
	\phantom{} \\ [1ex]
	
	
\subsubsection{}
\emph{Consider the same strategic situation, except for the payoff structure. If a player makes
	an effort, (s)he incurs accost of $2.5$ points. The sales points are awarded in the same
	way as in (a).} \\
	
	The updated matrix is below. Once again each player has a symmetric options. 
	The addition of the penalty makes it such that  non-effort becomes the strictly dominant strategy for both.

	\begin{center}
		\begin{tabular}{|
				>{\columncolor[HTML]{EFEFEF}}l |l|l|}
			\hline
			& \cellcolor[HTML]{EFEFEF}E & \cellcolor[HTML]{EFEFEF}N \\ \hline
			E & (-1.5,-1.5)               & (0.5,0)                   \\ \hline
			N & (0,0.5)                   & (1,1)                     \\ \hline
		\end{tabular}
	\end{center}
	
	\phantom{} \\
	\hrule
	
\subsection{}
\emph{Construct the matrix representation of the following game.
There is a duopoly market in which firm $1$ and firm $2$ are the only producers. Each firm
chooses a price in ${1, 2,3,4}$ simultaneously. The payoff of a firm $i\in {1,2}$ is its sales, which
is its price multiplied by its demand. The demand of a firm $i \in {1,2}$ is determined as
follows. Let $p_i$ be firm $i$’s price and $p_j$ be the opponent’s price. The demand of firm $i$ is}

\[
D_i(p_i,p_j)= \begin{cases}
	(4.6-p_i) & \text{if }p_i<p_j \\
	\frac{1}{2}(4.6-p_i) & \text{if }p_i=p_j \\
	0 & \text{if }p_i>p_j
\end{cases}\]

Once again the results are symmetric and the axes arbitrary; matrix is presented below.

\begin{center}
	\begin{tabular}{|
			>{\columncolor[HTML]{EFEFEF}}c |c|c|c|c|}
		\hline
		& \cellcolor[HTML]{EFEFEF}1 & \cellcolor[HTML]{EFEFEF}2 & \cellcolor[HTML]{EFEFEF}3 & \cellcolor[HTML]{EFEFEF}4 \\ \hline
		1 & (1.8,1.8)  & (3.6,0)     & (3.6,0)     & (3.6,0)     \\ \hline
		2 & (0,3.6)    & (2.6,2.6)   & (5.2,0)     & (5.2,0)     \\ \hline
		3 & (0,3.6)    & (0,5.2)     & (2.4,2.4)   & (4.8,0)     \\ \hline
		4 & (0,3.6)    & (0,5.2)     & (0,4.8)     & (1.2,1.2)   \\ \hline
	\end{tabular}
\end{center}


\phantom{} \\
\hrule

\subsection{}
\textit{\textbf{Dividing Money.} Two people have \$5 to divide between themselves. They use the
following process to divide the money. Each person names a number of dollars (a positive
integer), at most equal to 5. If the sum of the amounts that the people name is at most 5
then each person receives the amount of money she names (and the remainder is
destroyed). If the sum of the amounts that the people name exceeds 5 and the amounts
named are different, then the person who names the smaller amount receives that amount
and the other person receives the remaining money. If the sum of the amounts that the
people name exceeds 5 and the amounts named are the same then each person receives
\$2.50. \\
Model the situation as a strategic game, determine the best response of each player to each
of the other player’s actions. Is there a combination of pure strategies that are best
responses to each other?} \\

\begin{center}
	\begin{tabular}{|
			>{\columncolor[HTML]{DAE8FC}}c |c|c|c|c|c|}
		\hline
		\cellcolor[HTML]{EFEFEF} & \cellcolor[HTML]{FFCCC9}1     & \cellcolor[HTML]{FFCCC9}2     & \cellcolor[HTML]{FFCCC9}3         & \cellcolor[HTML]{FFCCC9}4     & \cellcolor[HTML]{FFCCC9}5     \\ \hline
		1                        & (1,1)                         & (1,2)                         & (1,3)                             & (1,4)                         & (1,4)                         \\ \hline
		2                        & (2,1)                         & (2,2)                         & (2,3)                             & (2,3)                         & (2,3)                         \\ \hline
		3                        & (3,1)                         & \cellcolor[HTML]{ECF4FF}(3,2) & \cellcolor[HTML]{ECF4FF}(2.5,2.5) & \cellcolor[HTML]{ECF4FF}(3,2) & (3,2)                         \\ \hline
		4                        & \cellcolor[HTML]{ECF4FF}(4,1) & \cellcolor[HTML]{ECF4FF}(3,2) & (2,3)                             & (2.5,2.5)                     & \cellcolor[HTML]{ECF4FF}(4,1) \\ \hline
		5                        & \cellcolor[HTML]{ECF4FF}(4,1) & \cellcolor[HTML]{ECF4FF}(3,2) & (2,3)                             & (1,4)                         & (2.5,2.5)                     \\ \hline
	\end{tabular} \\ \phantom{}
\end{center}

The above matrix represents the game with the optimal response for blue given the column red's decision.
There is no dominant pure strategy. Evaluated from the perspective of row player (blue) 
choosing 3 or 4 have the same number of outcomes in which blue maximizes their return.
4 has a higher expected value overall.
However, considering that 1 and 2 are strongly dominated it seems reasonable to expect player 2
to never pick those numbers.
Additionally, 5 is weakly dominated by 4.
This being the case, since 3 is dominant in all cases except when the opponent chooses 1 or 5, 
it is likely the best choice.

\phantom{}
\hrule

\subsection{}
\textit{\textbf{Swimming with sharks}. You and a friend are spending two days at an isolated beach and
	would like to go for a swim. With probability $\pi$, the water is infested with sharks, in which
	case anyone swimming will be attacked. You each have preferences represented by the
	expected value of a payoff function that assigns $-c$ to being attacked by a shark, $0$ to sitting
	on the beach, and 1 to a day’s worth of undisturbed swimming. \\ \\
	If one of you is attacked by sharks on the first day, then you both deduce that a swimmer
	will surely be attacked the next day, and hence do not go swimming the next day. If either
	of you goes swimming on the first day and is not attacked, you will both go swimming on
	the second day. If no one goes swimming on the first day, you both retain the belief that
	the probability of the water’s being infested is $\pi$. \\ \\
	Model this situation as a strategic game in which you and your friend each decides whether
	to go swimming on your first day at the beach.} \\
	
	The matrix representation of this game can be seen below. It is built under the assumption that if neither swim on the first day that both will swim on the second. \\
	
	\begin{center}
		\begin{tabular}{|
				>{\columncolor[HTML]{DAE8FC}}c |c|c|}
			\hline
			\cellcolor[HTML]{EFEFEF} & \cellcolor[HTML]{FFCCC9}0 & \cellcolor[HTML]{FFCCC9}1   \\ \hline
			0                        & ($1-\pi,\ 1-\pi$)         & ($1-\pi,\ 2-\pi(c+1)$)      \\ \hline
			1                        & ($2-\pi(c+1),\ 1-\pi$)    & ($2-\pi(c+1),\ 2-\pi(c+1)$) \\ \hline
		\end{tabular}
	\end{center}

\end{document}