\documentclass[12pt]{amsart}
\usepackage[left=0.5in, right=0.5in, bottom=0.75in, top=0.75in]{geometry}
\usepackage[english]{babel}
\usepackage[utf8x]{inputenc}
\usepackage{amsmath,amssymb,amsthm}
\usepackage{enumerate}
\usepackage{graphicx}

\renewcommand{\thesubsection}{\arabic{subsection}}
\renewcommand\thesubsubsection{\quad(\alph{subsubsection})}

\begin{document}
\raggedbottom

\noindent{\large OPER 618 - Game Theory and Math Programming %
	- Homework 1 }
\hspace{\fill} {\large B. Hosley}
\bigskip


%%%%%%%%%%%%%%%%%%%%%%%
%\setcounter{section}{1}
%\setcounter{subsection}{}
\subsection{}
\emph{ Give three examples of game-like situations from your everyday life. Be sure in each case to
	identify the players, the nature of the interaction, the actions available to each player, and
	each player’s preferences over the possible outcomes (i.e., the action tuples).}

\subsection{}
\emph{Construct a matrix representation for each of the following games:}

\subsubsection{}
\emph{There are two salespeople, Ms. A and Mr. B. They have the same abilities, and their
	sales performance depends only on their effort. They both have the same feasible
	strategies: to make an effort (strategy $E$), or to not make an effort (strategy $N$). They
	chose one of these strategies simultaneously. Their payoffs are the following “points”,
	which are based on their relative performance. \\
	If both players pick the same strategy, then the sales figures are the same, and
	therefore each player gets $1$ point. If one player makes an effort while the other player
	does not, then the player who makes an effort has the higher sales figure, so that (s)he
	gets $3$ points and the other gets 0 points.}
	
\subsubsection{}
\emph{Consider the same strategic situation, except for the payoff structure. If a player makes
	an effort, (s)he incurs accost of $2.5$ points. The sales points are awarded in the same
	way as in (a).}

\subsection{}
\emph{Construct the matrix representation of the following game.
There is a duopoly market in which firm $1$ and firm $2$ are the only producers. Each firm
chooses a price in ${1, 2,3,4}$ simultaneously. The payoff of a firm $i\in {1,2}$ is its sales, which
is its price multiplied by its demand. The demand of a firm $i \in {1,2}$ is determined as
follows. Let $p_i$ be firm $i$’s price and $p_j$ be the opponent’s price. The demand of firm $i$ is}

\[
D_i(p_i,p_j)= \begin{cases}
	(4.6-p_i) & \text{if }p_i<p_j \\
	\frac{1}{2}(4.6-p_i) & \text{if }p_i=p_j \\
	0 & \text{if }p_i>p_j
\end{cases}\]

\subsection{}
\textit{\textbf{Dividing Money.} Two people have \$5 to divide between themselves. They use the
following process to divide the money. Each person names a number of dollars (a positive
integer), at most equal to 5. If the sum of the amounts that the people name is at most 5
then each person receives the amount of money she names (and the remainder is
destroyed). If the sum of the amounts that the people name exceeds 5 and the amounts
named are different, then the person who names the smaller amount receives that amount
and the other person receives the remaining money. If the sum of the amounts that the
people name exceeds 5 and the amounts named are the same then each person receives
\$2.50. \\
Model the situation as a strategic game, determine the best response of each player to each
of the other player’s actions. Is there a combination of pure strategies that are best
responses to each other?}

\subsection{}
\textit{\textbf{Swimming with sharks}. You and a friend are spending two days at an isolated beach and
	would like to go for a swim. With probability $\pi$, the water is infested with sharks, in which
	case anyone swimming will be attacked. You each have preferences represented by the
	expected value of a payoff function that assigns $-c$ to being attacked by a shark, $0$ to sitting
	on the beach, and 1 to a day’s worth of undisturbed swimming. \\
	If one of you is attacked by sharks on the first day, then you both deduce that a swimmer
	will surely be attacked the next day, and hence do not go swimming the next day. If either
	of you goes swimming on the first day and is not attacked, you will both go swimming on
	the second day. If no one goes swimming on the first day, you both retain the belief that
	the probability of the water’s being infested is $\pi$. \\
	Model this situation as a strategic game in which you and your friend each decides whether
	to go swimming on your first day at the beach.}

\end{document}