\documentclass[12pt]{amsart}
\usepackage[left=0.5in, right=0.5in, bottom=0.75in, top=0.75in]{geometry}
\usepackage[english]{babel}
\usepackage[utf8x]{inputenc}
\usepackage{amsmath,amssymb,amsthm}
\usepackage{enumerate}
\usepackage{graphicx}
\usepackage{multirow}

\renewcommand{\thesubsection}{\arabic{subsection}}
\renewcommand{\thesubsubsection}{\quad(\alph{subsubsection})}

\begin{document}
\raggedbottom

\noindent{\large OPER 618 - Game Theory and Math Programming %
	- Homework 2 }
\hspace{\fill} {\large B. Hosley}
\bigskip


%%%%%%%%%%%%%%%%%%%%%%%
%\setcounter{section}{}
%\setcounter{subsection}{}
\subsection{}
\textit{\textbf{Selfish and altruistic social behavior}. Two people enter a bus. Two adjacent cramped seats
	are free. Each person must decide whether to sit or stand. Sitting alone is more comfortable
	than sitting next to the other person, which is more comfortable than standing.}
	
	\subsubsection{}
	\textit{Suppose that each person cares only about her own comfort. Model the situation as a
	strategic game. Is this game the Prisoner’s Dilemma? Find its Nash equilibrium
	(equilibria?).}
	
	\begin{center}
		\def\arraystretch{1.25}%
		\begin{tabular}{cccc}
			&                            & \multicolumn{2}{c}{Player 2}                           \\
			& \multicolumn{1}{c|}{}      & \multicolumn{1}{c|}{Sit}  & \multicolumn{1}{c|}{Stand} \\ \cline{2-4} 
			\multirow{2}{*}{Player 1} & \multicolumn{1}{c|}{Sit}   & \multicolumn{1}{c|}{1, 1} & \multicolumn{1}{c|}{2, 0}  \\ \cline{2-4} 
			& \multicolumn{1}{c|}{Stand} & \multicolumn{1}{c|}{0, 2} & \multicolumn{1}{c|}{0, 0}  \\ \cline{2-4} 
		\end{tabular}
	\end{center}
	
	This is not a prisoner's dilemma type of game, 
	it is more similar to a coordination game.
	The Nash equilibrium in this game is [sit, sit].
	
	\subsubsection{}
	\textit{Suppose that each person is altruistic, ranking the outcomes according to the other
	person’s comfort, and, out of politeness, prefers to stand than to sit if the other person
	stands. Model the situation as a strategic game. Is this game the Prisoner’s Dilemma?
	Find its Nash equilibrium (equilibria?).}
	
	\begin{center}
		\def\arraystretch{1.25}%
		\begin{tabular}{cccc}
			&                            & \multicolumn{2}{c}{Player 2}                           \\
			& \multicolumn{1}{c|}{}      & \multicolumn{1}{c|}{Sit}  & \multicolumn{1}{c|}{Stand} \\ \cline{2-4} 
			\multirow{2}{*}{Player 1} & \multicolumn{1}{c|}{Sit}   & \multicolumn{1}{c|}{1, 1} & \multicolumn{1}{c|}{2, 0}  \\ \cline{2-4} 
			& \multicolumn{1}{c|}{Stand} & \multicolumn{1}{c|}{0, 2} & \multicolumn{1}{c|}{3, 3}  \\ \cline{2-4} 
		\end{tabular}
	\end{center}
	
	This situation is a prisoner's dilemma, there are Nash equilibria at [sit, sit] and [stand, stand].
	
	\subsubsection{}
	\textit{Compare the people’s comfort in the equilibria of the two games.} \\

	Because the rewards in each situation are ordinal, they may not accurately represent the proportional
	reward between each of the choices. Similarly, the scale may be different between the two games being compared.
	The mutually selfish equilibria seem that they should be equivalent in terms of utility between the two games, 
	and if the assumption that the [sit, sit] outcome is indeed identical between the two situations
	then the ordinality of the utilities should be transitively preserved.
	If this is the case, then the highest utility is the mutually altruistic action(s), which ironically
	represent the lowest comfort game state. \\


\clearpage


\subsection{}
\textit{\textbf{Finding Nash equilibria for normal form games with continuous action spaces}. 
	Find the Nash equilibria of the two-player strategic game in which each player’s set of actions is the
	set of nonnegative numbers and the players’ payoff functions are}
	$u_1(a_1,a_2) =a_1(a_2-a_1)$ \textit{and} $u_2(a_1,a_2) = a_2(1-a_1-a_2)$. \\

	For player 1 we have,
	\begin{align*}
		u_1(a_1,a_2) &= a_1(a_2-a_1) \\
					 &= a_1a_2-a_1^2 \\
		\frac{\partial\,u_1(a_1,a_2)}{\partial\,a_2} &= a_2-2a_1.
	\end{align*}
	
	For player 2 we have,
	\begin{align*}
		u_2(a_1,a_2) &= a_2(1-a_1-a_2) \\
					 &= a_2-a_1a_2-a_2^2 \\
		\frac{\partial\,u_2(a_1,a_2)}{\partial\,a_2} &= 1-a_1-2a_2.
	\end{align*}
	
	Solving the two partials,
	\begin{align*}
		2a_1 &= a_2 = \frac{1-a_1}{2} \\
		4a_1 &= 1-a_1 \\
		5a_1 &= 1.
	\end{align*}

	And finally, we find a single Nash equilibrium where the combined player actions are represented by the vector
	\[\mathbf A = \begin{bmatrix} 0.2 \\0.4 \end{bmatrix}.\]\\

\subsection{}
\textit{\textbf{The Odd Couple}. Felix and Oscar share an apartment. They have decidedly different views
	on cleanliness and, hence, on whether or not they would be willing to put in the hours of
	work necessary to clean the apartment. Suppose that it takes at least twelve hours of work
	(per week) to keep the apartment clean, nine hours to make it livable, and anything less
	than nine hours leaves the apartment filthy. Suppose that each person can devote either
	three, six, or nine hours to cleaning.}
	
\textit{Felix and Oscar agree that a livable apartment is worth 2 on the utility index. They
	disagree on the value of clean apartment – Felix thinks it is worth 10 utils (i.e., utility units),
	while Oscar thinks it is worth only 5. They also disagree on the unpleasantness of a filthy
	apartment. Felix thinks it is worth -10 utils, while Oscar thinks it is worth -5. Each person’s
	payoff is the utility from the apartment minus the number of hours worked.}

\textit{Model this interaction as a normal form game and provide its matrix representation.
	Identify what each of the following solution concepts can tell you about how Felix and Oscar
	should play the game: (1) Nash equilibrium in pure strategies, (2) maxmin strategies, (3)
	minmax strategies, (4) minimax regret, (5) iterated dominance. Discuss the relationship
	between the various strategies.} \\
	
	Matrix representation: \\
	
	\begin{center}
		\def\arraystretch{1.25}%
		\begin{tabular}{ccccc}
			&                        & \multicolumn{3}{c}{Oscar}                                                               \\
			& \multicolumn{1}{c|}{}  & \multicolumn{1}{c|}{3}       & \multicolumn{1}{c|}{6}      & \multicolumn{1}{c|}{9}     \\ \cline{2-5} 
			\multirow{3}{*}{Felix} & \multicolumn{1}{c|}{3} & \multicolumn{1}{c|}{-13, -8} & \multicolumn{1}{c|}{-1, -4} & \multicolumn{1}{c|}{7, -4} \\ \cline{2-5} 
			& \multicolumn{1}{c|}{6} & \multicolumn{1}{c|}{-4, -1}  & \multicolumn{1}{c|}{4, -1}  & \multicolumn{1}{c|}{4, -4} \\ \cline{2-5} 
			& \multicolumn{1}{c|}{9} & \multicolumn{1}{c|}{1, 2}    & \multicolumn{1}{c|}{1, -1}  & \multicolumn{1}{c|}{1, -4} \\ \cline{2-5} 
		\end{tabular} \\[3ex]
	\end{center}
	
	\textbf{NE with Pure-strategy} occurs at actions [3, 9], [6, 6], and [9, 3].
	
	\textbf{Maxmin}:
	Felix will choose 9hrs;	
	Oscar, 6 or 9hrs possibly favoring 6 as it has a higher utility overall.
	
	\textbf{Minmax}:
	Felix: 3 hrs; 
	Oscar: 3 hrs.
	
	\textbf{Minimax Regret}:
	Felix: 9 hrs;
	Oscar: 9 hrs.
	
	\textbf{Iterated Dominance}:
	
	Oscar-9 is weakly dominated by 6,
	
	\begin{center}
		\def\arraystretch{1.25}%
		\begin{tabular}{cccc}
			&                        & \multicolumn{2}{c}{Oscar}                                  \\
			& \multicolumn{1}{c|}{}  & \multicolumn{1}{c|}{3}       & \multicolumn{1}{c|}{6}      \\ \cline{2-4} 
			\multirow{3}{*}{Felix} & \multicolumn{1}{c|}{3} & \multicolumn{1}{c|}{-13, -8} & \multicolumn{1}{c|}{-1, -4} \\ \cline{2-4} 
			& \multicolumn{1}{c|}{6} & \multicolumn{1}{c|}{-4, -1}  & \multicolumn{1}{c|}{4, -1}  \\ \cline{2-4} 
			& \multicolumn{1}{c|}{9} & \multicolumn{1}{c|}{1, 2}    & \multicolumn{1}{c|}{1, -1}  \\ \cline{2-4} 
		\end{tabular}
	\end{center}
	
	Felix-3 is strictly dominated by the other two actions
	
	\begin{center}
		\def\arraystretch{1.25}%
		\begin{tabular}{cccc}
			&                        & \multicolumn{2}{c}{Oscar}                                \\
			& \multicolumn{1}{c|}{}  & \multicolumn{1}{c|}{3}      & \multicolumn{1}{c|}{6}     \\ \cline{2-4} 
			\multirow{2}{*}{Felix} & \multicolumn{1}{c|}{6} & \multicolumn{1}{c|}{-4, -1} & \multicolumn{1}{c|}{4, -1} \\ \cline{2-4} 
			& \multicolumn{1}{c|}{9} & \multicolumn{1}{c|}{1, 2}   & \multicolumn{1}{c|}{1, -1} \\ \cline{2-4} 
		\end{tabular}
	\end{center}
	
	Oscar-6 is weakly dominated by 3
	
	\begin{center}
		\def\arraystretch{1.25}%
		\begin{tabular}{ccc}
			&                        & Oscar                       \\
			& \multicolumn{1}{c|}{}  & \multicolumn{1}{c|}{3}      \\ \cline{2-3} 
			\multirow{2}{*}{Felix} & \multicolumn{1}{c|}{6} & \multicolumn{1}{c|}{-4, -1} \\ \cline{2-3} 
			& \multicolumn{1}{c|}{9} & \multicolumn{1}{c|}{1, 2}   \\ \cline{2-3} 
		\end{tabular} \\[2ex]
	\end{center}

	Felix-9 is strictly dominant here, leading to a final [9,3] cleaning regime. 
	Consequently, this is the only set of actions that has a positive utility for both players. \\
	
	It seems that while some of the strategies select the same actions as the EID selected strategy,
	most of the competitive strategies appear to provide at least one of the players with a suboptimal 
	choice regarding their own interest.
	This is consistent with the intuition that this particular game has a significant cooperative aspect.

\subsection{}
\textit{\textbf{Dominance}. Consider the normal form game in the following matrix}

\begin{center}
	\def\arraystretch{1.25}%
	\begin{tabular}{cccccc}
		&                            &                            & \multicolumn{2}{c}{Player 2}                            &                            \\
		& \multicolumn{1}{c|}{}      & \multicolumn{1}{c|}{$A_2$} & \multicolumn{1}{c|}{$B_2$} & \multicolumn{1}{c|}{$C_2$} & \multicolumn{1}{c|}{$D_2$} \\ \cline{2-6} 
		& \multicolumn{1}{c|}{$A_1$} & \multicolumn{1}{c|}{5, 0}  & \multicolumn{1}{c|}{1, 1}  & \multicolumn{1}{c|}{0, 2}  & \multicolumn{1}{c|}{0, 2}  \\ \cline{2-6} 
		\multirow{2}{*}{Player 1} & \multicolumn{1}{c|}{$B_1$} & \multicolumn{1}{c|}{1, 6}  & \multicolumn{1}{c|}{1, 4}  & \multicolumn{1}{c|}{2, 5}  & \multicolumn{1}{c|}{1, 3}  \\ \cline{2-6} 
		& \multicolumn{1}{c|}{$C_1$} & \multicolumn{1}{c|}{2, 0}  & \multicolumn{1}{c|}{4, 0}  & \multicolumn{1}{c|}{1, 0}  & \multicolumn{1}{c|}{0, 4}  \\ \cline{2-6} 
		& \multicolumn{1}{c|}{$D_1$} & \multicolumn{1}{c|}{0, 1}  & \multicolumn{1}{c|}{1, 0}  & \multicolumn{1}{c|}{0, 1}  & \multicolumn{1}{c|}{4, 0}  \\ \cline{2-6} 
	\end{tabular}
\end{center}

	\subsubsection{}
	\textit{Find the mixed strategy for Player 2 that maximally dominates strategy $B_2$. \\[3ex] %
%
	\underline{Hint}: You can readily solve this problem via a linear program that maximizes the
	minimum amount of domination over all possible Player 1 actions. As an example
	calculation, if Player 2 plays $(p_{A_2},p_{C_2},p_{D_2}) = (0.5,0.5,0)$, 
	the level of domination is $1 − 1 = 0$ for $A_1, 5.5 - 4 = 1.5$ for 
	$B_1, 0 - 0 = 0$ for $C_1$, and $1 − 0 = 1$ for $D_1$, resulting in a
	minimal level of domination of $0$ for this mixed strategy.}
	
	\begin{align*} 
		\text{Maximize} &\left\{
		\begin{bmatrix}
			0 & 2 & 2 \\
			6 & 5 & 3 \\
			0 & 0 & 4 \\
			1 & 1 & 0 
		\end{bmatrix} 
		\begin{bmatrix}
			p_{A_2} \\ p_{C_2} \\ p_{D_2}
		\end{bmatrix} -
		\begin{bmatrix}
			1 \\ 4 \\ 0 \\ 0
		\end{bmatrix}
		\right\} \\
		\text{s.t. }\qquad 
		&p_{A_2} + p_{C_2} + p_{D_2} = 1 \\
		&p_{A_2},p_{C_2},p_{D_2} \geq 0 
	\end{align*}
	
	\(\mathbf P_2 = \begin{bmatrix}0.1212225 & 0.6919 & 0.1868775\end{bmatrix}\) \\

	\textbf{Note*} After several failures with pyomo I had to resort to an 
	embarrassingly inelegant solution in Mac Numbers\textsuperscript{\texttrademark}.

	\subsubsection{}
	\textit{After eliminating $B_2$ from consideration, find the mixed strategy for Player 1 that
	maximally dominates strategy $C_1$.}
	
	\begin{center}
		\def\arraystretch{1.25}%
		\begin{tabular}{ccccc}
			&                            & \multicolumn{3}{c}{Player 2}                                                         \\
			& \multicolumn{1}{c|}{}      & \multicolumn{1}{c|}{$A_2$} & \multicolumn{1}{c|}{$C_2$} & \multicolumn{1}{c|}{$D_2$} \\ \cline{2-5} 
			\multirow{4}{*}{Player 1} & \multicolumn{1}{c|}{$A_1$} & \multicolumn{1}{c|}{5, 0}  & \multicolumn{1}{c|}{0, 2}  & \multicolumn{1}{c|}{0, 2}  \\ \cline{2-5} 
			& \multicolumn{1}{c|}{$B_1$} & \multicolumn{1}{c|}{1, 6}  & \multicolumn{1}{c|}{2, 5}  & \multicolumn{1}{c|}{1, 3}  \\ \cline{2-5} 
			& \multicolumn{1}{c|}{$C_1$} & \multicolumn{1}{c|}{2, 0}  & \multicolumn{1}{c|}{1, 0}  & \multicolumn{1}{c|}{0, 4}  \\ \cline{2-5} 
			& \multicolumn{1}{c|}{$D_1$} & \multicolumn{1}{c|}{0, 1}  & \multicolumn{1}{c|}{0, 1}  & \multicolumn{1}{c|}{4, 0}  \\ \cline{2-5} 
		\end{tabular}
	\end{center}

	\(\mathbf P_1 = \begin{bmatrix}0.15 & 0.68 & 0.17\end{bmatrix}\)

	\subsubsection{}
	\textit{After eliminating $C_1$ from consideration, eliminate iteratively the other strictly and
	weakly dominated strategies.} \\
	
	\begin{center}
		\def\arraystretch{1.25}%
		\begin{tabular}{ccccc}
			&                            & \multicolumn{3}{c}{Player 2}                                                         \\
			& \multicolumn{1}{c|}{}      & \multicolumn{1}{c|}{$A_2$} & \multicolumn{1}{c|}{$C_2$} & \multicolumn{1}{c|}{$D_2$} \\ \cline{2-5} 
			\multirow{3}{*}{Player 1} & \multicolumn{1}{c|}{$A_1$} & \multicolumn{1}{c|}{5, 0}  & \multicolumn{1}{c|}{0, 2}  & \multicolumn{1}{c|}{0, 2}  \\ \cline{2-5} 
			& \multicolumn{1}{c|}{$B_1$} & \multicolumn{1}{c|}{1, 6}  & \multicolumn{1}{c|}{2, 5}  & \multicolumn{1}{c|}{1, 3}  \\ \cline{2-5} 
			& \multicolumn{1}{c|}{$D_1$} & \multicolumn{1}{c|}{0, 1}  & \multicolumn{1}{c|}{0, 1}  & \multicolumn{1}{c|}{4, 0}  \\ \cline{2-5} 
		\end{tabular}\\[2ex]
	\end{center}
	
	$D_2$ is weakly dominated by $C_2$, and after that elimination, 
	$D_1$ is dominated by both of the other two strategies.
	
	\begin{center}
		\def\arraystretch{1.25}%
		\begin{tabular}{cccc}
			&                            & \multicolumn{2}{c}{Player 2}                            \\
			& \multicolumn{1}{c|}{}      & \multicolumn{1}{c|}{$A_2$} & \multicolumn{1}{c|}{$C_2$} \\ \cline{2-4} 
			\multirow{2}{*}{Player 1} & \multicolumn{1}{c|}{$A_1$} & \multicolumn{1}{c|}{5, 0}  & \multicolumn{1}{c|}{0, 2}  \\ \cline{2-4} 
			& \multicolumn{1}{c|}{$B_1$} & \multicolumn{1}{c|}{1, 6}  & \multicolumn{1}{c|}{2, 5} 
		\end{tabular}
	\end{center}
	
 
	\subsubsection{}
	\textit{How many pure strategy Nash equilibria exist in the resulting, reduced game?} \\

	There are no pure strategy Nash equilibria remaining.


\subsection{}
\textit{\textbf{Nash Equilibria}. Consider the normal form game in the following matrix}

\begin{center}
	\def\arraystretch{1.25}%
	\begin{tabular}{ccccc}
		&                            & \multicolumn{3}{c}{Player 2}                                                         \\
		& \multicolumn{1}{c|}{}      & \multicolumn{1}{c|}{$A_2$} & \multicolumn{1}{c|}{$B_2$} & \multicolumn{1}{c|}{$C_2$} \\ \cline{2-5} 
		& \multicolumn{1}{c|}{$A_1$} & \multicolumn{1}{c|}{1, 4}  & \multicolumn{1}{c|}{2, 5}  & \multicolumn{1}{c|}{5, 3}  \\ \cline{2-5} 
		Player 1 & \multicolumn{1}{c|}{$B_1$} & \multicolumn{1}{c|}{7, 2}  & \multicolumn{1}{c|}{1, 0}  & \multicolumn{1}{c|}{4, 0}  \\ \cline{2-5} 
		& \multicolumn{1}{c|}{$C_1$} & \multicolumn{1}{c|}{5, 10} & \multicolumn{1}{c|}{0, 2}  & \multicolumn{1}{c|}{5, 3}  \\ \cline{2-5} 
	\end{tabular} \\[2ex]
\end{center}

\textit{Eliminate iteratively the strictly dominated strategies. Find all the Nash equilibria (in pure
	and mixed strategies) from the reduced game and provide the mixed Nash equilibria
	payoffs.} \\
	
	First, $C_2$ is eliminated being strictly dominated by $A_2$,
	which makes $C_1$ strictly dominated by $B_1$.

	\begin{center}
		\def\arraystretch{1.25}%
		\begin{tabular}{cccc}
			&                            & \multicolumn{2}{c}{Player 2}                            \\
			& \multicolumn{1}{c|}{}      & \multicolumn{1}{c|}{$A_2$} & \multicolumn{1}{c|}{$B_2$} \\ \cline{2-4} 
			\multirow{2}{*}{Player 1} & \multicolumn{1}{c|}{$A_1$} & \multicolumn{1}{c|}{1, 4}  & \multicolumn{1}{c|}{2, 5}  \\ \cline{2-4} 
			& \multicolumn{1}{c|}{$B_1$} & \multicolumn{1}{c|}{7, 2}  & \multicolumn{1}{c|}{1, 0}  \\ \cline{2-4} 
		\end{tabular} \\[1ex]
	\end{center}
	
	$[A_1,B_2]$ and $[B_1,A_2]$ are both pure strategy Nash equilibria; with expected payoffs of $[2,5]$ and $[7,2]$ respectively.
	
	Using $p_i$ as the probability that player $i$ chooses action $A_i$.
	\begin{align*}
		4p_i + 2(1-p) &= 5p_1 + 0 \\
		2-2p_1 &= p_1 \\
		p_1 &= \frac{2}{3}
		\intertext{and }
		p_2+5(1-p_2) &= 7p_2 + (1-p_2) \\
		6p_2 &= 4-4p_2 \\
		10p_2 &= 4 \\
		p_2 &= \frac{2}{5}
	\end{align*}
	
	giving a third Nash equilibrium with the mixed strategy of 
	$s_1=\begin{bmatrix} 2/3 \\ 1/3	\end{bmatrix}$ and
	$s_2=\begin{bmatrix} 2/5 \\ 3/5	\end{bmatrix}$ with an
	expected payoff of $[2.2,10/3]$.

\subsection{}
\textit{\textbf{Correlated equilibrium}. For the reduced game from the previous problem, identify a
	correlated equilibrium that maximizes the minimum utility increase relative to the mixed strategy Nash equilibrium utility, over both players.} \\
	
	Original solution was found via the awful Numbers\textsuperscript{\texttrademark} solution that I kludged together.
	The result obtained was evaluated using a graphing method, seen below.
	The updated strategies of 
	$s_1=\begin{bmatrix} 0.765 \\ 0.235	\end{bmatrix}$ and
	$s_2=\begin{bmatrix} 0.985 \\ 0.015	\end{bmatrix}$
	which increases the utility of both players by 0.2.
	
	\begin{center}
		\includegraphics[width=0.7\linewidth]{"Screenshot 2023-07-16 at 4.20.52 PM"}
	\end{center}


\end{document}