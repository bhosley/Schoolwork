%\documentclass[12pt]{journal}
\documentclass[sigplan,nonacm]{acmart}
\usepackage[english]{babel}
\usepackage[utf8x]{inputenc}

\usepackage{amsmath,amssymb,amsthm}
\usepackage{tcolorbox}
\newtcbox{\inlinecode}{on line, boxrule=0pt, boxsep=0pt, top=2pt, left=2pt, bottom=2pt, right=2pt, colback=gray!15, colframe=white, fontupper={\ttfamily \footnotesize}}

%\usepackage{enumerate}
\usepackage{graphicx}
\usepackage{booktabs,tabularx}
\usepackage{caption}
\usepackage{float}


\begin{document}
\title{}

\author{Brandon Hosley}
\orcid{0000-0002-2152-8192}
%\authornotemark[1]
%\authornote{text}
\email{brandon.hosley.1@us.af.mil}
\affiliation{%
	\institution{Air Force Institute of Technology}
	\streetaddress{1751 11th St.}
	\city{Wright-Patterson Air Force Base}
	\state{Ohio}
	\country{USA}
	\postcode{45433}
}

\begin{abstract}
	\textbf{Executive Summary:} 

\end{abstract}

\received{\today}
\maketitle

\section{Introduction}
\subsection{Overview}

This proposal will outline the research performed to fulfill the requirements of 
CSCE 632 - Statistical Machine Learning; 
to demonstrate competency of the topics presented in the course.
In this proposal we will outline the background and motivation behind the project,
briefly discussing the larger problem to be solved, and how this smaller project
is intended to fit into that objective.

An outline of the data science trajectory, data source, and data preparation that
will be used in this project.
To conclude this proposal we will discuss what the intended contribution of this project
to the body of Computer Vision writ large will be.


\subsection{Background and Motivation}

The motivation behind this work is to explore possible data transformations that may
advance a larger project of wide-angle homographic transformations

\section{Data Science Trajectory}



\section{Data Acquisition and Preparation}

The dataset to be used was made 

\section{Data Exploration}

approximately 20GB of

%%%%%%%%%%%%%%%%%%%
%%%%%%%%%%%%%%%%%%%
In particular, the web-nature data contains 161 car makes with 1,687 car models, covering most of the commercial car models in the recent ten years. There are a total of 136, 727 images capturing the entire cars and 27, 618 images capturing the car parts, where most of them are labeled with attributes and viewpoints. The surveillance-nature data contains 50, 000 car images captured in the front view.

Car Hierarchy - Make, model, year

For example, we define twelve types of cars, which are MPV, SUV, hatchback, sedan, minibus, fastback, estate, pickup, sports, crossover, convertible, and hardtop convertible,

Viewpoints We also label five viewpoints for each car model, including front (F), rear (R), side (S), front-side (FS), and rear-side (RS).

eight car parts for each car model, including four exterior parts (i.e. headlight, taillight, fog light, and air intake) and four interior parts (i.e. console, steering wheel, dashboard, and gear lever).
%%%%%%%%%%%%%%%%%%%
%%%%%%%%%%%%%%%%%%%

\section{Expected Contributions}


%\section{Publication Venues}
%\section{Additional Methods}

\end{document}