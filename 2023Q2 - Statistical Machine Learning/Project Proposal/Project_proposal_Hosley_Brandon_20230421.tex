%\documentclass[12pt]{journal}
\documentclass[sigplan,nonacm]{acmart}
\usepackage[english]{babel}
\usepackage[utf8x]{inputenc}

\usepackage{amsmath,amssymb,amsthm}
\usepackage{tcolorbox}
\newtcbox{\inlinecode}{on line, boxrule=0pt, boxsep=0pt, top=2pt, left=2pt, bottom=2pt, right=2pt, colback=gray!15, colframe=white, fontupper={\ttfamily \footnotesize}}

%\usepackage{enumerate}
\usepackage{graphicx}
\usepackage{booktabs,tabularx}
\usepackage{caption}
\usepackage{float}


\begin{document}
\title{}

\author{Brandon Hosley}
\orcid{0000-0002-2152-8192}
%\authornotemark[1]
%\authornote{text}
\email{brandon.hosley.1@us.af.mil}
\affiliation{%
	\institution{Air Force Institute of Technology}
	\streetaddress{1751 11th St.}
	\city{Wright-Patterson Air Force Base}
	\state{Ohio}
	\country{USA}
	\postcode{45433}
}


\begin{abstract}

\end{abstract}

% Note that keywords are not normally used for peerreview papers.

\received{\today}

% make the title area
\maketitle

\section{Introduction}
\section{Data Science Trajectory}
\section{Data Acquisition and Preparation}
\section{Data Exploration}
\section{Expected Contributions}

%\section{Publication Venues}
%\section{Additional Methods}

\end{document}