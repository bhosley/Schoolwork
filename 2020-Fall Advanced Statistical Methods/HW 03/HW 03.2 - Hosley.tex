\documentclass[a4paper,man,natbib]{apa6}
\usepackage[english]{babel}

\usepackage[cache=false]{minted}
\usemintedstyle{vs}
\usepackage{xcolor}
\definecolor{bg}{rgb}{.95,.95,.95}

\graphicspath{ {./images/} }
\usepackage{graphicx}
\usepackage{caption}

\usepackage{setspace}
%\usepackage{titlesec}
%\titleformat{\subsection}[runin]% runin puts it in the same paragraph
%	{\normalfont\bfseries}% formatting commands to apply to the whole heading
%	{\thesubsection}% the label and number
%	{0.5em}% space between label/number and subsection title
%	{}% formatting commands applied just to subsection title
%	[]% punctuation or other commands following subsection title
% End Packages %

\title{Advanced Statistical Methods Homework 3}
\shorttitle{DAT 530 HW2}
\author{Brandon Hosley}
\date{\today}
\affiliation{University of Illinois - Springfield}
%\abstract{}

\begin{document}
\maketitle
\singlespacing

\section{Introduction to Statistical Learning \\ Chapter 3: Problem 15}
This problem involves the \textbf{\textcolor{red}{Boston}} data set, 
which we saw in the lab for this chapter. 
We will now try to predict per capita crime rate
using the other variables in this data set. 
In other words, per capita crime rate is the response, 
and the other variables are the predictors.

\begin{minted}[bgcolor=bg]{r}
# Load Boston data set
library(MASS)
mount(Boston)
\end{minted}

\subsection{(a)} 
\emph{
	For each predictor, 
	fit a simple linear regression model to predict the response. 
	Describe your results. 
	In which of the models is there a statistically significant association
	between the predictor and the response? 
	Create some plots to back up your assertions.}
\begin{minted}[bgcolor=bg]{r}
par(mfrow=c(2,7))
plot(crim,  crim); abline(lm(crim~crim)) 
plot(zn,    crim); abline(lm(crim~zn)) 
plot(indus, crim); abline(lm(crim~indus)) 
plot(chas,  crim); abline(lm(crim~chas))
plot(nox,   crim); abline(lm(crim~nox)) 
plot(rm,    crim); abline(lm(crim~rm))  
plot(age,   crim); abline(lm(crim~age))  
plot(dis,   crim); abline(lm(crim~dis))  
plot(rad,   crim); abline(lm(crim~rad))  
plot(tax,   crim); abline(lm(crim~tax))    
plot(ptratio, crim); abline(lm(crim~ptratio))
plot(black, crim); abline(lm(crim~black))
plot(lstat, crim); abline(lm(crim~lstat))
plot(medv,  crim); abline(lm(crim~medv))
\end{minted}
\includegraphics[width=\linewidth]{LinearMatrix}
It appears that only certain predictors have slopes not near-zero or near-infinite:
zn, indus, age, rad, lstat, medv.

\subsection{(b)}
\emph{
	Fit a multiple regression model to predict the 
	response using all of the predictors. 
	Describe your results. 
	For which predictors can we reject the null hypothesis 
	$H_0 : \beta_j = 0$?
}
\begin{minted}[bgcolor=bg]{r}
summary(lm(crim~., data=Boston))
\end{minted}
Zn, dis, rad, black, and medv are the predictors that allow rejection of null hypothesis with a fairly high confidence.

\subsection{(c)}
\emph{
	How do your results from (a) compare to your results from (b)?
	Create a plot displaying the univariate regression coefficients
	from (a) on the x-axis, and the multiple regression coefficients
	from (b) on the y-axis. That is, each predictor is displayed as a
	single point in the plot. Its coefficient in a simple linear regression model is shown on the x-axis, and its coefficient estimate
	in the multiple linear regression model is shown on the y-axis.
} \\
\begin{minted}[bgcolor=bg]{r}
predictors <- names(Boston)
y <- coefficients(lm(crim~., data=Boston))
x <- vector(mode="numeric", length=1)
x <- append(x,coefficients(lm(crim~zn))[[2]])
x <- append(x,coefficients(lm(crim~indus))[[2]])
x <- append(x,coefficients(lm(crim~chas))[[2]])
x <- append(x,coefficients(lm(crim~nox))[[2]])
x <- append(x,coefficients(lm(crim~rm))[[2]])
x <- append(x,coefficients(lm(crim~age))[[2]])
x <- append(x,coefficients(lm(crim~dis))[[2]])
x <- append(x,coefficients(lm(crim~rad))[[2]])
x <- append(x,coefficients(lm(crim~tax))[[2]])
x <- append(x,coefficients(lm(crim~ptratio))[[2]])
x <- append(x,coefficients(lm(crim~black))[[2]])
x <- append(x,coefficients(lm(crim~lstat))[[2]])
x <- append(x,coefficients(lm(crim~medv))[[2]])

df <- data.frame(predictors,x,y)

library(ggplot2)
library(ggrepel)
p <- ggplot(df, aes(x,y)) 
p <- p + geom_point() 
p <- p + geom_text_repel(aes(x,y,label=predictors)) 
p <- p + coord_cartesian(xlim=c(-5,5),ylim=c(-5,5))
p
\end{minted}
\includegraphics[width=0.5\linewidth]{c-ratio}
\subsection{(d)}
\emph{
	Is there evidence of non-linear association between any of the
	predictors and the response? To answer this question, for each
	predictor $X$, fit a model of the form} \\
	$Y = \beta_0 + \beta_1X + \beta_2X^2 + \beta_3X^3 + \epsilon$.
	
\end{document}