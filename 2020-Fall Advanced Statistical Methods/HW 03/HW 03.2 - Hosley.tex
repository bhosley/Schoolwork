\documentclass[a4paper,man,natbib]{apa6}
\usepackage[english]{babel}

\usepackage[cache=false]{minted}
\usemintedstyle{vs}
\usepackage{xcolor}
\definecolor{bg}{rgb}{.95,.95,.95}

\graphicspath{ {./images/} }
\usepackage{graphicx}
\usepackage{caption}

\usepackage{setspace}
\usepackage{titlesec}
\titleformat{\subsection}[runin]% runin puts it in the same paragraph
	{\normalfont\bfseries}% formatting commands to apply to the whole heading
	{\thesubsection}% the label and number
	{0.5em}% space between label/number and subsection title
	{}% formatting commands applied just to subsection title
	[]% punctuation or other commands following subsection title
% End Packages %

\title{Advanced Statistical Methods Homework 3}
\shorttitle{DAT 530 HW2}
\author{Brandon Hosley}
\date{\today}
\affiliation{University of Illinois - Springfield}
%\abstract{}

\begin{document}
\maketitle
\singlespacing

\section{Introduction to Statistical Learning \\ Chapter 3: Problem 15}
This problem involves the \textbf{\textcolor{red}{Boston}} data set, 
which we saw in the lab for this chapter. 
We will now try to predict per capita crime rate
using the other variables in this data set. 
In other words, per capita crime rate is the response, 
and the other variables are the predictors.

\begin{minted}[bgcolor=bg]{r}
# Load Boston data set
library(MASS)
mount(Boston)
\end{minted}

\subsection{(a)} 
\emph{
	For each predictor, 
	fit a simple linear regression model to predict the response. 
	Describe your results. 
	In which of the models is there a statistically significant association
	between the predictor and the response? 
	Create some plots to back up your assertions.} \\
\begin{minted}[bgcolor=bg]{r}
par(mfrow=c(2,7))
plot(crim,  crim); abline(lm(crim~crim)) 
plot(zn,    crim); abline(lm(zn~crim)) 
plot(indus, crim); abline(lm(indus~crim)) 
plot(chas,  crim); abline(lm(chas~crim))
plot(nox,   crim); abline(lm(nox~crim)) 
plot(rm,    crim); abline(lm(rm~crim))  
plot(age,   crim); abline(lm(age~crim))  
plot(dis,   crim); abline(lm(dis~crim))  
plot(rad,   crim); abline(lm(rad~crim))  
plot(tax,   crim); abline(lm(tax~crim))    
plot(ptratio, crim); abline(lm(ptratio~crim))
plot(black, crim); abline(lm(black~crim))
plot(lstat, crim); abline(lm(lstat~crim))
plot(medv,  crim); abline(lm(medv~crim))
\end{minted}
\includegraphics[width=\linewidth]{LinearMatrix} \\
It appears that only certain predictors have slopes not near-zero or near-infinite:
zn, indus, age, rad, lstat, medv.

\subsection{(b)}
\emph{Fit a multiple regression model to predict the response using
	all of the predictors. Describe your results. For which predictors
	can we reject the null hypothesis $H_0 : \beta_j = 0$?
}
\begin{minted}[bgcolor=bg]{r}
pairs(Boston)
\end{minted}

\subsection{(c)}
\emph{} \\

\subsection{(d)}
\emph{} \\


\end{document}