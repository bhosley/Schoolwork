\documentclass[]{article}
\usepackage{titling}
\usepackage{hyperref}


%opening
\title{\textbf{DAT 530 Advanced Statistical Methods}\\
	\normalsize{Project Review: Week 5} }
\author{\textbf{Reviewer: }
	Brandon Hosley}

\begin{document}
\setlength{\droptitle}{-10em} 
\pretitle{\begin{flushleft}\LARGE} % makes document title flush right
	\posttitle{\end{flushleft}}
\preauthor{\begin{flushleft}\large} % makes author flush right
	\postauthor{\end{flushleft}}
\predate{\begin{flushleft}\large} % makes date title flush right
	\postdate{\end{flushleft}}
\maketitle

\vspace{-2em}

\subsection*{Project Title:}
Bluebonnet Data

\subsection*{Author(s):}
David Zask, Chase Rendall, and Devin Fagan

\subsection*{Source:}
\href{https://nycdatascience.com/blog/student-works/bluebonnet-data/}{NYC Data Academy}

\subsection*{The Problem the Author(s) is Trying to Solve in the Project:}
The authors intend to provide recommendations to Bluebonnet, 
an organization dedicated to bring data scientist to work with Democratic party campaigns in smaller elections.
The authors recommend a model for determining the contestability of a political district, which may provide a better return on campaign investments.

\subsection*{Machine Learning (ML) Algorithm(s) used:}
\begin{itemize}
	\item \textbf{Principle Component Analysis}
	\item K-Means Clustering
\end{itemize}

\subsection*{A Brief Description of One of the ML Algorithms used:}
\textbf{Principle Component Analysis} is the process of determining candidate features that may be most effective in generating predictive models.
This process helps dimensionality reduction, improving the overall efficiency of a model by eliminating less effective, less correlative features.
Using this method to add preference to features with a higher absolute covariance, the researcher may select the best predictors in a number that reduced computation wasted on other factors, and preventing overfitting.
Potentially, this may also allow future researchers be more parsimonious in their data collection efforts.

\subsection*{Metrics Used to Evaluate the ML Algorithms:}
The authors offer the results of a confusion matrix. Additionally, they suggest that uncontested districts that were predicted to vote in contrast to their actual results may be strong options for recruiting new candidates.

\end{document}
