\documentclass[]{article}
\usepackage{titling}
\usepackage{hyperref}


%opening
\title{\textbf{DAT 530 Advanced Statistical Methods}\\
	\normalsize{Project Review: Week X} }
\author{\textbf{Reviewer: }
	Brandon Hosley}

\begin{document}
\setlength{\droptitle}{-10em} 
\pretitle{\begin{flushleft}\LARGE} % makes document title flush right
	\posttitle{\end{flushleft}}
\preauthor{\begin{flushleft}\large} % makes author flush right
	\postauthor{\end{flushleft}}
\predate{\begin{flushleft}\large} % makes date title flush right
	\postdate{\end{flushleft}}
\maketitle

\vspace{-2em}

\subsection*{Project Title:}
Business Analytics Using Forecasting

\subsection*{Author(s):}
 Maxim Castaneda, Riku Li, Uniss Tseng, and William Feng

\subsection*{Source:}
\href{https://www.galitshmueli.com/data-mining-project/enhancing-supply-chain-efficiency-through-demand-forecasting-nivea}{Galit Shmueli Student Projects}

\subsection*{The Problem the Author(s) is Trying to Solve in the Project:}
The authors seek to generate a model that will allow Nivea to forecast product supply and demand. Ultimately, the goal is to be able to engage changes in the supply-chain early enough to meet future fluctuations in consumer demand 

\subsection*{Machine Learning (ML) Algorithm(s) used:}
\begin{itemize}
	\item Linear Regression
	\item Moving Average
	\item \textbf{ARIMA}
	\item Statistical Ensemble
\end{itemize}

\subsection*{A Brief Description of One of the ML Algorithms used:}
Auto Regressive Integrated Moving Average (\textbf{ARIMA}) is a technique for making time series predictions. It provides a confidence interval for the next unit of time.
The predictions are made using a moving average based on residuals and assumes that residuals will tend to a normal distribution.

\subsection*{Metrics Used to Evaluate the ML Algorithms:}
Root Mean Squared Error (RMSE)

\end{document}
