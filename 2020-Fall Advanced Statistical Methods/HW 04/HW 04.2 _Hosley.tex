\documentclass[a4paper,man,natbib]{apa6}
\usepackage[english]{babel}

\usepackage[cache=false]{minted}
\usemintedstyle{vs}
\usepackage{xcolor}
\definecolor{bg}{rgb}{.95,.95,.95}

\graphicspath{ {./images/} }
\usepackage{graphicx}
\usepackage{caption}

\usepackage{setspace}
%\usepackage{titlesec}
%\titleformat{\subsection}[runin]% runin puts it in the same paragraph
%	{\normalfont\bfseries}% formatting commands to apply to the whole heading
%	{\thesubsection}% the label and number
%	{0.5em}% space between label/number and subsection title
%	{}% formatting commands applied just to subsection title
%	[]% punctuation or other commands following subsection title
% End Packages %

\title{Advanced Statistical Methods Homework 3}
\shorttitle{DAT 530 HW2}
\author{Brandon Hosley}
\date{\today}
\affiliation{University of Illinois - Springfield}
%\abstract{}

\begin{document}
\maketitle
\singlespacing

\section{Introduction to Statistical Learning \\ Chapter 4.7 : Problem 13}
\emph{
Using the \textbf{\textcolor{red}{Boston}} data set, 
fit classification models in order to predict
whether a given suburb has a crime rate above or below the median.
Explore logistic regression, LDA, and KNN models using various sub-sets of the predictors. Describe your findings.}

Prepare the data set:

\begin{minted}[bgcolor=bg]{r}
library(MASS)
attach(Boston)
library(Metrics)

dim(Boston)
cor(Boston[,-14])

summary(crim)
b <- Boston
medCrim = median(b$crim)
b$highCrim <- ifelse(b$crim < medCrim, 0, 1)
summary(b$highCrim)

set.seed(123)
train_ind <- sample(seq_len(nrow(b)), size = floor(0.8 * nrow(b)))
train <- b[train_ind, ]
test <- b[-train_ind, ]
\end{minted}

\subsection{(a)} 
\emph{Logistic Regression}
%\includegraphics[width=\linewidth]{}
\begin{minted}[bgcolor=bg]{r}
glm.fits=glm(crim~rad+tax+lstat, data=train, family=binomial)
summary(glm.fits)
\end{minted}

Train the model on the training data. Then we will test it against the test data.

\begin{minted}[bgcolor=bg]{r}
glm.pred=predict(glm.fits, test, type="response")
mse(test$crim,glm.pred)
[1] 0.006966051
\end{minted}

The mean-squared error result from the model applied to test data is very low.
Even given the scale of the data it suggests a fairly effective model based on the provided predictors.

\subsection{(b)}
\emph{LDA}
%\includegraphics[width=\linewidth]{}
\begin{minted}[bgcolor=bg]{r}
\end{minted}

\begin{minted}[bgcolor=bg]{r}
glm.fits=glm(crim~rad+tax+lstat, data=b, family=binomial)
summary(glm.fits)
\end{minted}

\subsection{(c)}
\emph{KNN}
%\includegraphics[width=\linewidth]{}
\begin{minted}[bgcolor=bg]{r}
\end{minted}
	
\end{document}