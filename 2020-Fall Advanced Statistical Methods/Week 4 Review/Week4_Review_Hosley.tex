\documentclass[]{article}
\usepackage{titling}
\usepackage{hyperref}


%opening
\title{\textbf{DAT 530 Advanced Statistical Methods}\\
	\normalsize{Project Review: Week 4} }
\author{\textbf{Reviewer: }
	Brandon Hosley}

\begin{document}
\setlength{\droptitle}{-10em} 
\pretitle{\begin{flushleft}\LARGE} % makes document title flush right
	\posttitle{\end{flushleft}}
\preauthor{\begin{flushleft}\large} % makes author flush right
	\postauthor{\end{flushleft}}
\predate{\begin{flushleft}\large} % makes date title flush right
	\postdate{\end{flushleft}}
\maketitle

\vspace{-2em}

\subsection*{Project Title:}
Which Kickstarter projects reach their funding goals?

\subsection*{Author(s):}
Michael Griffin

\subsection*{Source:}
\href{https://nycdatascience.com/blog/student-works/which-kickstarter-projects-reach-their-funding-goals/}{NYC Data Academy}

\subsection*{The Problem the Author(s) is Trying to Solve in the Project:}
The author seeks to determine what factors can be used to predict the likelihood of reaching funding goals for projects appearing on GoFundMe.com.

\subsection*{Machine Learning (ML) Algorithm(s) used:}
\begin{itemize}
	\item Natural Language Processing
	\item Random Forest
	\item sklearn AutoML
\end{itemize}

\subsection*{A Brief Description of One of the ML Algorithms used:}


\subsection*{Metrics Used to Evaluate the ML Algorithms:}
An $F_1$ score is used to evaluate the model. 
The scores presented suggest the model is slightly overfit as the score associated with training data is significantly higher than the score associated with the test group.


\end{document}
