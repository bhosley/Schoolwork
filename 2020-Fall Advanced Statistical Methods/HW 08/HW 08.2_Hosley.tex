\documentclass[a4paper,man,natbib]{apa6}
\usepackage[english]{babel}

\usepackage[cache=false]{minted}
\usemintedstyle{vs}
\usepackage{xcolor}
\definecolor{bg}{rgb}{.95,.95,.95}

\graphicspath{ {./images/} }
\usepackage{graphicx}
\usepackage{caption}

\usepackage{setspace}
%\usepackage{titlesec}
%\titleformat{\subsection}[runin]% runin puts it in the same paragraph
%	{\normalfont\bfseries}% formatting commands to apply to the whole heading
%	{\thesubsection}% the label and number
%	{0.5em}% space between label/number and subsection title
%	{}% formatting commands applied just to subsection title
%	[]% punctuation or other commands following subsection title
% End Packages %

\title{Advanced Statistical Methods Homework 8 \\ Support Vector Machine}
\shorttitle{DAT 530 HW7}
\author{Brandon Hosley}
\date{\today}
\affiliation{University of Illinois - Springfield}
%\abstract{}

\begin{document}
\maketitle
\singlespacing

\section{Intro to the Algorithm}

Support Vector Machines are a type of classification algorithm that is most commonly used for binary classification problems. Classification is performed by calculating a boundary to act as a decision line. Typically the boundary is optimized by calculating it to be as far from the sample points as possible.

\subsection{Intuition}

The boundary can be thought of as a hyperplane of one dimension less than the number of factors being used. A dataset with two factors may be represented as a two-dimensional graph, the classifier will be a one-dimensional hyperplane (a line) dividing the data, and the classifier will optimized to be a far from each cluster as possible.

The maximum margin can be achieved by plotting the hyperplane in a way such that at least three of the closest points to the line are equidistant; with at least one point coming from each class. In the case of overlapping classes or classes too close and too dense the hyperplane is calculated such that a fixed margin will contain the fewest possible points.

\begin{center}
	\includegraphics[width=0.5\linewidth]{Margins}
\end{center}


\section{Applying the Algorithm to }
\subsection{Description of the Problem}
\subsection{Summary of the Dataset}
\subsection{Summary of the Results}

\end{document}
