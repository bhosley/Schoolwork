\documentclass[a4paper,man,natbib]{apa6}
\usepackage[english]{babel}

\usepackage[cache=false]{minted}
\usemintedstyle{vs}
\usepackage{xcolor}
\definecolor{bg}{rgb}{.95,.95,.95}

\graphicspath{ {./images/} }
\usepackage{graphicx}
\usepackage{caption}

\usepackage{setspace}
\usepackage{amsmath}


\title{Advanced Statistical Methods Homework 8 \\ Support Vector Machine}
\shorttitle{DAT 530 HW7}
\author{Brandon Hosley}
\date{\today}
\affiliation{University of Illinois - Springfield}
%\abstract{}

\begin{document}
\maketitle
\singlespacing

\section{Intro to the Algorithm}

Support Vector Machines are a type of classification algorithm that is most commonly used for binary classification problems. Classification is performed by calculating a boundary to act as a decision line. Typically the boundary is optimized by calculating it to be as far from the sample points as possible.

\subsection{Intuition}

The boundary can be thought of as a hyperplane of one dimension less than the number of factors being used. A dataset with two factors may be represented as a two-dimensional graph, the classifier will be a one-dimensional hyperplane (a line) dividing the data, and the classifier will optimized to be a far from each cluster as possible.

The maximum margin can be achieved by plotting the hyperplane in a way such that at least three of the closest points to the line are equidistant; with at least one point coming from each class. In the case of overlapping classes or classes too close and too dense the hyperplane is calculated such that a fixed margin will contain the fewest possible points.

\begin{center}
	\includegraphics[width=0.5\linewidth]{Margins}
\end{center}

In the case of a linear classifier of the classic $y=wx + b$ notation we get the following equation. $\lambda$ will represent the size of the margin.

\vspace{1em}
$ \left[ \frac{1}{n} \sum_{n}^{i=1} \text{max} (0,1-y_i(w \cdot x_i-b)) \right] + \lambda \|w\|^2 $
\vspace{1em}

\subsection{e1071}

The library used for this project will be e1071, which includes an SVM module. The type of SVM trained by e1071 is one in which the final model is a voting ensemble of $k(k-1)/2$ binary classifiers rather than a $k-1$ dimensional hyperplane. The ensemble method likely generalizes better as a library and is likely more robust under a large variety of user abilities and knowledge levels.

\clearpage

\section{Applying the Algorithm to Boston}
\subsection{Description of the Problem}

The \textcolor{red}{Boston} data set has been a staple of this course. As we have gained a significant familiarity with the dataset it seems like a good option for exercising SVM use. A primary consideration when choosing a dataset for demonstrating SVMs is the limitation of input and output data. Input data should be quantitative and may be either continuous or discrete. The output is binary. 

A common use of the \textcolor{red}{Boston} dataset, and what we will use it for in this assignment is using other attributes to predict crime in certain districts of the city. The data provided in \textcolor{red}{Boston} is Quantitative with the exception of 'chas' which is a binary variable representing whether or not the district borders the Charles River. It is provided as a dummy variable; for this exercise we will not be using it.

\subsection{Summary of the Dataset and Preliminaries}



\subsection{Training an SVM Model}



\subsection{Summary of the Results}



\end{document}
