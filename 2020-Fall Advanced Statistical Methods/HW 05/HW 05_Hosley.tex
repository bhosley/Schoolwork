\documentclass{beamer}
\usetheme{Darmstadt}
\usecolortheme{beaver}
\usepackage{booktabs}
\usepackage{hyperref}
\usepackage{multirow}

\usepackage[utf8]{inputenc}
\usepackage{graphicx}
\graphicspath{ {./images/} }

%Information to be included in the title page:
\title{Advanced Statistical Methods \\ Homework 5}
\author{Brandon Hosley}
\institute{University of Illinois - Springfield}
\date{\today}

\begin{document}
\frame{\titlepage}

\begin{frame}{Overview}
\tableofcontents
\end{frame}

\section[Q1A]{Q1A: Issues/mistakes with cross-validation}

\begin{frame}{Issues/mistakes with cross-validation}
	\begin{itemize}[<+->]
		\item K-fold cross validation biases toward increased prediction error.
		\item Filtering data before placing into validation groups can cause problems with fitting; \\ over-fitting to 0\% training error.
	\end{itemize}
	\vspace{1em}
	\centering 
	\includegraphics[width=0.75\linewidth]{KfoldCV}
	\vspace{1em}
\end{frame}

\section[Q1B]{Q1B: Issues/mistakes with bootstrap?}
\begin{frame}{}
	\begin{itemize}[<+->]
		\item Datasets possess significant overlap.
		\item Severely underestimates the prediction error.
		\item 
	\end{itemize}
\end{frame}

\section[Q2]{Q2: Hastie and Tibshirani Summary}

\begin{frame}{Tibshirani Lecture: Resampling Methods}
	Testing the accuracy of our model. \\
	The goal is to minimize testing error: \\
	\vspace{1em}
	\centering 
	\includegraphics[width=0.75\linewidth]{TrainVsTest}
	\vspace{1em}
\end{frame}

\begin{frame}{Tibshirani Lecture: Resampling Methods}
	Possible approaches: \vspace{1em}
	\begin{description}
		\item[Validation Set] Random splitting of data to provide a set for testing error.
		\item[K-fold Cross Validation] Splitting data into $K$ parts, using one as the validation set, and the other $K-1$ as training sets. Afterward select the model that provided the lowest test error.
		\item[Bootstrap] Multiple data sets produced from the original by sampling from the original with some data replaced with random selections from the original data-set.
	\end{description}
\end{frame}

\end{document}
