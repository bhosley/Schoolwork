\documentclass[]{article}
\usepackage{titling}
\usepackage{hyperref}


%opening
\title{\textbf{DAT 530 Advanced Statistical Methods}\\
	\normalsize{Project Review: Week 3} }
\author{\textbf{Reviewer: }
	Brandon Hosley}

\begin{document}
\setlength{\droptitle}{-10em} 
\pretitle{\begin{flushleft}\LARGE} % makes document title flush right
	\posttitle{\end{flushleft}}
\preauthor{\begin{flushleft}\large} % makes author flush right
	\postauthor{\end{flushleft}}
\predate{\begin{flushleft}\large} % makes date title flush right
	\postdate{\end{flushleft}}
\maketitle

\vspace{-2em}

\subsection*{Project Title:}
Investment Opportunity in Ames, Iowa

\subsection*{Author(s):}
Jessie Wang

\subsection*{Source:}
\href{https://nycdatascience.com/blog/student-works/investment-opportunity-in-ames-iowa/}{NYC Data Science Academy}.

\subsection*{The Problem the Author(s) is Trying to Solve in the Project:}
This author looks to apply data science principles to approach real estate in a growing community as an investment opportunity.

\subsection*{Machine Learning (ML) Algorithm(s) used:}
\begin{itemize}
	\item Mean and Median Imputation
	\item Correlation Matrix
	\item Lasso and Ridge regressions
	\item Random Forest
\end{itemize}

\subsection*{A Brief Description of One of the ML Algorithms used:}
% Random Forest Model

\subsection*{Metrics Used to Evaluate the ML Algorithms:}
Root Mean Square Error is used to evaluate the author's model. 
The author was not explicit about which value the RMSE was compared against. 
It appears that it may have been on the same scale as the log-value of the homes.
If this is the case, the model has a very good score.
If the model was intended to predict the absolute home value, the RMSE is so that it suggests some sort of mistake.

\end{document}
