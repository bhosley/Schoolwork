\documentclass[]{article}
\usepackage{titling}
\usepackage{hyperref}

%opening
\title{\textbf{DAT 530 Advanced Statistical Methods}\\
	\normalsize{Project Review} }
\author{\textbf{Reviewer: }
	Brandon Hosley}

\begin{document}
\setlength{\droptitle}{-10em} 
\pretitle{\begin{flushleft}\LARGE} % makes document title flush right
	\posttitle{\end{flushleft}}
\preauthor{\begin{flushleft}\large} % makes author flush right
	\postauthor{\end{flushleft}}
\predate{\begin{flushleft}\large} % makes date title flush right
	\postdate{\end{flushleft}}
\maketitle

\vspace{-2em}

\subsection*{Project Title:}
Consumer Behavior Insights for Instacart

\subsection*{Author(s):}
Sunny Lee, Josefa Sullivan, Michael Emmert and Jordan Runge

\subsection*{Source:}
\href{https://nycdatascience.com/blog/student-works/recommendations-for-online-groceries/}{NYC Data Science Academy}.

\subsection*{The Problem the Author(s) is Trying to Solve in the Project:}
To examine online grocer customer's settled and exploratory purchasing habits. \\
The Authors then use the patterns that they discover to build and test a product recommender.
 
\subsection*{Machine Learning (ML) Algorithm(s) used:}
\begin{itemize}
	\item XGBoost (eXtreme Gradient Boost)
	\item K-means followed by Hierarchical Clustering
	\item Associative Rule Mining
\end{itemize}

\subsection*{A Brief Description of One of the ML Algorithms used:}
 XGBoost (eXtreme Gradient Boost)

\subsection*{Metrics Used to Evaluate the ML Algorithms:}
The Author's build and deploy a recommender, but fail to provide any metrics by which to test the efficacy of the recommendations. From the screen captures provided, it is doubtful that the recommendations actually made it to a test group. It may not have been feasible to track which items ultimately end up purchased after recommendation.

\end{document}
