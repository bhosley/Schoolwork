\documentclass[]{article}
\usepackage{titling}
\usepackage{hyperref}


%opening
\title{\textbf{DAT 530 Advanced Statistical Methods}\\
	\normalsize{Project Review: Week 1} }
\author{\textbf{Reviewer: }
	Brandon Hosley}

\begin{document}
\setlength{\droptitle}{-10em} 
\pretitle{\begin{flushleft}\LARGE} % makes document title flush right
	\posttitle{\end{flushleft}}
\preauthor{\begin{flushleft}\large} % makes author flush right
	\postauthor{\end{flushleft}}
\predate{\begin{flushleft}\large} % makes date title flush right
	\postdate{\end{flushleft}}
\maketitle

\vspace{-3em}

\subsection*{Project Title:}
Machine Listening and Music Production: A Discrimination-Generation Feedback Loop Using Amper Music and Icelandic Indie Audio Data

\subsection*{Author(s):}
Alexander Sigman

\subsection*{Source:}
\href{https://nycdatascience.com/blog/student-works/machine-listening-and-music-production-a-discrimination-generation-feedback-loop-using-amper-music-and-icelandic-indie-audio-data/}{NYC Data Science Academy}.

\subsection*{The Problem the Author(s) is Trying to Solve in the Project:}
The author seeks to train a "Discrimination-Generation Feedback Loop", more commonly called a Generative Adversarial Network (GAN). However, the author is not attempting to truly train a generative algorithm, rather they use an already built generative algorithm to train their classifier, use that model on Icelandic Indie music, and finally use the model to fine tune the hyper-parameters of the generative algorithm.

\subsection*{Machine Learning (ML) Algorithm(s) used:}
\begin{itemize}
	\item Amper Music's \emph{Composer}: uses a generative algorithm
	\item K-Nearest Neighbors (KNN)
	\item Multilayer-Perceptron (MLP)
\end{itemize}

\subsection*{A Brief Description of One of the ML Algorithms used:}
MLP outperformed KNN for this author's intended classification on the Icelandic Indie music, and as such was chosen as the primary algorithm used for the feedback-loop portion of their work. \\
Multilayer-Perceptron is a feed-forward artificial-neural-network. The author uses the SciKit default hyper-parameters to train this model (number of layers, optimizer, learning rate, etc). The maximum iterations is set to 800.

\subsection*{Metrics Used to Evaluate the ML Algorithms:}
SciKit's Accuracy Score applied to Genre, Subgenre, Mood, Sentiment, Valence, and Momentum features. 


\end{document}
