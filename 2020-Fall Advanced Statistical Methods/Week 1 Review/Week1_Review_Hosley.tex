\documentclass[]{article}
\usepackage{titling}


%opening
\title{\textbf{DAT 530 Advanced Statistical Methods}\\
	\normalsize{Project Review} }
\author{\textbf{Reviewer: }
	Brandon Hosley}

\begin{document}
\setlength{\droptitle}{-10em} 
\pretitle{\begin{flushleft}\LARGE} % makes document title flush right
	\posttitle{\end{flushleft}}
\preauthor{\begin{flushleft}\large} % makes author flush right
	\postauthor{\end{flushleft}}
\predate{\begin{flushleft}\large} % makes date title flush right
	\postdate{\end{flushleft}}
\maketitle

\vspace{-2em}

\subsection*{Project Title:}
[** Project title goes here **]

\subsection*{Author(s):}
[** Names go here **]

\subsection*{Source:}
[** source goes here; e.g., the URL of the project **]

\subsection*{The Problem the Author(s) is Trying to Solve in the Project:}
[** A brief description goes here **]

\subsection*{Machine Learning (ML) Algorithm(s) used:}
• ML Algorithm 1 (just the name of the algorithm; such as, Random Forest) \\
• ML Algorithm 2 \\
• More … 

\subsection*{A Brief Description of One of the ML Algorithms used:}
[** Pick one of the ML algorithms you wrote down from above and write a brief description on it **] \\
At the beginning, I would suggest that you pick the ML algorithm that you are not familiar with (e.g., Random Forest). Google it. Then write a *short* paragraph on it. \\
• You will find, for most common ML algorithms, there are lots of short YouTube videos on them. Watch them! \\
• As time goes by, you will eventually know most of the ML algorithms (or, at least, heard of most of them). Then you can think of writing a slightly in-depth description. 

\subsection*{Metrics Used to Evaluate the ML Algorithms:}
[** They goes here; e.g., Root Mean Squared Error (RMSE) for Regression **]


\end{document}
