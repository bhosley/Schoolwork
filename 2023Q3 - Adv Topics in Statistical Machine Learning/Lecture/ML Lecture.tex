\documentclass[12pt]{amsart}
\usepackage[left=0.5in, right=0.5in, bottom=0.75in, top=0.75in]{geometry}
\usepackage[english]{babel}
\usepackage[utf8x]{inputenc}
\usepackage{amsmath,amssymb,amsthm}
\usepackage{enumerate}
\usepackage{graphicx}
\usepackage{multirow}

\renewcommand{\thesubsection}{\arabic{subsection}}
\renewcommand{\thesubsubsection}{\quad(\alph{subsubsection})}

\begin{document}
\raggedbottom

\noindent{\large CSCE 832 - Advanced Topics in Statistical Machine Learning %
	- Student Lecture }
\hspace{\fill} {\large B. Hosley}
\bigskip


%%%%%%%%%%%%%%%%%%%%%%%
%\setcounter{section}{}
%\setcounter{subsection}{}
\subsection{Scheduling}

No preference. \\

\subsection{Data Imbalance}

\subsubsection{Overview}
There are many potential issues associated with data that has significant imbalance between classes. 
While the data-level techniques were addressed in the previous course, 
this time we will examine algorithmic-level techniques.

\subsubsection{Relevance}
It is highly probable that students will encounter data imbalance of various sizes,
a survey of different approaches will hopefully point other students in useful directions
for their theses, or in future projects.

\subsubsection{Rationale}
This topic is a continuation of my previous and probably future class project.
While I do not expect that my dissertation with be very focused on imbalance, 
I do expect that the same statistical principles that make imbalance a problem
will similarly be an obstacle within the areas of interest.

\subsubsection{Sources}

\subsubsection{Activities}
The coding exercise will likely be comprised of a skeleton code in the style of what 
is normally presented in this course.
The student code may be a segment of implementing and adjusting the parameters 
of threshold moving, cost-sensitive learning, and novel loss functions that address this problem.



\subsection{GANs (or some related sub-topic)}
\subsubsection{Overview}
Training a pair of neural networks, one that generates a synthetic sample and another that classifies
that synthetic example. The pair engage in an 'arms-race' until acceptable synthetic samples are
produced by the generator network.

\subsubsection{Relevance}
GANs have become a fairly ubiquitous tool, even if students never use one in a meaningful capacity,
it is important that a base level of understanding is achieved for today's practitioners;
the increased popularity presents an increased risk of stakeholders requesting this tool to be
implemented at inappropriate time.

\subsubsection{Rationale}
My research interests have taken me in the direction of game theory applications to machine learning.
As a result, adversarial interactions are a natural area of study.

\subsubsection{Sources}

\subsubsection{Activities}
Student coding exercise will likely be something to the effect of taking a small plot of gaussian noise
and generating artificial handwritten number(s) a la the MNIST dataset.



\subsection{Model Explainability}
% Explainability techniques (e.g. Layer-wise Relevance Propagation)
\subsubsection{Overview}
Explaining the inner workings of a model can increase the confidence not only of the stakeholder,
but also the analyst producing the model. For example, decision trees might be considered the gold standard
for explainability, neural networks are a long way off in this regard.

\subsubsection{Relevance}
Many models (especially those of the neural network variety) are frequently referred to as being a black box. 
I believe this often gives the wrong idea of what is occurring in the hidden layers.
Students in this course should recognize that this is an allusion to the currently very difficult 
assignment of meaning to network weights.

\subsubsection{Rationale}


\subsubsection{Sources}
\subsubsection{Activities}
 

\subsection{Model Compression}




\end{document}