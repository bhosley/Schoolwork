\clearpage
\appendix

\section*{Appendix}
\addcontentsline{toc}{section}{Appendix}
\glsresetall


% ------------------------------ Contribution 1 Appendix ---------------------------------

\subsection*{Observation Schema Examples}
\label{con1:app:obs_spaces}

To support varying assumptions about observability and coordination, 
we implement three variants of the agent observation space. 
Each schema is derived from the default \gls{lbf} observation format.

\paragraph{Default Format.}
The baseline observation space in \gls{lbf} is a fixed-length vector composed of tuples. 
Each tuple is of the form \((x, y, \ell)\), where \(x, y\) denote coordinates on the map 
and \(\ell\) is the agent or food level. The overall vector is a concatenation of:
\[
\text{obs}_i = \left( \gls{s}_i,\, f_1, f_2, 
    \dots, f_{N_f},\, \gls{s}_1, \gls{s}_2, 
    \dots, \gls{s}_{N_i-1} \right)
\]
Here:
\begin{itemize}
    \item \(\gls{s}_i\): the observing agent's own state \((x, y, \ell)\),
    \item \(f_j\): state of food item \(j\), zero-padded if fewer than max,
    \item \(\gls{s}_k\): state of ally \(k\), for all other agents, zero-padded as needed.
\end{itemize}
The vector is padded to accommodate the maximum number of agents and food items expected 
in the environment.

\paragraph{Schema 1: Full Observability.}
This schema preserves the full \gls{lbf} observation space. 
The agent sees all food items and all teammates. 
It is structurally identical to the default observation structure and is only 
applicable when the maximum team size is known and fixed during training and evaluation.

\paragraph{Schema 2: Truncated Observability.}
In this variant, the number of ally observation slots is fixed to a smaller value \(k\), 
and allies are randomly sampled from the set of available teammates. 
The order of ally observations is randomized each time to prevent learned 
slot-specific associations. If fewer than \(k\) allies are visible, 
the remaining slots are zero-padded. The food tuple list remains unchanged.

\paragraph{Schema 3: Ally-Ignorant Observability.}
This schema omits ally information entirely. 
The observation consists only of the agent's own state and the food tuples. 
This reflects a strict form of partial observability in which agents cannot 
coordinate via observation.

\paragraph{Example.}
Consider a simplified environment:
\begin{itemize}
    \item Agent 0 at position (2, 3), level 2
    \item Food: one item at (5, 5), level 1
    \item Allies: Agent 1 at (4, 4), level 2; Agent 2 at (1, 6), level 3
    \item Max team size = 4 (so 3 allies max), max food = 2
    \item Truncated schema uses \(k = 1\)
\end{itemize}

\textbf{Full Observability:}
\[
[2, 3, 2,\ 5, 5, 1,\ 0, 0, 0,\ 4, 4, 2,\ 1, 6, 3,\ 0, 0, 0]
\]

\textbf{Truncated Observability (sampled Agent 1):}
\[
[2, 3, 2,\ 5, 5, 1,\ 0, 0, 0,\ 4, 4, 2]
\]

\textbf{Truncated Observability (sampled Agent 2):}
\[
[2, 3, 2,\ 5, 5, 1,\ 0, 0, 0,\ 1, 6, 3]
\]

\textbf{Ally-Ignorant Observability:}
\[
[2, 3, 2,\ 5, 5, 1,\ 0, 0, 0]
\]

\noindent In each case, unused slots (extra food or allies) are filled with \((0, 0, 0)\). 
The truncated schema introduces variation across episodes and removes positional 
indexing over teammates, while the ally-ignorant schema fully removes inter-agent awareness.

\clearpage

% Contribution 1 Supporting curves
% \usepackage{pgffor} %enables for loop used in appendix

\subsection*{Complete Training Curves}
The following figures present exhaustive training curves for all 
agent configurations across the three environments. Each chart compares tabula rasa 
training to retraining from various pretraining durations.

\foreach \i in {3,4,5,6,7,8} {
    \begin{figure}[h]
        \centering
        \includegraphics[width=0.9\linewidth]{Waterworld-\i-agent.png}
        \caption{Training curves for Waterworld, \i\ agents.}
    \end{figure}
}
\foreach \i in {4,5,6,7} {
    \begin{figure}[h]
        \centering
        \includegraphics[width=0.9\linewidth]{Multiwalker-\i-agent.png}
        \caption{Training curves for Multiwalker, \i\ agents.}
    \end{figure}
}
\foreach \i in {3,4,5,6,7} {
    \begin{figure}[h]
        \centering
        \includegraphics[width=0.9\linewidth]{LBF-\i-agent.png}
        \caption{Training curves for LBF, \i\ agents.}
    \end{figure}
}

% ------------------------------ Contribution 2 Appendix ---------------------------------

% ------------------------------ Contribution 3 Appendix ---------------------------------
