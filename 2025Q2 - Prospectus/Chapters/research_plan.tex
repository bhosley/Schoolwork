
In this chapter we propose contributions divided into three sections.
The sections are intended to represent coherent groups of publishable results.
The target venue of publication is \emph{Autonomous Agents and Multi-Agent 
Systems}~\cite{zotero-2605} or publication with similar objectives.

\Cref{fig:timeline} details the proposed timeline for the contributions listed 
in the following sections. The bars of the gantt chart represent the period 
time during with the subject of the bar is expected to be a primary focus. 
The end of the bar coincides with the point at which the associated paper 
is finished and has been submitted to some publication.

\begin{figure}[htbp]
    \begin{center}
    \begin{ganttchart}[y unit title=0.4cm, y unit chart=0.5cm,
    vgrid,hgrid, title label anchor/.style={below=-1.6ex},
    title left shift=.05, title right shift=-.05, title height=1,
    progress label text={}, bar height=0.8, bar top shift=0.1,
    group right shift=0, group top shift=.6,
    inline,
    group height=.3]{1}{18}
    
        %labels
        \gantttitle{2025}{18} \\
        \gantttitle{Apr}{2} 
        \gantttitle{May}{2} 
        \gantttitle{Jun}{2} 
        \gantttitle{Jul}{2} 
        \gantttitle{Aug}{2} 
        \gantttitle{Sep}{2} 
        \gantttitle{Oct}{2} 
        \gantttitle{Nov}{2} 
        \gantttitle{Dec}{2} \\

        %tasks -6
        \ganttbar[bar/.style={fill=blue!25}]{C1: Writing}{1}{2} \\
        \ganttbar[bar/.style={fill=blue!35}]{C1: Editing}{3}{7} \\
        
        \ganttbar[bar/.style={fill=green!15}]{C2: Lit Review}{3}{5} \\
        \ganttbar[bar/.style={fill=green!20}]{C2: Coding}{4}{6} \\
        \ganttbar[bar/.style={fill=green!25}]{C2: Writing}{7}{10} \\
        \ganttbar[bar/.style={fill=green!20}]{C2: Exp. Running}{7}{10} \\
        \ganttbar[bar/.style={fill=green!35}]{C2: Editing}{11}{16} \\
        
        \ganttbar[bar/.style={fill=teal!15}]{C3: Lit Review}{10}{12} \\
        \ganttbar[bar/.style={fill=teal!20}]{C3: Coding}{11}{13} \\
        \ganttbar[bar/.style={fill=teal!25}]{C3: Writing}{12}{16} \\
        \ganttbar[bar/.style={fill=teal!20}]{C3: Exp. Running}{14}{16} \\
        \ganttbar[bar/.style={fill=teal!35}]{C3: Editing}{17}{18} \\
        
        \ganttbar[bar/.style={fill=red!25}]{Def. Prep.}{16}{18} 
        %\ganttbar{Defense Prep}{}{}
        % \ganttbar[bar inline label node/.style={left=10mm},]{Prospectus}{12}{12} \\
        % \ganttbar[bar inline label node/.style={left=15mm},]{Specialty Exam}{14}{14} \\
        % \ganttmilestone{Specialty Defense NLTD}{23} \\
    \end{ganttchart}
    \end{center}
    \caption{Planned Timeline}
    \label{fig:timeline}
\end{figure}


\section{Contribution 1}

\emph{Proposed research for this section is covered in \cref{ch:contribution_1}}

\section{Contribution 2}

\emph{Proposed research for this section is covered in \cref{ch:contribution_2}}


\section{Contribution 3}

% \emph{Proposed research for this section is covered in \cref{ch:contribution_3}}

\subsection{Motivation}
The third contribution investigates a progressive learning strategy in which policy 
networks grow in capacity during training. Rather than training a large network 
from the outset, which increases sample inefficiency and risk of overfitting, 
this approach proposes beginning with a smaller network and increasing its size over 
time through structured transformations. Inspired by ideas such as Net2Net~\cite{chen2016}, 
this work evaluates the use of tensor projection techniques to enable seamless 
expansion while preserving prior network behavior.

This contribution builds on lessons from the previous two. Contribution 1 
showed that smaller-team training can accelerate convergence, while Contribution 2 
intends to demonstrate the importance of input design for scalable policy reuse. 
Contribution 3 investigates whether network capacity itself can be staged similarly,
starting small to learn core dynamics, then growing to support more nuanced policies,
without discarding prior learning.

\subsection{Methodology}
The experiment will begin with small-capacity networks trained using standard 
PPO in PettingZoo-compatible environments. At predefined training milestones, 
the network will be expanded by projecting its weights into a higher-dimensional tensor space. 
Multiple projection strategies will be compared, including:
\begin{itemize}
    \item Identity-based expansion (e.g., block-diagonal initialization).
    \item Learned projection layers.
    \item Randomized low-rank initialization with partial freezing.
\end{itemize}
Training will then resume from the expanded model, and performance will 
be compared against fixed-size networks trained for the same duration.

\subsection{Resources}
The project will reuse the infrastructure from Contributions 1 and 2, 
including RLlib for training and Weights and Biases for tracking. 
Tensor projection layers will be implemented directly in PyTorch and 
integrated into the RLlib model registry.

We have not yet identified any specific features that we need in an environment
and thus anticipate being able to reuse the environments used in the 
earlier contributions.

\subsection{Anticipated Obstacles}
The primary challenge lies in selecting projection methods that preserve 
learned function approximations across architectural transitions. 
Naive expansion may disrupt policy performance or destabilize learning. 
Additional tuning may be needed to maintain gradient flow and convergence 
after projection. Furthermore, identifying optimal growth schedules will 
likely require additional experiments.

\subsection{Expected Contributions}
\begin{description}
    \item[Progressive Network Design:] Demonstrate the feasibility of growing 
        policy networks during training using tensor projections.
    \item[Transition Timing:] Identify when architectural expansion yields the 
        greatest benefit for training efficiency or final performance.
    \item[Sample Efficiency:] Compare progressive-growth models against fixed-size 
        baselines to assess gains in learning cost or generalization.
    \item[Broader Scalability Strategy:] Position architectural growth as a 
        complementary tool alongside policy reuse and input-invariant designs.
\end{description}

\section{Expected Challenges}

