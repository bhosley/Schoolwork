
In this chapter we propose contributions divided into three sections.
The sections are intended to represent coherent groups of publishable results.
The target venue of publication is \emph{Autonomous Agents and Multi-Agent 
Systems}~\cite{zotero-2605} or publication with similar objectives.

\Cref{fig:timeline} details the proposed timeline for the contributions listed 
in the following sections. The bars of the gantt chart represent the period 
time during with the subject of the bar is expected to be a primary focus. 
The end of the bar coincides with the point at which the associated paper 
is finished and has been submitted to some publication.

\begin{figure}[htbp]
    \begin{center}
    \begin{ganttchart}[x unit=0.8cm, y unit title=0.4cm, y unit chart=0.5cm,
    vgrid,hgrid, title label anchor/.style={below=-1.6ex},
    title left shift=.05, title right shift=-.05, title height=1,
    progress label text={}, bar height=0.8, bar top shift=0.1,
    group right shift=0, group top shift=.6,
    inline,
    group height=.3]{1}{18}
    
        %labels
        \gantttitle{2025}{18} \\
        \gantttitle{Apr}{2} 
        \gantttitle{May}{2} 
        \gantttitle{Jun}{2} 
        \gantttitle{Jul}{2} 
        \gantttitle{Aug}{2} 
        \gantttitle{Sep}{2} 
        \gantttitle{Oct}{2} 
        \gantttitle{Nov}{2} 
        \gantttitle{Dec}{2} \\

        %tasks -6
        \ganttbar[bar/.style={fill=blue!25}]{C1: Writing}{1}{2} \\
        \ganttbar[bar/.style={fill=blue!35}]{C1: Editing}{3}{7} \\
        
        \ganttbar[bar/.style={fill=green!15}]{C2: Lit Review}{3}{5} \\
        \ganttbar[bar/.style={fill=green!20}]{C2: Coding}{4}{8} \\
        \ganttbar[bar/.style={fill=green!25}]{C2: Writing}{7}{10} \\
        \ganttbar[bar/.style={fill=green!20}]{C2: Exp. Running}{7}{10} \\
        \ganttbar[bar/.style={fill=green!35}]{C2: Editing}{11}{16} \\
        
        \ganttbar[bar/.style={fill=teal!15}]{C3: Lit Review}{11}{12} \\
        \ganttbar[bar/.style={fill=teal!20}]{C3: Coding}{12}{13} \\
        \ganttbar[bar/.style={fill=teal!25}]{C3: Writing}{12}{16} \\
        \ganttbar[bar/.style={fill=teal!20}]{C3: Exp. Running}{14}{16} \\
        \ganttbar[bar/.style={fill=teal!35}]{C3: Editing}{17}{18} \\
        
        \ganttbar[bar/.style={fill=red!25}]{Def. Prep.}{16}{18} 
        %\ganttbar{Defense Prep}{}{}
        % \ganttbar[bar inline label node/.style={left=10mm},]{Prospectus}{12}{12} \\
        % \ganttbar[bar inline label node/.style={left=15mm},]{Specialty Exam}{14}{14} \\
        % \ganttmilestone{Specialty Defense NLTD}{23} \\
    \end{ganttchart}
    \end{center}
    \caption{Planned Timeline}
    \label{fig:timeline}
\end{figure}


% \section{Contribution 1}

% \emph{Proposed research for this section is covered in \cref{ch:contribution_1}}

% \section{Contribution 2}

% \emph{Proposed research for this section is covered in \cref{ch:contribution_2}}


% \section{Contribution 3}

% \emph{Proposed research for this section is covered in \cref{ch:contribution_3}}



\subsection{Expected Contributions}
\begin{description}
    \item[Progressive Network Design:] Demonstrate the feasibility of growing 
        policy networks during training using tensor projections.
    \item[Transition Timing:] Identify when architectural expansion yields the 
        greatest benefit for training efficiency or final performance.
    \item[Sample Efficiency:] Compare progressive-growth models against fixed-size 
        baselines to assess gains in learning cost or generalization.
    \item[Broader Scalability Strategy:] Position architectural growth as a 
        complementary tool alongside policy reuse and input-invariant designs.
\end{description}

% --- Rewritten Expected Challenges section ---
\section{Expected Challenges}

This section outlines anticipated challenges for each of the three planned contributions.

\subsection*{Contribution 1 - Finalizing and Publishing Results}
\begin{itemize}
    \item Manuscript draft is under review; committee feedback may require 
        reworking structure or emphasizing different findings.
    \item Journal selection is still open; I have three targets in mind, 
        but the final choice will depend on feedback and timing.
    \item The submission and revision process could delay follow-on work, 
        depending on how smoothly editorial cycles go.
\end{itemize}

\subsection*{Contribution 2 - Modeling, Implementation, and Evaluation}
\begin{itemize}
    \item The architecture is mostly outlined, but some pieces, like how masking 
        interacts with the invariant layer, may need adjustment during implementation.
    \item I'll need a custom environment to support heterogeneous observation subspaces. 
        Modifying an existing PettingZoo task looks feasible, but it will take time.
    \item I'd prefer to move away from RLlib due to instability during C1 
        (notably the RLModule API). This may mean adopting and integrating a new training stack,
        or increased training time.
    \item The planned experiments involve a wide range of team sizes, 
        sensor configurations, and evaluation conditions. 
        It will be compute-heavy and time-intensive to run them all.
\end{itemize}

\subsection*{Contribution 3 - Progressive Network Expansion}
\begin{itemize}
    \item The biggest unknown is how to grow network capacity without hurting performance. 
        I want to be well informed and intentional in choosing the right projection method.
    \item Even if the method underperforms, I want the paper to be publishable. 
        That may mean shifting focus toward some other data gathered during the experiment,
        I am not yet satisfied with a plan, so I considered this an unresolved potential obstacle.
    \item I still need to finalize the relevant literature and pick a projection 
        method that balances feasibility and novelty.
    \item Based on C1, I lack confidence that RLlib's framework will be amendable to this type of
        manipulation. I'll likely need to build the training setup directly in 
        PyTorch or a lightweight framework.
\end{itemize}
