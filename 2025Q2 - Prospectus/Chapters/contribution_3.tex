\subsection{Motivation}

Contribution 3 investigates whether graph-based policy architectures improve 
the efficiency, robustness, and cost-effectiveness of learning in \gls{marl}. 
Whereas Contribution 2 evaluated a mean-field input-invariant approach for its 
scalability and generality, this contribution explores a complementary hypothesis: 
that relational inductive bias introduced through \glspl{gnn}can enhance 
policy learning by explicitly modeling inter-agent dependencies.

This work is motivated by the observation that \gls{marl} environments 
often involve structured, dynamic interactions among agents. 
Encoding these interactions directly via message passing and attention 
mechanisms may yield more expressive and generalizable policies than flat, 
unordered input designs.

A significant shortcoming of mean-field solutions is that they often neglect
information pertaining to agent-group interactions~\cite{yang2021a}.
Graph-based solutions are often chosen for the ability to maintain such information
and their order invariance~\cite{liu2020b, yang2021a}.


\subsubsection{Novelty}

While the primary novelty of this contribution is still under active exploration, 
we anticipate several avenues through which meaningful contributions may emerge.

First, we apply graph-based policy architectures to a custom heterogeneous environment 
that introduces substantially more agent diversity than the original PIC implementation. 
This alone provides an opportunity to evaluate the limits of relational policy learning 
under asymmetric observation, action, and role-based heterogeneity.

Second, we will characterize the performance and robustness of PIC-style architectures 
under conditions that deviate significantly from the assumptions of earlier benchmarks. 
This includes testing sensitivity to observation degradation, varying team compositions, 
and dynamic environments.

Finally, depending on experimental outcomes, we may explore architectural or 
environmental modifications that address observed failure cases or extend the 
capabilities of existing \gls{gnn}-based models in \gls{marl}.

This contribution is thus framed as both a performance evaluation and a 
potential launching point for novel design improvements in the use of relational 
architectures for heterogeneous multi-agent learning.


\subsection{Methodology}

We implement and evaluate a graph-based policy architecture adapted from 
\gls{pic}~\cite{liu2020b}, which combines a \gls{gnn}-based encoder for 
agent observations with a transformer-based centralized critic. 

We adopt PIC as our benchmark GNN-based architecture due to its 
balance of simplicity and representational power. 
Initial exploratory efforts to implement a minimal 
graph neural network for this setting yielded architectures 
that were functionally and structurally similar to PIC. 
As a result, we concluded that further reduction would risk 
eliminating key relational mechanisms that motivate the study. 
PIC is a mature, well-documented method that supports centralized 
critics and inter-agent message passing without excessive overhead, 
making it a natural choice for evaluating graph-based inductive biases in heterogeneous MARL.

This model is trained using the \gls{ctde} paradigm and benchmarked against 
both input-invariant architectures from Contribution 2 and non-shared 
baselines (e.g., \gls{happo}~\cite{zhong2024}).

We conduct controlled experiments in the same custom heterogeneous 
environment developed for earlier contributions. 
Metrics include convergence rate (agent-steps to reach baseline reward), 
final performance under full and partial observability, 
and sensitivity to agent dropout or observation channel degradation.


\subsection{Resources}

The architecture is implemented in PyTorch and integrated into the RLlib 
framework using custom model registration. 
Existing logging and evaluation infrastructure from Contributions 1 and 2 is reused. 
No additional environment modifications are required beyond those already completed.


\subsection{Anticipated Obstacles}

Graph-based policies introduce additional architectural complexity, 
requiring careful tuning of message-passing depth, 
graph connectivity assumptions, and attention mechanisms. 
Furthermore, scalability with respect to team size and training 
cost may present practical challenges, especially in the 
presence of high-dimensional relational embeddings. 
Finally, interpreting learned message-passing behavior may be 
difficult without additional instrumentation.
