\documentclass[]{article}
\usepackage{graphicx}
\usepackage{caption}
\graphicspath{ {./images/} }

% Minted
\usepackage[cache=false]{minted}
	\usemintedstyle{vs}
	\usepackage{xcolor}
		\definecolor{light-gray}{gray}{0.97}

\usepackage{enumitem}
\usepackage{hyperref}

\title{Data Mining: Exercise 2-2}
\author{Brandon Hosley}
\date{\today}

\begin{document}
\maketitle

%\vspace{3em}

\subsection*{Q1: Modify/rewrite the grouping-set-query in the example with ROLLUP (Let’s call it rollup-query). Run it, check the results, and explain the differences.}

\begin{itemize}[before=\itshape,font=\normalfont]
	\setlength\itemsep{0.5em}
	\item Replace GROUPING SETS part with ROLLUP:
	\begin{itemize}
		\item Delete GROUPING SETS line.
		\item Add ‘WITH ROLLUP’ in GROUP BY line.
		\item For syntax, see \href{https://cwiki.apache.org/confluence/display/Hive/Enhanced+Aggregation\%2C+Cube\%2C+Grouping+and+Rollup}{link}.
	\end{itemize}
	\item First line of the result should be like below
	\begin{itemize}
		\item NULL NULL NULL 8000
	\end{itemize}
	\item Explain the differences of the results between the grouping-set-query and the roll-up-query with your query results. 
	\begin{itemize}
		\item Not with general cases like A, B, C or a, b, c 
	\end{itemize}
	\item Regarding query results, please include first 10 lines or so only.
\end{itemize}

\begin{minted}[breaklines,bgcolor=light-gray]{shell-session}
SELECT driverid, event, city, count(*) as occurance
FROM csc533.bhosl2_geolocation
GROUP BY driverid, event, city WITH ROLLUP
LIMIT 10;
\end{minted}

\includegraphics[width=\linewidth]{image1}

The Rollup function differentiates from the group by implicitly grouping by a series of sets in which the the last item is dropped by each subsequent set until no items remain.

\subsection*{Q2: Modify/rewrite the rollup-query (or grouping-set-query) with GROUPING SETS (Let’s call it rollup-like-query) to produce same results of rollup-query.}

\begin{itemize}[before=\itshape,font=\normalfont]
	\item Query result should be same.
	\item Regarding query results, please include first 10 lines or so only.
\end{itemize}

\begin{minted}[breaklines,bgcolor=light-gray]{shell-session}
SELECT driverid, event, city, count(*) as occurance
FROM csc533.bhosl2_geolocation
GROUP BY driverid, event, city
GROUPING SETS ((driverid, event, city), (driverid, event), driverid, ())
LIMIT 10;
\end{minted}

\includegraphics[width=\linewidth]{image2}

The above query replicates a rollup by specifying the sets that would be derived in a rollup.

\subsection*{Q3: Modify/rewrite the grouping-set-query with CUBE. Run it, check the results, and explain the differences and similarities.}

\begin{itemize}[before=\itshape,font=\normalfont]
	\item Explain the differences and similarities of the queries and query results among GROUPING SETS, ROLLUP and CUBE
	\begin{itemize}
		\item Explain with your query results. (not general cases like A, B, C or a, b, c)
	\end{itemize}
	\item Regarding query results, please include first 10 lines or so only.
\end{itemize}

\begin{minted}[breaklines,bgcolor=light-gray]{shell-session}
SELECT driverid, event, city, count(*) as occurance
FROM csc533.bhosl2_geolocation
GROUP BY driverid, event, city WITH CUBE
LIMIT 10;
\end{minted}

\includegraphics[width=\linewidth]{image3}

When compared to a rollup, a cube produces a series of subsets, but instead of a decreasing subset, the cube produces each possible combination, to include an empty set.

\end{document}