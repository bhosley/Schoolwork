\documentclass[]{article}
\usepackage{graphicx}
\usepackage{caption}
\graphicspath{ {./images/} }

% Minted
\usepackage[cache=false]{minted}
	\usemintedstyle{vs}
	\usepackage{xcolor}
		\definecolor{light-gray}{gray}{0.97}

\usepackage{enumitem}

\title{Data Mining: Exercise 2-1}
\author{Brandon Hosley}
\date{\today}

\begin{document}
\maketitle

\vspace{3em}

\subsection*{Q1: Import Data}
\emph{Import the data, trucks.csv and create a DataFrame, named df4. \\
	Should be loaded using csv formatting and have headers, so that we can specify columns later} \\
	\includegraphics[width=\linewidth]{image1}
\begin{minted}[breaklines,bgcolor=light-gray]{shell-session}
>>> df4 = spark.read.format("csv").option("header", "true").load("/user/bhosl2/Geolocation/geolocation.csv")
\end{minted}

\clearpage
\subsection*{Q2: }
\emph{Write scripts to create another DataFrame, df5, 
	that has normalized/scaled feature vector and
	show 10 rows of the DataFrame.}
	\begin{itemize}[before=\itshape,font=\normalfont]
		\item It should have 6 columns: truckid, driverid, event, velocity, features, and scaled Features Velocity column has numerical data.
		\item The data of some columns (truckid, driverid, event, and velocity) should copied from the DataFrame, df4
		\item The column, ‘features’, should have a feature vector transformed from velocity column. The feature vector should have float-type data. e.g. [0.123]
		\item The column, ‘scaledFeatures’, should have a scaled feature vector. The feature vector should have rescaled values (float-type) to a specific range, [0, 1] with MinMaxScale from features column.
	\end{itemize} 

\begin{minted}[breaklines,bgcolor=light-gray]{shell-session}

\end{minted}

\includegraphics[width=\linewidth]{image2}

\clearpage
\subsection*{Q3: }
\emph{Write scripts to create another DataFrame, df6, 
	that has a binary sparse vector transformed from 
	categorical variable (column) and show 10 rows of the DataFrame.} \\
	\begin{itemize}[before=\itshape,font=\normalfont]
		\item It should have 6 columns: truckid, driverid, event, city, cityIndex, and cityVec
		\item The data of some columns (truckid, driverid, event, and city) should copied from the DataFrame, df4
		\item City column has categorical data.
		\item The column, ‘cityIndex’, should have float-type index data, e.g. 0.0 transformed/encoded from city column.
		\item The column, ‘cityVec’, should have a binary sparse vector, e.g. (10, [7], [1.0]), transformed from cityIndex column to be used in some data mining/machine learning algorithms
`	\end{itemize} 


\begin{minted}[breaklines,bgcolor=light-gray,fontsize=\footnotesize]{shell-session}
from pyspark.ml.feature import StringIndexer, OneHotEncoderEstimator

df6 = df4.select("truckid", "driverid", "event", "city")
indexer = StringIndexer(inputCol="city", outputCol="cityIndex")
df6 = indexer.fit(df6).transform(df6)
encoder = OneHotEncoderEstimator(inputCols=["cityIndex"], outputCols=["cityVec"])
df6 = encoder.fit(df6).transform(df6)
df6.show(10)
\end{minted}

\includegraphics[width=\linewidth]{image3}

\end{document}