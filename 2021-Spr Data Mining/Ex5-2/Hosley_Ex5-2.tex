\documentclass[]{article}
\usepackage{graphicx}
\usepackage{caption}
\graphicspath{ {./images/} }

% Minted
\usepackage[cache=false]{minted}
	\usemintedstyle{vs}
	\usepackage{xcolor}
		\definecolor{light-gray}{gray}{0.97}

\usepackage{amsmath}
\usepackage{enumitem}
\usepackage{hyperref}
\usepackage{comment}

\title{Data Mining: Exercise 5-1}
\author{Brandon Hosley}
\date{\today}

\begin{document}
\maketitle

\section*{Assignment 1: Do exercise in Section 1.2.1}

\section*{Assignment 2: Do exercise in Section 2.1.1}

\begin{minted}[breaklines,bgcolor=light-gray]{shell-session}

\end{minted}
\includegraphics[width=\linewidth]{image1}


\section*{Assignment 3: Do exercise in Section 3.1.1}

\begin{minted}[breaklines,bgcolor=light-gray]{shell-session}

\end{minted}
\includegraphics[width=\linewidth]{image1}


\section*{Assignment 4: Write queries to answer which hours of the day had the highest number of pickups? (Assume K=8, Seed=1)}

\begin{minted}[breaklines,bgcolor=light-gray]{shell-session}

\end{minted}
\includegraphics[width=\linewidth]{image1}

\section*{Assignment 5: Write queries to answer how many pickups occurred in each cluster? (Assume K=8, Seed=1)}

\begin{minted}[breaklines,bgcolor=light-gray]{shell-session}

\end{minted}
\includegraphics[width=\linewidth]{image1}

\section*{Assignment 6:}
\textbf{
For the k-means algorithm, it is interesting to note that by choosing the initial cluster centers 
carefully, we may be able to not only speed up the convergence of the algorithm, but also 
guarantee the quality of the final clustering. The k-means++ algorithm is a variant of k-means, 
which chooses the initial centers as follows. First, it selects one center uniformly at random from 
the objects in the data set. Iteratively, for each object p other than the chosen center, it chooses 
an object as the new center. This object is chosen at random with probability proportional to 
dist(p)2, where dist(p)) is the distance from p, to the closest center that has already been chosen. 
The iteration continues until k centers are selected.
Explain why this method will not only speed up the convergence of the k-means algorithm, but 
also guarantee the quality of the final clustering results.
}



\end{document}


\begin{minted}[breaklines,bgcolor=light-gray,fontsize=\footnotesize]{shell-session}
\end{minted}
\includegraphics[width=\linewidth]{image1}