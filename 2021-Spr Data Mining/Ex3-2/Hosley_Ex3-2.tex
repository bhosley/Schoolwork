\documentclass[]{article}
\usepackage{graphicx}
\usepackage{caption}
\graphicspath{ {./images/} }

% Minted
\usepackage[cache=false]{minted}
	\usemintedstyle{vs}
	\usepackage{xcolor}
		\definecolor{light-gray}{gray}{0.97}

\usepackage{enumitem}
\usepackage{hyperref}

\title{Data Mining: Exercise 3-2}
\author{Brandon Hosley}
\date{\today}

\begin{document}
\maketitle

\section*{Do the Five Assignments in the Exercise Guide}

\begin{minted}[breaklines,bgcolor=light-gray,fontsize=\footnotesize]{shell-session}
from pyspark.ml.fpm import FPGrowth
from pyspark.sql.functions import col, split

df0 = spark.read.text("/user/data/CSC533DM/groceries.csv").toDF("csv")
df = df0.withColumn("items", split(col("csv"), ",").cast("array<string>"))
\end{minted}

\subsection*{Q1: Reading data}
\begin{enumerate}[before=\itshape,label=\arabic*.]
	\item Write spark codes to read above data.
	\begin{enumerate}[before=\itshape,label=\alph*.]
		\item Read four tables only for this exercise
		\begin{enumerate}[before=\itshape,label=\roman*.]
			\item orders
			\item products
			\item departments
			\item order\_products\_train
		\end{enumerate}
		\item Read “headers” as well
		\begin{enumerate}[before=\itshape,label=\roman*.]
			\item Each csv file of the dataset has a header line.
		\end{enumerate}
	\end{enumerate}
	\item Take screenshots of running your code in your pyspark shell in your terminal.
\end{enumerate} 

\subsection*{Q2: Training a model using FPGrowth}
\begin{enumerate}[before=\itshape,label=\arabic*.]
	\item Write spark codes to train the data to calculate frequent itemsets
	\begin{enumerate}[before=\itshape,label=\alph*.]
		\item Review Ex3-1
		\item Pyspark use different library to use FPGrowth
		\item Read data from baskets view
		\begin{enumerate}[before=\itshape,label=\roman*.]
			\item Using this query: select items from baskets
			\item Should understand how the data will be organized or stored in DataFrame
		\end{enumerate}
		\item Use MinSupport = 0.001, MinConfidence = 0
		\item Create temporary view
	\end{enumerate}
	\item Take screenshot(s) of running your codes in your pyspark shell in your terminal.
	\item Take a screen shot of output of frequentItemsets
	\begin{enumerate}[before=\itshape,label=\alph*.]
		\item Using this command: model.freqItemsets.show(15, truncate=False)
		\item Show 15 rows, not 10 rows. (below output shows 10 rows)
	\end{enumerate}
\end{enumerate} 

\subsection*{Q3: Finding frequent itemsets}
\begin{enumerate}[before=\itshape,label=\arabic*.]
	\item Take a screenshot of running above codes in your pyspark shell in your terminal.
	\begin{enumerate}[before=\itshape,label=\alph*.]
		\item SQL query: select items, freq from mostPopularItemInABasket where size(items) \\ $>$ 2 order by freq desc limit 20
	\end{enumerate}
	\item Take a screen shot of output of frequentItemsets
	\begin{enumerate}[before=\itshape,label=\alph*.]
		\item Using this command: mostPopularItemInABasket\_2.show(15, truncate=False)
		\item Show 15 rows, not 10 rows. (above output shows 10 rows)
	\end{enumerate}
\end{enumerate} 


\subsection*{Q4: Generating association rules}
\begin{enumerate}[before=\itshape,label=\arabic*.]
	\item Write spark codes
	\begin{enumerate}[before=\itshape,label=\alph*.]
		\item Review Ex3-1
		\item Store generated association rule in a DataFrame (name: ifThen)
		\item Create a temporary view (name: ifThen)
	\end{enumerate}
	\item Take screenshot(s) of running your codes in your pyspark shell in your terminal.
	\item Take a screenshot of output of frequentItemsets
	\begin{enumerate}[before=\itshape,label=\alph*.]
		\item Show three columns only (search ‘.select’ command)
		\begin{enumerate}[before=\itshape,label=\roman*.]
			\item Antecedent
			\item Consequent
			\item Confidence
		\end{enumerate}
		\item Show 10 rows, not 5 rows. (below output shows 5 rows with all columns)
	\end{enumerate}
\end{enumerate} 


\subsection*{Q5: Identifying items frequently purchased together (association rules)}

\begin{enumerate}[before=\itshape,label=\arabic*.]
	\item Write spark codes to find items frequently purchased together using above SQL query
	\begin{enumerate}[before=\itshape,label=\alph*.]
		\item Store the results of the query as a DataFrame, named ‘assocRules’
	\end{enumerate}
	\item Take screenshot(s) of running your codes in your pyspark shell in your terminal.
	\item Take a screenshot of output of frequentItemsets
	\begin{enumerate}[before=\itshape,label=\alph*.]
		\item Show 10 rows, not 5 rows. (below output shows 5 rows)
	\end{enumerate}
\end{enumerate} 


\begin{minted}[breaklines,bgcolor=light-gray,fontsize=\footnotesize]{shell-session}

\end{minted}

\includegraphics[width=\linewidth]{image1}



\end{document}

\begin{minted}[breaklines,bgcolor=light-gray]{shell-session}
\end{minted}
\includegraphics[width=\linewidth]{image1}