\documentclass[]{article}
\usepackage{graphicx}
\usepackage{caption}
\graphicspath{ {./images/} }

% Minted
\usepackage[cache=false]{minted}
	\usemintedstyle{vs}
	\usepackage{xcolor}
		\definecolor{light-gray}{gray}{0.97}

\usepackage{enumitem}
\usepackage{hyperref}

\title{Data Mining: Exercise 3-2}
\author{Brandon Hosley}
\date{\today}

\begin{document}
\maketitle

\section*{Do the Five Assignments in the Exercise Guide}

\begin{minted}[breaklines,bgcolor=light-gray]{shell-session}
from pyspark.ml.fpm import FPGrowth
from pyspark.sql.functions import col, split, count, collect_set
\end{minted}

\subsection*{Q1: Reading data}
\begin{enumerate}[before=\itshape,label=\arabic*.]
	\item Write spark codes to read above data.
	\begin{enumerate}[before=\itshape,label=\alph*.]
		\item Read four tables only for this exercise
		\begin{enumerate}[before=\itshape,label=\roman*.]
			\item orders
			\item products
			\item departments
			\item order\_products\_train
		\end{enumerate}
		\item Read “headers” as well
		\begin{enumerate}[before=\itshape,label=\roman*.]
			\item Each csv file of the dataset has a header line.
		\end{enumerate}
	\end{enumerate}
	\item Take screenshots of running your code in your pyspark shell in your terminal.
\end{enumerate} 

\begin{minted}[breaklines,bgcolor=light-gray]{shell-session}
departments = spark.read.format('csv').options(header='true').load( '/user/data/CSC533DM/instacart/departments.csv')
order_products_train = spark.read.format('csv').options(header='true').load( '/user/data/CSC533DM/instacart/order_products__train.csv')
orders = spark.read.format('csv').options(header='true').load( '/user/data/CSC533DM/instacart/orders.csv')
products = spark.read.format('csv').options(header='true').load( '/user/data/CSC533DM/instacart/products.csv')

departments.createOrReplaceTempView("departments")
order_products_train.createOrReplaceTempView("order_products_train")
orders.createOrReplaceTempView("orders")
products.createOrReplaceTempView("products")
\end{minted}
\includegraphics[width=\linewidth]{image1}

\subsection*{Q2: Training a model using FPGrowth}
\begin{enumerate}[before=\itshape,label=\arabic*.]
	\item Write spark codes to train the data to calculate frequent itemsets
	\begin{enumerate}[before=\itshape,label=\alph*.]
		\item Review Ex3-1
		\item Pyspark use different library to use FPGrowth
		\item Read data from baskets view
		\begin{enumerate}[before=\itshape,label=\roman*.]
			\item Using this query: select items from baskets
			\item Should understand how the data will be organized or stored in DataFrame
		\end{enumerate}
		\item Use MinSupport = 0.001, MinConfidence = 0
		\item Create temporary view
	\end{enumerate}
	\item Take screenshot(s) of running your codes in your pyspark shell in your terminal.
	\item Take a screen shot of output of frequentItemsets
	\begin{enumerate}[before=\itshape,label=\alph*.]
		\item Using this command: model.freqItemsets.show(15, truncate=False)
		\item Show 15 rows, not 10 rows. (below output shows 10 rows)
	\end{enumerate}
\end{enumerate} 

\begin{comment}
df_orders_dow = spark.sql("""
	select count(order_id) as total_orders, 
		(case 
			when order_dow = '0' then 'Sunday' 
			when order_dow = '1' then 'Monday' 
			when order_dow = '2' then 'Tuesday' 
			when order_dow = '3' then 'Wednesday' 
			when order_dow = '4' then 'Thursday' 
			when order_dow = '5' then 'Friday' 
			when order_dow = '6' then 'Saturday' 
		end) as day_of_week 
	from orders 
	group by order_dow 
	order by total_orders desc
""")

df_orders_dow.show()

df_orders_hour = spark.sql("""
	select count(order_id) as total_orders, order_hour_of_day as hour 
		from orders
	group by order_hour_of_day 
	order by order_hour_of_day
""")

df_orders_hour.show(24)

df_orders_shelf = spark.sql("""
	select d.department, count(distinct p.product_id) as products
		from products p
	inner join departments d
		on d.department_id = p.department_id
	group by d.department
	order by products desc
	limit 10
""")

df_orders_shelf.show()

rawData = spark.sql("select p.product_name, o.order_id from products p inner join order_products_train o where o.product_id = p.product_id")
baskets = rawData.groupBy('order_id').agg(collect_set('product_name').alias('items'))
baskets.createOrReplaceTempView('baskets')
baskets.show(2, truncate=False)

\end{comment}

\begin{minted}[breaklines,bgcolor=light-gray]{shell-session}
# Extract out the items 
baskets_ds = spark.sql("select items from baskets")

# Use FPGrowth
fpgrowth = FPGrowth().setItemsCol("items").setMinSupport(0.001).setMinConfidence(0)
model = fpgrowth.fit(baskets_ds)

# Calculate frequent itemsets
mostPopularItemInABasket = model.freqItemsets

# Create temporary view
mostPopularItemInABasket.createOrReplaceTempView("mostPopularItemInABasket")

# Show Output of:
model.freqItemsets.show(15, truncate=False)
\end{minted}

\includegraphics[width=\linewidth]{image2}


\subsection*{Q3: Finding frequent itemsets}
\begin{enumerate}[before=\itshape,label=\arabic*.]
	\item Take a screenshot of running above codes in your pyspark shell in your terminal.
	\begin{enumerate}[before=\itshape,label=\alph*.]
		\item SQL query: select items, freq from mostPopularItemInABasket where size(items) \\ $>$ 2 order by freq desc limit 20
	\end{enumerate}
	\item Take a screen shot of output of frequentItemsets
	\begin{enumerate}[before=\itshape,label=\alph*.]
		\item Using this command: mostPopularItemInABasket\_2.show(15, truncate=False)
		\item Show 15 rows, not 10 rows. (above output shows 10 rows)
	\end{enumerate}
\end{enumerate} 

\begin{minted}[breaklines,bgcolor=light-gray]{shell-session}
model.freqItemsets.show(10, truncate=False)
mostPopularItemInABasket_2 = spark.sql("""
select items, freq from mostPopularItemInABasket where size(items) > 2 order by freq desc limit 20
""")
mostPopularItemInABasket_2.show(10, truncate=False)
\end{minted}
\includegraphics[width=\linewidth]{image3}


\subsection*{Q4: Generating association rules}
\begin{enumerate}[before=\itshape,label=\arabic*.]
	\item Write spark codes
	\begin{enumerate}[before=\itshape,label=\alph*.]
		\item Review Ex3-1
		\item Store generated association rule in a DataFrame (name: ifThen)
		\item Create a temporary view (name: ifThen)
	\end{enumerate}
	\item Take screenshot(s) of running your codes in your pyspark shell in your terminal.
	\item Take a screenshot of output of frequentItemsets
	\begin{enumerate}[before=\itshape,label=\alph*.]
		\item Show three columns only (search ‘.select’ command)
		\begin{enumerate}[before=\itshape,label=\roman*.]
			\item Antecedent
			\item Consequent
			\item Confidence
		\end{enumerate}
		\item Show 10 rows, not 5 rows. (below output shows 5 rows with all columns)
	\end{enumerate}
\end{enumerate} 

\begin{minted}[breaklines,bgcolor=light-gray]{shell-session}
# Store generated association rules in a DataFrame
ifThen = model.associationRules
# Create a view of ifThen
ifThen.createOrReplaceTempView("ifThen")
ifThen.show(10, truncate=False)
\end{minted}
\includegraphics[width=\linewidth]{image4}

\subsection*{Q5: Identifying items frequently purchased together (association rules)}

\begin{enumerate}[before=\itshape,label=\arabic*.]
	\item Write spark codes to find items frequently purchased together using above SQL query
	\begin{enumerate}[before=\itshape,label=\alph*.]
		\item Store the results of the query as a DataFrame, named ‘assocRules’
	\end{enumerate}
	\item Take screenshot(s) of running your codes in your pyspark shell in your terminal.
	\item Take a screenshot of output of frequentItemsets
	\begin{enumerate}[before=\itshape,label=\alph*.]
		\item Show 10 rows, not 5 rows. (below output shows 5 rows)
	\end{enumerate}
\end{enumerate} 

\begin{minted}[breaklines,bgcolor=light-gray]{shell-session}

\end{minted}
\includegraphics[width=\linewidth]{image1}


\end{document}

\begin{minted}[breaklines,bgcolor=light-gray,fontsize=\footnotesize]{shell-session}
\end{minted}
\includegraphics[width=\linewidth]{image1}