\documentclass[]{article}
\usepackage{graphicx}
\usepackage{caption}
\graphicspath{ {./images/} }

% Minted
\usepackage[cache=false]{minted}
	\usemintedstyle{vs}
	\usepackage{xcolor}
		\definecolor{light-gray}{gray}{0.97}

\usepackage{amsmath}
\usepackage{enumitem}
\usepackage{hyperref}
\usepackage{comment}

\title{Data Mining: Exercise 5-1}
\author{Brandon Hosley}
\date{\today}

\begin{document}
\maketitle

\section*{Preliminary: }

\begin{minted}[breaklines,bgcolor=light-gray]{python}
raw_df = spark.read.csv("/user/data/CSC533DM/titanic.csv", header=True, inferSchema=True)
raw_df.describe().show()

filtered_df = raw_df.select(['Survived', 'Pclass', 'Gender', 'Age', 'SibSp', 'Parch', 'Fare'])
\end{minted}
\includegraphics[width=\linewidth]{image0}

Outliers will be calculated based on their original distribution in the data set, but removed from a common set. The intention of doing it this was is to find and remove outliers such that order will not matter with earlier outliers removing data points that would otherwise be used to calculate subsequent ranges and outliers.

\section*{Assignment 1: Remove outliers from Fare}

\begin{minted}[breaklines,bgcolor=light-gray]{python}
## Fare
quantiles = filtered_df.approxQuantile("Fare", [0.25, 0.75], 0.0)
Q1, Q3 = quantiles[0], quantiles[1]
IQR = Q3-Q1
fareLowerRange, fareUpperRange = Q1 - 1.5 * IQR, Q3 + 1.5 * IQR
fareLowerRange, fareUpperRange

trimmed_df = filtered_df \
	.filter(filtered_df.Fare > fareLowerRange) \
	.filter(filtered_df.Fare < fareUpperRange)
trimmed_df.count()
\end{minted}
\includegraphics[width=\linewidth]{image1}


\section*{Assignment 2: Remove outliers from Age}

\begin{minted}[breaklines,bgcolor=light-gray]{python}
# Check for Nulls
filtered_df.filter(filtered_df.Age.isNull()).count()
# Impute missing values with arithmetic mean
from pyspark.ml.feature import Imputer
imputer = Imputer(strategy='mean', inputCols=['Age'], outputCols=['ImputedAge'])
filtered_df = imputer.fit(filtered_df).transform(filtered_df)

# Find IQR 
quantiles = filtered_df.approxQuantile('ImputedAge', [0.25, 0.75], 0.0)
Q1, Q3 = quantiles[0], quantiles[1]
IQR = Q3-Q1
ageLowerRange, ageUpperRange = Q1 - 1.5 * IQR, Q3 + 1.5 * IQR
# Count outliers
filtered_df.filter(filtered_df.ImputedAge < ageLowerRange).count() + \
filtered_df.filter(filtered_df.ImputedAge > ageUpperRange).count()

# Re-trim
trimmed_df = filtered_df \
	.filter(filtered_df.Fare > fareLowerRange) \
	.filter(filtered_df.Fare < fareUpperRange) \
	.filter(filtered_df.ImputedAge > ageLowerRange) \
	.filter(filtered_df.ImputedAge < ageUpperRange)
trimmed_df.count()
\end{minted}
First we check for nulls. \\
Then we impute the arithmetic mean into the null values. \\
Then we calculate the IQR. \\
Last we count number of outliers. \\ 
\includegraphics[width=\linewidth]{image2.1}

Since there are outliers we then update the trimming filter. \\
It is necessary to do it this way as the previously trimmed dataframe does not have the imputed ages available for filtering.
\includegraphics[width=\linewidth]{image2.2}


\section*{Assignment 3: Remove outliers from Siblings-Spouses}

\begin{minted}[breaklines,bgcolor=light-gray]{python}
# Check for Nulls
filtered_df.filter(filtered_df.SibSp.isNull()).count()

# Find IQR 
quantiles = filtered_df.approxQuantile('SibSp', [0.25, 0.75], 0.0)
Q1, Q3 = quantiles[0], quantiles[1]
IQR = Q3-Q1
sibSpLowerRange, sibSpUpperRange = Q1 - 1.5 * IQR, Q3 + 1.5 * IQR
# Count outliers
filtered_df.filter(filtered_df.SibSp < sibSpLowerRange).count() + \
filtered_df.filter(filtered_df.SibSp > sibSpUpperRange).count()

# Trim
trimmed_df = trimmed_df \
	.filter(trimmed_df.SibSp > sibSpLowerRange) \
	.filter(trimmed_df.SibSp < sibSpUpperRange)
trimmed_df.count()
\end{minted}

\includegraphics[width=\linewidth]{image3.1}
No null values means no need to impute missing data.

\includegraphics[width=\linewidth]{image3.2}


\section*{Assignment 4: Remove outliers from }

\begin{minted}[breaklines,bgcolor=light-gray]{python}

\end{minted}
\includegraphics[width=\linewidth]{image1}




\section*{Assignment 5: Redo HW 4-2 with the new, no-outliers dataset. Explain the differences in results.}


\end{document}


\begin{minted}[breaklines,bgcolor=light-gray,fontsize=\footnotesize]{python}
\end{minted}
\includegraphics[width=\linewidth]{image1}