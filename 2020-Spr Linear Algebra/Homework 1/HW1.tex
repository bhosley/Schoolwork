\documentclass[a4paper,man,natbib]{apa6}
\usepackage[english]{babel}
\usepackage[utf8x]{inputenc}

% Common Packages - Delete Unused Ones %
\usepackage{setspace}
\usepackage{amsmath}
\usepackage[cache=false]{minted}
\usepackage{graphicx}
\usepackage{caption}
\graphicspath{ {./images/} }
\usepackage{multirow}
% End Packages %

\title{Homework 1}
\shorttitle{HW1}
\author{Brandon Hosley}
\date{2020 01 13}
\affiliation{DAT 332 - Matrix Analysis and Numeric Optimization \\
	Liang Kong}

%\abstract{}

\begin{document}
\maketitle
\raggedbottom

\subsection{Problem 1}
\singlespacing
\emph{
Four products are processed successively on two machines.
The manufacturing times in hours per unit of each product 
are tabulated below for the two machines. }
\begin{center}	
	\begin{tabular}{|c|c|c|c|c|}
		\hline
		\multirow{2}{*}{Machine} & \multicolumn{4}{|c|}{Time per unit (hours)} \\
		\cline{2-5}
		& Product 1 & Product 2 & Product 3 & Product 4 \\ \hline
		1 & 2 & 3 & 4 & 2 \\ \hline
		2 & 3 & 2 & 1 & 2 \\ \hline
	\end{tabular} 
\end{center}
\emph{
The total cost of producing a unit of each product is based directly on the machine time.  
Assume the cost per hour for  machines 1 and 2 are \$10 and \$15.  
The total hours budgeted for all the products on machine 1 and 2 are 500 and 380.  
If the sale price per unit for products 1, 2, 3, and 4 are \$65, \$70, \$55, and  \$45, 
formulate the problem as a linear programming model to maximize total net profit. 
} \\
\doublespacing
Begin with a matrix of the each product by how much time is required for each machine to produce it. \\
\begin{tabular}{ll}	
	& \\
	$ \begin{bmatrix}
	2 & 3 & 4 & 2 \\
	3 & 2 & 1 & 2 \\
	\end{bmatrix} $ &
	$ \begin{bmatrix} \$10 \\ \$15 \end{bmatrix} \times M_1 \rightarrow M_2 $ \\
	& (Machine time cost multiplied by Production times = Part Production Cost) \\ 
	$ \begin{bmatrix}
	\$20 & \$30 & \$40 & \$20 \\
	\$45 & \$30 & \$15 & \$30 \\
	\end{bmatrix} $ &
	$ \begin{bmatrix}
	\$65 & \$70 & \$55 & \$45 \\
	\$65 & \$70 & \$55 & \$45 \\
	\end{bmatrix} - M_2 \rightarrow M_3 $ \\
	& (Per-part Profit minus production cost = Net Profit Per Part) \\ 
	$ \begin{bmatrix}
	\$45 & \$40 & \$15 & \$25 \\
	\$20 & \$40 & \$40 & \$30 \\
	\end{bmatrix} $ &
	$ \frac{M_3}{M_1} \rightarrow M_4 $ \\
	& (Net Profit divided by production hours = Profit/Hour) \\
	$ \begin{bmatrix}
	\$22\frac{1}{2} & \$13\frac{1}{3} & \$3\frac{3}{4} & \$12\frac{1}{2} \\
	\$6\frac{2}{3}  & \$20            & \$40           & \$15            \\
	\end{bmatrix} $ & \\
	& \\
\end{tabular}
Maximum profit is obtained when each machine is dedicated to producing its most profitable product for as much time as it is alloted. 
Machine one will produce 250 units of product 1 for a profit of \$11,250.
Machine two will produce 380 units of product 3 for a profit of \$15,200.
Together this will net a profit of \$26,450.
\singlespacing
\newpage
\subsection{Problem 2 (page 55 \#1)}
\emph{
Farmer Jones must determine how many acres of corn and wheat to plant this year. 
An Acre of wheat yields 25 bushels of wheat and requires 10 hours of labor per week.
And acre of corn yields 10 bushels of corn and requires 4 hours of labor per week.
All wheat can be sold at \$4 a bushel, and all corn can be sold at \$3 a bushel.
Seven acres of land and 40 hours per week of labor are available.
Government regulations require that at least 30 bushels of corn be produced during the current year.
Let $x_1 = $ number of acres of corn planted, 
and $x_2 = $ number of acres of wheat planted.
Using these decision variables, formulate an LP whose solution will tell Farmer Jones how to maximize the total revenue from wheat and corn. 
}\\

\subsubsection{Solving Long-Hand:}
\doublespacing
\begin{tabular}{l l}
	$ x_1 + x_2 = 7 $ & : Total Acreage is the sum of Corn and Wheat (7 Acres) \\
	$ 4x_1 + 10x_2 \leq 40 $ & : Working hours must be less than 40 hours \\
	$ 10x_1 + 25x_2 = b_t $ & : Number of bushels produced total \\	
	$ 10x_1 \geq 30 $ & : Meeting Government requirements \\
	$ 30x_1 + 100x_2 = P_\$ $ & : Total profit \\
	\hline
	$ x_2 = 7 - x_1 $ & : For Substitution \\
	$ 30x_1 + 100\left( 7-x_1 \right) = P_\$$ & : Total profit in terms of $x_1$ \\
	$ 700 - 70x_1 = P_\$$ & : Simplified \\
	\hline
	$4x_1+10\left( 7-x_1 \right)\leq40$ & : Comply to Labor Laws \\
	$ 70 - 6x_1 \leq 40 $ & :  \\
	$ -6x_1 \leq -30 $ & :  \\
	$ x_1 \geq 5 $ & : Simplified \\
	& \\
\end{tabular}

\subsubsection{Using Matrix:} \hfill\\

\begin{tabular}{ll}	
	$ \begin{bmatrix}
		1 & 1  & 7  \\
		4 & 10 & 40 \\
	\end{bmatrix} $ &
	$ r_2 - 4r_1 \rightarrow r_2 $ \\
	& \\
	$ \begin{bmatrix}
		1 & 1  & 7  \\
		0 & 6 & 12 \\
	\end{bmatrix} $ &
	$ \frac{r_2}{6} \rightarrow r_2 $ \\
	& \\
	$ \begin{bmatrix}
		1 & 1 & 7 \\
		0 & 1 & 2 \\
	\end{bmatrix} $ &
	$ r_1 - r_2 \rightarrow r_1 $ \\
	& \\
	$ \begin{bmatrix}
		1 & 0 & 5 \\
		0 & 1 & 2 \\
	\end{bmatrix} $ & \\
\end{tabular}
	
\singlespacing
In both cases we are given 
5 acres of corn and 
2 acres of wheat 
yielding a profit of \$350

\newpage
\subsection{Problem 3 (page 55 \#4)}
\emph{
TruckCo manufactures two types of trucks: 1 and 2.
Each truck must go through the painting shop and assembly shop.
If the painting shop were completely devoted to painting 
Type 1 trucks, then 800 per day could be painted;
Type 2 trucks, then 700 per day could be painted.
If the assembly shop were completely dedicated to assembling
Type 1 engines, then 1,500 per day could be assembled;
Type 2 engines, then 1,200 per day could be assembled.
Each Type 1 truck contributes \$300 to profit;
each Type 2 truck contributes \$500.
Formulate an LP that will maximize TruckCo's profit. 
}\\
\doublespacing
\begin{tabular} {l l}
	$ p = 300x_1 + 500x_2 $ & : Equation for total profit \\
	$ 1 = \frac{x_1}{800} + \frac{x_2}{700} $ & 
	: Equation for daily production from the factory. \\
	$ \frac{x_2}{700} = 1 - \frac{x_1}{800} $ & : Solve for $x_2$ \\ 
	$ x_2 = 700 - \frac{7}{8}x_1 $ & \\
	$ p = 300x_1 + 500 \left( 700 - \frac{7}{8}x_1 \right) $ & 
	: Substitute $x_2$ into the profit equation. \\
	$ p = 350,000 - 137\frac{1}{2}x_1 $ & : Simplify \\	
	& \\
\end{tabular}
Each type 1 truck is \$137.50 less profit than the type 2. 
It would be ideal for TruckCo to produce 0 Type 1 trucks, and 
700 Type 2 trucks for a total daily profit of \$350,000. \\


\bibliographystyle{apacite}
\bibliography{} %link to relevant .bib file
\end{document}