\documentclass[a4paper,man,natbib]{apa6}
\usepackage[english]{babel}
\usepackage[utf8x]{inputenc}

% Common Packages - Delete Unused Ones %
\usepackage{setspace}
\usepackage{amsmath}
\usepackage[cache=false]{minted}
\usepackage{graphicx}
\usepackage{caption}
\graphicspath{ {./images/} }
\usepackage{algorithm}
\usepackage[noend]{algpseudocode}
% End Packages %

\title{Homework 3}
\shorttitle{HW3}
\author{Brandon Hosley}
\date{2018 11 12}
\affiliation{Brian-Thomas Rogers}
%\abstract{}

\begin{document}
\maketitle
\raggedbottom
\subsection{Objective}
\noindent
This program will take a set of numbers \\ 
$S = \{...\}$ \\ \noindent
and check for any possible pair of subsets \\
$ A \subset S $ and $ B \subset S $ \\ \noindent
where \\
$x = \Sigma A = \Sigma B$ and $x \in S-(A \cup B)$ \\ \noindent
If such a pair of subsets exist the program will output a truthy boolean. If no such sets exist the program will conclude with no output.

\subsection{Restrictions and Considerations}
Because this project is an exercise in the Functional Programming Paradigm several restrictions have been made to the implementation of this program. First, the operations available will be limited to: \\
\singlespacing
\begin{itemize}
	\item Named Variables
	\item Integer Primitives
	\item Basic Comparisons ( = , < , > , and , or , null?)
	\item If-Else Operators
	\item Arithmetic Operators
	\item Scheme Operations 
		\subitem car
		\subitem cdr
	\item The User-Defined "has-subtarget?" Function
\end{itemize}
\doublespacing

\subsection{Algorithm}
\begin{algorithm}[H]
	\begin{algorithmic}
		\State \textbf{Function} Find Subset:
		\State \textbf{Input:} \{...\} $S$, Integer $A$, Integer $B$, \{..\} $C$
		\If {$S$ has \emph{any} elements}
			\State Iterative Recursive Call:
			\State $\cdot$ Find Subset ( $S\{\}-S_0$ , $A+S_0$ , $B$ , $C\{\}$ )
			\State $\cdot$ Find Subset ( $S\{\}-S_0$ , $A$ , $B+S_0$ , $C\{\}$ )
			\State $\cdot$ Find Subset ( $S\{\}-S_0$ , $A$ , $B$ , $C\{\}+S_0$ )
		\ElsIf {Input is Testable*}
			\State \textbf{Call Function:} Has Target Number(<Target Number , Set):
				\If {Any Number in <Set> = <Target Number>}
					\State \textbf{Return:} \#t
				\EndIf
		\Else {$\implies$ Function has no further iterations, and is not testable.}
			\State Return Nothing 
		\EndIf
		\State \textbf{End} Find Subset
		\State
		\State \textbf{*}Input is \emph{Testable} if:
			\State $A>0$
			\State $A=B \implies B>0$
			\State List $C$ is not Empty
	\end{algorithmic}
	\caption{}
\end{algorithm}

\clearpage
\subsection{Efficiency}
This program was built around solving the problem at hand with the simplest and shortest code possible; as a result the efficiency does suffer. Factors that reduced or impacted efficiency follow:
\begin{itemize}
	\item The program will continue even after the first \#t is returned. As such it will examine each testable iteration.
	\item The process for producing iterations preserves order and will test combinations that are equivalent separately.
	\subitem $S\{1,1,2,2\}$ will return \#t because $\{1,1\}$ and $\{2\}$ both sum to 2.
	\subitem $S\{1_a,1_b,2_c,2_d\}$ will produce subsets $A=\{1_a,1_b\}$ which will evaluate \#t for both $B=\{2_c\}$ and $B=\{2_d\}$ which in turn will produce another pair of true results when the values given to A and B are swapped.
	\subitem As a result four \#t are returned even though each is the result of equivalent subsets.
	\item Once called, the function will "spool-up" a number of recursive calls until they are resolvable and can be disposed of. Once "spooled" the stack will generally contain $3n+1$ calls to the 'has-subtarget?' function.
\end{itemize}
$F(n) = \mathcal{O}(3^n) = \Theta(3^n) = \Omega(3^n)$
\clearpage
\subsection{Source Code}
\begin{minted}{scheme}
(define has-target-num?
   (lambda (Sum Set)
      (if (member Sum Set) (display #t) (display ""))))

(define (has-subtarget? Lst SubA SubB Holder)
  (cond
    ((eqv? Holder 0)(has-subtarget? Lst SubA SubB '()))
    ((not(null? Lst)) ;do recursion?
      (begin
        (has-subtarget? (cdr Lst) (+ SubA (car Lst)) SubB Holder)
        (has-subtarget? (cdr Lst) SubA (+ SubB (car Lst)) Holder)
        (has-subtarget? (cdr Lst) SubA SubB (append Holder (list(car Lst))))))
    ((and(not(null? Holder))(and(not(null? SubA))(= SubA SubB))) ;testable?
      (has-target-num? SubA Holder))
    ))
\end{minted}
\end{document}