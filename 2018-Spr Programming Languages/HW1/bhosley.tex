\documentclass[a4paper,man,natbib]{apa6}
\usepackage[english]{babel}
\usepackage[utf8x]{inputenc}
% Common Packages - Delete Unused Ones %
\usepackage{setspace}
\usepackage[linguistics]{forest}
\usepackage{amsmath}
% End Packages %

\title{Homework 1}
\shorttitle{HW1}
\author{Brandon Hosley}
\date{2018 09 20}
\affiliation{Brian-Thomas Rogers}
%\abstract{}

\begin{document}
\maketitle
\singlespacing
\subsection{1}
\emph{Using the grammar in Example 3.4 (page 125) show a parse tree and a leftmost derivation for the following statement (make sure you do not omit parentheses in your derivation): C = (A+B)*(C+A)*(C+B)}
\\
\begin{forest}
[<assign>
	[<id> [A ]]
	[\textbf{=},before computing xy={s/.average={s}{siblings}} ] 
	[<expr> [<term>
		[<term> 
			[<term> 
				[<factor> [(<expr>)
					[<term> [<factor> [<id> [A ]]]]
					[+ ] 
					[<expr> [<term> [<factor> [<id> [B ]]]]]
				]]
			]
			[\textbf{*},before computing xy={s/.average={s}{siblings}} ]
			[<factor> 
				[(<expr>)
					[<term> [<factor> [<id> [C ]]]]
					[+ ] 
					[<expr> [<term> [<factor> [<id> [A ]]]]]
				]
			]
		]
		[\textbf{*},before computing xy={s/.average={s}{siblings}} ]
		[<factor> 
			[(<expr>)
				[<term> [<factor> [<id> [B ]]]]
				[+ ] 
				[<expr> [<term> [<factor> [<id> [C ]]]]]
			]
		]
	]]
]
\end{forest}

\clearpage
\subsection{2}
\emph{Consider the following grammar(S is the start symbol; 0 and 1 are terminal symbols; A and B are nonterminals) \\ S  0B | 1A \\ A  0B \\ B  1A | 1 } \\
\noindent\emph{Which of the following sentences are in the language generated by the grammar?  Show derivations.  If a sentence cannot be generated by the grammar, explain why.} \\[2ex]
01010100 \\
S => 0B \\
=> 01A \\
=> 010B \\
=> 0101A \\ 
=> 01010B \\
=> 010101A \\
=> 0101010B - Here we run into the problem. B \textbf{cannot derive} the final 0. \\[2ex]
01010101 \\
S => 0B \\
=> 01A \\
=> 010B \\
=> 0101A \\
=> 01010B \\
=> 010101A \\
=> 01010101 - This sentence \textbf{is derivable} using the defined grammar. \\[2ex]
10101010 \\
S => 1A \\
=> 10B \\
=> 101A \\
=> 1010B \\
=> 10101A \\
=> 101010B \\
=> 1010101A \\
=> 10101010B - This sentence is \textbf{not derivable}. \\[2ex]
10010011 \\
S => 1A \\
=> 10B - \textbf{Cannot derive} the next 0 using the defined grammar. \\
\clearpage
\subsection{3}
\emph{For the following grammar and the right sentential form F * (id + id) draw a parse tree and show all phrases, simple phrases, and the handle (E, T, and F are nonterminal symbols; id is a terminal symbol). Explain. \\ E  E + T | T \\ T  T * F | F \\ F  (E) | id } \\[2ex]

\begin{forest}
for tree={if n children=0{tier=terminal}{}} 
[E [T 
	[T [F ]]
	[* ]
	[F [(E) 
		[E [T [F [$(id$ ]]]]
		[+ ]
		[T [F [$id)$ ]]]
	]]
]]
\end{forest}

Phrases: \\
E => E + T \\
=> E + T * F \\ 
=> E + T * id \\
=> E + F * id \\
=> E + id * id \\
=> T + id * id \\
=> F + id * id \\
=> id + id * id \\[2ex]

Simple Phrases: \\
E => T => F => id and (E) \\[2ex]

The handle for this is E => T \\
This is the first derivation made when parsing this tree. I found this unusual because while that is the definition given in the textbook, there were no examples given in which at least one part of the final sentential goal was immediately available. T * F is the first derivation with any expansion to the initial.
\clearpage
\subsection{4}
\emph{Transform the following left recursive EBNF grammar into an equivalent non-left recursive grammar (S and A are nonterminal symbols; S is the start symbol; a and b are terminal symbols):\\S $\rightarrow$ aSb | bAS \\ A $\rightarrow$ AaA | bAA | AAa | bAb }
\\[3ex]
S $\rightarrow$ aSb | bAS \\
A $\rightarrow$ bAAA' | bAbA' \\
A' $\rightarrow$ aAA' | AaA' | $\varepsilon$ \\

\nocite{sebasta}
\bibliographystyle{apacite}
\bibliography{./CS} %link to relevant .bib file
\end{document}