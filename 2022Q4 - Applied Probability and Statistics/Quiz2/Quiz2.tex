\documentclass[answers]{exam}
\usepackage[english]{babel}
\usepackage[utf8x]{inputenc}
\usepackage{amsmath,amssymb,amsthm}

\title{STAT 587 - Introduction to Probability and Statistics%
	\\ Quiz 2}
\author{Brandon Hosley}
\date{\today}

\begin{document}
\maketitle
\begin{questions}

%%%%%%%%%%%%%%%%%%%%%%%%%%%
%	\begin{ Question 1}	  %
%%%%%%%%%%%%%%%%%%%%%%%%%%%
\question 
Suppose \(X_1,\ldots,X_n \overset{iid}{\sim} Normal(\mu,\sigma^2)\) with density,
\[
f(x) =\frac{1}{\sigma\sqrt{2\pi}}e^{\frac{-(x-\mu)^2}{2\sigma^2}} 
\quad x\in(-\infty,\infty).
\]
Recall that an unbiased estimator for \(\sigma^2\),
\[
S^2 = \frac{1}{n-1}\sum_{i=1}^{n}(X_i-\bar{X})^2,
\]
has the variance \(\operatorname{Var}(S^2) = \frac{2\sigma^4}{n-1}\).
\begin{parts}
	\part Does \(S^2\) attain the Cramer-Rao Lower Bound for \(\sigma^2\)?
	\part Would the MLE for \(\sigma^2\),
	\[
	\hat{\sigma}^2 = \frac{1}{n}\sum_{i=1}^{n} (X_i-\bar{X})^2,
	\]
	be the UMVUE if it attained the Cramer-Rao Lower Bound?
\end{parts}
\textit{Show your work!}

\begin{solution}
\begin{parts}
	\part
	First find the LLF
	\begin{align*}
		\ln f(x)
		= \ln\bigg(\frac{1}{\theta\sqrt{2\pi}}\bigg)
			-\bigg(\frac{(x-\mu)^2}{2\theta^2}\bigg) 
		= -\frac{1}{2}\ln(2\pi)-\ln(\sigma)-\frac{(x-\mu)^2}{2\sigma^2} .
	\end{align*}
	Next, the second partial of the LLF,
	\begin{align*}
		\frac{\partial^2}{\partial(\sigma^2)^2} \bigg(
			-\frac{1}{2}\ln(2\pi)-\ln(\sigma)-\frac{(x-\mu)^2}{2\sigma^2} \bigg)
		&= \frac{\partial^2}{\partial(\sigma^2)^2} \bigg(
			-\ln(\sigma)-\frac{(x-\mu)^2}{2\sigma^2} \bigg) \\
		&= \frac{\partial^2}{\partial(\sigma^2)^2} \bigg(
			-\frac{1}{2}\ln(\sigma^2)-\frac{(x-\mu)^2}{2\sigma^2} \bigg) \\
		&= \frac{\partial}{\partial(\sigma^2)} \bigg(-\frac{1}{2(\sigma^2)}\bigg) -
			\frac{\partial^2}{\partial(\sigma^2)^2} \bigg(\frac{(x-\mu)^2}{2(\sigma^2)} \bigg) \\
		&= \frac{1}{2(\sigma^2)^2} - \frac{\partial}{\partial(\sigma^2)} 			 
			\bigg(\frac{(x-\mu)^2}{2(\sigma^2)^2} \bigg) \\
		&= \frac{1}{2\sigma^4} + \frac{(x-\mu)^2}{(\sigma^2)^3} \\
		&= \frac{1}{2\sigma^4} + \frac{(x-\mu)^2}{\sigma^6} .
	\end{align*}
	Find the first component of the CRLB,
	\begin{align*}
		E\bigg[\bigg(\frac{1}{2(\sigma^2)^2}\ln f(x)\bigg)^2\bigg]
		&= -E\bigg[\frac{1}{2\sigma^4} + \frac{(x-\mu)^2}{\sigma^6}\bigg] \\
		&= -\frac{1}{2\sigma^4} + \frac{E[(x-\mu)^2]}{\sigma^6} \\
		&= -\frac{1}{2\sigma^4} + \frac{(\sigma)^2}{\sigma^6} \\
		&= \frac{(\sigma)^2}{\sigma^6} - \frac{1}{2\sigma^4} \\
		&= \frac{1}{2\sigma^4} .
	\end{align*}
	Insert into CRLB function to complete,
	\begin{align*}
		\text{CLRB } 
		= \frac{1}{n\,E\bigg[\bigg(\frac{1}{2(\sigma^2)^2}\ln f(x)\bigg)^2\bigg]}
		= \frac{2\sigma^4}{n} .
	\end{align*}
	Comparing this bound with \(\operatorname{Var}(S^2)\)
	\begin{align*}
		\operatorname{Var}(S^2)
		= \frac{2\sigma^4}{n-1}
		= \bigg(\frac{n}{n-1}\bigg)\frac{2\sigma^4}{n}
		> \frac{2\sigma^4}{n} \quad \text{for all }n>1.
	\end{align*}

\part
	Finally,
	\begin{align*}
		\hat{\sigma^2} 
		= \frac{2\sigma^4}{n}
	\end{align*}
	showing that \(\hat{\sigma^2}\) is equal to the CRLB and is thus the UMVUE.
\end{parts}
\end{solution}
%\end{ Question 1}

\clearpage

%%%%%%%%%%%%%%%%%%%%%%%%%%%
%	\begin{ Question 2}	  %
%%%%%%%%%%%%%%%%%%%%%%%%%%%
\question 
Suppose \(X_1,\ldots,X_n \overset{iid}{\sim} \text{Shifted Exponential}(\theta,\gamma) \ (x>\gamma)\),
where \(E(X)=\theta+\gamma\) and \(\text{Var}(X_1)=\theta^2\).
We have observed the following data:
\begin{flalign*}
	X_1 &= 5.31 && \\
	X_2 &= 8.76 && \\
	X_3 &= 8.61 && \\
	X_4 &= 6.67 && \\
	X_5 &= 7.45 &&
\end{flalign*}
Derive the method of moments estimators for $\theta$ and $\gamma$. 
Are the estimates associated with the observed data reasonable, under the assumed distribution? \textit{Show your work!}

\begin{solution}
	First to find \(E[X]\).
	\[E(X)=\theta+\gamma=\frac{1}{n}\sum_{i=1}^{n}X_i
	=\frac{1}{5}\sum_{i=1}^{5}X_i = 7.36\,.\]
	Second,
	\[\text{Var}(X)=\theta^2=\frac{1}{n}\sum_{i=1}^{n}(X_i-\bar{X})^2
	=1.64184\,.\]
	Next, solve for \(\theta\),
	\[\hat{\theta}=\sqrt{\theta^2}=\sqrt{1.64184}=1.28134\,.\]
	And finally, for \(\gamma\)
	\[\hat{\gamma}=E(X)-\theta=7.36-1.28134=6.08\,.\]
	These estimates are not reasonable, as the lower bound \((\gamma)\) has a value higher than the lowest observation. Clearly, the range would need to be adjusted in order ot be useful.
\end{solution}
%\end{ Question 2}

\end{questions}
\end{document}