\documentclass[answers]{exam}
\usepackage[english]{babel}
\usepackage[utf8x]{inputenc}

% Common Packages - Delete Unused Ones %
\usepackage{setspace}
\usepackage{amsmath}
%\usepackage[cache=false]{minted}
\usepackage{graphicx}
\usepackage{caption}
\graphicspath{ {./images/} }
\usepackage{multirow}
% End Packages %

\usepackage{amssymb}
\usepackage[table,xcdraw]{xcolor}

\title{Homework 2}
%\shorttitle{HW2}
\author{Brandon Hosley}
\date{\today}
%\affiliation{STAT 587 - Introduction to Probability and Statistics}

%\abstract{}

\begin{document}
\maketitle
%\raggedbottom
%\singlespacing

\begin{questions}
\question
Let $X$ be a non-negative continuous random variable with CDF $F(x)$ and 
$E[X]<\infty$. Use integration by parts to show that $E[X] = \int_{0}^{\infty}[1−F(x)] dx$.  Note that the function $1−F(x)$ is called the \textbf{survival function}. \\
Remember integration by parts: $\int_{b}^{a} u \frac{dv}{dx} dx = uv a_b^a -\int_{b}^{a}v\frac{du}{dx} dx$.  Which $u$ and $v$ should you select to prove the result above?
	
\begin{solution}[1in]
	\begin{align*}
		\int_{0}^{\infty} xf(x)dx &=  \left. -x( 1-F(x) ) \right|_0^\infty + \int_{0}^{\infty}( 1-F(x) )dx \\
		& \\
		\left. -x( 1-F(x) ) \right|_0^\infty &= \left. xF(x)-x \right|_0^\infty \equiv 0
	\end{align*}
\end{solution}
	
\question 
At a computer store, the annual demand for a particular software package is a discrete random variable $X$. The store owner orders four copies of the package at \$10 per copy and charges customers \$35 per copy. At the end of the year the package is obsolete and the owner loses the investment on unsold copies. The pdf of $X$ is given by the following table:\\
	\begin{tabular}{l|lllll}
		\textbf{x }  & 0   & 1   & 2   & 3   & 4   \\ \hline
		$f(x)$ & 0.1 & 0.3 & 0.3 & 0.2 & 0.1
	\end{tabular}

\begin{solution}
	\begin{align*}
		E[x] &= 35(\sum f(x) x) - 10(4) \\
		&= 35(0*0.1 + 1*0.3 + 2*0.3+ 3*0.2 + 4*0.1) - 40 \\
		&= 35(0 + 0.3 + 0.6 + 0.6 + 0.4) -40 \\
		&= 35(1.9) - 40 \\
		&= 66.5 - 40 \\
		&= 26.5
	\end{align*}
\end{solution}

\question
For the FMGF $G_X^{(r)}(t) = r!(2−t)^{−r−1}$, find $E[X(X-1)(X-2)] =$

\begin{solution}
	$$
	E[X(X-1)(X-2)] = G_X^{(3)}(t) = 3!(2-3)^{-2} = 6(-1)^{-2} = 6
	$$
\end{solution}


\question
A discrete random variable $X$ has a pdf of the form $f(x) = c(8−x)$ for $x = 0,1,2,3,4,5$ and zero otherwise.  Find the CDF $F_X(x)$.

\begin{solution}
	\begin{tabular}{rllllll}
		x:  & 0    & 1     & 2     & 3     & 4     & 5    \\  
		$f(x)$: & 8c   & 7c    & 6c    & 5c    & 4c    & 3c  \hspace{3em} $= 33c$\hspace{2em}$c=1/33$ \\
		& 8/33 & 7/33  & 6/33  & 5/33  & 4/33  & 3/33 \\
		CDF: & 8/33 & 15/33 & 21/33 & 26/33 & 30/33 & 1   
	\end{tabular}
\end{solution}

\question
Let $X$ be continuous with pdf $f(x)=3x^2$ if $0<x<1$ and zero otherwise.  Use Chebyshev's inequality to obtain a lower bound on $P(5/8<X<7/8)$.

\begin{solution}
	\begin{flalign*}
		& \mu =E(x)= \int_{0}^{1}x\cdot3x^2dx = \int_{0}^{1}3x^3dx  =\left. \frac{3}{4}x^4 \right|_0^1 = \mathbf{\frac{3}{4}} & \\
		& \sigma^2 = E(x^2)-\mu^2 = \int_{0}^{1}x^2\cdot3x^2dx - \bigg(\frac{3}{4}\bigg)^2 = \int_{0}^{1}3x^4dx - \frac{9}{16} = \left.\bigg( \frac{3}{5}x^5 - \frac{9}{16}  \bigg)\right|_0^1 = \frac{3}{5}x^5 - \frac{9}{16} = \mathbf{ \frac{3}{80}} & \\
		&  & \\
		& P(5/8 < x < 7/8) =  P(-1/8 < x-\mu < 1/8) & \\
		& \hspace{7em} = P(|x-3/4|\geq1/8) \hspace{2em} k\sigma = 1/8 & \\
		&  & \\
		& P(|x-3/4|\geq1/8) > 1 -\frac{1}{k^2} & \\
		& k = \frac{1}{8} \bigg/ \sqrt{\frac{3}{80}} = \frac{\sqrt{80}}{8\sqrt{3}}& \\
		& k^2 = \frac{80}{(8^2)3} = 5/12 & \\
		& P(|x-3/4|\geq1/8) > 1 -\frac{1}{k^2} = P(|x-3/4|\geq1/8) > 1 -\frac{12}{5}  = -7/5& \\
	\end{flalign*}
\end{solution}

\question
A discrete random variable has pdf $f(x)$. If $f(x)=k(1/2)^x$ for $x=1,2,3$ and zero otherwise, find $k$.

\begin{solution}
	\begin{align*}
		(x) &= k(1/2)^x \\
		1 &= \frac{k}{2} + \frac{k}{4} + \frac{k}{8} \\
		&= \frac{4k + 2k + k}{8} \\
		&= \frac{7}{8}k \\
		k &= \frac{8}{7}
	\end{align*}
\end{solution}


\question
Assume that $X$ is a continuous random variable with MGF $M_X(t)= \frac{e^{−2t}}{(1−t)}$. \\ Find $E[X]$.
 
 \begin{solution}
 	\begin{align*}
 		M_x(t) = \frac{e^{-2t}}{1-t} \\
 		E[x] = M'_x(0) \\
 		\\
 		M'_x(t) &= \frac{d}{dt}\bigg(\frac{e^{-2t}}{1-t}\bigg) \\
 		&= \frac{(1-t)(-2e^{-2t})-e^{-2t}(-1) }{ (1-t)^2} \\
 		&= \frac{(1-t)(-2e^{-2t})+e^{-2t}  }{  (1-t)^2} \\
 		&= \frac{2te^{-2t} -e^{-2t}  }{  (1-t)^2} \\
 		&= \frac{e^{-2t}(2t-1) }{  (1-t)^2} \\
 		\\
 		M'_x(0) &= \frac{e^{0}(-1) }{  (1)^2} \\
 		&= \frac{-1}{1} = -1\\
 	\end{align*}
 	
 	Confirmed by:
 	
	 Referencing Bain and Enqelhardt,
	 For Two-Parameter Exponential function:
	 \begin{align*}
	 	\text{MGF:} &: \frac{e^{nt}}{1-\theta t} \\
	 	E[X] = \mu &: n + \theta \\
	 	\text{Var}[X] = \sigma^2 &: \theta^2
	 \end{align*}
	 $n = -2, \theta = 1, n-\theta = -1$
\end{solution}


\question 
Suppose that $X$ is a discrete random variable with MGF $M_x(t) = (1/8)e^t+(1/4)e^{2t}+(5/8)e^{5t}$.  What is the distribution of $X$?

\begin{solution}
	$$
	M_x(t) = E[e^tx] = \sum_{x\in X}P(X=x)e^{xt}
	$$
	\begin{tabular}{r|lll}
		$x$ & 1 & 2 & 5 \\
		$f(x)$ & 1/8 & 1/4 & 5/8 
	\end{tabular}
\end{solution}

\question
Suppose $E[X]= \mu \text{and Var}[X] = \sigma^2$. Find the approximate mean and variance of $h(x) = e^x$.

\begin{solution}
	\begin{align*}
		E[h(x)] &\approx h(\mu)+\frac{1}{2}h^{(2)}(\mu)\text{ Var}[x] \approx e^\mu+\frac{1}{2}e^\mu\sigma^2 \\
		\text{Var}[h(x)] &\approx (e^\mu)^2\sigma^2 \approx e^{2\mu}\sigma^2
	\end{align*}
\end{solution}

\question
A random variable $X$ has the pdf:
$$
f(x) =
\begin{cases}
	x^2 & \text{if } 0<x\leq1 \\
	2/3 & \text{if } 1<x\leq2 \\
	0 & \text{otherwise} 
\end{cases}
$$ 
Find the median of $X$.

\begin{solution}
	\begin{align*}
		\int_{0}^{1}x^2dx &= \left. \frac{1}{3}x^3 \right|_0^1 = \frac{1}{3} \\
		\int_{1}^{2}\frac{2}{3}dx &= \left. \frac{2}{3}x \right|_1^2 = \frac{2}{3} \\
	\end{align*}
	The two piece-wise functions sum to 1. Since the second function is $>1/2$ the median $x_m$ will occur in that function's domain.
	\begin{align*}
		\left. \frac{2}{3}x \right|_{x_m}^2 &= \frac{1}{2} \\
		\frac{4}{3} - \frac{2x}{3} &= \frac{1}{2} \\
		2x &= 4 - \frac{3}{2} \\
		x &= \frac{5}{4} 
	\end{align*}
\end{solution}

\end{questions}
\end{document}