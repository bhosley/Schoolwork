\documentclass[answers]{exam}
\usepackage[english]{babel}
\usepackage[utf8x]{inputenc}
\usepackage{amsmath,amssymb,amsthm}
\usepackage{booktabs}
\usepackage{hyperref}

\title{STAT 587 - Introduction to Probability and Statistics%
	\\ Final Exam}
\author{Brandon Hosley}
\date{\today}

\usepackage{comment}
\begin{comment}
	Choose two of questions 1 through 4 to complete
	Choose two of questions 5 through 8 to complete
	Choose two of questions 9 through 12 to complete
	Choose two of questions 13 through 15 to complete (only three problems to choose from in this section)
	Complete questions 16 and 17 Submit work for question 18.
\end{comment}

\begin{document}
\maketitle
\begin{questions}

%%%%%%%%%%%%%%%%%%%%%%%%%%%
% 	\begin{ Question 1}	  %
%%%%%%%%%%%%%%%%%%%%%%%%%%%

%\end{ Question 1}

%%%%%%%%%%%%%%%%%%%%%%%%%%%
% 	\begin{ Question 2}	  %
%%%%%%%%%%%%%%%%%%%%%%%%%%%
\setcounter{question}{2}
%\end{ Question 2}

%%%%%%%%%%%%%%%%%%%%%%%%%%%
% 	\begin{ Question 3}	  %
%%%%%%%%%%%%%%%%%%%%%%%%%%%
\question 
The speed of a molecule in a uniform gas at equilibrium is a random variable
\(V\) whose density function is given by
\[f(v)=av^2e^{-bv^2}\text{ for } v>0,\]
where \(a=\sqrt{\frac{2}{\pi}}\left(\frac{m}{kT}\right)^{3/2}\), \(b=\frac{m}{2kT}\) 
and \(k\), \(T\), and \(m\) denote Boltzmann’s constant, the absolute temperature, 
and the mass of the molecule, respectively. 
Find \(E[W]\) the expected value of the kinetic energy of the molecule defined as the 
random variable \(W=mV^2/2\).

\begin{solution}
	\(E[W]=\frac{m}{2}E[V^2]\)
	First, find \(E[V^2]\)
	\begin{align*}
		E[V^2]
		&= \int_{0}^{\infty} v^2\, av^2e^{-bv^2}\, dv \\
		&= a\int_{0}^{\infty} v^4e^{-bv^2}\, dv \\
		\intertext{using an identity from link below,}
			%
		&= a\frac{(3)!!}{2^3b^2}\sqrt{\frac{\pi}{b}} \\
		&= \sqrt{\frac{2}{\pi}}\left(\frac{m}{kT}\right)^{3/2}  \frac{3}{8} \left(\frac{2kT}{m}\right)^2 \sqrt{\pi} \sqrt{\frac{2kT}{m}} \\
		&= 3 \sqrt{\frac{2kT}{m}} \left(\frac{m}{kT}\right)^{3/2} \left(\frac{kT}{m}\right)^2 \\
		&= 3 \left(\frac{m}{kT}\right) \left(\frac{kT}{m}\right)^2 \\
		&= 3 \frac{kT}{m}.
		\intertext{Then substitute, this back in,}
		E[W]
		&= \frac{m}{2}E[V^2] \\
		&= \frac{m}{2} 3 \frac{kT}{m}\\
		&= \frac{3}{2}kT.
	\end{align*}

\href{https://en.wikipedia.org/wiki/List_of_integrals_of_exponential_functions#Definite_integrals}{Referenced link.}
\end{solution}
%\end{ Question 3}

%%%%%%%%%%%%%%%%%%%%%%%%%%%
% 	\begin{ Question 4}	  %
%%%%%%%%%%%%%%%%%%%%%%%%%%%
\question 
A process for refining sugar yields up to 1 ton of pure sugar per day, 
but the actual amount produced, \(Y\), is a random variable 
because of machine breakdowns and other slowdowns. 
Suppose that \(Y\) has density function given by
\[ f(y) = \begin{cases} 2y & \text{for } 0\leq y\leq1\\ 0 & \text{otherwise} \end{cases}\]
The company is paid at the rate of \$300 per ton for the refined sugar, 
but it also has a fixed overhead cost of \$100 per day. 
Thus, the daily profit, in hundreds of dollars, is \(U=3Y-1\). 
Find the probability density function for \(U\).
\begin{solution}
	\begin{align*}
		U &= 3Y-1 \\
		3Y &= U+1 \\
		Y &= (U+1)/3 \\
		f(u) &= (u+1)/3
	\end{align*}
	Checking to confirm that it is a valid distribution,
	\begin{align*}
		\int_{0}^{1} \frac{2}{3}(u+1)\, du 
		= \left. \frac{u^2}{3}+\frac{2u}{3} \right|_{0}^{1}
		= 1.
	\end{align*}
\end{solution}
%\end{ Question 4}


%%%%%%%%%%%%%%%%%%%%%%%%%%%
% 	\begin{ Question 5}  %
%%%%%%%%%%%%%%%%%%%%%%%%%%%

%\end{ Question 5}

%%%%%%%%%%%%%%%%%%%%%%%%%%%
% 	\begin{ Question 6}	  %
%%%%%%%%%%%%%%%%%%%%%%%%%%%

%\end{ Question 6}

\setcounter{question}{6}
%%%%%%%%%%%%%%%%%%%%%%%%%%%
% 	\begin{ Question 7}   %
%%%%%%%%%%%%%%%%%%%%%%%%%%%
\question 
Suppose that \(X_1,\ldots,X_5\overset{iid}{\sim}Uniform(5,\theta_U)\)
and you observe the following values:
\begin{flalign*}
	X_1 = 5.31 && \\
	X_2 = 8.76 && \\
	X_3 = 8.61 && \\
	X_4 = 6.67 && \\
	X_5 = 7.45 && 
\end{flalign*}
The method of moments estimator for \(\theta_U\) is reasonable 
(find the estimator and use it to construct an estimate for \(\theta_U\)
\begin{solution}
	\begin{align*}
		E[X] = \frac{5+\theta_U}{2} =\frac{1}{5}\sum_{i=1}^{5}x_i = 7.36
	\end{align*}
	\begin{align*}
		\operatorname{Var}[X]= \frac{(5+\theta_U)^2}{12} =\frac{1}{5}\sum_{i=1}^{5}(x_i-\bar{x}) =1.64184
	\end{align*}
	Solved by \(E[X]\) we get \(\theta_U=9.72\).
	However, solved by \(\operatorname{Var}[X]\) we get \(\theta_U=9.438703\).

\end{solution}
%\end{ Question 7} 
\clearpage

%%%%%%%%%%%%%%%%%%%%%%%%%%%
% 	\begin{ Question 8}   %
%%%%%%%%%%%%%%%%%%%%%%%%%%%
\question 
Suppose you have iid random variables from a \(Gamma(\alpha=3,\beta)\)
distribution. Find the CRLB for \(\beta\).
\begin{solution}
	\begin{align*}
		E[X^2] = E[X]^2 -\operatorname{Var}[X] = (3\beta)^2 -3\beta^2 = 6\beta^2
	\end{align*}
	\begin{align*}
		\text{LLF} 
		= 2\ln(x)-\left(\frac{x}{\beta}\right)\ln(e) -\ln(\Gamma(3)\beta^3)
		= 2\ln x -\frac{x}{\beta}-\ln(2\beta^3)
	\end{align*}

	\begin{align*}
		\frac{\partial^2}{\partial\beta^2} \left( 2\ln x -\frac{x}{\beta}-\ln(2\beta^3) \right)
		&= \frac{\partial^2}{\partial\beta^2} \left( -\frac{x}{\beta}-\ln(2\beta^3) \right) \\
		&= \frac{\partial}{\partial\beta} \left( \frac{2x}{\beta^2}-\frac{1}{2\beta^3} \right) \\
		&= -\frac{6x}{\beta^3}+\frac{2}{\beta^4}
	\end{align*}

	\begin{align*}
		\frac{1}{nE[LLF]}
		= n\frac{6\beta^2}{2\beta^4}
		= \frac{\beta^2}{3n}
	\end{align*}
\end{solution}
%\end{ Question 8}
	
%%%%%%%%%%%%%%%%%%%%%%%%%%%
% 	\begin{ Question 9}   %
%%%%%%%%%%%%%%%%%%%%%%%%%%%
\setcounter{question}{9}
%\end{ Question 9}

%%%%%%%%%%%%%%%%%%%%%%%%%%%
% 	\begin{ Question 10}  %
%%%%%%%%%%%%%%%%%%%%%%%%%%%
\question 
Suppose that the random variable \(Y\) is an observation from a 
normal distribution with unknown mean \(\mu\) and variance 1. 
Find a 95\% confidence interval for \(\mu\). If \(Y=8.33\)
8.33, what is the upper bound of this interval?
\begin{solution}
	\(95\% \Rightarrow \mu \pm Z_{0.975} \sigma = \mu \pm 1.96 \). \\
	\(Y=8.33 \Rightarrow (6.37,10.29)\). \\
	The upper interval is 10.29
\end{solution}
%\end{ Question 10}

%%%%%%%%%%%%%%%%%%%%%%%%%%%
% 	\begin{ Question 11}  %
%%%%%%%%%%%%%%%%%%%%%%%%%%%
\question 
In laboratory work, it is desirable to run careful checks on the variability of readings 
produced on standard samples. In a study of the amount of calcium in drinking 
water undertaken as part of a water quality assessment, the same standard 
sample was run through the laboratory six times at random intervals. The six 
readings, in parts per million, were 9.54, 9.61, 9.32, 9.48, 9.70, and 9.26. Estimate 
the population variance \(\sigma^2\) for readings on this standard, using a 90\% confidence 
interval. What is the lower bound of this interval?
\begin{solution}
	\(\sigma^2 = \frac{1}{6}\sum_{i=1}^{6}(x_i-9.485)^2 = 0.02379\). \\
	\(\sigma=\sqrt{\sigma^2}= \sqrt{0.02379} =0.154245\). \\
	90\% CI \(\Rightarrow 9.485\pm 0.1542445 Z_{0.95} = 9.485 \pm 0.2573 = (9.2277,9.7423)\).
\end{solution}
%\end{ Question 11}

%%%%%%%%%%%%%%%%%%%%%%%%%%%
% 	\begin{ Question 12}  %
%%%%%%%%%%%%%%%%%%%%%%%%%%%
\setcounter{question}{12}
%\end{ Question 12}

\clearpage

%%%%%%%%%%%%%%%%%%%%%%%%%%%
% 	\begin{ Question 13}  %
%%%%%%%%%%%%%%%%%%%%%%%%%%%
\question 
Consider the biased coin discussed in Example 9.2.5, 
where the probability of a head, \(p\), is known to be 0.2, 0.30, or 0.80. 
The coin is tossed repeatedly, and we let \(X\) be the 
number of tosses required to obtain the first head. For a test of
\(H_0 : p=0.20\), suppose we use a rejection region of the form
\(\{1,2,3,17,\ldots\}\). Find \(P[\text{TI}]=\alpha=\)
\begin{solution}
	\(1-\sum_{n\in RR}(p)(1-p)^{(1-n)} = 1-\sum_{n\in RR}(0.2)(0.80)^{(1-n)}\) \\
	
	Constructing a table similar to Example 9.2.5: \\
	\begin{center}
		\begin{tabular}{c|rrr}
			& \multicolumn{3}{c}{\(p\)} \\ \toprule
			\(n\) & 0.2000 & 0.3000 &  0.8000 \\ \midrule
			1   & 0.2000 & 0.3000 &  0.8000 \\
			2   & 0.1600 & 0.2100 &  0.1600 \\
			3   & 0.1280 & 0.1470 &  0.0320 \\
			4   & 0.1024 & 0.1029 &  0.0064 \\
			5   & 0.0819 & 0.0720 &  0.0013 \\
			6   & 0.0655 & 0.0504 &  0.0003 \\
			7   & 0.0524 & 0.0353 &  0.0001 \\
			8   & 0.0419 & 0.0247 &  0.0000 \\
			9   & 0.0336 & 0.0173 &  0.0000 \\
			10   & 0.0268 & 0.0121 &  0.0000 \\
			11   & 0.0215 & 0.0085 &  0.0000 \\
			12   & 0.0172 & 0.0059 &  0.0000 \\
			13   & 0.0137 & 0.0042 &  0.0000 \\
			14   & 0.0110 & 0.0029 &  0.0000 \\
			15   & 0.0088 & 0.0020 &  0.0000 \\
			16   & 0.0070 & 0.0014 &  0.0000 \\ \bottomrule
		\end{tabular}
	\end{center}

	\(\alpha=0.6185\)

\end{solution}
%\end{ Question 13} 

%%%%%%%%%%%%%%%%%%%%%%%%%%%
% 	\begin{ Question 14}  %
%%%%%%%%%%%%%%%%%%%%%%%%%%%
\setcounter{question}{14}
%\end{ Question 14}
	
%%%%%%%%%%%%%%%%%%%%%%%%%%%
% 	\begin{ Question 15}  %
%%%%%%%%%%%%%%%%%%%%%%%%%%%
\question 
We are interested in testing whether or not a coin is balanced based
 on the number of heads \(Y\) on 36 tosses of the coin. 
 (\(H_0 : p=.5\) versus \(H_a:p\neq.5\)). 
 If we use the rejection region \(|y-18|\geq4\), 
 what is the value of \(\beta\) if \(p=.7\)?
\begin{solution}
	\(p=0.7\)\\
	\(\bar{x}=np=36(0.7)=25.2\) \\
	\(\sigma^2=npq=36(0.21)=7.56\) \\
	\(\frac{24.2-14}{2.75}=4.072\) and \(\frac{24.2-22}{2.75}=1.163\) \\
	\(Z_{4.072}=0.50\) and \(Z_{1.163}=0.377\) \\
	\(0.123\).
\end{solution} 
%\end{ Question 15}

\clearpage

%%%%%%%%%%%%%%%%%%%%%%%%%%%
% 	\begin{ Question 16}  %
%%%%%%%%%%%%%%%%%%%%%%%%%%%
\question 
Recall that for a collection iid of \(Poisson(\gamma)\) random variables
\(X_1,\ldots,X_n\), we rejected the null hypothesis for the likelihood ratio test,
\begin{align*}
	H_0 &: \gamma = \gamma_0 \\
	H_a &: \gamma \neq \gamma_0
\end{align*}
when \(\hat{\gamma}=\frac{1}{n}\sum_{i=1}^{n}X_i\) was such that \(2n(\gamma_0\hat{\gamma}+\hat{\gamma}\log(\frac{\hat{\gamma}}{\gamma_0}))\geq\chi^2_{1-\alpha,1}\).
Using the likelihood ratio, select an analogous rejection region result for 
\(\hat{p}=\frac{1}{n}\sum_{i=1}^{n}X_i/k\), when our observations are instead iid
\(Binom(k,p)\) where the number of trials \(k\) is known and \(n=1\). Please, show your work!
\begin{solution}
	The provided rejection region is \(2n\) times the log likelihood of the gamma function.
	\begin{align*}
		\text{LLF(Binom)}
		&= \ln\left(p^{X_1}(1-p)^{(k-X_1)}\right) \\
		&= \ln\left( \left(\frac{p_0}{\hat{p}}\right)^{X_1} \left(\frac{1-p_0}{1-\hat{p}}\right)^{(k-X_1)}\right) \\
		&= \ln\left(\frac{p_0}{\hat{p}}\right)^{X_1} \ln\left(\frac{1-p_0}{1-\hat{p}}\right)^{(k-X_1)} \\
		&= X_1\ln\left(\frac{p_0}{\hat{p}}\right) + (k-X_1)\ln\left(\frac{1-p_0}{1-\hat{p}}\right) \\
		&\Rightarrow 2\left(X_1\ln\left(\frac{p_0}{\hat{p}}\right) + (k-X_1)\ln\left(\frac{1-p_0}{1-\hat{p}}\right)\right) \\
	\end{align*}
	
\end{solution}
%\end{ Question 16}

%%%%%%%%%%%%%%%%%%%%%%%%%%%
% 	\begin{ Question 17}  %
%%%%%%%%%%%%%%%%%%%%%%%%%%%
\question 
For the following hypothesis test,
\begin{align*}
	H_0 &: \frac{\sigma_1^2}{\sigma_2^2} = 1 \\
	H_a &: \frac{\sigma_1^2}{\sigma_2^2} \neq 1 
\end{align*}
calculate the probability of Type II error at a significance level \(\alpha=0.1\) 
with an alternative variance ratio \(\sigma_a^2=\frac{\sigma_1^2}{\sigma_2^2}=1.5\),
when \(n_1=n_2=42\), where \(n_1\) and \(n_2\) are the number of observations associated 
with sample 1 and sample 2, respectively. Please, show your work!

\begin{solution} \\
	Rejection region with \(\alpha=0.1\): \bigskip\\
	\begin{tabular}{rcl}
		$\leq F_{0.05,41,41}$& & $\geq F_{0.95,41,41}$ \\
		0.59466 & & 1.68164 \\
		x1.5 & & x1.5 \\
		\(p=0.64193\) & & \(p=0.00190\) \\
	\end{tabular} \bigskip\\
	\(p=0.64\)
\end{solution}
%\end{ Question 17}

\end{questions}
\end{document}