\documentclass[answers]{exam}
\usepackage[english]{babel}
\usepackage[utf8x]{inputenc}
\usepackage{amsmath,amssymb,amsthm}

\title{STAT 587 - Introduction to Probability and Statistics%
	\\ Homework 5}
\author{Brandon Hosley}
\date{\today}

\begin{document}
\maketitle
\begin{questions}

%%%%%%%%%%%%%%%%%%%%%%%%%%%
%	\begin{ Question 1}	  %
%%%%%%%%%%%%%%%%%%%%%%%%%%%
\question 
Let X and Y have joint pdf \(f(x,y)=4e^{-2(x+y)}\) for \(0<x,0<y\), and zero otherwise. 
Find the pdf of \(U=X/Y\).
\begin{solution}
	Let \(U = \frac{X}{Y}\) and \(V=Y\),
	\begin{align*}
		f_{U,V}(u,v) &= f_{X,Y}(uv,v) = 4e^{-2(uv+v)} \\
		f_U(u) &= \int_{0}^{\infty} 4e^{-2(uv+v)} \ dv \\
		&= 4 \int_{0}^{\infty} e^{-2(u+1)v} \ dv \\
		\intertext{integrand is the kernel of exponential pdf with $\beta=1/2(u+1)$} 
		%b161 in Statistical inference text
		&= \frac{4}{4(u+1)^2} \\
		&= \frac{1}{(u+1)^2} \hspace{3em}\text{for }0<u
	\end{align*}
\end{solution}
%\end{ Question 1}

%%%%%%%%%%%%%%%%%%%%%%%%%%%
%	\begin{ Question 2}	  %
%%%%%%%%%%%%%%%%%%%%%%%%%%%
\question 
The measured radius of a circle, \(R\), has pdf \(f(r)=6r(1-r)\) for \(0<r<1\). 
Find the distribution of the circumference.
\begin{solution} \\
	\(F_R(r)=3r^2-2r^3 \quad c=2\pi r \Rightarrow r=\frac{c}{2\pi}\) \\ 
	\(0 < c < 2\pi\)
	\[f_C(c) 
	= \frac{d}{dc} F_R(\frac{c}{2\pi})
	= \frac{d}{dc} \bigg(\frac{3c^2}{(2\pi)^2}-\frac{2c^3}{(2\pi)^3} \bigg)
	= \frac{d}{dc} \frac{3c^2(2\pi)-2c^3}{(2\pi)^3}
	= \frac{6c(2\pi)-6c^2}{(2\pi)^3}
	= \frac{6c(2\pi-c)}{(2\pi)^3}
	\]
	
\end{solution}
%\end{ Question 2}

%%%%%%%%%%%%%%%%%%%%%%%%%%%
%	\begin{ Question 3}	  %
%%%%%%%%%%%%%%%%%%%%%%%%%%%
\question 
If \(X \sim \operatorname{BETA}(p,q)\), derive \(E[X^n]\).
\begin{solution}
	Using Pochhammer notation where:
	\[ a^{(b)} = \frac{\Gamma(a+b)}{\Gamma(a)}\]
	Then,
	\begin{align*}
		E[X^n]_{\text{BETA}}
		= \frac{p^{(n)}}{(p+q)^{(n)}}
		= \frac{\Gamma(p+n)}{\Gamma(p)(p+q)^{(n)}}
		= \frac{\Gamma(p+n)\Gamma(p+q)}{\Gamma(p)\Gamma(p+q+n)}.
	\end{align*}
\end{solution}
%\end{ Question 3}

%%%%%%%%%%%%%%%%%%%%%%%%%%%
%	\begin{ Question 4}	  %
%%%%%%%%%%%%%%%%%%%%%%%%%%%
\question 
Let \(Z \sim N(0,1)\). Find \(P[Z^2<3.84]\) using tabled values of the chi-square distribution.
\begin{solution}
	\(Z \sim N(0,1)\) \\
	\(Z^2 = Q\) \\
	\(Q \sim \chi^2_1\) \\
	Table: \(\chi^2=3.84,df=1: \quad P=0.05\) \\
	Positive z: 1-P = 0.95
\end{solution}
%\end{ Question 4}

%%%%%%%%%%%%%%%%%%%%%%%%%%%
%	\begin{ Question 5}	  %
%%%%%%%%%%%%%%%%%%%%%%%%%%%
\question 
In reliability studies, the cumulative hazard function is defined as
\[H(t)=\int_{0}^{t}\frac{f(t)}{1-F(t)} \ dt\]
What distribution does the transformation \(Y=H(T)\), 
follow when is \(T\) a continuous random variable with support on the positive real line?

(Hint: Find a more concise function to describe \(H(t)\) and then use the CDF method.)
\begin{solution}
	Reference: https://web.stanford.edu/~lutian/coursepdf/unit1.pdf \\
	\(h(t) = f(t)/F(t)\) \\
	$Y = H(t) \sim Exp(1)$
\end{solution}
%\end{ Question 5}

%%%%%%%%%%%%%%%%%%%%%%%%%%%
%	\begin{ Question 6}	  %
%%%%%%%%%%%%%%%%%%%%%%%%%%%
\question 
Let \(X_1,X_2,\ldots,X_n\) be a random sample of size \(n\) from a normal distribution, 
\(X_i \sim N(\mu,\sigma^2)\)and define \(U=\sum_{i=1}^{n}X_i\) and \(W=\sum_{i=1}^{n}X_i^2\). 
Let \(c\) be a constant, and define \(Y_i=1\) if \(X_i\leq c\) and zero otherwise. 
Find a statistic that is a function of \(Y_1,Y_2,\ldots,Y_n\) and also unbiased for 
\(F_X(c)=\Phi\bigg(\frac{c-\mu}{\sigma}\bigg)\).
\begin{solution}
	For \(F_X(c) = \Phi\bigg(\frac{c-\mu}{\sigma}\bigg)\),
	\(c-\mu\) moves the c to the 'middle' Arithmetic mean of the distribution.
	Division by $\sigma$ with modify the kurtosis, but for the purposes of the $Y_i$
	function, the spread is not a concern. \\
	Because the values of \(Y_i\) are binary the estimator will be
	\[\frac{1}{n}\sum_{i=1}^{n}Y_i\]
\end{solution}
%\end{ Question 6}

%%%%%%%%%%%%%%%%%%%%%%%%%%%
%	\begin{ Question 7}	  %
%%%%%%%%%%%%%%%%%%%%%%%%%%%
\question 
Let a random variable \(Y=\frac{2}{\beta}X\), where 
\(X \sim \operatorname{GAMMA}(\alpha,\beta)\) for \(\beta\neq2\). 
What distribution does \(Y\) follow?
\begin{solution}
	\(Y=\frac{2}{\beta}X \sim Gamma(\alpha,\beta) \\
	\text{CDF}
	=\bigg(\frac{2}{\beta}\bigg)\frac{1}{\Gamma(\alpha)}\gamma(\alpha,\beta x)
	=\frac{1}{\Gamma(\alpha)} \bigg(\frac{2}{\beta}\bigg) \int_{0}^{\beta}t^{\alpha-1}e^{-t} \ dt
	= \frac{1}{\Gamma(\alpha)} \int_{0}^{1/2}u^{\alpha-1}e^{-u} \ du
	=\frac{1}{\Gamma(\alpha)}\gamma(\alpha,x/2)	\)
	\\
	\(\text{CDF } \chi^2_{2\alpha} = 
		\frac{1}{\Gamma(\alpha)}\gamma(\alpha,x/2)\)
	
\end{solution}
%\end{ Question 7}

%%%%%%%%%%%%%%%%%%%%%%%%%%%
%	\begin{ Question 8}	  %
%%%%%%%%%%%%%%%%%%%%%%%%%%%
\question 
Suppose we have two sets of random samples 
\(X_1,\ldots,X_n \stackrel{\text{iid}}{\sim} N(\mu_1,\sigma_1^2)\) and 
\(Y_1,\ldots,Y_n \stackrel{\text{iid}}{\sim} N(\mu_2,\sigma_21^2)\). 
Let \(T\) be a statistic such that
	\[T = \frac{S_1^2}{S_2^2}  \hspace{0.8\linewidth}\]
where \(S_1^2\) and \(S_2^2\) are the sample variances associated with the first and second random samples, respectively. 
Using the CDF method, construct the cumulative distribution function for \(T\), 
in terms of the cumulative distribution function for the \(F\) distribution 
with degrees of freedom \(v_1\) and \(v_2\), denoted as \(F_{F,v_1,v_2}(x)\).
\begin{solution}
	\begin{align*}
		T &= \frac{S_1^2}{S_2^2} \\
		\intertext{In terms of F-Distro}
		CDF of F_{F,v_1,v_2}(x)
		&= I(\frac{v_1}{2},\frac{v_2}{2}) \\
		&= \int_{0}^{x}t^{v_1/2-1}(1-t)^{v_2/2-1} \ dt \\
		&= \frac{\Gamma(v_1/2)\Gamma(v_2/2)}{\Gamma((v_1+v_2)/2)}
	\end{align*}
\end{solution}
%\end{ Question 8}

%%%%%%%%%%%%%%%%%%%%%%%%%%%
%	\begin{ Question 9}	  %
%%%%%%%%%%%%%%%%%%%%%%%%%%%
\question 
Let \(X\) be a random variable that is uniformly distributed, 
\(X\sim\operatorname{UNIF}(0,1)\). 
Use the CDF technique to determine the pdf of \(Z=1-e^{-X}\).
\begin{solution} \\
	\(e^-X=1-z\)\\
	\(x=-\ln(1-z)\) \\
	\begin{align*}
		F_Z(z) &= F_X(-\ln(1-z)) = -\ln(1-z) \\
		f_Z(z) &= -\ln(1-z) \ dz = \frac{-1}{(1-z)}
		\quad 0<z<(1-e)^-(x=1) = (1-e)^-1
	\end{align*}
\end{solution}
%\end{ Question 9}

%%%%%%%%%%%%%%%%%%%%%%%%%%%
%  \begin{ Question 10}	  %
%%%%%%%%%%%%%%%%%%%%%%%%%%%
\question 
Suppose you have random variables \(X_1,\ldots,X_n \sim \operatorname{BINOM}(k,p)\). Derive the distribution associated with \(T=\sum_{i=1}^{n}X_i\).
\begin{solution}
	Sums of binomial distributions of the same probability are also 
	binomial distributions with that probability.
	In this case, \[Binom(nk,p)\].
\end{solution}
%\end{ Question 10}

\end{questions}
\end{document}