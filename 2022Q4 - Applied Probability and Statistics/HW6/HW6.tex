\documentclass[answers]{exam}
\usepackage[english]{babel}
\usepackage[utf8x]{inputenc}
\usepackage{amsmath,amssymb,amsthm}

\title{STAT 587 - Introduction to Probability and Statistics%
	\\ Homework 6}
\author{Brandon Hosley}
\date{\today}

\begin{document}
\maketitle
\begin{questions}

%%%%%%%%%%%%%%%%%%%%%%%%%%%
%	\begin{ Question 1}	  %
%%%%%%%%%%%%%%%%%%%%%%%%%%%
\question 
Let \(X_1,\dots,X_n\) be a random sample of size \(n\) from an exponential distribution, 
\(X_i\sim\operatorname{Shifted Exponential}(\beta=1,\eta)\). 
A test of \(H_0\ :\,\eta\leq\eta_0\) vs. \(H_a\ :\,\eta>\eta_0\)
is desired, based on \(X_{(1)}\). 
Find a critical region of size \(\alpha\) of the form \(\{x_{(1)}\geq c\}\).
\begin{solution}
	S
\end{solution}
%\end{ Question 1}

%%%%%%%%%%%%%%%%%%%%%%%%%%%
%	\begin{ Question 2}	  %
%%%%%%%%%%%%%%%%%%%%%%%%%%%
\question 
Let \(X_1,\dots,X_n\) be a random sample from an exponential distribution, 
\(X\sim\operatorname{EXP}(\theta)\). 
Find a one-sided lower 95\% confidence limit for
\(P(X>t)=e^{-t/\theta}\) where \(t\) is an arbitrary known value.

%
%  Missing info is in problem 9 or 4?
% 

\begin{solution}
	S
\end{solution}
%\end{ Question 2}

%%%%%%%%%%%%%%%%%%%%%%%%%%%
%	\begin{ Question 3}	  %
%%%%%%%%%%%%%%%%%%%%%%%%%%%
\question 
Let \(X\sim\operatorname{BIN}(n,p)\) and \(\hat{p}=X/n\).
Find a constant \(c\) so that
\(E\left[c\hat{p}(1-\hat{p})\right]=p(1-p)\).
\begin{solution}
	S
\end{solution}
%\end{ Question 3}

%%%%%%%%%%%%%%%%%%%%%%%%%%%
%	\begin{ Question 4}	  %
%%%%%%%%%%%%%%%%%%%%%%%%%%%
\question 
Let \(X_1,\ldots,X_n\) be a random sample from an exponential distribution,
\(X\sim\operatorname{EXP}(\theta)\). 
If \(\bar{x}=17.9\) with \(n=50\), then find a one-sided lower 95\%
confidence limit for \(\theta\).
\begin{solution}
	S
\end{solution}
%\end{ Question 4}

%%%%%%%%%%%%%%%%%%%%%%%%%%%
%	\begin{ Question 5}	  %
%%%%%%%%%%%%%%%%%%%%%%%%%%%
\question 
Consider a random sample of size \(n\) from a Binomial distribution,
\(X_i\sim\operatorname{BIN}(1,p)\). 
Find the CRLB for the variances of unbiased estimators of \(p\).
\begin{solution}
	S
	
	\begin{align*}
		\ln f(x) &= \ln\binom{n}{x}p^y(1-p)^{n-x} \\
		\intertext{using Sterling approximation}
			&= y \ln(p) + (n-x)ln(1-p) \\
	\intertext{First partial,}
		\frac{\partial}{\partial(p)}\ln f(x) &= \frac{x}{p} -\frac{(n-x)}{(1-p)} \\
	\intertext{then the second partial}
		\frac{\partial^2}{\partial(p)^2}\ln f(x)
			&= \frac{-x}{p^2} -\frac{(n-x)}{(1-p)^2} \\
	\intertext{Expected value}
		E\left[\frac{\partial^2}{\partial(p)^2}\ln f(x)\right]
		&= \frac{-np}{p^2} -\frac{(n-np)}{(1-p)^2}
		= \frac{-n}{p} -\frac{n}{(1-p)} 
		= \frac{-n}{p(1-p)} \\
	\intertext{insert into CRLB}
		\frac{1}{nE\left[\frac{\partial^2}{\partial(p)^2}\ln f(x)\right]}
		&= \frac{1}{-n\left(\frac{-n}{p(1-p)}\right)}
		= \frac{p(1-p)}{n^2}
	\end{align*}
\end{solution}
%\end{ Question 5}

%%%%%%%%%%%%%%%%%%%%%%%%%%%
%	\begin{ Question 6}	  %
%%%%%%%%%%%%%%%%%%%%%%%%%%%
\question 
Consider a random sample of size from a uniform distribution,
\(X_i\sim\operatorname{UNIF}(0,\theta)\), \(\theta>0\),
and let \(X_{(n)}\) be the largest order statistic. 
Find constant \(c\) such that \((x_{(n)},cx_{(n)})\) is a \(100(1-\alpha)\%\) 
confidence interval for \(\theta\).
\begin{solution}
	S
\end{solution}
%\end{ Question 6}

%%%%%%%%%%%%%%%%%%%%%%%%%%%
%	\begin{ Question 7}	  %
%%%%%%%%%%%%%%%%%%%%%%%%%%%
\question 
Suppose that 45 workers in a textile mill are selected at random in a study of accident rate. 
The number of accidents per worker is assumed to be Poisson distributed with mean \(\mu\). 
The average number of accidents per worker is \(\bar{x}=1.7\).
Find an approximate one-sided lower 90\% confidence limit for \(\mu\) 
using eq (11.3.20).
\begin{solution}
	S
\end{solution}
%\end{ Question 7}

%%%%%%%%%%%%%%%%%%%%%%%%%%%
%	\begin{ Question 8}	  %
%%%%%%%%%%%%%%%%%%%%%%%%%%%
\question 
Consider the biased coin discussed in Example 9.2.5, where the probability of a head, 
\(p\), is known to be 0.2, 0.30, or 0.80. 
The coin is tossed repeatedly, 
and we let \(X\) be the number of tosses required to obtain the first head. 
For a test of \(H_0\ :\,p=0.30\), suppose we use a rejection region of the form
\(\{1,14,15,\ldots\}\). Find \(P\left[\text{TI}\right]=\alpha=\)
\begin{solution}
	S
	
	%
	%  P[TI] = Probability Type 1 error
	%
\end{solution}
%\end{ Question 8}

%%%%%%%%%%%%%%%%%%%%%%%%%%%
%	\begin{ Question 9}	  %
%%%%%%%%%%%%%%%%%%%%%%%%%%%
\question 
Let \(X_1,\ldots,X_n\) be a random sample from \(\operatorname{EXP}(\theta)\) where \(\hat{\theta_1}=\bar{X}\) and \(\hat{\theta_2}=\frac{n\bar{X}}{(n+1)}\). 
Compare the MSEs of \(\hat{\theta_1}\) and \(\hat{\theta_2}\) for \(n=2\).
\begin{solution}
	\begin{align}
		\hat{\theta_1} &= \bar{X} \\
		\hat{\theta_2} & = \frac{n\bar{X}}{(n+1)} 
		= \frac{2}{3}\bar{X} \quad \text{for } n=2 \\
		\Rightarrow \bar{X} &> \frac{2}{3}\bar{X} \\
		\intertext{thus,}
		\hat{\theta_1} &> \hat{\theta_2}
	\end{align}
	
	
\end{solution}
%\end{ Question 9}

%%%%%%%%%%%%%%%%%%%%%%%%%%%
%  \begin{ Question 10}	  %
%%%%%%%%%%%%%%%%%%%%%%%%%%%
\question 
Consider the biased coin discussed in Example 9.2.5, where the probability of a
head, \(p\), is known to be 0.2, 0.30, or 0.80. 
The coin is tossed repeatedly, and we let \(X\) be the number of tosses 
required to obtain the first head. For a test of \(H_0\ :\,p=0.30\), 
suppose we use a rejection region of the form \(\{1,14,15,\ldots\}\). 
Find \(P\left[\text{TII}\right]\) for \(p=0.2\).
\begin{solution}
	S
	
	%
	%  P[TII] = Probability Type 2 error
	%
\end{solution}
%\end{ Question 10}

\end{questions}
\end{document}