\documentclass[answers]{exam}
\usepackage[english]{babel}
\usepackage[utf8x]{inputenc}
\usepackage{amsmath,amssymb,amsthm}

\title{STAT 587 - Introduction to Probability and Statistics%
	\\ Homework 6}
\author{Brandon Hosley}
\date{\today}

\begin{document}
\maketitle
\begin{questions}

%%%%%%%%%%%%%%%%%%%%%%%%%%%
%	\begin{ Question 1}	  %
%%%%%%%%%%%%%%%%%%%%%%%%%%%
\question 
Q
\begin{solution}
	S
\end{solution}
%\end{ Question 1}

%%%%%%%%%%%%%%%%%%%%%%%%%%%
%	\begin{ Question 2}	  %
%%%%%%%%%%%%%%%%%%%%%%%%%%%
\question 
Q

Missing info is in problem 9
\begin{solution}
	S
\end{solution}
%\end{ Question 2}

%%%%%%%%%%%%%%%%%%%%%%%%%%%
%	\begin{ Question 3}	  %
%%%%%%%%%%%%%%%%%%%%%%%%%%%
\question 
Q
\begin{solution}
	S
\end{solution}
%\end{ Question 3}

%%%%%%%%%%%%%%%%%%%%%%%%%%%
%	\begin{ Question 4}	  %
%%%%%%%%%%%%%%%%%%%%%%%%%%%
\question 
Q
\begin{solution}
	S
\end{solution}
%\end{ Question 4}

%%%%%%%%%%%%%%%%%%%%%%%%%%%
%	\begin{ Question 5}	  %
%%%%%%%%%%%%%%%%%%%%%%%%%%%
\question 
Q
\begin{solution}
	S
\end{solution}
%\end{ Question 5}

%%%%%%%%%%%%%%%%%%%%%%%%%%%
%	\begin{ Question 6}	  %
%%%%%%%%%%%%%%%%%%%%%%%%%%%
\question 
Consider a random sample of size from a uniform distribution,
\(X_i\sim\operatorname{UNIF}(0,\theta)\), \(\theta>0\),
and let \(X_{(n)}\) be the largest order statistic. 
Find constant \(c\) such that \((x_{(n)},cx_{(n)})\) is a \(100(1-\alpha)\%\) 
confidence interval for \(\theta\).
\begin{solution}
	S
\end{solution}
%\end{ Question 6}

%%%%%%%%%%%%%%%%%%%%%%%%%%%
%	\begin{ Question 7}	  %
%%%%%%%%%%%%%%%%%%%%%%%%%%%
\question 
Suppose that 45 workers in a textile mill are selected at random in a study of accident rate. 
The number of accidents per worker is assumed to be Poisson distributed with mean \(\mu\). 
The average number of accidents per worker is \(\bar{x}=1.7\).
Find an approximate one-sided lower 90\% confidence limit for \(\mu\) 
using eq (11.3.20).
\begin{solution}
	S
\end{solution}
%\end{ Question 7}

%%%%%%%%%%%%%%%%%%%%%%%%%%%
%	\begin{ Question 8}	  %
%%%%%%%%%%%%%%%%%%%%%%%%%%%
\question 
Consider the biased coin discussed in Example 9.2.5, where the probability of a head, 
\(p\), is known to be 0.2, 0.30, or 0.80. 
The coin is tossed repeatedly, 
and we let \(X\) be the number of tosses required to obtain the first head. 
For a test of \(H_0\ :\,p=0.30\), suppose we use a rejection region of the form
\(\{1,14,15,\ldots\}\). Find \(P\left[TI\right]=\alpha=\)
\begin{solution}
	S
\end{solution}
%\end{ Question 8}

%%%%%%%%%%%%%%%%%%%%%%%%%%%
%	\begin{ Question 9}	  %
%%%%%%%%%%%%%%%%%%%%%%%%%%%
\question 
Let \(X_1,\ldots,X_n\) be a random sample from \(\text{\textbf{EXP}}(\theta)\) where \(\hat{\theta}_1=\bar{X}\) and \(\hat{\theta}_2=\frac{n\bar{X}}{(n+1)}\). 
Compare the MSEs of \(\hat{\theta_1}\) and \(\hat{\theta_1}\) for \(n=2\).
\begin{solution}
	S
\end{solution}
%\end{ Question 9}

%%%%%%%%%%%%%%%%%%%%%%%%%%%
%  \begin{ Question 10}	  %
%%%%%%%%%%%%%%%%%%%%%%%%%%%
\question 
Consider the biased coin discussed in Example 9.2.5, where the probability of a
head, \(p\), is known to be 0.2, 0.30, or 0.80. 
The coin is tossed repeatedly, and we let \(X\) be the number of tosses 
required to obtain the first head. For a test of \(H_0\ :\,p=0.30\), 
suppose we use a rejection region of the form \(\{1,14,15,\ldots\}\). 
Find \(P\left[TII\right]\) for \(p=0.2\).
\begin{solution}
	S
\end{solution}
%\end{ Question 10}

\end{questions}
\end{document}