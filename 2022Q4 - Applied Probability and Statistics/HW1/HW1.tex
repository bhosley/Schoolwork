% !TeX document-id = {8a6a6473-5fee-4dfd-a8cb-085b30fb17aa}
\documentclass[a4paper,man,natbib]{apa6}
\usepackage[english]{babel}
\usepackage[utf8x]{inputenc}

% Common Packages - Delete Unused Ones %
\usepackage{setspace}
\usepackage{amsmath}
%\usepackage[cache=false]{minted}
\usepackage{graphicx}
\usepackage{caption}
\graphicspath{ {./images/} }
\usepackage{multirow}
% End Packages %

\usepackage{amssymb}
\usepackage[table,xcdraw]{xcolor}

\title{Homework 1}
\shorttitle{HW1}
\author{Brandon Hosley}
\date{\today}
\affiliation{STAT 587 - Introduction to Probability and Statistics}

%\abstract{}

\begin{document}
\maketitle
\raggedbottom
\singlespacing

\subsection{Problem 1}
\emph{In how many ways can three boys and three girls sit in a row if boys and girls must alternate?}\vspace{1em}

The alternation constrain implies that all sample points will comply to one of two patterns:
$$
{b,g,b,g,b,g} \text{ or } {g,b,g,b,g,b}.
$$
So the number of sample points will be equal to 
$2\times\text{(Permutations of boys)}\times\text{(Permutations of Girls)}$ 
$=2(b!)(g!) = 2(3!)(3!) = 72$


\subsection{Problem 2}
\emph{$k\binom{n}{k}$}
\doublespacing
\begin{align*}
	k \binom{n}{k} &=  k \frac{n!}{k!(n-k)!} \\
	&= \frac{n!}{(k-1)!(n-k)!} \\
	&= n\frac{(n-1)!}{(k-1)!(n-k)!} \\
	&= n\frac{(n-1)!}{(k-1)!(n-k+(1-1))!} \\
	&= n\frac{(n-1)!}{(k-1)!(n-1-k+1)!} \\
	&= n\frac{(n-1)!}{(k-1)!((n-1)-(k-1))!} \\
	&= n\binom{n-1}{k-1}
\end{align*}
\singlespacing

\subsection{Problem 3}
\emph{When an experiment is performed, one and only one of the events $A_1,A_2$ or $A_3$ will occur. 
	Find $P(A_1), P(A_2)$ and $P(A_3)$ under the assumption: $P(A_3) = \frac{1}{2}$ and $P(A_1) = P(A_2)=$}

\begin{align*}
	P(A_1) + P(A_2) + P(A_3) &= 1 \\
	P(A_3) &= 1/2 \\
	P(A_1) + P(A_2) &= 1/2 \\
	P(A_1) &= P(A_2) \\
	P(A_1), P(A_2) &= 1/4 \\
\end{align*}

\subsection{Problem 4}
\emph{There are four basic blood groups: $O, A, B,$ and $AB$. 
	Ordinarily, anyone can receive the blood of a donor from their own group. Also, anyone can receive the blood of a donor from the $O$ group, and any of the four types can be used by a recipient from the $AB$ group. All other possibilities are undesirable. An experiment consists of drawing a pint of blood and determining its type for each of the next two donors who enter a blood bank. Suppose, for a particular group, the four blood types are equally likely to occur. Compute the probability that the second donor can receive blood from the first donor.} \vspace{1em}

Sample Space: \\
\begin{tabular}{|c|c|c|c|c|}
	\hline
	& Donor2 O & Donor2 A & Donor2 B & Donor2 AB \\
	\hline Donor1 O   & \checkmark &  \checkmark & \checkmark  & \checkmark \\
	\hline Donor1 A   & X & \checkmark  & X & \checkmark \\
	\hline Donor1 B   & X & X & \checkmark  & \checkmark  \\
	\hline Donor1 AB & X  & X & X &  \checkmark  \\
	\hline
\end{tabular}

	In the described situation the probability that Donor2 can receive from Donor1 appropriately is
	$\frac{9}{16} = 0.5625$

\subsection{Problem 5}
\emph{  $\sum_{i=0}^{n} \binom{2n}{2i} =$  }\vspace{1em}

\begin{align*}
	\sum_{i=0}^{n}\binom{2n}{2i}
	&= \sum_{i=0}^{n}\bigg[ \binom{2n-1}{2i-1}+\binom{2n-1}{2i} \bigg] &\text{Pascal's Identity}\\
	&= \sum_{i=-1}^{2n}\binom{2n-1}{i}  & \text{Condense using even-odd summation}\\
	&= \sum_{i=0}^{2n-1}\binom{2n-1}{i}  &\\
	&= 2^{2n-1}  &\text{Binomial Identity}
\end{align*}

\subsection{Problem 6}
\emph{A laboratory test for steroid use in professional athletes has detection rates given in the table.
	If the rate of steroid use among professional athletes is 1 in 50, what is the probability that a professional athlete chosen at random will have a negative test result for steroid use? (Hint: these are conditional probabilities, e.g.,} 
	$P(+|  \text{"use steriods"} )=0.9$
	
	\begin{align*}
		P(-) &= P(-|No)P(No) \\
		&= P(-|No)P(Yes)^c \\
		&= P(0.99)(0.98) \\
		&= 0.9702
	\end{align*}
	
\vspace{1em}

\subsection{Problem 7}
\emph{How many ways can 10 students be lined up to get on a bus if a particular pair of students refuse to follow each other in line?}\vspace{1em}

This can be represented as the permutation of all ten students subtracting the excluded events:
$$ P^{10}_{10} - (\text{Excluded Events})$$
The excluded events are any in which students a and b $S_a, S_b$ are placed immediately next to each other. Which may be represented by each position they can be inserted into a line of the other 8 students:
$$ (n_i, S_a, S_b, n_{8-i})\text{ and  }(n_i, S_b, S_a, n_{8-i}).$$
This represents 18 positions in which  $S_a, S_b$ may be inserted into each permutation of the other 8 students. Or;
$$P^{10}_{10} - (18)P^{8}_{8}  = 10! - (18)8! = 3628800 - (18)40320 = 3628800 - 725760 = 2903040$$

\subsection{Problem 8}
\emph{ 
	In a marble game a shooter may $(A)$ miss, $(B)$ hit one marble out and stick in the ring, or $(C)$ hit one marble out and leave the ring. If $B$ occurs, the shooter shoots again. Let  $P(A)=p_1, P(B)=p_2$ and $P(C)=p_3$ where these probabilities do not change from shot to shot. Show that the probability of getting one marble is greater than the probability of getting zero marbles if $p_1 < \frac{1-p_2}{2-p_2}$. }\vspace{1em}

\begin{align*}
	P(1) &> P(0) &\text{}\\
	p_2p_1 + p_3 &> p_1 &\text{Lose marbles from miss, }p_1\\
	p_2p_1 + 1- p_2 - p_1 &> p_1 &\text{Expand: }p_3 = 1- p_2 - p_1\\
	1-p_2  &> p_1 + p_1 - p_2p_1 & +p_1, -p_2p_1 \text{ to both sides.} \\
	1-p_2  &> p_1(2-p_2) & \text{ Factor } p_1 + p_1 - p_2p_1 = p_1(2-p_2) \\
	\frac{1-p_2}{2-p_2}  &> p_1 & \text{ Divide by } p_1(2-p_2) \\
	p_1  &< \frac{1-p_2}{2-p_2}   & \text{ Flip } \\
\end{align*}

\subsection{Problem 9}
\emph{Suppose $P(A_i )= \frac{1}{(3+i)}$ for $i=1,2,3,4$. 
	Find an upper bound for $P(A_1 \cup A_2 \cup A_3 \cup A_4)$. }\vspace{1em}

Assuming that $P(A_i )$ are independent:
\begin{align*}
	P(A_1 \cup A_2 \cup A_3 \cup A_4)
	&= 1-P((A_1 \cup A_2 \cup A_3 \cup A_4)^c) \\
	&= 1-P(A_1^c \cap A_2^c \cap A_3^c \cap A_4^c) \\
	&= 1-P(A_1^c)\cap P(A_2^c)\cap P(A_3^c) \cap P(A_4^c) \\
	&= 1-(3/4)(4/5)(5/6)(6/7) \\
	&= 1- 0.428571428571 \\
	&= 0.571428571429
\end{align*}

\subsection{Problem 10}
\emph{A balanced coin is tossed four times. In your work, list the possible outcomes. Compute the probability of getting exactly three heads:} \vspace{1em}

\begin{tabular}{|c|cccccccc|cccccccc|}
	\hline
	1 &
	\multicolumn{8}{c|}{H} &
	\multicolumn{8}{c|}{T} \\ \hline
	2 &
	\multicolumn{4}{c|}{H} &
	\multicolumn{4}{c|}{T} &
	\multicolumn{4}{c|}{H} &
	\multicolumn{4}{c|}{T} \\ \hline
	3 &
	\multicolumn{2}{c|}{H} &
	\multicolumn{2}{c|}{T} &
	\multicolumn{2}{c|}{H} &
	\multicolumn{2}{c|}{T} &
	\multicolumn{2}{c|}{H} &
	\multicolumn{2}{c|}{T} &
	\multicolumn{2}{c|}{H} &
	\multicolumn{2}{c|}{T} \\ \hline
	4 &
	\multicolumn{1}{c|}{H} &
	\multicolumn{1}{c|}{\cellcolor[HTML]{34FF34}T} &
	\multicolumn{1}{c|}{\cellcolor[HTML]{34FF34}H} &
	\multicolumn{1}{c|}{T} &
	\multicolumn{1}{c|}{\cellcolor[HTML]{34FF34}H} &
	\multicolumn{1}{c|}{T} &
	\multicolumn{1}{c|}{H} &
	T &
	\multicolumn{1}{c|}{\cellcolor[HTML]{34FF34}H} &
	\multicolumn{1}{c|}{T} &
	\multicolumn{1}{c|}{H} &
	\multicolumn{1}{c|}{T} &
	\multicolumn{1}{c|}{H} &
	\multicolumn{1}{c|}{T} &
	\multicolumn{1}{c|}{H} &
	T \\ \hline
\end{tabular}

\vspace{1em}
The probability of heads landing exactly 3 times is $4/16 = 0.25$

\bibliographystyle{apacite}
\bibliography{} %link to relevant .bib file
\end{document}