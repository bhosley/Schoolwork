% !TeX document-id = {8a6a6473-5fee-4dfd-a8cb-085b30fb17aa}
\documentclass[a4paper,man,natbib]{apa6}
\usepackage[english]{babel}
\usepackage[utf8x]{inputenc}

% Common Packages - Delete Unused Ones %
\usepackage{setspace}
\usepackage{amsmath}
%\usepackage[cache=false]{minted}
\usepackage{graphicx}
\usepackage{caption}
\graphicspath{ {./images/} }
\usepackage{multirow}
% End Packages %

\usepackage{amssymb}
\usepackage[table,xcdraw]{xcolor}

\title{Homework 1}
\shorttitle{HW1}
\author{Brandon Hosley}
\date{\today}
\affiliation{STAT 587 - Introduction to Probability and Statistics}

%\abstract{}

\begin{document}
\maketitle
\raggedbottom
\singlespacing

\subsection{Problem 1}
\emph{In how many ways can three boys and three girls sit in a row if boys and girls must alternate?}

\subsection{Problem 2}
\emph{$k\binom{n}{k}$}
\doublespacing
\begin{align*}
	k \binom{n}{k} &=  k \frac{n!}{k!(n-k)!} \\
	&= \frac{n!}{(k-1)!(n-k)!} \\
	&= n\frac{(n-1)!}{(k-1)!(n-k)!} \\
	&= n\frac{(n-1)!}{(k-1)!(n-k+(1-1))!} \\
	&= n\frac{(n-1)!}{(k-1)!(n-1-k+1)!} \\
	&= n\frac{(n-1)!}{(k-1)!((n-1)-(k-1))!} \\
	&= n\binom{n-1}{k-1}
\end{align*}
\singlespacing

\subsection{Problem 3}
\emph{When an experiment is performed, one and only one of the events $A_1,A_2$ or $A_3$ will occur. 
	Find $P(A_1), P(A_2)$ and $P(A_3)$ under the assumption: $P(A_3) = \frac{1}{2}$ and $P(A_1) = P(A_2)=$}

\begin{align*}
	P(A_1) + P(A_2) + P(A_3) &= 1 \\
	P(A_3) &= 1/2 \\
	P(A_1) + P(A_2) &= 1/2 \\
	P(A_1) &= P(A_2) \\
	P(A_1), P(A_2) &= 1/4 \\
\end{align*}

\subsection{Problem 4}
\emph{There are four basic blood groups: $O, A, B,$ and $AB$. 
	Ordinarily, anyone can receive the blood of a donor from their own group. Also, anyone can receive the blood of a donor from the $O$ group, and any of the four types can be used by a recipient from the $AB$ group. All other possibilities are undesirable. An experiment consists of drawing a pint of blood and determining its type for each of the next two donors who enter a blood bank. Suppose, for a particular group, the four blood types are equally likely to occur. Compute the probability that the second donor can receive blood from the first donor.} \vspace{1em}

Sample Space: \\
\begin{tabular}{|c|c|c|c|c|}
	\hline
	& Donor2 O & Donor2 A & Donor2 B & Donor2 AB \\
	\hline Donor1 O   & \checkmark &  \checkmark & \checkmark  & \checkmark \\
	\hline Donor1 A   & X & \checkmark  & X & \checkmark \\
	\hline Donor1 B   & X & X & \checkmark  & \checkmark  \\
	\hline Donor1 AB & X  & X & X &  \checkmark  \\
	\hline
\end{tabular}

	In the described situation the probability that Donor2 can receive from Donor1 appropriately is
	$\frac{9}{16} = 0.5625$

\subsection{Problem 5}
\emph{}

\subsection{Problem 6}
\emph{}

\subsection{Problem 7}
\emph{}

\subsection{Problem 8}
\emph{}

\subsection{Problem 9}
\emph{}

\subsection{Problem 10}
\emph{A balanced coin is tossed four times. In your work, list the possible outcomes. Compute the probability of getting exactly three heads:} \vspace{1em}

\begin{tabular}{|c|cccccccc|cccccccc|}
	\hline
	1 &
	\multicolumn{8}{c|}{H} &
	\multicolumn{8}{c|}{T} \\ \hline
	2 &
	\multicolumn{4}{c|}{H} &
	\multicolumn{4}{c|}{T} &
	\multicolumn{4}{c|}{H} &
	\multicolumn{4}{c|}{T} \\ \hline
	3 &
	\multicolumn{2}{c|}{H} &
	\multicolumn{2}{c|}{T} &
	\multicolumn{2}{c|}{H} &
	\multicolumn{2}{c|}{T} &
	\multicolumn{2}{c|}{H} &
	\multicolumn{2}{c|}{T} &
	\multicolumn{2}{c|}{H} &
	\multicolumn{2}{c|}{T} \\ \hline
	4 &
	\multicolumn{1}{c|}{H} &
	\multicolumn{1}{c|}{\cellcolor[HTML]{34FF34}T} &
	\multicolumn{1}{c|}{\cellcolor[HTML]{34FF34}H} &
	\multicolumn{1}{c|}{T} &
	\multicolumn{1}{c|}{\cellcolor[HTML]{34FF34}H} &
	\multicolumn{1}{c|}{T} &
	\multicolumn{1}{c|}{H} &
	T &
	\multicolumn{1}{c|}{\cellcolor[HTML]{34FF34}H} &
	\multicolumn{1}{c|}{T} &
	\multicolumn{1}{c|}{H} &
	\multicolumn{1}{c|}{T} &
	\multicolumn{1}{c|}{H} &
	\multicolumn{1}{c|}{T} &
	\multicolumn{1}{c|}{H} &
	T \\ \hline
\end{tabular}

\vspace{1em}
The probability of heads landing exactly 3 times is $4/16 = 0.25$

\bibliographystyle{apacite}
\bibliography{} %link to relevant .bib file
\end{document}