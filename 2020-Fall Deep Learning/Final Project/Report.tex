\documentclass[]{article}
\usepackage[english]{babel}
\usepackage{amsmath}
\usepackage{framed}
\usepackage[hypcap=false]{caption}
\usepackage{forest}
\usepackage{multicol}
\graphicspath{ {./images/} }
\usepackage{graphicx}
\usepackage{hyperref}
\hypersetup{
	hidelinks
	}

\title{CSC 535: Deep Learning \\ Final Project: \\ }
\author{Brandon Hosley}
\date{\today}

\begin{document}
	\maketitle
	\clearpage
	
\begin{abstract}
	
\end{abstract}
	
\section{Introduction} 

Dataset source:
Available here: https://vision.eng.au.dk/plant-seedlings-dataset/
\cite{Giselsson2017}
The Plant Seedlings Dataset contains images of approximately 960 unique plants belonging to 12 species at several growth stages.
It comprises annotated RGB images with a physical resolution of roughly 10 pixels per mm.
The database have been recorded at Aarhus University Flakkebjerg Research station in a collaboration between University of Southern Denmark and Aarhus University.


\includegraphics[width=\linewidth]{DataDistribution}
\includegraphics[width=\linewidth]{ImageGrid}

Problem: Uneven data-sets have a tendency to mis-classify minority sample type as majority types.
Solutions: Undersampling or oversampling

imblearn library

ADASYN: Adaptive Synthetic. sampling
\cite{Bai2008} 
A type of resampling in which the more difficult minority samples are duplicated,
the more difficult learning samples decreases the over-all bias, improves generalization and decreases erroneous classification.

SMOTE: Synthetic Minority Over-sampling Technique.
\cite{Bowyer2011} shows improved results over under-sampling (ROC)
under numerous types of classification problems smote decreases classification errors.

\section{Important Ideas}
	\subsection{}
	


	
\section{Conclusion}

\clearpage
\bibliographystyle{IEEEtran}
\bibliography{\jobname}
\end{document}