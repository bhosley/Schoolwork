\documentclass[]{article}
\usepackage[english]{babel}
\usepackage{amsmath}
\usepackage{framed}
\usepackage[hypcap=false]{caption}
\usepackage{forest}
\usepackage{multicol}
\graphicspath{ {./images/} }
\usepackage{graphicx}
\usepackage{hyperref}
\hypersetup{
	hidelinks
	}

\title{CSC 535: Deep Learning \\ Final Project: \\ }
\author{Brandon Hosley}
\date{\today}

\begin{document}
	\maketitle
	\clearpage
	
\begin{abstract}
	
\end{abstract}
	
\section{Introduction} 

Whether they are weeds invading cultivated land or potentially harmful species growing in pasture, control of unwanted flora is a paramount issue within agriculture.
Weeds are shown to be the greatest factor in decreased crop yield when compared to pests, and disease.
\cite{Oerke2006}
\cite{Rao2000}
\cite{Gianessi2007}
Kraehmer and Baur \cite{WeedAnatomy2013}
suggest that global crop loss and weed control measures each cost in the range of billions of dollars.
Specifically, global costs are estimated at 
\$40 Billion \cite{Monaco2002} for crop loss and 
\$30 Billion \cite{Lawes2008} for weed prevention.

A multitude of mitigation methods are available; 
the primary methods are manual removal and application of herbicides.
The easier and more cost effective of the two is a general application of herbicide to cultivated land.
However, general application of herbicides raises significant concerns regarding negative environmental or health effects. 
\cite{Sopena2009}






Interest in applying computer vision to the task of plant identification has led to the production of numerous datasets of plant images.
\cite{Sudars2020}




Dataset source:
Available here: https://vision.eng.au.dk/plant-seedlings-dataset/
\cite{Giselsson2017}
The Plant Seedlings Dataset contains images of approximately 960 unique plants belonging to 12 species at several growth stages.
It comprises annotated RGB images with a physical resolution of roughly 10 pixels per mm.
The database have been recorded at Aarhus University Flakkebjerg Research station in a collaboration between University of Southern Denmark and Aarhus University.


\includegraphics[width=\linewidth]{DataDistribution}
\includegraphics[width=\linewidth]{ImageGrid}

Problem: Uneven data-sets have a tendency to mis-classify minority sample type as majority types.
Solutions: Undersampling or oversampling

Original Solutions:

imblearn library

ADASYN: Adaptive Synthetic. sampling
\cite{Bai2008} 
A type of resampling in which the more difficult minority samples are duplicated,
the more difficult learning samples decreases the over-all bias, improves generalization and decreases erroneous classification.

SMOTE: Synthetic Minority Over-sampling Technique.
\cite{Bowyer2011} shows improved results over under-sampling (ROC)
under numerous types of classification problems smote decreases classification errors.

Actual solution:

The original 5539 sample size was large enough to repeatedly crash the Google Colab Kernel during pre-processing for the models. Successive attempts to pre-process a smaller data set eventually led to the final volume of 3000 samples. This number pulls 250 samples from each category, less than the minimum present in any one category. This limitation in the working environment effectively forced under-sampling and solved the imbalance problem.



MODELS:

% https://www.kaggle.com/gaborfodor/seedlings-pretrained-keras-models/data
Xception \cite{Chollet2016}
was selected as the highest performing image classification model available in the Keras application library

?? Using a logistic regression to fit outputs of Xception to the actual results 
fit is solved using the L-BFGS-B – Software for Large-scale Bound-constrained Optimization \cite{Zhu1997}

MobileNetV2 \cite{Howard2017} \cite{Sandler2019}
was selected for testing purposes as a number of conceivable applications of this type of identification would benefit from deployment to edge computing devices or small mobile units; this would be easier/more accessible with a smaller/lighter model type.
Mobilenet was designed with the small, efficient, low power deployment in mind 

\section{Important Ideas}
	\subsection{}
	


	
\section{Conclusion}

\clearpage
\bibliographystyle{IEEEtran}
\bibliography{\jobname}
\end{document}